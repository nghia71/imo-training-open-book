\documentclass[../03-arithmetic-functions.tex]{subfiles}

\begin{document}

\begin{example*}[\gls{ROU 2015 TST}/D2/P1]\label{example:ROU-2015-TST-D2-P1}[\textbf{\nameref{definition:25M}}]
    Cho \( a \in \mathbb{Z} \) và \( n \in \mathbb{N}_{>0} \). Chứng minh rằng:
    \[
        \sum_{k=1}^{n} a^{\gcd(k,n)}
    \]
    luôn chia hết cho \( n \), trong đó \( \gcd(k,n) \) là ước chung lớn nhất của \( k \) và \( n \).
\end{example*}

\begin{story*}
    Ta nhóm các số \( k \) có cùng \(\gcd(k,n)=d\) và viết tổng dưới dạng \(\sum_{d \mid n} \phi(d) \, a^{n/d}\). Sau đó, áp dụng định lý Euler cho các \(n = p^s\) và sử dụng nguyên lý đồng dư Trung Hoa cho trường hợp tổng quát khi \(n\) có ít nhất hai thừa số nguyên tố.
\end{story*}

\bigbreak

\begin{soln}
    Ta chứng minh khẳng định sau.
    \begin{claim*}
        Nếu \( p \) là số nguyên tố và \( \gcd(a,p)=1 \), thì:
        \[
            a^{p^k} \equiv a^{p^{k-1}} \Mod{p^k}.
        \]
    \end{claim*}

    \begin{subproof}
        Theo định lý Euler, \( a^{\phi(p^k)} \equiv 1 \Mod{p^k}\), với \(\phi(p^k)=p^{k-1}(p-1)\). Từ đó:
        \[
            a^{p^k} = a^{p \cdot p^{k-1}} \equiv a^{p^{k-1}} \Mod{p^k}.
        \]
    \end{subproof}

    \textit{Trường hợp 1:} \( n = p^s \), với \( p \) nguyên tố.  
    Từ \nameref{theorem:gcd-power-sum},
    \[
        \sum_{k=1}^{p^s} a^{\gcd(k,p^s)} = \sum_{d \mid p^s} \phi(d) \, a^{p^s/d}.
    \]
    Các ước \( d \) gồm \( p^0, p^1, \dots, p^s \), do đó:
    \[
        S = a^{p^s} + (p-1) \, a^{p^{s-1}} + (p^2 - p) \, a^{p^{s-2}} + \dots + (p^s - p^{s-1}) \, a.
    \]
    Ta chứng minh bằng quy nạp theo \( s \) rằng \( p^s \mid S \).  
    \(\bullet\) Cơ sở \( s = 1 \): \( a^p + (p-1) \, a = p \, a \equiv 0 \Mod{p} \).  
    \(\bullet\) Giả thiết quy nạp và khẳng định cho thấy mỗi bước đều bảo toàn tính chia hết \( p^s \).  

    \textit{Trường hợp 2:} \( n \) có ít nhất hai thừa số nguyên tố (một dạng “tổng quát”).  
    Giả sử \( n = p^s m \) với \( \gcd(p,m) = 1 \). Theo \nameref{theorem:gcd-power-sum},
    \[
        \sum_{d \mid n} \phi(d) \, a^{n/d} = \sum_{\substack{d \mid n \\ p \nmid d}} \phi(d) \, a^{n/d} + \sum_{\substack{d \mid n \\ p \mid d}} \phi(d) \, a^{n/d}.
    \]
    Phần đầu chia hết cho \( m = n/p^s \) (theo giả thiết quy nạp với số nhỏ hơn), phần sau chia hết cho \( p^s \) (theo bước lũy thừa nguyên tố). Cuối cùng, vì \( \gcd(m, p^s) = 1 \), suy ra \( n \mid \sum_{k=1}^n a^{\gcd(k,n)} \) nhờ nguyên lý đồng dư Trung Hoa.

    Kết luận:
    \[
        n \mid \sum_{k=1}^{n} a^{\gcd(k,n)}.
    \]
\end{soln}

\footnotetext{\href{https://artofproblemsolving.com/community/c6h1097351p4930928}{Dựa theo lời giải của \textbf{andria}.}}

\end{document}