\documentclass[../03-arithmetic-functions.tex]{subfiles}

\begin{document}

\begin{exercise*}[\gls{TWN 2015 TST}2/Q2/P1]\label{example:GBR-2015-TST2-Q2-P3}[\textbf{\nameref{definition:20M}}]
    Với mỗi số nguyên dương \( n \), định nghĩa:
    \[
        a_n = \sum_{k=1}^{\infty} \left\lfloor \frac{n + 2^{k-1}}{2^k} \right\rfloor,
    \]
    trong đó \( \left\lfloor x \right\rfloor \) là phần nguyên của \( x \), tức là số nguyên lớn nhất không vượt quá \( x \).
\end{exercise*}

\begin{remark*}
    Thử viết \( n \) trong cơ số 2 rồi phân tích biểu thức \( \left\lfloor \tfrac{n + 2^{k-1}}{2^k} \right\rfloor \).  
    Có thể tách từng chữ số nhị phân và xem cách các số hạng góp phần vào tổng.
\end{remark*}

\end{document}