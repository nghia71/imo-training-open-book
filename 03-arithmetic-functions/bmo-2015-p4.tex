\documentclass[../03-arithmetic-functions.tex]{subfiles}

\begin{document}

\begin{example*}[\gls{BMO 2015}/P4]\label{example:BMO-2015-P4}[\textbf{\nameref{definition:20M}}]
    Chứng minh rằng trong bất kỳ dãy \( 20 \) số nguyên dương liên tiếp nào cũng tồn tại một số nguyên \( d \) sao cho với mọi số nguyên dương \( n \),
    bất đẳng thức sau luôn đúng:
    \[
        n \sqrt{d} \cdot \left\{n \sqrt{d} \right\} > \frac{5}{2},
    \]
    trong đó \( \left\{ x \right\} \) ký hiệu phần thập phân của \( x \), tức là \( \left\{ x \right\} = x - \lfloor x \rfloor \).
\end{example*}

\begin{story*}
    Ta chọn \( d \) thuộc dạng \( 5(4k + 3) \) trong dãy 20 số nguyên liên tiếp. Dựa vào tính chất \( p \equiv -1 \Mod{4} \) đối với một ước nguyên tố của \( 4k+3 \), ta chứng minh rằng mỗi \( n\sqrt{d} \) có phần thập phân đủ lớn để \( n\sqrt{d} \cdot \{n\sqrt{d}\} > \tfrac{5}{2} \). Kỹ thuật chính là sử dụng chia hết theo mô-đun 4 và 5 để loại trừ những khả năng nhỏ hơn.
\end{story*}

\bigbreak

\begin{soln}\footnotemark
    Trong 20 số nguyên liên tiếp luôn tồn tại một số có dạng \( 20k + 15 = 5(4k + 3) \). Ta sẽ chứng minh rằng \( d = 5(4k + 3) \) thỏa mãn yêu cầu bài toán.

    Vì \( d \equiv -1 \Mod{4} \), nên \( d \) không phải là một số chính phương. Với mọi \( n \in \mathbb{N} \), tồn tại một số nguyên \( a \) sao cho:
    \[
        a < n\sqrt{d} < a + 1 \implies a^2 < n^2 d < (a + 1)^2.
    \]

    Ta sẽ chứng minh rằng \( n^2 d \geq a^2 + 5 \). Thật vậy: ta sử dụng \nameref{theorem:prime-divisor-4k3}.

    Gọi \( p \mid (4k + 3) \) sao cho \( p \equiv -1 \Mod{4} \). Với dạng này, các số \( a^2 + 1 \) và \( a^2 + 4 \) không chia hết cho \( p \). Do \( p \mid n^2 d \), suy ra:
    \[
        n^2 d \ne a^2 + 1, \quad n^2 d \ne a^2 + 4.
    \]

    Mặt khác, vì \( 5 \mid n^2 d \) và \( 5 \nmid a^2 + 2, \quad 5 \nmid a^2 + 3 \), ta có:
    \[
        n^2 d \ne a^2 + 2, \quad n^2 d \ne a^2 + 3, \quad  n^2 d > a^2 \implies n^2 d \geq a^2 + 5.
    \]

    Do đó:
    \[
        n\sqrt{d} \cdot \left\{n\sqrt{d}\right\}
        = n\sqrt{d}(n\sqrt{d} - a)
        = n^2 d - a n\sqrt{d}.
    \]

    Ta đánh giá:
    \[
        n^2 d \geq a^2 + 5, \quad n\sqrt{d} < a + 1 \implies a n\sqrt{d} < a(a + 1).
    \]

    Vì \( a^2 + 5 > a(a + 1) \), ta có:
    \[
        n\sqrt{d} \cdot \left\{n\sqrt{d}\right\} > a^2 + 5 - a(a + 1) = 5 - a.
    \]

    Khi \( a \leq 2 \), ta có \( n\sqrt{d} > a + 1 \geq 3 \), và vì \( \left\{ n\sqrt{d} \right\} > 0 \), nên bất đẳng thức đúng.

    Còn khi \( a \geq 3 \), ta có:
    \[
        n\sqrt{d} \cdot \left\{n\sqrt{d}\right\} > 5 - a \geq \frac{5}{2},
    \]
    vì biểu thức là số dương tăng theo \( a \).

    Kết luận: Trong mọi dãy 20 số nguyên dương liên tiếp luôn tồn tại một số \( d \) sao cho với mọi \( n \in \mathbb{N} \), ta có:
    \[
        n\sqrt{d} \cdot \left\{n\sqrt{d}\right\} > \frac{5}{2}.
    \]
\end{soln}

\footnotetext{\href{https://www.hms.gr/32bmo2015/sols.pdf}{Lời giải chính thức.}}

\end{document}