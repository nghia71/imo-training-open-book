\documentclass[../03-arithmetic-functions.tex]{subfiles}

\begin{document}

\begin{example*}[\gls{ROU 2015 TST}/D3/P3]\label{example:ROU-2015-TST-D3-P3}[\textbf{\nameref{definition:30M}}]
    Với hai số nguyên dương \( k \leq n \), ký hiệu \( M(n,k) \) là bội chung nhỏ nhất của dãy số \( n, n-1, \dots, n - k + 1 \).  
    Gọi \( f(n) \) là số nguyên dương lớn nhất thỏa mãn:
    \[
        M(n,1) < M(n,2) < \cdots < M(n,f(n)).
    \]
    Chứng minh rằng:
    \begin{itemize}[topsep=0pt, partopsep=0pt, itemsep=0pt]
        \item Với mọi số nguyên dương \( n \), ta có \( f(n) < 3\sqrt{n} \).
        \item Với mọi số nguyên dương \( N \), tồn tại hữu hạn số \( n \) sao cho \( f(n) \leq N \), tức là \( f(n) > N \) với mọi \( n \) đủ lớn.
    \end{itemize}    
\end{example*}

\begin{story*}
    \textbf{Phần (a):} Giả sử \( f(n) \ge 3\sqrt{n} \) và xét dãy số \( s = \lfloor \sqrt{n} \rfloor \). Ta phân tích sự mâu thuẫn khi xét dãy các số nguyên trong khoảng \( \{a+1, a+2, \dots, n\} \), trong đó \( a = s(s-1) \), để thấy rằng có các số nguyên chia hết cho một số không thể là phần tử của dãy.

    \textbf{Phần (b):} Giả sử \( n > N! + N \), ta chứng minh rằng \( f(n) > N \) bằng cách sử dụng các tính chất chia hết với \( k! \) và mâu thuẫn với giả thuyết khi \( n - k > N! \).
\end{story*}

\begin{soln}
    \textbf{Phần (a):} Giả sử ngược lại rằng \( f(n) \ge 3\sqrt{n} \). Đặt \( s = \lfloor \sqrt{n} \rfloor \), và \( a = s(s - 1) \). Ta có:
    \[
        a = s(s - 1) < s(s + 1) < n \implies a < n.
    \]
    Xét tập các số nguyên \( \{a+1, a+2, \dots, n\} \). Tập này chứa cả:
    \[
        s^2 = s \cdot s > s(s-1) = a \implies s^2 \in [a+1, n],
    \]
    và
    \[
        (s+1)(s-1) = s^2 - 1 \implies (s+1)(s-1) \in [a+1, n].
    \]
    Khi đó, \( M(n, n - a) = \mathrm{lcm}(n, n-1, \dots, a+1) \) chia hết cho \( s^2 \) và \( s^2 - 1 \), do đó cũng chia hết cho \( a = s(s-1) \).  
    Suy ra:
    \[
        \mathrm{lcm}(n, n-1, \dots, a+1) \mid a \implies M(n, n - a + 1) = \mathrm{lcm}(M(n, n - a), a) = M(n, n - a),
    \]
    mâu thuẫn với giả thiết chuỗi \( M(n, 1) < M(n, 2) < \dots \). Vậy \( f(n) < 3\sqrt{n} \).
        
    \textbf{Phần (b):} Cho \( N \in \mathbb{N}_{>0} \). Ta chứng minh nếu \( n > N! + N \) thì \( f(n) > N \).  
    Giả sử tồn tại \( k \le N \) sao cho \( M(n,k) = M(n,k+1) \).  
    Điều đó có nghĩa:
    \[
        (n-k) \mid M(n,k) \implies M(n,k) \mid n(n-1)\cdots(n-k+1),
    \]
    suy ra:
    \[
        (n-k) \mid k!.
    \]
    Nhưng nếu \( n-k > N! \ge k! \), mâu thuẫn.  
    
    Vậy với mọi \( k \le N \), ta có:
    \[
        M(n,k) < M(n,k+1) \implies f(n) > N.
    \]
    Từ đó suy ra chỉ có hữu hạn \( n \) với \( f(n) \le N \). Điều này chứng minh vế thứ hai.
\end{soln}

\footnotetext{\href{https://artofproblemsolving.com/community/c6h1097389p6337541}{Dựa theo lời giải của \textbf{Aiscrim}.}}

\end{document}