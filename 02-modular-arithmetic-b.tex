\documentclass[../imo-training-open-book.tex]{subfiles}

\begin{document}

\newpage

\section{Các ví dụ}

\subfile{./02-modular-arithmetic-b/ger-2015-mo-p2.tex} \newpage

\section{Bài tập}

\subfile{./02-modular-arithmetic-b/rou-2014-mo-g9-p2.tex}

\newpage

\section{Định lý, bổ đề, và hằng đẳng thức}

\begin{theorem}[Ước nguyên tố dạng \(4k+3\)]
    \label{theorem:prime-divisor-4k3}
    Mỗi số nguyên dương có dạng \(4s + 3\) đều có ít nhất một ước nguyên tố cũng có dạng đó, tức là \(\equiv -1 \Mod{4}\)
    (see \nameref{theorem:dirichlet-ap}).
\end{theorem}

\end{document}