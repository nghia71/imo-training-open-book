\documentclass[openany,twoside,a4paper]{book}

\usepackage[main=vietnamese,english]{babel}
\usepackage[utf8]{inputenc}
\usepackage[sexy]{evan}
\usepackage{array}
\usepackage{background}
\usepackage{booktabs}
\usepackage{cancel}
\usepackage{CJKutf8}
\usepackage{figchild}
\usepackage{hyperref}
% \usepackage{glossaries}
\usepackage[automake]{glossaries-extra}
\usepackage{listings}
\usepackage{matchsticks}
\usepackage{mathdots}
\usepackage{pdfpages}
\usepackage{subfiles}
\usepackage{wrapfig}

\usepackage{tocloft}
\renewcommand{\cftchappresnum}{}
	\renewcommand{\cftchapaftersnumb}{}
	\setlength{\cftchapindent}{0em}
	\setlength{\cftchapnumwidth}{3em} %
    \addtolength{\cftsecnumwidth}{10pt}

\newcommand\DeactivateBG{\backgroundsetup{contents={}}}
\newcommand\ActivateBG{
    \backgroundsetup{
        position=current page.west,
        angle=90,
        nodeanchor=west,
        vshift=-5mm,
        opacity=1,
        scale=1,
        contents=Phiên bản \date{\today}.
    }
}

\makeglossaries

\subfile{./tex/glossary.tex}

\title{Các bài toán thi Olympic Quốc Gia và Thế Giới}

\date{\today}
\author{Ban biên soạn\\ \\Tạp chí Pi\\ Hội toán học Việt Nam}

\begin{document}

\DeactivateBG

\maketitle

\tableofcontents

\ActivateBG

\chapter*{Mở đầu}
\subfile{./tex/intro.tex}

\addcontentsline{toc}{chapter}{Introduction}

\part{Lý thuyết chung}

\subfile{./tex/common-tools.tex}

\part{Lý thuyết số}

\chapter{Tính chia hết}
\subfile{./nbt/01-divisibility.tex}

\chapter{Cơ bản về số học đồng dư}
\subfile{./nbt/02-modular-arithmetic-b.tex}

\chapter{Các hàm số học}
\subfile{./nbt/03-arithmetic-functions.tex}

\chapter{Phương trình Diophantine}

\chapter{Số học đồng dư nâng cao}

\chapter{Luỹ thừa lớn nhất}
\subfile{./nbt/05-largest-exponent.tex}

\chapter{Đa thức nguyên}

\chapter{Phần dư bậc hai}
\subfile{./nbt/07-quadratic-residues.tex}

\chapter{Chứng minh kiến tạo}

\appendix

\chapter*{Tiêu chuẩn Xếp hạng MOHS}
\subfile{./tex/mohs.tex}

\clearpage

\printglossaries

\end{document}