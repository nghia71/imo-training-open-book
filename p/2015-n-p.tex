\documentclass[./m.tex]{subfiles}

\begin{document}

\section{2015}

\begin{problem}\label{problem:BGR-2015-EGMO-TST-P4}
	Chứng minh rằng với mọi số nguyên dương \( m \), tồn tại vô số cặp số nguyên dương \( (x, y) \) nguyên tố cùng nhau sao cho:
	\[ x \mid y^2 + m,\ \text{và}\ y \mid x^2 + m. \]
\end{problem}

\begin{problem}\label{problem:BGR-2015-EGMO-TST-P6}
	Chứng minh rằng với mọi số nguyên dương \( n \geq 3 \),
	tồn tại \( n \) số nguyên dương phân biệt sao cho tổng các lập phương của chúng cũng là một lập phương hoàn hảo.
\end{problem}

\begin{problem}\label{problem:BMO-2015-P4}
    Chứng minh rằng trong bất kỳ dãy \( 20 \) số nguyên dương liên tiếp nào cũng tồn tại một số nguyên \( d \) sao cho với mọi số nguyên dương \( n \),
    bất đẳng thức sau luôn đúng:
    \[
        n \sqrt{d} \cdot \left\{n \sqrt{d} \right\} > \frac{5}{2},
    \]
    trong đó \( \left\{ x \right\} \) ký hiệu phần thập phân của \( x \), tức là \( \left\{ x \right\} = x - \lfloor x \rfloor \).
\end{problem}

\begin{problem}\label{problem:BxMO-2015-P3}
	Có tồn tại số nguyên tố nào có biểu diễn thập phân dưới dạng
	\[
		3811\underbrace{11\ldots1}_{n\ \text{chữ số}\ 1}?
	\]
\end{problem}

\begin{problem}\label{problem:CAN-2015-QRC-P3}
	Cho \( N \) là một số có ba chữ số phân biệt và khác 0. Ta gọi \( N \) là một số \textit{tầm thường} nếu nó có tính chất sau:
	khi viết ra tất cả 6 hoán vị có ba chữ số từ các chữ số của \( N \), trung bình cộng của chúng bằng chính \( N \).
	Ví dụ: \( N = 481 \) là số \textit{tầm thường} vì trung bình cộng của các số \( \{418, 481, 148, 184, 814, 841\} \) bằng \( 481 \).
	Hãy xác định số \textit{tầm thường} lớn nhất.	
\end{problem}

\begin{problem}\label{problem:CHN-2015-TST1-D1-P2}
	Cho dãy các số nguyên dương phân biệt \( a_1, a_2, a_3, \ldots \), và một hằng số thực \( 0 < c < \dfrac{3}{2} \).  
	Chứng minh rằng tồn tại vô hạn số nguyên dương \( k \) sao cho:
	\[
		\operatorname{lcm}(a_k, a_{k+1}) > c k.
	\]
\end{problem}

\begin{problem}\label{problem:CHN-2015-TST1-D2-P2}
	Cho \( n \) là một số nguyên dương. Chứng minh rằng với mọi số nguyên dương \( a, b, c \) không vượt quá \( 3n^2 + 4n \),
	tồn tại các số nguyên \( x, y, z \) có giá trị tuyệt đối không vượt quá \( 2n \) và không đồng thời bằng 0, sao cho
	\[
		ax + by + cz = 0.
	\]
\end{problem}

\begin{problem}\label{problem:EGMO-2015-P3}
	Cho \( n, m \) là các số nguyên lớn hơn \( 1 \), và \( a_1, a_2, \dots, a_m \) là các số nguyên dương không vượt quá \( n^m \).
	Hãy chứng minh rằng tồn tại các số nguyên dương \( b_1, b_2, \dots, b_m \leq n \) sao cho:
	\[
		\gcd(a_1 + b_1, a_2 + b_2, \dots, a_m + b_m) < n.
	\]
\end{problem}

\begin{problem}\label{problem:EGMO-2015-P3-strong}
	Với \( m, n > 1 \), giả sử \( a_1, \dots, a_m \) là các số nguyên dương sao cho có ít nhất một \( a_i \leq n^{2^{m-1}} \). 
	Chứng minh rằng tồn tại các số nguyên \( b_1, \dots, b_m \in \{1, 2\} \) sao cho:
	\[
		\gcd(a_1 + b_1, \dots, a_m + b_m) < n.
	\]
\end{problem}

\begin{problem}\label{problem:GBR-2015-TST-N3-P3}
    Cho các số nguyên dương phân biệt đôi một \( a_1 < a_2 < \cdots < a_n \), trong đó \( a_1 \) là số nguyên tố và \( a_1 \geq n + 2 \).
    Trên đoạn thẳng \( I = \left[0, \prod_{i=1}^n a_i \right] \) trên trục số thực,
    đánh dấu tất cả các số nguyên chia hết cho ít nhất một trong các số \( a_1, a_2, \ldots, a_n \).
    Các điểm này chia đoạn \( I \) thành nhiều đoạn con.

    Chứng minh rằng tổng bình phương độ dài các đoạn con đó chia hết cho \( a_1 \).
\end{problem}

\begin{problem}\label{problem:GER-2015-MO-P2}
    Một số nguyên dương \( n \) được gọi là \textbf{trơn} nếu tồn tại các số nguyên \( a_1, a_2, \dots, a_n \) sao cho:
    \[
        a_1 + a_2 + \dots + a_n = a_1 \cdot a_2 \cdot \dots \cdot a_n = n.
    \]
    
    Hãy tìm tất cả các số trơn.
\end{problem}

\begin{problem}\label{problem:HUN-2015-TST-KMA-633}
    Chứng minh rằng nếu \( n \) là một số nguyên dương đủ lớn,
    thì trong bất kỳ tập hợp gồm \( n \) số nguyên dương khác nhau nào cũng tồn tại bốn số sao cho bội chung nhỏ nhất của chúng lớn hơn \( n^{3{,}99} \).
\end{problem}

\begin{problem}\label{problem:IND-2015-N2}
	Với mọi số tự nhiên \( n > 1 \), phân số \( \frac{1}{n} \) với số chữ số thập phân hữu hạn dưới dạng thập phân vô hạn 
	ví dụ như: \( 0.5 \) được viết là \( \frac{1}{2} = 0.4\overline{9} \).
	Hãy xác định độ dài phần không tuần hoàn trong biểu diễn thập phân vô hạn của \( \frac{1}{n} \).
\end{problem}

\begin{problem}\label{problem:IND-2015-N6}
	Chứng minh rằng từ một tập gồm \( 11 \) số chính phương, ta luôn có thể chọn ra sáu số \( a^2, b^2, c^2, d^2, e^2, f^2 \) sao cho:
	\[
		a^2 + b^2 + c^2 \equiv d^2 + e^2 + f^2 \Mod{12}
	\]
\end{problem}

\begin{problem}\label{problem:IND-2015-TST2-P1}
	Cho số nguyên \( n \geq 2 \), và đặt:
	\[
		A_n = \{2^n - 2^k \mid k \in \mathbb{Z},\, 0 \leq k < n \}.
	\]
	Tìm số nguyên dương lớn nhất không thể biểu diễn được dưới dạng tổng của một hay nhiều (không nhất thiết khác nhau) phần tử trong tập \( A_n \).
\end{problem}

\begin{problem}\label{problem:IRN-2015-N2}
	Gọi \( M_0 \subset \mathbb{N} \) là một tập hợp hữu hạn, không rỗng chứa một số số tự nhiên.
	Ali tạo ra các tập \( M_1, M_2, \dots, M_n \) theo quy trình sau:
	Tại bước \( n \), Ali chọn một phần tử \( b_n \in M_{n-1} \), sau đó định nghĩa tập:
	\[
		M_n = \left\{ b_n m + 1 \mid m \in M_{n-1} \right\}.
	\]

	Chứng minh rằng tồn tại một bước nào đó mà trong tập tạo ra, không có phần tử nào chia hết cho phần tử nào khác trong cùng tập.
\end{problem}

\begin{problem}\label{problem:IRN-2015-TST-D3-P2}
	Giả sử \( a_1, a_2, a_3 \) là ba số nguyên dương cho trước. Xét dãy số được xác định bởi công thức:
	\[
		a_{n+1} = \text{lcm}[a_n, a_{n-1}] - \text{lcm}[a_{n-1}, a_{n-2}] \quad \text{với } n \geq 3,
	\]
	trong đó \( [a, b] \) ký hiệu bội chung nhỏ nhất của \( a \) và \( b \), và chỉ được áp dụng với các số nguyên dương.

	Chứng minh rằng tồn tại một số nguyên dương \( k \leq a_3 + 4 \) sao cho \( a_k \leq 0 \).
\end{problem}

\begin{problem}\label{problem:JPN-2015-MO-P1}
	Tìm tất cả các số nguyên dương \( n \) sao cho 
	\[
		\frac{10^n}{n^3 + n^2 + n + 1}
	\]
	là một số nguyên.
\end{problem}

\begin{problem}\label{problem:KOR-2015-MO-P8}
	Cho \( n \) là một số nguyên dương. Các số \( a_1, a_2, \dots, a_k \) là các số nguyên dương không lặp lại, không lớn hơn \( n \),
	và nguyên tố cùng nhau với \( n \). Nếu \( k > 8 \), hãy chứng minh rằng:
	\[
		\sum_{i=1}^k \left| a_i - \frac{n}{2} \right| < \frac{n(k - 4)}{2}.
	\]
\end{problem}

\begin{problem}\label{problem:MEMO-2015-I-P4}
	Tìm tất cả các cặp số nguyên dương \( (m, n) \) sao cho tồn tại hai số nguyên \( a, b > 1 \), nguyên tố cùng nhau, thỏa mãn:
	\[
		\frac{a^m + b^m}{a^n + b^n} \in \mathbb{Z}.
	\]	
\end{problem}

\begin{problem}\label{problem:ROU-2015-MO-G10-P2}
    Xét một số tự nhiên \( n \) sao cho tồn tại một số tự nhiên \( k \) và \( k \) số nguyên tố phân biệt sao cho \( n = p_1 \cdot p_2 \cdots p_k \).
    \begin{itemize}[topsep=0pt, partopsep=0pt, itemsep=0pt]
        \item Tìm số lượng các hàm \( f : \{1, 2, \ldots, n\} \longrightarrow \{1, 2, \ldots, n\} \) sao cho tích \( f(1) \cdot f(2) \cdots f(n) \) chia hết \( n \).
        \item Với \( n = 6 \), hãy tìm số lượng các hàm \( f : \{1, 2, 3, 4, 5, 6\} \longrightarrow \{1, 2, 3, 4, 5, 6\} \)
        sao cho tích \( f(1)\cdot f(2)\cdot f(3)\cdot f(4)\cdot f(5)\cdot f(6) \) chia hết cho \( 36 \).
    \end{itemize}
\end{problem}

\begin{problem}\label{problem:ROU-2015-TST-D1-P4}
    Gọi \( k \) là một số nguyên dương sao cho \( k \equiv 1 \Mod{4} \), và \( k \) không phải là số chính phương. Đặt
    \[
        a = \frac{1 + \sqrt{k}}{2}.
    \]
    
    Chứng minh rằng
    \[
        \left\{ \left\lfloor a^2 n \right\rfloor - \left\lfloor a \left\lfloor a n \right\rfloor \right\rfloor : n \in \mathbb{N}_{>0} \right\} = \{ 1, 2, \ldots, \left\lfloor a \right\rfloor \}.
    \]
\end{problem}

\begin{problem}\label{problem:ROU-2015-TST-D2-P1}
    Cho \( a \in \mathbb{Z} \) và \( n \in \mathbb{N}_{>0} \). Chứng minh rằng:
    \[
        \sum_{k=1}^{n} a^{\gcd(k,n)}
    \]
    luôn chia hết cho \( n \), trong đó \( \gcd(k,n) \) là ước chung lớn nhất của \( k \) và \( n \).
\end{problem}

\begin{problem}\label{problem:ROU-2015-TST-D3-P3}
    Với hai số nguyên dương \( k \leq n \), ký hiệu \( M(n,k) \) là bội chung nhỏ nhất của dãy số \( n, n-1, \dots, n - k + 1 \).  
    Gọi \( f(n) \) là số nguyên dương lớn nhất thỏa mãn:
    \[
        M(n,1) < M(n,2) < \cdots < M(n,f(n)).
    \]
    Chứng minh rằng:
    \begin{itemize}[topsep=0pt, partopsep=0pt, itemsep=0pt]
        \item Với mọi số nguyên dương \( n \), ta có \( f(n) < 3\sqrt{n} \).
        \item Với mọi số nguyên dương \( N \), tồn tại hữu hạn số \( n \) sao cho \( f(n) \leq N \), tức là \( f(n) > N \) với mọi \( n \) đủ lớn.
    \end{itemize}    
\end{problem}

\begin{problem}\label{problem:RUS-2015-TST-D10-P2}
	Cho số nguyên tố \( p \ge 5 \). Chứng minh rằng tập \( \{1,2,\ldots,p - 1\} \) có thể được chia thành hai tập con không rỗng
	sao cho tổng các phần tử của một tập con và tích các phần tử của tập con còn lại cho cùng một phần dư modulo \( p \).	
\end{problem}

\begin{problem}\label{problem:THA-2015-MO-P1}
	Cho số nguyên tố \( p \), và dãy số nguyên dương \( a_1, a_2, a_3, \dots \) thỏa mãn:
	\[
		a_n a_{n+2} = a_{n+1}^2 + p \quad \text{với mọi số nguyên dương } n.
	\]
	
	Chứng minh rằng với mọi \( n \), ta có:
	\[
		a_{n+1} \mid a_n + a_{n+2}.
	\]
\end{problem}

\begin{problem}\label{problem:TWN-2015-TST2-Q2-P1}
    Với mỗi số nguyên dương \( n \), định nghĩa:
    \[
        a_n = \sum_{k=1}^{\infty} \left\lfloor \frac{n + 2^{k-1}}{2^k} \right\rfloor,
    \]
    trong đó \( \left\lfloor x \right\rfloor \) là phần nguyên của \( x \), tức là số nguyên lớn nhất không vượt quá \( x \).
\end{problem}

\end{document}