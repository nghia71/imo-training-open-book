\documentclass[./m.tex]{subfiles}

\begin{document}

\section{2014}

\begin{problem}\label{problem:ROU-2014-MO-G9-P2}
    Cho \( a \) là một số tự nhiên lẻ không phải là một số chính phương, và \( m, n \in \mathbb{N} \). Khi đó:
    \begin{itemize}[topsep=0pt, partopsep=0pt, itemsep=0pt]
        \item \( \left\{ m\left( a+\sqrt{a} \right) \right\} \neq \left\{ n\left( a-\sqrt{a} \right) \right\} \)
        \item \( \left[ m\left( a+\sqrt{a} \right) \right] \neq \left[ n\left( a-\sqrt{a} \right) \right] \)
    \end{itemize}
    Trong đó, \( \{\cdot\} \) ký hiệu phần thập phân (phần lẻ), và \( [\cdot] \) ký hiệu phần nguyên.
\end{problem}

\begin{problem}\label{problem:ROU-2014-MO-G10-P1}
    Cho \( n \) là một số tự nhiên. Tính giá trị của biểu thức
    \[
        \sum_{k=1}^{n^2} \#\left\{ d \in \mathbb{N} \,\middle|\, 1 \le d \le k \le d^2 \le n^2,\ k \equiv 0 \Mod{d} \right\}.
    \]
    Trong đó, ký hiệu \( \# \) biểu thị số phần tử của tập hợp.
\end{problem}

\end{document}