\documentclass[../04-diophantine-equations.tex]{subfiles}

\begin{document}

\begin{example*}[\gls{KOR 2015 MO}/P1]\label{example:KOR-2015-MO-P1}[\textbf{\nameref{definition:30M}}]
	Với mỗi số nguyên dương \( m \), \( (x, y) \) là một cặp số nguyên dương thỏa mãn hai điều kiện:
	\begin{enumerate}[topsep=0pt, partopsep=0pt, itemsep=0pt]
		\item[(i)] \( x^2 - 3y^2 + 2 = 16m \),
		\item[(ii)] \( 2y \le x - 1 \).
	\end{enumerate}
	Chứng minh rằng số lượng các cặp như vậy là số chẵn hoặc bằng 0.
\end{example*}

\begin{story*}
    Ta xây dựng ánh xạ \( \mathrm{T}(x, y) = (2x - 3y, x - 2y) \). Ánh xạ này bảo toàn cả hai điều kiện của bài toán.  
    Kiểm tra cho thấy:
    \begin{itemize}[topsep=0pt, partopsep=0pt, itemsep=0pt]
        \item Ánh xạ bảo toàn phương trình \( x^2 - 3y^2 + 2 = 16m \).
        \item Ánh xạ bảo toàn điều kiện \( 2y \le x - 1 \).
        \item Ánh xạ không có điểm bất biến, vì giải hệ \( \mathrm{T}(x, y) = (x, y) \) dẫn đến mâu thuẫn.
    \end{itemize}
    Vì vậy, mọi nghiệm đều xuất hiện thành từng cặp hoán vị, suy ra số lượng nghiệm là chẵn hoặc bằng 0.
\end{story*}

\bigbreak

\begin{soln}\footnotemark
    Nếu không có nghiệm thì rõ ràng số lượng là 0. Giả sử tồn tại ít nhất một nghiệm \( (x, y) \in \mathbb{Z}_{>0}^2 \) thỏa mãn hai điều kiện.
    Định nghĩa ánh xạ:
    \[
        \mathrm{T}(x, y) = (x', y') = (2x - 3y,\; x - 2y).
    \]

    \textbf{Bước 1.} Kiểm tra \( (x', y') \) vẫn là số nguyên dương:  
    Từ bất đẳng thức \( 2y \le x - 1 \Rightarrow x \ge 2y + 1 \), suy ra:
    \[
        \left.
        \begin{array}{rcl}
            x' &=& 2x - 3y \ge 2(2y + 1) - 3y = 4y + 2 - 3y = y + 2 \ge 3\\
            y' &=& x - 2y \ge 2y + 1 - 2y = 1.
        \end{array}
        \right\} \implies x', y' \in \mathbb{Z}_{>0}
    \]

    \textbf{Bước 2.} Kiểm tra bảo toàn phương trình:
    \[
        \begin{aligned}
            &(2x - 3y)^2 - 3(x - 2y)^2 + 2
            = 4x^2 - 12xy + 9y^2 - 3(x^2 - 4xy + 4y^2) + 2\\
            &= 4x^2 - 12xy + 9y^2 - 3x^2 + 12xy - 12y^2 + 2
            = x^2 - 3y^2 + 2 = 16m \implies (i)
        \end{aligned}
    \]

    \textbf{Bước 3.} Kiểm tra điều kiện (ii) sau ánh xạ:
    \[  
        \begin{aligned}
            &2y' = 2(x - 2y) = 2x - 4y,\quad x' - 1 = 2x - 3y - 1.\\
            &2x - 4y \le 2x - 3y - 1 \iff -4y \le -3y - 1 \iff -y \le -1 \iff y \ge 1 \implies y \in \mathbb{Z}_{>0}
        \end{aligned}
    \]

    \textbf{Bước 4.} Chứng minh \( \mathrm{T} \) là một ánh xạ nghịch đảo (involution):
    \[
        \begin{aligned}
            \mathrm{T}(\mathrm{T}(x, y)) &= \mathrm{T}(2x - 3y,\; x - 2y) = (2(2x - 3y) - 3(x - 2y),\; (2x - 3y) - 2(x - 2y))\\
            &= (4x - 6y - 3x + 6y,\; 2x - 3y - 2x + 4y) = (x, y).
        \end{aligned}
    \]

    \textbf{Bước 5.} Không có điểm bất biến:  
    Giả sử \( \mathrm{T}(x, y) = (x, y) \Rightarrow x = 2x - 3y,\; y = x - 2y \).  
    Giải hệ này:
    \[
        \begin{aligned}
            &x = 3y,\quad y = 3y - 2y = y \implies x^2 - 3y^2 + 2 = 9y^2 - 3y^2 + 2 = 6y^2 + 2.\\
            &\implies 6y^2 + 2 \equiv 0 \pmod{16} \Rightarrow 6y^2 \equiv -2 \pmod{16},
        \end{aligned}
    \]
    nhưng \( 6y^2 \pmod{16} \in \{0, 6, 8, 14, 2, 10, 12\} \), không có giá trị nào cho ra \( -2 \equiv 14 \). Mâu thuẫn.

    Vậy không có điểm bất biến, và ánh xạ chia các nghiệm thành từng cặp hoán vị.  
    Do đó số nghiệm là số chẵn, hoặc bằng 0.
\end{soln}

\footnotetext{\href{https://artofproblemsolving.com/community/c6h1158057p5501392}{Lời giải chính thức.}}

\end{document}