\documentclass[../04-diophantine-equations.tex]{subfiles}

\begin{document}

\begin{example*}[\gls{CHN 2015 TST}3/D2/P3]\label{example:CHN-2015-TST3-D2-P3}[\textbf{\nameref{definition:30M}}]
	Với mọi số tự nhiên \( n \), định nghĩa:
	\[
		f(n) = \tau(n!) - \tau((n-1)!),
	\]
	trong đó \( \tau(a) \) là số ước số dương của \( a \).  
	
	Chứng minh rằng tồn tại vô hạn số \( n \) là hợp số sao cho với mọi số tự nhiên \( m < n \), ta có:
	\[
		f(m) < f(n).
	\]
\end{example*}

\begin{story*}
    Hàm \( f(n) \) đo mức độ tăng thêm về số lượng ước số của \( n! \) so với \( (n-1)! \).  
    Để \( f(n) \) đạt giá trị lớn hơn mọi giá trị trước đó, ta tìm một bước nhảy lớn trong độ phức tạp của \( n! \).  
    Sử dụng định lý Bertrand, chọn \( q \) là số nguyên tố lớn nhất trong đoạn \( (p, 2p) \), ta chứng minh \( f(2p) > f(q) \), với \( p \) là số nguyên tố lẻ.  
    Khi \( p \) chạy qua vô hạn số nguyên tố, ta thu được vô hạn số \( n \) hợp số thỏa mãn yêu cầu bài toán.
\end{story*}

\begin{soln}\footnotemark
	Cho \( p \) là một số nguyên tố lẻ. Theo \nameref{theorem:bertrand-postulate}, tồn tại số nguyên tố giữa \( p \) và \( 2p \).  
	Giả sử \( q \) là số nguyên tố lớn nhất trong đoạn \( (p, 2p) \).  
	Ta chứng minh khẳng định sau:

	\begin{claim*}
		\( f(2p) > f(q) \).
	\end{claim*}

	\begin{subproof}
		Ta có:
		\[
			\begin{aligned} 
				f(2p) &= \tau((2p)!) - \tau((2p-1)!) 
				= \frac{3}{2} \cdot \tau\bigl(2(2p-1)!\bigr) - \tau\bigl((2p-1)!\bigr) \\ 
			 	&= 3\tau\!\Bigl(\frac{2(2p-1)!}{q}\Bigr) - 2\tau\!\Bigl(\frac{(2p-1)!}{q}\Bigr) 
			  	>\; \tau\!\Bigl(\frac{(2p-1)!}{q}\Bigr) \;\geq\; \tau\bigl((q-1)!\bigr) \;=\; f(q).
			\end{aligned}
		\]
	\end{subproof}

	Gọi \( n \) là số nguyên dương nhỏ nhất thỏa mãn \( n \leq 2p \) và \( f(n) \) đạt giá trị lớn nhất trong dãy \( f(1), f(2), \dots, f(2p) \).  
			
	Nếu \( n \leq q \), thì \( f(n) \leq \tau\bigl((q-1)!\bigr) = f(q) < f(2p) \), mâu thuẫn.  
	Do đó, \( n > q \), và từ định nghĩa của \( q \), suy ra \( n \) là hợp số và \( f(n) > f(m) \) với mọi \( m < n \). 
	
	Vì \( n \in [p+1, 2p+1] \), ta thấy rằng khi cho \( p \) chạy qua vô hạn số nguyên tố lẻ, sẽ có vô hạn giá trị \( n \) là hợp số thoả mãn điều kiện của bài toán.
\end{soln}

\footnotetext{\href{https://artofproblemsolving.com/community/c6h1069931p21493104}{Lời giải của \textbf{chirita.andrei}.}}

\end{document}