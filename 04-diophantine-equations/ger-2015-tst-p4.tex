\documentclass[../04-diophantine-equations.tex]{subfiles}

\begin{document}

\begin{example*}[\gls{GER 2015 TST}/P4]\label{example:GER-2015-TST-P4}[\textbf{\nameref{definition:30M}}]
	Tìm tất cả các cặp số nguyên dương \((x, y)\) sao cho
	\[
		\sqrt[3]{7x^2 - 13xy + 7y^2} = |x - y| + 1.
	\]
\end{example*}

\begin{story*}
    Đặt \( u = |x - y| \), khi đó phương trình trở thành \( \sqrt[3]{7u^2 + v} = u + 1 \).  
    Giải phương trình, ta tìm được \( v = u^3 - 4u^2 + 3u + 1 \), với \( v = xy = y^2 + uy \).  
    Để phương trình có nghiệm, biểu thức trong căn phải là số chính phương, từ đó suy ra \( 4u + 1 = (2m + 1)^2 \), và \( u = m^2 + m \).  
    Trường hợp \( u = 0 \) dẫn đến nghiệm \( (1,1) \), còn các nghiệm khác theo dạng \( (m^3 + 2m^2 - m - 1, m^3 + m^2 - 2m - 1) \) với \( m \ge 2 \).
\end{story*}

\begin{soln}\footnotemark
	Không mất tính tổng quát, giả sử \( x \ge y \). Đặt \( u = x - y \), khi đó phương trình trở thành:
	\[
		\sqrt[3]{7(x - y)^2 + xy} = u + 1.
	\]
	Đặt \( v = xy \), ta được:
	\[
		7u^2 + v = (u + 1)^3 \Rightarrow v = u^3 - 4u^2 + 3u + 1.
	\]
	
	Mặt khác, \( x = y + u \Rightarrow v = x y = (y + u)y = y^2 + uy \). Suy ra:
	\[
		y^2 + uy = u^3 - 4u^2 + 3u + 1 
		\Rightarrow y = \frac{-u \pm \sqrt{u^2 + 4(u^3 - 4u^2 + 3u + 1)}}{2}.
	\]
	
	Yêu cầu biểu thức trong căn là số chính phương:
	\[
		u^2 + 4(u^3 - 4u^2 + 3u + 1) = (u - 2)^2 (4u + 1).
	\]
	Suy ra \( 4u + 1 = \ell^2 \), với \( \ell \) là số lẻ. Đặt \( \ell = 2m + 1 \Rightarrow u = m^2 + m \).

	\textit{Trường hợp \( m = 0 \)}: \( u = 0 \Rightarrow x = y \). Thay vào phương trình gốc:
	\[
		\sqrt[3]{7x^2 - 13x^2 + 7x^2} = \sqrt[3]{x^2} = |x - y| + 1 = 1 \Rightarrow x^2 = 1 \Rightarrow x = y = 1.
	\]
	
	\textit{Trường hợp \( m \ge 1 \)}:  
	\[
		u = m^2 + m, \quad y = m^3 + m^2 - 2m - 1, \quad x = y + u = m^3 + 2m^2 - m - 1.
	\]
	
	Với \( m \ge 2 \), \( y > 0 \). Như vậy, các nghiệm nguyên dương của bài toán là:
	\[
		(x, y) = (1, 1) \quad \text{và} \quad \left(m^3 + 2m^2 - m - 1,\; m^3 + m^2 - 2m - 1\right),\; m \ge 2,
	\]
	cùng các hoán vị \((y,x)\) (do phương trình đối xứng theo \( |x - y| \)).
\end{soln}

\footnotetext{\href{https://artofproblemsolving.com/community/c6h1113196p11637064}{Lời giải của \textbf{pad}.}}

\end{document}