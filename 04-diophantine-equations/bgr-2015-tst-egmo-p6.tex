\documentclass[../04-diophantine-equations.tex]{subfiles}

\begin{document}

\begin{exercise*}[\gls{BGR 2015 EGMO TST}/P6]\label{example:BGR-2015-EGMO-TST-P6}[\textbf{\nameref{definition:30M}}]
	Chứng minh rằng với mọi số nguyên dương \( n \geq 3 \),
	tồn tại \( n \) số nguyên dương phân biệt sao cho tổng các lập phương của chúng cũng là một lập phương hoàn hảo.
\end{exercise*}

\begin{remark*}
	Thử xây dựng một họ các bộ \( n \) số (có thể đồng dư hoặc biến thiên theo tham số) sao cho tổng lập phương thu được là \( (\text{một giá trị tham số})^3 \).  
	Các ví dụ quen thuộc là các “nhóm” số mà tổng lập phương khéo léo triệt tiêu và cộng dồn thành một khối lập phương.
\end{remark*}

\end{document}