\documentclass[../04-diophantine-equations.tex]{subfiles}

\begin{document}

\begin{example*}[\gls{USA 2015 MO}/P5]\label{example:USA-2015-MO-P5}\textbf{[\nameref{definition:30M}]}
	Cho các số nguyên dương phân biệt \( a, b, c, d, e \) sao cho
	\[
		a^4 + b^4 = c^4 + d^4 = e^5.
	\]
	Chứng minh rằng \( ac + bd \) là một hợp số.
\end{example*}

\begin{story*}
    Giả sử \( ac + bd \) là một số nguyên tố. Ta biến đổi biểu thức
    \[
        (a^4 + b^4)c^2d^2 - (c^4 + d^4)a^2b^2,
    \]
    thu được tích bốn nhân tử. Vì các biến phân biệt, không nhân tử nào bằng 0.  
    Ta suy ra \( ac + bd \mid e^5(cd - ab)(cd + ab) \). Nếu \( ac + bd \) là số nguyên tố, nó phải chia \( e^5 \), nhưng điều này mâu thuẫn với tính chất lũy thừa bậc 5 của \( e \), do đó \( ac + bd \) không thể là số nguyên tố, tức là hợp số.
\end{story*}

\bigbreak

\begin{soln}\footnotemark
    Giả sử ngược lại rằng \(ac + bd\) là một số nguyên tố.

    Không mất tính tổng quát, giả sử \(a > d\). Từ \(a^4 + b^4 = c^4 + d^4\), suy ra \(b < c\).  
    Xét biểu thức:
    \[
        (a^4 + b^4)c^2d^2 - (c^4 + d^4)a^2b^2
        = (a^2c^2 - b^2d^2)(a^2d^2 - b^2c^2).
    \]
    Vì \(a^4 + b^4 = c^4 + d^4 = e^5\), nên:
    \[
        e^5(cd - ab)(cd + ab) 
        = (ac - bd)(ac + bd)(ad - bc)(ad + bc).
    \]
    Nếu \(ac - bd = 0\) hoặc \(ad - bc = 0\), thì ta có tỉ lệ \(\tfrac{a}{b} = \tfrac{c}{d}\) hoặc \(\tfrac{a}{b} = \tfrac{d}{c}\), mâu thuẫn với giả thiết các số đều phân biệt.  
    Vậy mọi nhân tử đều khác 0.

    Khi đó, \( ac + bd \mid e^5(cd - ab)(cd + ab) \).  
    Nhưng vì \( ac + bd > cd + ab \), ta thấy \( ac + bd \nmid cd + ab \) và cũng không chia hết \( cd - ab \).  
    Suy ra \( ac + bd \mid e^5 \). Gọi \( ac + bd = p \), khi đó \( p \mid e^5 \Rightarrow p \le e \).  

    Nhưng:
    \[
        e = \sqrt[5]{a^4 + b^4} \le \sqrt[5]{2a^4} = a^{4/5} \cdot 2^{1/5} < a.
    \]
    Vì \( ac + bd > ab \), thì \( p > ab \), mâu thuẫn với \( p \le e < a \le ab \).  

    Vậy \( ac + bd \) không thể là số nguyên tố, tức là nó là hợp số.
\end{soln}

\footnotetext{\href{https://artofproblemsolving.com/wiki/index.php/2015_USAMO_Problems/Problem_5}{Lời giải chính thức.}}

\end{document}