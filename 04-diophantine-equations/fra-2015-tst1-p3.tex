\documentclass[../04-diophantine-equations.tex]{subfiles}

\begin{document}

\begin{example*}[\gls{FRA 2015 TST}/1/P3]\label{example:FRA-2015-TST1-P3}[\textbf{\nameref{definition:25M}}]
	Cho \( n \) là một số nguyên dương sao cho \( n(n + 2015) \) là một số chính phương.
    \begin{itemize}[topsep=0pt, partopsep=0pt, itemsep=0pt]
        \item Chứng minh rằng \( n \) không phải là số nguyên tố.
        \item Cho một ví dụ về số nguyên \( n \) như vậy.
    \end{itemize}
\end{example*}

\begin{story*}
    \begin{itemize}[topsep=0pt, partopsep=0pt, itemsep=0pt]
        \item \textbf{Phần (a):} Giả sử \( n \) là số nguyên tố và \( n(n + 2015) = m^2 \) là số chính phương.  
        Khi đó \( n \mid m^2 \Rightarrow n \mid m \), đặt \( m = nr \). Ta suy ra phương trình:  
        \[
            2015 = n(r^2 - 1).
        \]
        Vì \( n \mid 2015 \), nên chỉ có thể là một trong các ước nguyên tố của 2015 là \( 5, 13, 31 \).  
        Kiểm tra từng giá trị cho thấy không có giá trị nào cho \( r^2 \) nguyên, từ đó suy ra \( n \) không thể là số nguyên tố.
        
        \item \textbf{Phần (b):} Ta cần xây dựng một ví dụ cụ thể sao cho \( n(n + 2015) \) là số chính phương.  
        Sử dụng kỹ thuật đặt biến: nếu \( n(n + 2015) = \left(\frac{(2n + 2015)^2 - 2015^2}{4} \right) \), ta đưa phương trình về dạng hiệu hai bình phương.  
        Từ đó, chọn các cặp số \( (a, b) \) sao cho \( ab = 2015^2 \), rồi giải hệ để tìm được \( n \).  
        Chọn \( a = 2015 \cdot 5, b = 2015/5 \) ta tìm được \( n = 1612 \), là một nghiệm phù hợp.
    \end{itemize}
\end{story*}

\begin{soln}\footnotemark
	\textit{(a)} Giả sử \( n \) là số nguyên tố và tồn tại \( m \in \mathbb{Z} \) sao cho \( n(n + 2015) = m^2 \).  
	Khi đó, \( n \mid m^2 \Rightarrow n \mid m \). Đặt \( m = nr \), ta có:
	\[
		n(n + 2015) = n^2 r^2 \Rightarrow n + 2015 = nr^2 \Rightarrow 2015 = n(r^2 - 1).
	\]
	Vì \( 2015 = 5 \cdot 13 \cdot 31 \), nên \( n \in \{5, 13, 31\} \).  
	Ta kiểm tra từng giá trị:
	\begin{itemize}[topsep=0pt, partopsep=0pt, itemsep=0pt]
	    \item \( n = 5 \Rightarrow r^2 = 1 + \frac{2015}{5} = 1 + 403 = 404 \), không là số chính phương.
	    \item \( n = 13 \Rightarrow r^2 = 1 + \frac{2015}{13} = 1 + 155 = 156 \), không là số chính phương.
	    \item \( n = 31 \Rightarrow r^2 = 1 + \frac{2015}{31} = 1 + 65 = 66 \), không là số chính phương.
	\end{itemize}
	Không giá trị nào cho \( r^2 \) nguyên, nên \( n \) không thể là số nguyên tố.

	\textit{(b)} Ta xét biểu thức:
	\[
		(2m)^2 = 4n(n + 2015) = (2n + 2015)^2 - 2015^2.
	\]
	Đặt \( 2n + 2015 + 2m = a \), \( 2n + 2015 - 2m = b \), ta có:
	\[
		ab = 2015^2.
	\]
	Chọn:
	\[
		a = 2015 \cdot 5 = 10075,\quad b = \frac{2015}{5} = 403.
	\]
	Suy ra:
	\[
		2n + 2015 = \frac{a + b}{2} = \frac{10075 + 403}{2} = 5239.
	\]
	\[
		n = \frac{5239 - 2015}{2} = \frac{3224}{2} = 1612.
	\]
	Khi đó:
	\[
		n(n + 2015) = 1612 \cdot 3627 = \left( \frac{10075 - 403}{2} \right)^2 = m^2,
	\]
	là một số chính phương. Vậy \( n = 1612 \) là một ví dụ thoả mãn.
\end{soln}

\footnotetext{\href{http://maths-olympiques.fr/wp-content/uploads/2017/10/ofm-2014-2015-test-janvier-corrige.pdf}{Lời giải chính thức.}}

\end{document}