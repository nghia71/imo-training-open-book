\documentclass[../04-diophantine-equations.tex]{subfiles}

\begin{document}

\begin{exercise*}[\gls{ROU 2014 MO}/G9/P1]\label{example:ROU-2014-MO-G9-P1}[\textbf{\nameref{definition:30M}}]
    Tìm các số nguyên \( x, y, z \in \mathbb{Z} \) sao cho
    \[
        x^2 + y^2 + z^2 = 2^n(x + y + z)
    \]
    với \( n \in \mathbb{N} \).
\end{exercise*}

\begin{remark*}
    Thử đưa toàn bộ phương trình về một vế và nhóm các biểu thức theo từng biến.
    Xét biến đổi hoàn chỉnh bình phương để tạo điều kiện cho việc đánh giá hoặc tìm các nghiệm nhỏ.
    Ngoài ra, kiểm tra các trường hợp đặc biệt như \( x = y = z \) hoặc \( x + y + z = 0 \) có thể giúp tìm nghiệm đặc biệt.
\end{remark*}

% \begin{story*}
%     Phương trình cho một đẳng thức giữa tổng bình phương và tổng tuyến tính. Ta viết lại:
%     \[
%         x^2 - 2^n x + y^2 - 2^n y + z^2 - 2^n z = 0.
%     \]
%     Gom nhóm:
%     \[
%         (x - 2^{n-1})^2 + (y - 2^{n-1})^2 + (z - 2^{n-1})^2 = 3 \cdot 2^{2n - 2}.
%     \]
%     Từ đây, ta thấy nghiệm \( (x, y, z) \) nằm trong một mặt cầu bán kính cố định phụ thuộc vào \( n \).  
%     Vì bên trái là tổng của ba bình phương, số lượng nghiệm nguyên bị giới hạn. Ta có thể:
%     \begin{itemize}[topsep=0pt, partopsep=0pt, itemsep=0pt]
%         \item Với từng giá trị nhỏ của \( n \), liệt kê hữu hạn các bộ \( (x, y, z) \) thỏa mãn điều kiện.
%         \item Khi \( x + y + z = 0 \), thì vế phải bằng 0, nên ta tìm các bộ ba có tổng bằng 0 và tổng bình phương bằng 0 — chỉ xảy ra khi tất cả bằng 0.
%         \item Nếu \( x = y = z \), ta được phương trình một ẩn, có thể giải cụ thể.
%     \end{itemize}
%     Nói chung, bài toán dẫn đến phương pháp liệt kê theo \( n \) và sử dụng bất đẳng thức để khống chế giá trị các biến.
% \end{story*}

\end{document}