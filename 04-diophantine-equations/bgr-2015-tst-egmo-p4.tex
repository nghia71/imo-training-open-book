\documentclass[../04-diophantine-equations.tex]{subfiles}

\begin{document}

\begin{exercise*}[\gls{BGR 2015 EGMO TST}/P4]\label{example:BGR-2015-EGMO-TST-P4}[\textbf{\nameref{definition:30M}}]
	\footnotemark Chứng minh rằng với mọi số nguyên dương \( m \), tồn tại vô số cặp số nguyên dương \( (x, y) \) nguyên tố cùng nhau sao cho:
	\[ x \mid y^2 + m,\ \text{và}\ y \mid x^2 + m. \]
\end{exercise*}

\begin{remark*}
	Tìm cách xây dựng (hoặc suy luận) các nghiệm từ công thức tham số.  
	Thử coi \((x,y)\) vừa đủ điều kiện chia, kết hợp với sự nguyên tố cùng nhau của \((x,y)\).  
	Xem định lý Euclid về vô hạn số nguyên tố để tạo dãy vô hạn đáp ứng yêu cầu.
\end{remark*}

\footnotetext{\href{https://artofproblemsolving.com/community/c3943}{IMO SL 1992 P1.}}

\end{document}