\documentclass[../04-diophantine-equations.tex]{subfiles}

\begin{document}

\begin{example*}[\gls{CAN 2015 QRC}/P1]\label{example:CAN-2015-QRC-P1}[\textbf{\nameref{definition:25M}}]
	Tìm tất cả nghiệm nguyên của phương trình:
	\[
		7x^2y^2 + 4x^2 = 77y^2 + 1260.
	\]	
\end{example*}

\begin{story*}
    Phân tích phương trình dưới dạng tích hai biểu thức. Nhận thấy các hệ số đều chia hết cho 7 ngoại trừ hạng tử \( 4x^2 \), ta suy ra \( x \) phải chia hết cho 7.  
    Sau khi chuyển vế và nhóm các hạng tử hợp lý, ta có phương trình:
    \[
        (x^2 - 11)(7y^2 + 4) = 1216.
    \]
    Vì vế phải là số cố định, ta chỉ cần xét một số hữu hạn giá trị \( x \) và \( y \). Dùng thử các ước của 1216, ta tìm được các nghiệm \((x, y) = (\pm 7, \pm 2)\).
\end{story*}

\begin{soln}\footnotemark
    Các hệ số đều chia hết cho 7 ngoại trừ 4, nên \( x \) phải chia hết cho 7.

    Biến đổi phương trình:
    \[
        7x^2y^2 + 4x^2 = 77y^2 + 1260
        \quad \Rightarrow \quad
        (x^2 - 11)(7y^2 + 4) = 1216.
    \]
    
    Vì vế phải là hằng số, ta chỉ cần xét một số hữu hạn giá trị \( x \). Giả sử \( x = 7t \), thì:
    \[
        x^2 = 49t^2 \leq 1216 + 77y^2 + 4x^2 \Rightarrow x^2 < 122 \Rightarrow |x| < 11 \Rightarrow x \in \{0, \pm7\}.
    \]

    \begin{itemize}[topsep=0pt, partopsep=0pt, itemsep=0pt]
        \item Với \( x = 0 \Rightarrow 0 = 77y^2 + 1260 \), vô lý.
        \item Với \( x = \pm7 \):
        \[
            (x^2 - 11)(7y^2 + 4) = 1216 \Rightarrow (49 - 11)(7y^2 + 4) = 1216 \Rightarrow 38(7y^2 + 4) = 1216.
        \]
        \[
            7y^2 + 4 = \frac{1216}{38} = 32 \Rightarrow 7y^2 = 28 \Rightarrow y^2 = 4 \Rightarrow y = \pm2.
        \]
    \end{itemize}

    Kết luận, các nghiệm nguyên là \( (x, y) = (\pm 7, \pm 2) \).
\end{soln}

\footnotetext{\href{https://cms.math.ca/wp-content/uploads/2019/07/2015official_solutions.pdf}{Lời giải chính thức.}}

\end{document}