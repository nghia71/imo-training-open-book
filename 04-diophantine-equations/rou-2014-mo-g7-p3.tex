\documentclass[../04-diophantine-equations.tex]{subfiles}

\begin{document}

\begin{exercise*}[\gls{ROU 2014 MO}/G7/P3]\label{example:ROU-2014-MO-G7-P3}\textbf{[\nameref{definition:25M}]}
    Tìm tất cả các số nguyên dương \( n \) sao cho:
    \[
        17^n + 9^{n^2} = 23^n + 3^{n^2}.
    \]
\end{exercise*}

\begin{remark*}
    Nhận xét rằng \( 17^n < 23^n \) với \( n \ge 1 \), và \( 9^{n^2} > 3^{n^2} \) với \( n \ge 1 \). So sánh từng cặp và tìm điểm cân bằng.  
    Thử các giá trị nhỏ của \( n \), vì \( n^2 \) tăng rất nhanh trong luỹ thừa.
\end{remark*}

% \begin{story*}
%     Phương trình gồm hai cặp lũy thừa: một theo cơ số \( 17 \) và \( 23 \), và một theo \( 9 = 3^2 \) và \( 3 \).  
%     Với \( n \ge 1 \), ta có:
%     \begin{itemize}[topsep=0pt, partopsep=0pt, itemsep=0pt]
%         \item \( 17^n < 23^n \) do \( 17 < 23 \).
%         \item \( 9^{n^2} = (3^2)^{n^2} = 3^{2n^2} \) lớn hơn \( 3^{n^2} \).
%     \end{itemize}
%     Từ đó, ta viết lại phương trình:
%     \[
%         17^n - 23^n = 3^{n^2} - 9^{n^2} = 3^{n^2}(1 - 3^{n^2}).
%     \]
%     Vế trái là số âm (vì \( 17^n < 23^n \)), vế phải cũng âm (vì \( 3^{n^2} < 9^{n^2} \)), nên hai vế cùng dấu — hợp lý.

%     Tuy nhiên, vì cả \( 23^n - 17^n \) và \( 9^{n^2} - 3^{n^2} \) đều tăng rất nhanh, có thể chỉ có giá trị nhỏ \( n \) mới cho phương trình đúng.  
%     Thử các giá trị nhỏ như \( n = 1, 2 \) và đối chiếu hai vế, ta tìm được nghiệm duy nhất.
% \end{story*}

\end{document}