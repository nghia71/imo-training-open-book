\documentclass[../04-diophantine-equations.tex]{subfiles}

\begin{document}

\begin{exercise*}[\gls{ROU 2014 MO}/G8/P3]\label{example:ROU-2014-MO-G8-P3}\textbf{[\nameref{definition:15M}]}
    Tìm số nguyên nhỏ nhất \( n \) sao cho tập hợp 
    \[
        A = \{n, n + 1, n + 2, \ldots, 2n\}
    \]
    chứa năm phần tử \( a < b < c < d < e \) thỏa mãn
    \[
        \frac{a}{c} = \frac{b}{d} = \frac{c}{e}.
    \]
\end{exercise*}

\begin{remark*}
    Gọi tỉ số chung là \( r \), ta có:
    \[
        \frac{a}{c} = \frac{b}{d} = \frac{c}{e} = r \implies a = rc,\quad b = rd,\quad e = \frac{c}{r} \implies ae = c^2,\quad be = cd.
    \]
\end{remark*}

% \begin{story*}
%     Điều kiện đề bài tương đương với việc tồn tại số thực \( r \) sao cho:
%     \[
%         a = rc,\quad b = rd,\quad e = \frac{c}{r}.
%     \]
%     Nhân chéo các biểu thức:
%     \[
%         ae = c^2,\quad be = cd.
%     \]
%     Đây là các điều kiện ràng buộc chặt chẽ giữa các phần tử. Vì \( a < b < c < d < e \) thuộc đoạn \( [n, 2n] \), ta có thể kiểm tra bằng cách liệt kê.

%     Hướng tiếp cận hiệu quả:
%     \begin{itemize}[topsep=0pt, partopsep=0pt, itemsep=0pt]
%         \item Thử các giá trị nhỏ của \( n \), tạo tập \( A = \{n, n+1, \ldots, 2n\} \).
%         \item Duyệt tất cả các bộ 5 phần tử tăng dần trong \( A \), kiểm tra xem có thoả mãn các đẳng thức \( ae = c^2 \) và \( be = cd \) hay không.
%         \item Có thể dùng máy tính hoặc chương trình nhỏ để thực hiện kiểm tra.
%         \item Vì điều kiện rất mạnh, nghiệm nhỏ nhất sẽ xuất hiện ở giá trị \( n \) không quá lớn.
%     \end{itemize}
% \end{story*}

\end{document}