\documentclass[../09-contruction-methods.tex]{subfiles}

\begin{document}

\begin{example*}[\gls{IRN 2015 MO}/N5]\label{example:IRN-2015-MO-N5}[\textbf{\nameref{definition:20M}}]
    Cho \( p > 30 \) là một số nguyên tố. Chứng minh rằng tồn tại một số trong tập sau có dạng \( x^2 + y^2 \) với \( x, y \in \mathbb{Z} \):
    \[
        p + 1,\ 2p + 1,\ 3p + 1,\ \dots,\ (p - 3)p + 1
    \]
\end{example*}

\begin{story*}
    Ta cần chỉ ra rằng trong dãy \( p + 1, 2p + 1, \dots, (p - 3)p + 1 \) có ít nhất một số là tổng của hai bình phương.

    Ý tưởng chính:
    \begin{itemize}[topsep=0pt, partopsep=0pt, itemsep=0pt]
        \item Tìm \( x, y \in \mathbb{Z} \) sao cho \( x^2 + y^2 \equiv 1 \pmod{p} \).
        \item Do \( \mathbb{F}_p \) là trường, có thể chọn \( x, y \in \mathbb{F}_p \) phù hợp với điều kiện trên.
        \item Sau đó, xét xem giá trị \( x^2 + y^2 \) thuộc đoạn nào và đánh giá giá trị này là \( kp + 1 \) với \( k \le \frac{p - 3}{2} \).
    \end{itemize}
\end{story*}

\begin{soln}\footnotemark
    Thực ra, mệnh đề đúng với mọi \( p \ge 7 \), không chỉ \( p > 30 \).

    Chọn \( x, y \in \mathbb{F}_p \) sao cho:
    \[
        x \equiv \frac{3}{5} \pmod{p}, \quad y \equiv \frac{4}{5} \pmod{p}
    \]
	
    Vì \( \gcd(5, p) = 1 \), nên \( \tfrac{3}{5} \) và \( \tfrac{4}{5} \) tồn tại trong \( \mathbb{F}_p \). Khi đó:
    \[
        x^2 + y^2 \equiv \left( \frac{3}{5} \right)^2 + \left( \frac{4}{5} \right)^2 = \frac{9 + 16}{25} = 1 \pmod{p}
    \]

    Suy ra tồn tại \( x, y \in \mathbb{Z} \) sao cho \( x^2 + y^2 \equiv 1 \pmod{p} \), tức là tồn tại \( k \in \mathbb{N} \) sao cho:
    \[
        x^2 + y^2 = kp + 1
    \]

    Mặt khác, chọn \( x, y \in \left\{1, 2, \dotsc, \frac{p - 1}{2} \right\} \), thì:
    \[
        x^2 + y^2 \le 2 \cdot \left( \frac{p - 1}{2} \right)^2 = \frac{(p - 1)^2}{2} = \frac{p^2 - 2p + 1}{2}
    \]

    Vậy:
    \[
        kp + 1 \le \frac{p^2 - 2p + 1}{2} \Rightarrow k \le \frac{p - 1}{2}
    \]

    Mà \( 1 \le k \le \frac{p - 3}{2} \Rightarrow kp + 1 \in \{p + 1, 2p + 1, \dotsc, (p - 3)p + 1\} \)

    \textbf{Kết luận:} \textit{Tồn tại một số trong dãy \( p + 1, 2p + 1, \dotsc, (p - 3)p + 1 \) có dạng \( x^2 + y^2 \).}
\end{soln}

\footnotetext{\href{https://artofproblemsolving.com/community/c6h1139099p27598968}{Dựa theo lời giải của \textbf{math90}.}}

\end{document}