\documentclass[../01-divisibility.tex]{subfiles}

\begin{document}

\begin{exercise*}[\gls{RUS 2015 TST}/D10/P1]\label{example:RUS-2015-TST-D10-P1}[\textbf{\nameref{definition:15M}}]
    Chứng minh rằng tồn tại hai số nguyên dương \( a, b \) sao cho với mọi cặp số nguyên dương \( m, n \) nguyên tố cùng nhau, ta có:
    \[
        |a - m| + |b - n| > 1000.
    \]
\end{exercise*}

\begin{remark*}
    \begin{itemize}[topsep=0pt, partopsep=0pt, itemsep=0pt]
        \item Hãy thử chọn \( a \) và \( b \) đủ lớn và cùng chia hết cho một số lớn, ví dụ như \( a = b = 1000! \).
        \item Khi đó, \( m, n \) nguyên tố cùng nhau không thể cùng chia hết cho các thừa số nguyên tố của \( a \) hoặc \( b \), khiến khoảng cách tuyệt đối không thể nhỏ.
        \item Ý tưởng là buộc \( m \ne a \) và \( n \ne b \) bằng điều kiện chia hết.
    \end{itemize}
\end{remark*}

% \begin{story*}
%     Bài toán yêu cầu xây dựng hai số cố định \( a, b \) sao cho bất kỳ cặp số \( m, n \) nguyên tố cùng nhau nào cũng không thể nằm gần \( (a, b) \).
%     Ta xét việc chọn \( a = b = N! \) với \( N \) lớn (ví dụ \( N = 1000 \)).
%     Khi đó, mọi số nhỏ hơn hoặc bằng \( N \) đều chia hết \( a \) và \( b \),
%     trong khi \( m, n \) nguyên tố cùng nhau không thể đồng thời chia hết cho cùng các thừa số nguyên tố của \( N! \).
%     Vì vậy, \( m \ne a \) và \( n \ne b \), và sự khác biệt tuyệt đối với từng hoán vị là lớn.
%     Ta kiểm soát được giá trị \( |a - m| + |b - n| \) bằng việc ép \( m, n \) không thể gần \( a, b \). Từ đó ta đảm bảo giá trị luôn vượt quá 1000.
% \end{story*}

\end{document}