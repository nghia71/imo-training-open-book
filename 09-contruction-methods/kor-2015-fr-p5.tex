\documentclass[../09-contruction-methods.tex]{subfiles}

\begin{document}

\begin{exercise*}[\gls{KOR 2015 FR}/P1]\label{example:KOR-2015-FR-P1}[\textbf{\nameref{definition:20M}}]
	Cho số nguyên dương cố định \( k \). Xét hai dãy số \( A_n \) và \( B_n \) được định nghĩa như sau:
	\begin{align*}
		&A_1 = k, \quad A_2 = k, \quad A_{n+2} = A_n A_{n+1}, \\
		&B_1 = 1, \quad B_2 = k, \quad B_{n+2} = \frac{B_{n+1}^3 + 1}{B_n}.
	\end{align*}
	
	Chứng minh rằng với mọi số nguyên dương \( n \), biểu thức \( A_{2n}B_{n+3} \) là một số nguyên.
\end{exercise*}

\begin{remark*}
	Xét chứng minh bằng quy nạp trên \( n \), và tìm mối liên hệ giữa dãy \( B_n \) và \( A_n \) thông qua các biểu thức truy hồi.
	Đặc biệt lưu ý tính nguyên khi chia trong biểu thức định nghĩa \( B_{n+2} \).
\end{remark*}

% \begin{story*}
%     Ta cần chứng minh rằng \( A_{2n}B_{n+3} \) là số nguyên với mọi \( n \in \mathbb{Z}_{>0} \), trong đó:
%     \begin{itemize}[topsep=0pt, partopsep=0pt, itemsep=0pt]
%         \item Dãy \( A_n \) là cấp số nhân theo đệ quy bậc hai không tuyến tính: \( A_{n+2} = A_n A_{n+1} \), với \( A_1 = A_2 = k \);
%         \item Dãy \( B_n \) có quy luật phi tuyến: \( B_{n+2} = \frac{B_{n+1}^3 + 1}{B_n} \), với \( B_1 = 1, B_2 = k \).
%     \end{itemize}
    
%     Hướng giải đề xuất:
%     \begin{itemize}[topsep=0pt, partopsep=0pt, itemsep=0pt]
%         \item Dùng quy nạp để chứng minh tính nguyên của \( A_{2n}B_{n+3} \).
%         \item Thử biểu diễn các \( B_n \) đầu tiên theo \( A_n \) hoặc tìm một biểu thức đối ứng để liên hệ \( B_{n+3} \) với \( A_{2n} \).
%         \item Phân tích kỹ mẫu số \( B_n \) trong công thức truy hồi \( B_{n+2} = \frac{B_{n+1}^3 + 1}{B_n} \) để đảm bảo không gây chia cho số không hoặc chia không nguyên.
%     \end{itemize}
%     Ngoài ra, có thể khai thác tính chất chia hết của \( A_n \) và cấu trúc lũy thừa trong \( B_n \) để điều khiển biểu thức \( A_{2n}B_{n+3} \).
% \end{story*}

\end{document}