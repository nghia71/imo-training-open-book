\documentclass[../08-quadratic-residues.tex]{subfiles}

\begin{document}

\begin{example*}[\gls{EGMO 2015}/P5]\label{example:EGMO-2015-P5}\textbf{[\nameref{definition:25M}]}
    Cho \( m, n \) là các số nguyên dương với \( m > 1 \). Anastasia chia tập \( \{1, 2, \dots, 2m\} \) thành \( m \) cặp số.
    Boris chọn một số trong mỗi cặp và tính tổng các số được chọn.
    Chứng minh rằng Anastasia có thể chọn cách chia sao cho Boris không thể chọn được tổng bằng \( n \).
\end{example*}

\begin{story*}
    Bài toán yêu cầu chứng minh rằng luôn tồn tại một cách chia các số từ \( 1 \) đến \( 2m \) thành \( m \) cặp sao cho \textit{mọi cách chọn 1 số từ mỗi cặp} không bao giờ có tổng đúng bằng \( n \). Hướng tiếp cận:
    \begin{itemize}[topsep=0pt, partopsep=0pt, itemsep=0pt]
        \item Xét ba cách chia mẫu đặc biệt: tuần tự, xen kẽ, và phản xạ.
        \item Với mỗi cách chia, ta kiểm soát được tập hợp các tổng mà Boris có thể thu được: ràng buộc bởi biên độ hoặc phần dư modulo.
        \item Tùy theo giá trị của \( n \), ta chọn ra cách chia phù hợp để loại trừ.
    \end{itemize}
\end{story*}

\begin{soln}\footnotemark
    Xét ba cách chia tập \( \{1, 2, \dots, 2m\} \) như sau:

    \textbf{Chia kiểu 1 (tuần tự):}
    \[
        P_1 = \{ \{1,2\}, \{3,4\}, \dots, \{2m - 1, 2m\} \}
    \]
    Tổng nhỏ nhất: chọn phần tử nhỏ hơn mỗi cặp \( \Rightarrow s = m^2 \)  
    Tổng lớn nhất: chọn phần tử lớn hơn mỗi cặp \( \Rightarrow s = m^2 + m \)  
    \textit{Suy ra: nếu \( n < m^2 \) hoặc \( n > m^2 + m \), thì \( n \) không thể đạt được.}

    \textbf{Chia kiểu 2 (ghép đối xứng trung tâm):}
    \[
        P_2 = \{ \{1, m+1\}, \{2, m+2\}, \dots, \{m, 2m\} \}
    \]
    Tổng luôn có dạng \( s \equiv \sigma \pmod{m} \), với \( \sigma = \sum_{i=1}^{m} i = \frac{m(m+1)}{2} \)  
    \textit{Vậy nếu \( n \in [m^2, m^2 + m] \) mà \( n \not\equiv \sigma \pmod{m} \), thì \( n \) không thể đạt được.}

    \textbf{Chia kiểu 3 (đối xứng ngoài – trong):}
    \[
        P_3 = \{ \{1, 2m\}, \{2, 2m - 1\}, \dots, \{m, m+1\} \}
    \]
    Gọi \( d \) là số cặp trong đó Boris chọn phần tử lớn hơn  
    Tổng khi đó: \( s = \sigma + d(m - d) \)  
    \textit{Ta chứng minh được rằng:}  
    \[
        s \equiv \sigma \pmod{m} \Rightarrow s \in \left\{ \frac{m(m+1)}{2}, \; \frac{3m^2 + m}{2} \right\}
    \]
    Vì khoảng \( [m^2, m^2 + m] \) không giao với hai giá trị trên, nên \( n \in [m^2, m^2 + m] \) và \( n \equiv \sigma \pmod{m} \Rightarrow n \) không đạt được.

    \textbf{Kết luận:} Trong mọi trường hợp, Anastasia luôn có thể chọn một cách chia để Boris không thể đạt được tổng \( n \).
\end{soln}

\footnotetext{\href{https://www.egmo.org/egmos/egmo4/solutions.pdf}{Lời giải chính thức.}}

\end{document}