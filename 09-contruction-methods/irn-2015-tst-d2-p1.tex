\documentclass[../09-contruction-methods.tex]{subfiles}

\begin{document}

\begin{example*}[\gls{IRN 2015 TST}/D2/P1]\label{example:IRN-2015-TST-D2-P1}[\textbf{\nameref{definition:35M}}]
	Cho trước số tự nhiên \( n \). Tìm giá trị nhỏ nhất của \( k \) sao cho với mọi tập \( A \) gồm \( k \) số tự nhiên,
	luôn tồn tại một tập con của \( A \) có số phần tử chẵn và tổng các phần tử chia hết cho \( n \).
\end{example*}

\begin{story*}
	Bài toán yêu cầu tìm \( k \) nhỏ nhất sao cho mọi tập \( A \) gồm \( k \) số tự nhiên đều có một tập con chẵn phần tử có tổng chia hết cho \( n \).  
	Hướng giải gồm ba trường hợp tách biệt:
	\begin{itemize}[topsep=0pt, partopsep=0pt, itemsep=0pt]
		\item Nếu \( n \) là số lẻ: chọn \( A \) gồm \( 2n \) phần tử, chia làm hai nửa, áp dụng định lý tổng chia hết.
		\item Nếu \( n \equiv 0 \pmod{4} \): chọn \( A \) gồm \( n + 1 \) phần tử, chia thành hai phần, mỗi phần có \( k + 1 \) phần tử (với \( k = n/2 \)).
		\item Nếu \( n \equiv 2 \pmod{4} \): viết \( n = 2k \) với \( k \) lẻ, dùng cách nhóm các phần tử thành các cặp để áp dụng định lý tổng chia hết modulo \( k \), rồi kết hợp với chẵn-lẻ để điều khiển số phần tử.
	\end{itemize}
	Trong cả ba trường hợp, áp dụng khéo léo bổ đề về tồn tại tập con có tổng chia hết cho \( n \) giúp tìm ra tập con thỏa yêu cầu về tổng và số phần tử.
\end{story*}

\begin{soln}\footnotemark
	Trước hết ta chứng minh định lý sau.
	\begin{theorem*}[subset-divisibility]
		Cho \( n \) là một số nguyên dương. Khi đó, trong mọi tập \( X = \{x_1, x_2, \dots, x_n\} \) gồm \( n \) số nguyên,
		tồn tại một tập con \( A \subseteq X \) sao cho tổng các phần tử của \( A \) chia hết cho \( n \).
	\end{theorem*}
	\begin{subproof}
		Xét các tổng sau:
		\[
			A_1 = \{a_1\},\quad A_2 = \{a_1 + a_2\},\quad \dots,\quad A_n = \{a_1 + a_2 + \dots + a_n\}
		\]
		Nếu tồn tại \( i \) sao cho \( \overline{A_i} \equiv 0 \Mod{n} \), ta đã xong. Ngược lại, tồn tại hai chỉ số \( i < j \) sao cho:
		\[
			\overline{A_i} \equiv \overline{A_j} \Mod{n} \implies \overline{A_j - A_i} = a_{i+1} + \dots + a_j \equiv 0 \Mod{n}
		\]
	\end{subproof}
	
	Xét ba trường hợp:

	\textit{Trường hợp 1:} \( n = 2k \) là số chẵn và \( k \) là số lẻ.

	Lấy \( n+1 \) số tự nhiên bất kỳ. Chia chúng thành hai tập: \( A \): chứa các số lẻ, và \( B \): chứa các số chẵn.

	Vì \( t + s = n + 1 \) là số lẻ, nên một trong hai số \( t, s \) là chẵn, số còn lại là lẻ. Giả sử \( t \) chẵn.

	Chia các phần tử trong \( A \) thành \( \frac{t}{2} \) cặp. Bỏ một phần tử ra khỏi \( B \), phần còn lại chia thành \( \frac{s - 1}{2} \) cặp.

	Gọi \( \overline{X} \) là tổng các phần tử trong tập \( X \). Có tổng cộng:
	\[
		\frac{t + s - 1}{2} = \frac{n}{2} = k
	\]
	tổng từ các cặp. Áp dụng bổ đề, tồn tại tổ hợp các cặp sao cho tổng các phần tử chia hết cho \( k \). Mỗi tổng là tổng của hai số, nên toàn bộ tổng chia hết cho \( 2k = n \). Tập con này có số phần tử chẵn.

	\textit{Trường hợp 2:} \( 4 \mid n = 2k \).

	Gọi \( A = \{a_1, \dots, a_{n+1}\} \). Xét tập con \( A_1 = \{a_1, \dots, a_{k+1}\} \). Vì \( k \) chẵn, áp dụng trường hợp (1), tồn tại tập con \( X_1 \subseteq A_1 \) gồm số phần tử chẵn có tổng chia hết cho \( k \), tức \( \overline{X_1} = kt \).

	Phần còn lại của \( A \) có ít nhất \( k + 1 \) phần tử. Áp dụng lại ta có \( X_2 \subseteq A \setminus X_1 \) sao cho \( \overline{X_2} = kl \). Nếu \( t \) hoặc \( l \) chẵn thì xong. Nếu cả hai lẻ thì:
	\[
		\overline{X_1 \cup X_2} = k(t + l) = 2ks = ns
	\]
	và \( |X_1 \cup X_2| \) chẵn.

	\textit{Trường hợp 3:} \( n \) là số lẻ.

	Xét tập \( A = \{a_1, \dots, a_{2n}\} \), chia thành hai nửa:
	\[
		A_1 = \{a_1, \dots, a_n\},\quad A_2 = \{a_{n+1}, \dots, a_{2n}\}
	\]

	Áp dụng bổ đề cho \( A_1 \) và \( A_2 \), được \( X_1, X_2 \) sao cho \( \overline{X_1} \equiv \overline{X_2} \equiv 0 \Mod{n} \). Nếu \( X_1 \) hoặc \( X_2 \) có số phần tử chẵn, ta xong. Nếu cả hai lẻ, thì \( X_1 \cup X_2 \) có số phần tử chẵn và tổng chia hết cho \( n \).

	\textbf{Kết luận:} Trong cả ba trường hợp, luôn tồn tại một tập con chẵn phần tử có tổng chia hết cho \( n \). Do đó, giá trị nhỏ nhất của \( k \) là:
	\[
		k = \begin{cases}
			2n & \text{n lẻ} \\
			n + 1 & \text{n chẵn}
		\end{cases}
	\]
\end{soln}

\footnotetext{\href{https://artofproblemsolving.com/community/c6h1087607p5239629}{Dựa theo lời giải của \textbf{andria}.}}

\end{document}