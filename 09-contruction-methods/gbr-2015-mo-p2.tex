\documentclass[../09-contruction-methods.tex]{subfiles}

\begin{document}

\begin{exercise*}[\gls{GBR 2015 MO}/P2]\label{example:GBR-2015-MO-P2}[\textbf{\nameref{definition:15M}}]
	Tại trường tiểu học Oddesdon có một số lẻ lớp học. Mỗi lớp có một số lẻ học sinh. Từ mỗi lớp, một học sinh sẽ được chọn để tạo thành hội đồng học sinh.
	Hãy chứng minh rằng hai mệnh đề sau là tương đương:
    \begin{enumerate}[topsep=0pt, partopsep=0pt, itemsep=0pt]
        \item Có nhiều cách lập hội đồng học sinh sao cho số nam sinh là số lẻ hơn là số cách lập hội đồng sao cho số nữ sinh là số lẻ.
        \item Có một số lẻ lớp có nhiều nam sinh hơn nữ sinh.
    \end{enumerate}
\end{exercise*}

\begin{remark*}
	Xét biểu diễn các cách chọn theo tổ hợp nhị phân và theo mô hình đại số tổ hợp (như dùng định lý về parity hoặc đa thức tạo),
    đặc biệt lưu ý rằng số học sinh trong mỗi lớp là số lẻ.
\end{remark*}

% \begin{story*}
% 	Bài toán thuộc kiểu chuyển mệnh đề logic về tổ hợp, liên quan đến tính chẵn lẻ của số nam hoặc nữ được chọn từ từng lớp.
%     Vì số lớp và số học sinh mỗi lớp đều lẻ, có thể biểu diễn số cách chọn bằng tổng các số hạng với trọng số tương ứng, từ đó đưa về mô hình đại số nhị phân.
%     Trọng tâm là nhận ra rằng parities (số chẵn/lẻ) của các hoán vị lựa chọn có thể được tính qua định lý XOR hoặc các mô hình đại số khác để đối chiếu hai mệnh đề.
% \end{story*}

\end{document}