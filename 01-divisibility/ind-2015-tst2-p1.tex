\documentclass[../01-divisibility.tex]{subfiles}

\begin{document}

\begin{example*}[\gls{IND 2015 TST}2/P1]\label{example:IND-2015-TST2-P1}\textbf{[\nameref{definition:25M}]}\footnotemark
	Cho số nguyên \( n \geq 2 \), và đặt:
	\[
		A_n = \{2^n - 2^k \mid k \in \mathbb{Z},\, 0 \leq k < n \}.
	\]
	Tìm số nguyên dương lớn nhất không thể biểu diễn được dưới dạng tổng của một hay nhiều (không nhất thiết khác nhau) phần tử trong tập \( A_n \).
\end{example*}

\begin{story*}
	Bài toán liên quan đến cấu trúc của các số \( 2^n - 2^k \), với chiến lược chính là:
	\begin{itemize}[topsep=0pt, itemsep=0pt]
	    \item Dùng quy nạp để chứng minh mọi số lớn hơn một ngưỡng đều biểu diễn được.
	    \item Tìm giá trị nhỏ nhất thoả mãn đồng dư \( \equiv 1 \Mod{2^n} \) không thể rút xuống nhỏ hơn.
	    \item Sử dụng biểu diễn nhị phân duy nhất để tìm giá trị giới hạn không biểu diễn được.
	\end{itemize}
\end{story*}

\begin{soln}
	\textbf{Bước 1:} Mọi số lớn hơn \( (n - 2) \cdot 2^n + 1 \) đều biểu diễn được bằng tổng các phần tử trong \( A_n \). Ta chứng minh bằng quy nạp theo \( n \).

	\textit{Cơ sở:} \( n = 2 \Rightarrow A_2 = \{3, 2\} \). Mọi số dương \( \ne 1 \) đều biểu diễn được.

	\textit{Giả sử đúng với \( n - 1 \)}. Xét \( n > 2 \), và \( m > (n - 2) \cdot 2^n + 1 \).

	\textit{Trường hợp 1:} \( m \) chẵn. Khi đó
	\[
		\frac{m}{2} > (n - 3) \cdot 2^{n-1} + 1.
	\]
	Theo giả thiết quy nạp, \( \frac{m}{2} \) biểu diễn được từ \( A_{n-1} \), tức là từ các \( 2^{n-1} - 2^{k_i} \). Nhân hai vế:
	\[
		m = \sum (2^n - 2^{k_i + 1}) \in A_n.
	\]

	\textit{Trường hợp 2:} \( m \) lẻ. Khi đó
	\[
		\frac{m - (2^n - 1)}{2} > (n - 3) \cdot 2^{n-1} + 1,
	\]
	nên phần còn lại biểu diễn được từ \( A_{n-1} \), cộng thêm \( 2^n - 1 \in A_n \) để được \( m \).

	\textbf{Bước 2:} Chứng minh số \( (n - 2) \cdot 2^n + 1 \) không biểu diễn được.

	Gọi \( N \) là số nhỏ nhất \( \equiv 1 \Mod{2^n} \) biểu diễn được bằng tổng các phần tử trong \( A_n \). Khi đó:
	\[
		N = \sum (2^n - 2^{k_i}) = n \cdot 2^n - \sum 2^{k_i}.
	\]

	Nếu có \( k_i = k_j \), ta có thể thay \( 2 \cdot (2^n - 2^k) \to 2^n - 2^{k+1} \), từ đó giảm \( N \) đi \( 2^n \), mâu thuẫn với tính nhỏ nhất của \( N \). Vậy các \( k_i \) là phân biệt:
	\[
		\sum 2^{k_i} \le 2^0 + \dots + 2^{n-1} = 2^n - 1.
	\]

	Từ đó:
	\[
		N = n \cdot 2^n - (2^n - 1) = (n - 1) \cdot 2^n + 1.
	\]

	Suy ra số lớn nhất không biểu diễn được là:
	\[
		(n - 2) \cdot 2^n + 1.
	\]
\end{soln}

\footnotetext{\href{https://www.imo-official.org/problems/IMO2014SL.pdf}{IMO SL 2014 N1.}}

\end{document}