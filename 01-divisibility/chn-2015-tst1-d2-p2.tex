\documentclass[../01-divisibility.tex]{subfiles}

\begin{document}

\begin{example*}[\gls{CHN 2015 TST}1/D2/P2]\label{example:CHN-2015-TST1-D2-P2}\textbf{[\nameref{definition:10M}]}
	Cho trước một số nguyên dương \( n \). Chứng minh rằng: Với mọi số nguyên dương \( a, b, c \) không vượt quá \( 3n^2 + 4n \),
	tồn tại các số nguyên \( x, y, z \) có giá trị tuyệt đối không vượt quá \( 2n \) và không đồng thời bằng 0, sao cho
	\[
		ax + by + cz = 0.
	\]
\end{example*}

\begin{story*}
	Bài toán yêu cầu tìm một bộ số nguyên nhỏ \( (x, y, z) \) không đồng thời bằng 0 sao cho tổ hợp tuyến tính \( ax + by + cz = 0 \).  
	Ta xét tập giá trị mà biểu thức \( ax + by + cz \) có thể đạt được khi \( x, y, z \in [-n, n] \), và so sánh với số lượng bộ ba như vậy.  
	Vì số tổ hợp \( (x, y, z) \) nhiều hơn số giá trị có thể nhận, theo nguyên lý Dirichlet sẽ tồn tại hai bộ khác nhau cho cùng một giá trị.  
	Từ đó, trừ hai biểu thức tương ứng ta được tổ hợp không tầm thường thoả mãn đẳng thức cần tìm.
\end{story*}

\begin{soln}\footnotemark
	\textbf{Bước 1:} Xét tập giá trị
	\[
		A = \{ ax + by + cz \mid x, y, z \in \mathbb{Z} \cap [-n, n] \}.
	\]

	Ta có \( |A| \leq (2n \cdot a + 2n \cdot b + 2n \cdot c) + 1 \le 6n(3n^2 + 4n) + 1 = 6n^3 + 8n^2 + 1 \).

	\textbf{Bước 2:} Có tổng cộng \( (2n + 1)^3 \) bộ \( (x, y, z) \in [-n, n]^3 \), mà
	\[
		(2n+1)^3 > 6n^3 + 8n^2 + 1,
	\]
	nên theo \nameref{theorem:pigeonhole-principle}, tồn tại hai bộ khác nhau
	\[
		(x, y, z), (x', y', z') \in ([-n, n] \cap \mathbb{Z})^3 \quad \text{với} \quad ax + by + cz = ax' + by' + cz'.
	\]

	\textbf{Bước 3:} Trừ hai vế:
	\[
		a(x - x') + b(y - y') + c(z - z') = 0 \implies ax + by + cz = 0,
	\]
	với \( x - x', y - y', z - z' \in [-2n, 2n] \), không đồng thời bằng 0.

	\textbf{Kết luận:} Tồn tại bộ \( (x, y, z) \) thoả mãn yêu cầu đề bài.
\end{soln}

\footnotetext{\href{https://artofproblemsolving.com/community/c6h1063058p4617378}{Lời giải của \textbf{nayel}.}}

\end{document}