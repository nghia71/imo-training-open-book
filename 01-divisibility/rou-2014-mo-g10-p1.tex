\documentclass[../01-divisibility.tex]{subfiles}

\begin{document}

\begin{exercise*}[\gls{ROU 2014 MO}/G10/P1]\label{example:ROU-2014-MO-G10-P1}\textbf{[\nameref{definition:20M}]}
    Cho \( n \) là một số tự nhiên. Tính giá trị của biểu thức
    \[
        \sum_{k=1}^{n^2} \#\left\{ d \in \mathbb{N} \,\middle|\, 1 \le d \le k \le d^2 \le n^2,\ k \equiv 0 \Mod{d} \right\}.
    \]
    Trong đó, ký hiệu \( \# \) biểu thị số phần tử của tập hợp.
\end{exercise*}

\begin{remark*}
    Thay đổi góc nhìn: với mỗi số chia \( d \), tìm những giá trị \( k \) thỏa \( d \mid k \), \( k \le d^2 \), và tổng quát lại theo \( d \) thay vì \( k \).
\end{remark*}

% \begin{story*}
%     Với mỗi số \( k \), ta đếm các ước \( d \) sao cho \( d \mid k \) và \( d \le k \le d^2 \le n^2 \). Đặt \( k = dm \), ta có:
%     \[
%         1 \le m \le \min\left(d, \left\lfloor \frac{n^2}{d} \right\rfloor\right).
%     \]
    
%     Tổng ban đầu tương đương:
%     \[
%         \sum_{d=1}^{n} \min\left(d, \left\lfloor \frac{n^2}{d} \right\rfloor\right).
%     \]
% \end{story*}

\end{document}