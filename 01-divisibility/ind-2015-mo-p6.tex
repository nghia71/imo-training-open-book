\documentclass[../01-divisibility.tex]{subfiles}

\begin{document}

\begin{example*}[\gls{IND 2015 MO}/P6]\label{example:IND-2015-N6}\textbf{[\nameref{definition:15M}]}
	Chứng minh rằng từ một tập gồm \( 11 \) số chính phương, ta luôn có thể chọn ra sáu số \( a^2, b^2, c^2, d^2, e^2, f^2 \) sao cho:
	\[
		a^2 + b^2 + c^2 \equiv d^2 + e^2 + f^2 \Mod{12}
	\]
\end{example*}

\begin{story*}
	Bài toán yêu cầu tìm hai bộ ba số chính phương có tổng đồng dư modulo 12.  
	Ta xét phần dư của số chính phương theo modulo 12 (chỉ có thể là \( 0, 1, 4, 9 \)), sau đó áp dụng nguyên lý Dirichlet để buộc tồn tại phần dư xuất hiện nhiều lần.  
	Từ đó, xét các tổ hợp 3 phần tử và so sánh tổng modulo 12 giữa các nhóm khác nhau để tìm hai tổng bằng nhau.
\end{story*}

\begin{soln}
	Các phần dư khả dĩ của một số chính phương modulo 12 là:
	\[
		x^2 \equiv 0, 1, 4, 9 \Mod{12}.
	\]

	Gọi \( S \) là tập gồm 11 số chính phương. Mỗi phần tử của \( S \) thuộc một trong 4 lớp dư trên.

	Ta cần tìm hai tập rời nhau \( A, B \subset S \), mỗi tập gồm 3 phần tử, sao cho
	\[
		\sum_{x \in A} x \equiv \sum_{y \in B} y \Mod{12}.
	\]

	\textit{Trường hợp 1:} Có ít nhất 6 phần tử trong \( S \) có cùng phần dư.  
	Chia chúng thành 2 bộ ba giống nhau → tổng bằng nhau → đẳng thức modulo 12 hiển nhiên đúng.

	\textit{Trường hợp 2:} Có một phần dư xuất hiện 4 hoặc 5 lần.  
	Theo Dirichlet, phần dư khác xuất hiện ít nhất 2 lần. Từ đó có thể chọn hai bộ ba từ hai phần dư khác nhau nhưng tạo ra tổng đồng dư nhau.

	\textit{Trường hợp 3:} Mỗi phần dư xuất hiện tối đa 3 lần.  
	Khi đó số lượng bộ ba là hữu hạn nhưng đáng kể: có tổng cộng \( \binom{11}{3} = 165 \) cách chọn bộ ba.  
	Mà chỉ có hữu hạn tổng có thể xảy ra modulo 12 với các phần dư \( \{0,1,4,9\} \), nên tồn tại hai bộ ba khác nhau có tổng đồng dư nhau.

	\textbf{Kết luận:} Trong mọi trường hợp, luôn tồn tại hai bộ ba số chính phương rời nhau sao cho tổng của chúng đồng dư modulo 12.
\end{soln}

\footnotetext{\href{https://artofproblemsolving.com/community/c6h623456p3730860}{Dựa theo lời giải của \textbf{Sahil}.}}

\end{document}