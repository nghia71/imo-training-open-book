\documentclass[../01-divisibility.tex]{subfiles}

\begin{document}

\begin{example*}[\gls{THA 2015 MO}/P1]\label{example:THA-2015-MO-P1}\textbf{[\nameref{definition:25M}]}
	Cho số nguyên tố \( p \), và dãy số nguyên dương \( a_1, a_2, a_3, \dots \) thỏa mãn:
	\[
		a_n a_{n+2} = a_{n+1}^2 + p \quad \text{với mọi số nguyên dương } n.
	\]
	
	Chứng minh rằng với mọi \( n \), ta có:
	\[
		a_{n+1} \mid a_n + a_{n+2}.
	\]
\end{example*}

\begin{story*}
	Bài toán yêu cầu chứng minh một dạng chia hết xuất phát từ một đệ quy bậc hai có điều chỉnh bởi hằng số \( p \).
	Chiến lược:
	\begin{itemize}[topsep=0pt, partopsep=0pt, itemsep=0pt]
		\item Sử dụng hai lần liên tiếp định nghĩa đệ quy để trừ hai phương trình.
		\item Từ đó rút ra được mối liên hệ giữa \( a_n + a_{n+2} \) và \( a_{n+1} \), rồi chứng minh \( a_{n+1} \mid a_n + a_{n+2} \).
		\item Để chứng minh chia hết, sử dụng phép phản chứng với giả thiết \( \gcd(a_{n+1}, a_{n+2}) > 1 \) và dẫn đến mâu thuẫn về chia hết cho modulo \( p^2 \).
	\end{itemize}
\end{story*}

\begin{soln}(Cách 1)\footnotemark
	Từ giả thiết:
	\[
		a_n a_{n+2} = a_{n+1}^2 + p,\quad a_{n+1} a_{n+3} = a_{n+2}^2 + p.
	\]

	Trừ hai vế:
	\[
		a_n a_{n+2} - a_{n+1}^2 = a_{n+1} a_{n+3} - a_{n+2}^2 \implies
		a_{n+2}(a_n + a_{n+2}) = a_{n+1}(a_{n+1} + a_{n+3}).
	\]

	Giả sử \( \gcd(a_{n+1}, a_{n+2}) = d > 1 \). Từ:
	\[
		a_n a_{n+2} = a_{n+1}^2 + p \implies d \mid p \implies d = p.
	\]

	Xét tiếp:
	\[
		a_{n+2} a_{n+4} = a_{n+3}^2 + p,\quad a_{n+1} a_{n+3} = a_{n+2}^2 + p.
	\]

	Vì \( p \mid a_{n+2} \), thì:
	\[
		a_{n+2} \equiv 0 \Mod{p} \implies a_{n+2}^2 \equiv 0 \Mod{p^2}
	\implies a_{n+1} a_{n+3} \equiv p \Mod{p^2}.
	\]

	Nhưng:
	\[
		a_{n+1}, a_{n+3} \equiv 0 \Mod{p} \implies a_{n+1} a_{n+3} \equiv 0 \Mod{p^2}
	\implies p \equiv 0 \Mod{p^2}
	\]
	là mâu thuẫn. Vậy \( \gcd(a_{n+1}, a_{n+2}) = 1 \).

	Từ:
	\[
		a_{n+2}(a_n + a_{n+2}) = a_{n+1}(a_{n+1} + a_{n+3})
	\]
	và \( \gcd(a_{n+1}, a_{n+2}) = 1 \implies a_{n+1} \mid a_n + a_{n+2} \)

	\textbf{Kết luận:} Ta đã chứng minh:
	\[
		\boxed{a_{n+1} \mid a_n + a_{n+2} \quad \text{với mọi } n}
	\]
\end{soln}

\footnotetext{\href{https://artofproblemsolving.com/community/c1068820h2236874p29520199}{Lời giải của \textbf{rstenetbg}.}}

\end{document}