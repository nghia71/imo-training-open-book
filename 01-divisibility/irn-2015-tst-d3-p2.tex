\documentclass[../01-divisibility.tex]{subfiles}

\begin{document}

\begin{example*}[\gls{IRN 2015 TST}/D3-P2]\label{example:IRN-2015-TST-D3-P2}\textbf{[\nameref{definition:30M}]}
	Giả sử \( a_1, a_2, a_3 \) là ba số nguyên dương cho trước. Xét dãy số được xác định bởi công thức:
	\[
		a_{n+1} = \text{lcm}[a_n, a_{n-1}] - \text{lcm}[a_{n-1}, a_{n-2}] \quad \text{với } n \geq 3,
	\]
	trong đó \( [a, b] \) ký hiệu bội chung nhỏ nhất của \( a \) và \( b \), và chỉ được áp dụng với các số nguyên dương.

	Chứng minh rằng tồn tại một số nguyên dương \( k \leq a_3 + 4 \) sao cho \( a_k \leq 0 \).
\end{example*}

\begin{story*}
	Bài toán yêu cầu chứng minh rằng chuỗi được xây dựng theo công thức liên quan đến LCM sẽ đạt giá trị không dương trong thời gian hữu hạn.
	Chiến lược:
	\begin{itemize}[topsep=0pt, itemsep=0pt]
		\item Xây dựng chuỗi phụ \( b_n = \dfrac{a_n}{\text{lcm}(a_{n-2}, a_{n-3})} \) để theo dõi sự suy giảm.
		\item Chứng minh \( b_n \) nguyên và giảm dần.
		\item Từ đó suy ra tồn tại một \( k \leq a_3 + 3 \) sao cho \( a_k \leq 0 \).
	\end{itemize}
\end{story*}

\begin{soln}\footnotemark
	Ta chứng minh điều mạnh hơn: tồn tại \( k \leq a_3 + 3 \) sao cho \( a_k \leq 0 \).

	\textbf{Bước 1:} Đặt
	\[
		b_n = \frac{a_n}{\text{lcm}(a_{n-2}, a_{n-3})} \quad \text{với mọi } n \geq 5.
	\]
	Ta chứng minh rằng \( b_n \in \mathbb{N} \) và giảm dần.

	Với \( n \geq 4 \), \( a_{n-2} \mid a_n \) và \( a_{n-3} \mid a_n \), nên \( \text{lcm}(a_{n-2}, a_{n-3}) \mid a_n \), suy ra \( b_n \in \mathbb{N} \).

	\begin{claim*}
		Với mọi \( n \geq 5 \), ta có \( b_{n+1} < b_n \).
	\end{claim*}
	\begin{subproof}
		Ta có
		\[
		a_{n+1} = \text{lcm}(a_n, a_{n-1}) - \text{lcm}(a_{n-1}, a_{n-2}).
		\]
		Thay \( a_n = b_n \cdot \text{lcm}(a_{n-2}, a_{n-3}) \), ta suy ra:
		\[
		a_{n+1} = \text{lcm}(b_n, a_{n-2}, a_{n-3}, a_{n-1}) - \text{lcm}(a_{n-1}, a_{n-2}).
		\]
		Vì \( a_{n-3} \mid a_{n-1} \), nên
		\[
		a_{n+1} = \text{lcm}(b_n, a_{n-2}, a_{n-1}) - \text{lcm}(a_{n-1}, a_{n-2}) \implies b_{n+1} < b_n.
		\]
	\end{subproof}

	\textbf{Bước 2:} Ước lượng \( b_5 \).

	Ta có
	\[
	a_4 = \text{lcm}(a_3, a_2) - \text{lcm}(a_2, a_1) = c \cdot a_2,\quad \text{với } c \leq a_3 - 1,
	\]
	và
	\[
	a_5 = \text{lcm}(a_4, a_3) - \text{lcm}(a_3, a_2) = \text{lcm}(c a_2, a_3) - \text{lcm}(a_3, a_2).
	\]
	Suy ra:
	\[
	b_5 = \frac{a_5}{\text{lcm}(a_3, a_2)} \leq c - 1 \leq a_3 - 2.
	\]

	\textbf{Kết luận:} Dãy \( b_n \) nguyên, giảm dần, bắt đầu từ \( b_5 \leq a_3 - 2 \), nên sau tối đa \( a_3 - 2 \) bước sẽ đạt giá trị không dương. Do đó, tồn tại \( k \leq a_3 + 3 \) sao cho \( a_k = 0 \leq 0 \).
\end{soln}

\footnotetext{\href{https://artofproblemsolving.com/community/c6h1100830p25301244}{Lời giải của \textbf{guptaamitu1}.}}

\end{document}