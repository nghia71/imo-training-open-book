\documentclass[../01-divisibility.tex]{subfiles}

\begin{document}

\begin{example*}[\gls{CHN 2015 TST}1/D1/P2]\label{example:CHN-2015-TST1-D1-P2}\textbf{[\nameref{definition:10M}]}
	Cho dãy các số nguyên dương phân biệt \( a_1, a_2, a_3, \ldots \), và một hằng số thực \( 0 < c < \dfrac{3}{2} \).  
	Chứng minh rằng tồn tại vô hạn số nguyên dương \( k \) sao cho:
	\[
		\operatorname{lcm}(a_k, a_{k+1}) > c k.
	\]
\end{example*}

\begin{story*}
	Hướng giải bắt đầu từ giả định phản chứng: từ một chỉ số \( K \) trở đi, các giá trị \( \operatorname{lcm}(a_k, a_{k+1}) \) đều không vượt quá \( ck \).  
	Ta liên hệ bất đẳng thức giữa \( \gcd \) và \( \operatorname{lcm} \) để tạo ra cận dưới cho tổng \( \frac{1}{a_k} + \frac{1}{a_{k+1}} \), rồi cộng lại trên khoảng lớn.  
	Sử dụng đặc điểm phân biệt của các phần tử trong dãy, tổng này không thể vượt quá một giới hạn. So sánh với cận dưới tăng theo \( \ln N \) sẽ dẫn đến mâu thuẫn, từ đó khẳng định điều cần chứng minh.
\end{story*}

\begin{soln}\footnotemark
	Giả sử phản chứng rằng tồn tại số nguyên \( K \) sao cho
	\[
		\operatorname{lcm}(a_k, a_{k+1}) \le ck \quad \text{với mọi } k \ge K.
	\]

	\begin{claim*}
		Với mọi \( k \ge K \), ta có
		\[
			\frac{1}{a_k} + \frac{1}{a_{k+1}} \ge \frac{3}{ck}.
		\]
	\end{claim*}

	\begin{subproof}
		Ta dùng
		\[
			\frac{1}{a_k} + \frac{1}{a_{k+1}} = 
			\frac{a_k + a_{k+1}}{\gcd(a_k, a_{k+1}) \cdot \operatorname{lcm}(a_k, a_{k+1})}.
		\]
		Vì \( a_k \ne a_{k+1} \), ta có \( a_k + a_{k+1} > 2\min \ge 2\gcd \), mà tổng chia hết cho \( \gcd \), nên
		\[
			a_k + a_{k+1} \ge 3\gcd(a_k, a_{k+1}) \implies \frac{1}{a_k} + \frac{1}{a_{k+1}} \ge \frac{3}{\operatorname{lcm}(a_k, a_{k+1})} \ge \frac{3}{ck}.
		\]
	\end{subproof}

	Cộng hai vế từ \( k = K \) đến \( N \), ta có
	\[
		\frac{3}{c} \sum_{k=K}^N \frac{1}{k}
		\le \sum_{k=K}^N \left( \frac{1}{a_k} + \frac{1}{a_{k+1}} \right)
		\le 2 \sum_{j=1}^{\max a_j} \frac{1}{j}.
	\]

	Mà \( \sum \frac{1}{k} \sim \ln N \), \( \sum \frac{1}{j} \sim \ln (\max a_j) \le \ln(N+1) \), nên
	\[
		\frac{3}{c} \ln N \le 2 \ln(N + 1).
	\]

	Chia hai vế cho \( \ln N \) rồi lấy giới hạn:
	\[
		\frac{3}{c} \le 2 \quad \text{(vô lý vì \( c < \frac{3}{2} \))}.
	\]

	Suy ra giả thiết sai. Vậy tồn tại vô hạn chỉ số \( k \) sao cho
	\[
		\boxed{\operatorname{lcm}(a_k, a_{k+1}) > ck.}
	\]
\end{soln}

\footnotetext{\href{https://artofproblemsolving.com/community/c6h1062614p18497986}{Lời giải của \textbf{TheUltimate123}.}}

\end{document}