\documentclass[../01-divisibility.tex]{subfiles}

\begin{document}

\begin{example*}[\gls{IMO 2023}/P1]\label{example:IMO-2023-P1}\textbf{[\nameref{definition:5M}]}
    Xác định tất cả các số nguyên hợp dương \( n \) thỏa mãn tính chất sau: nếu các ước số dương của \( n \) là \( 1 = d_1 < d_2 < \dots < d_k = n \),
    thì \( d_i \mid (d_{i+1} + d_{i+2}) \quad \text{với mọi } 1 \leq i \leq k - 2. \)
\end{example*}

\begin{story*}
    Bài toán yêu cầu khảo sát cấu trúc các ước của một số nguyên hợp \( n \) sao cho mỗi ước \( d_i \) chia hết tổng của hai ước kế tiếp lớn hơn nó.  
    Một số hướng tiếp cận hiệu quả gồm:
    \begin{itemize}[topsep=0pt, itemsep=0pt]
        \item Thử các trường hợp \( n = p^r \), với \( p \) nguyên tố.
        \item Phản chứng nếu \( n \) có nhiều hơn một thừa số nguyên tố.
        \item Khai thác đối xứng \( d_i d_{k+1-i} = n \) và dùng quy nạp theo chia hết.
    \end{itemize}
\end{story*}

\begin{soln}(Cách 1)\footnotemark
    \textbf{Bước 1:} Giả sử \( n = p^r \). Khi đó \( d_i = p^{i-1} \), và
    \[
        d_i \mid d_{i+1} + d_{i+2} \Leftrightarrow p^{i-1} \mid p^i + p^{i+1} = p^i(1 + p),
    \]
    luôn đúng với mọi \( i \). Vậy mọi lũy thừa của số nguyên tố đều thỏa mãn.

    \textbf{Bước 2:} Giả sử \( n \) có ít nhất hai thừa số nguyên tố. Gọi \( p < q \) là hai thừa số nguyên tố nhỏ nhất.

    Khi đó tồn tại đoạn gồm các ước:
    \[
        d_j = p^{j-1},\quad d_{j+1} = p^j,\quad d_{j+2} = q,
    \]
    và ở cuối:
    \[
        d_{k-j-1} = \frac{n}{q},\quad d_{k-j} = \frac{n}{p^j},\quad d_{k-j+1} = \frac{n}{p^{j-1}}.
    \]

    Từ giả thiết:
    \[
        \frac{n}{q} \mid \left( \frac{n}{p^j} + \frac{n}{p^{j-1}} \right) \implies p^j \mid q(p + 1) \implies p \mid q,
    \]
    mâu thuẫn vì \( p \ne q \).

    \textbf{Kết luận:} \( n \) phải là lũy thừa của một số nguyên tố.
\end{soln}

\bigbreak

\begin{soln}(Cách 2)\footnotemark[\value{footnote}]
    \begin{claim*}
        \( d_i \mid d_{i+1} \) với mọi \( 1 \leq i \leq k - 1 \).
    \end{claim*}
    \begin{subproof}
        Chứng minh bằng quy nạp.

        Cơ sở: \( d_1 = 1 \Rightarrow d_1 \mid d_2 \).

        Giả sử \( d_{i-1} \mid d_i \), từ đề bài:
        \[
            d_{i-1} \mid d_i + d_{i+1} \implies d_{i-1} \mid d_{i+1}.
        \]
        Do \( d_{i-1} \mid d_i \) và \( d_i \mid d_{i+1} \), suy ra \( d_{i-1} \mid d_{i+1} \), nên \( d_i \mid d_{i+1} \).
    \end{subproof}

    Do đó, mọi ước \( d_i \) là bội của \( d_2 \), và \( d_2 \) là số nguyên tố nhỏ nhất chia \( n \), suy ra \( n \) là lũy thừa của một số nguyên tố.
\end{soln}

\newpage

\begin{soln}(Cách 3)\footnotemark[\value{footnote}]
    \textbf{Bước 1:} Sử dụng đối xứng \( d_i d_{k+1-i} = n \). Đặt
    \[
        d_{k-i-1} \mid d_{k-i} + d_{k-i+1} \Leftrightarrow \frac{n}{d_{i+2}} \mid \left( \frac{n}{d_{i+1}} + \frac{n}{d_i} \right).
    \]

    Nhân hai vế với \( d_i d_{i+1} d_{i+2} \), ta có:
    \[
        d_i d_{i+1} \mid d_i d_{i+2} + d_{i+1} d_{i+2} \implies d_i \mid d_{i+1} d_{i+2}. \quad (1)
    \]

    Mặt khác, từ đề bài:
    \[
        d_i \mid d_{i+1}^2 + d_{i+1} d_{i+2}.
    \]

    Kết hợp với (1) suy ra \( d_i \mid d_{i+1}^2 \).

    \textbf{Bước 2:} Gọi \( p = d_2 \) là ước nguyên tố nhỏ nhất. Ta dùng quy nạp:

    \begin{claim*}
        \( p \mid d_i \) với mọi \( i \geq 2 \).
    \end{claim*}
    \begin{subproof}
        Cơ sở đúng với \( d_2 \).

        Giả sử \( p \mid d_j \), từ \( d_j \mid d_{j+1}^2 \) và \( p \mid d_j \), do \( p \) nguyên tố nên \( p \mid d_{j+1} \).
    \end{subproof}

    Do đó, mọi \( d_i \) chia hết cho \( p \). Nếu tồn tại ước nguyên tố khác \( q \ne p \), thì \( p \mid q \), mâu thuẫn.

    \textbf{Kết luận:} \( n \) là lũy thừa của một số nguyên tố.
\end{soln}

\footnotetext{\samepage \href{https://www.imo-official.org/problems/IMO2023SL.pdf}{Shortlist 2023 with solutions.}}

\end{document}