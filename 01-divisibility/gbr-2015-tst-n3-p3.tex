\documentclass[../03-arithmetic-functions.tex]{subfiles}

\begin{document}

\begin{exercise*}[\gls{GBR 2015 TST}/N3/P3]\label{example:GBR-2015-TST-N3-P3}\textbf{[\nameref{definition:25M}]}
    Cho các số nguyên dương phân biệt đôi một \( a_1 < a_2 < \cdots < a_n \), trong đó \( a_1 \) là số nguyên tố và \( a_1 \geq n + 2 \).
    Trên đoạn thẳng \( I = \left[0, \prod_{i=1}^n a_i \right] \) trên trục số thực, đánh dấu tất cả các số nguyên chia hết cho ít nhất một trong các số \( a_1, a_2, \ldots, a_n \).
    Các điểm này chia đoạn \( I \) thành nhiều đoạn con.

    Chứng minh rằng tổng bình phương độ dài các đoạn con đó chia hết cho \( a_1 \).
\end{exercise*}

\begin{remark*}
    Hãy xét các đoạn con nằm giữa hai số nguyên liên tiếp không bị đánh dấu. Gọi độ dài mỗi đoạn là \( \ell_i \), ta cần chứng minh \( \sum \ell_i^2 \equiv 0 \pmod{a_1} \).
\end{remark*}

% \begin{story*}
%     Bài toán này gợi ý sử dụng \textbf{hàm đặc trưng chia hết} để xác định vị trí các điểm chia đoạn.  
%     Từ đó, ta có thể mô hình hóa mỗi đoạn con bằng độ dài giữa hai số nguyên không bị đánh dấu liên tiếp, tức là thuộc phần bù của hợp các tập bội.

%     Ý tưởng then chốt gồm:
%     \begin{itemize}[topsep=0pt, partopsep=0pt, itemsep=0pt]
%         \item Mỗi đoạn con là khoảng giữa hai số nguyên không chia hết cho bất kỳ \( a_i \), tức là nằm ngoài tập hợp các bội.
%         \item Tổng bình phương độ dài các đoạn này có thể được biểu diễn bằng \textbf{tổng trên phần bù của hợp các cấp số cộng}, liên quan đến hàm số đặc trưng và đếm tổ hợp.
%         \item Cuối cùng, bằng cách xét tính đối xứng theo modulo \( a_1 \) hoặc tính chu kỳ theo bội chung, ta chứng minh được tổng đó chia hết cho \( a_1 \).
%     \end{itemize}
% \end{story*}

\end{document}