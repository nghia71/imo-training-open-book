\documentclass[../01-divisibility.tex]{subfiles}

\begin{document}

\begin{example*}[\gls{CAN 2015 QRC}/P3]\label{example:CAN-2015-QRC-P3}\textbf{[\nameref{definition:10M}]}
	Gọi \( N \) là một số có ba chữ số phân biệt và khác 0. Ta gọi \( N \) là một số \textit{mediocre} nếu nó có tính chất sau:
	khi viết ra tất cả 6 hoán vị có ba chữ số từ các chữ số của \( N \), trung bình cộng của chúng bằng chính \( N \).
	Ví dụ: \( N = 481 \) là số mediocre vì trung bình cộng của các số \( \{418, 481, 148, 184, 814, 841\} \) bằng \( 481 \).

	Hãy xác định số mediocre lớn nhất.	
\end{example*}

\begin{story*}
	Ý tưởng chính là sử dụng đối xứng trong các hoán vị chữ số: tổng của 6 hoán vị luôn là \( 222(a + b + c) \), do đó trung bình là \( 37(a + b + c) \).  
	Điều này cho ta phương trình liên hệ giữa chữ số hàng trăm và hai chữ số còn lại. Ta biến đổi về phương trình Diophantine tuyến tính và thử giá trị hợp lý, bắt đầu từ chữ số hàng trăm lớn nhất, để tìm nghiệm thoả mãn điều kiện phân biệt và khác 0.
\end{story*}

\begin{soln}\footnotemark
	Giả sử \( abc \) là một số mediocre. Sáu hoán vị gồm \( \{abc, acb, bac, bca, cab, cba\} \). Theo \nameref{lemma:perm-average}, tổng của chúng là:
	\[
		222(a + b + c) \implies \frac{222(a + b + c)}{6} = 37(a + b + c).
	\]
	
	Vì \( abc \) là mediocre, nên:
	\[
		100a + 10b + c = 37(a + b + c) \implies 63a = 27b + 36c.
	\]

	\textit{Trường hợp 1:} \( a = 9, 8, 7 \) đều bị loại vì không tạo thành chữ số phân biệt hợp lệ.

	\textit{Trường hợp 2:} Xét \( a = 6 \), ta có
	\[
		378 = 27b + 36c \implies 3b + 4c = 42.
	\]

	Giải phương trình nguyên:
	\[
		b = \frac{42 - 4c}{3}
	\]
	đòi hỏi \( 4c \equiv 0 \Mod{3} \implies c \in \{3, 6, 9\} \), lần lượt cho:
	\begin{itemize}[topsep=0pt, partopsep=0pt, itemsep=0pt]
	    \item \( c = 3 \implies b = 10 \) (không hợp lệ)
	    \item \( c = 6 \implies b = 6 \implies abc = 666 \) (không phân biệt)
	    \item \( c = 9 \implies b = 2 \implies abc = 629 \) (thoả mãn)
	\end{itemize}

	\textbf{Kết luận:} Số mediocre lớn nhất là \( \boxed{629} \).
\end{soln}

\footnotetext{\href{https://cms.math.ca/wp-content/uploads/2019/07/2015official_solutions.pdf}{Lời giải chính thức.}}

\end{document}