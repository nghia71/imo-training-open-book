\documentclass[../01-divisibility.tex]{subfiles}

\begin{document}

\begin{exercise*}[\gls{HUN 2015 TST}/KMA/633]\label{example:HUN-2015-TST-KMA-633}\textbf{[\nameref{definition:35M}]}
    Chứng minh rằng nếu \( n \) là một số nguyên dương đủ lớn,
    thì trong bất kỳ tập hợp gồm \( n \) số nguyên dương khác nhau nào cũng tồn tại bốn số sao cho bội chung nhỏ nhất của chúng lớn hơn \( n^{3{,}99} \).
\end{exercise*}

\begin{remark*}
    Gợi ý: Hãy sắp xếp các số đã cho theo thứ tự tăng dần, ước lượng kích thước của bội chung nhỏ nhất,
    và tìm cách chọn bốn số sở hữu lũy thừa nguyên tố lớn vượt mức \( n^{3.99} \).
\end{remark*}

% \begin{story*}
%     Một hướng tiếp cận là sắp xếp các số theo thứ tự tăng dần và xét các nhóm bốn số liên tiếp hoặc gần nhau.
%     Nếu chúng có các lũy thừa nguyên tố lớn (đặc biệt là của các số nguyên tố lớn), thì bội chung nhỏ nhất giữa chúng có thể tăng rất nhanh.
%     Ý tưởng là khai thác sự phân bố của các số nguyên tố và lũy thừa cao trong tập hợp đã cho, để đảm bảo tồn tại tổ hợp bốn số có LCM vượt mức \( n^{3.99} \).
% \end{story*}

\end{document}