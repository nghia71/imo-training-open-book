\documentclass[../05-modular-arithmetic-a.tex]{subfiles}

\begin{document}

\begin{example*}[\gls{IND 2015 TST}3/P2]\label{example:IND-2015-TST3-P2}\textbf{[\nameref{definition:30M}]}
	Tìm tất cả các bộ ba \( (p, x, y) \) bao gồm một số nguyên tố \( p \) và hai số nguyên dương \( x \) và \( y \)
	sao cho \( x^{p-1} + y \) và \( x + y^{p-1} \) đều là lũy thừa của \( p \).
\end{example*}

\begin{story*}
    Ta xét các trường hợp riêng biệt:  
    \begin{itemize}[topsep=0pt, partopsep=0pt, itemsep=0pt]
        \item Với \( p = 2 \), dễ dàng nhận thấy rằng luôn tồn tại vô số nghiệm.
        \item Với \( p > 2 \), ta phân tích các biểu thức \( x^{p-1} + y = p^a \) và \( x + y^{p-1} = p^b \), kết hợp với phân tích p-adic và định lý Fermat nhỏ để suy ra các giới hạn.
        \item Qua các phép thử, ta nhận thấy nghiệm duy nhất là \( (3, 2, 5) \) với \( p = 3 \).
    \end{itemize}
\end{story*}

\bigbreak

\begin{soln}\footnotemark
	\textbf{Trường hợp \( p = 2 \):} Với \( x, y \in \mathbb{Z}_{>0} \), ta có:
	\[
		x^{p-1} + y = x + y,\quad x + y^{p-1} = x + y
	\implies \text{cùng bằng tổng } x + y
	\]
	Vì lũy thừa của 2 xuất hiện dày đặc, nên luôn tồn tại \( (x, y) \) sao cho tổng là lũy thừa của 2. Trường hợp này cho vô số nghiệm.

	\textbf{Giả sử từ đây \( p > 2 \)}. Đặt:
	\[
		x^{p-1} + y = p^a,\quad x + y^{p-1} = p^b
	\]

	Giả sử \( x \le y \implies x^{p-1} + y \le x + y^{p-1} \implies a \le b \implies p^a \mid p^b \).

	Xét:
	\[
		p^b = x + y^{p-1} = x + (p^a - x^{p-1})^{p-1}
	\]

	Xét modulo \( p^a \), do \( p - 1 \) chẵn:
	\[
		x^{(p-1)^2} + x \equiv 0 \Mod{p^a}
	\implies x^{(p-1)^2 - 1} + 1 \equiv 0 \Mod{p^a}
	\implies p^a \mid x^{p(p - 2)} + 1
	\]

	Áp dụng định lý Fermat nhỏ: \( x^{p-1} \equiv 1 \Mod{p} \implies x^{p(p - 2)} \equiv 1 \Mod{p} \implies p \mid x + 1 \).

	Đặt \( x + 1 = p^r \implies r = \nu_p(x + 1) \), ta xét lũy thừa lớn nhất của \( p \) chia hết \( x^{p(p - 2)} + 1 \).  
	Sử dụng khai triển nhị thức:
	\[
		x = -1 \implies x^{p(p - 2)} = (-1)^{p(p - 2)} = 1 \implies x^{p(p - 2)} + 1 = 2 \text{ không chia hết cho } p
	\]

	Nhưng vì \( p \mid x + 1 \implies p^r \le x + 1 \le p^a \implies a = r \) hoặc \( r + 1 \).

	Nếu \( a = r \implies x + 1 = p^a \implies x = p^a - 1 \implies y = p^a - x^{p-1} \)

	Một vài thử nghiệm:
	- \( p = 3 \implies x = 2 \implies x^{p-1} = 4,\ y = 5 \implies x^{p-1} + y = 9 = 3^2,\ x + y^{p-1} = 2 + 25 = 27 = 3^3 \)

	Nếu \( p \ge 5 \), thử với \( x = p - 1 \implies x^{p-1} \ge (p - 1)^4 \gg p^2 \implies x^{p-1} + y > p^a \implies \) mâu thuẫn.

	\textbf{Kết luận:}
	\[
		\boxed{
			\begin{array}{l}
				\text{Các bộ ba thỏa mãn là:} \\
				(p, x, y) = (2, x, y), \text{ với } x + y \text{ là lũy thừa của } 2, \text{ và} \\
				(3, 2, 5)
			\end{array}
		}
	\]
\end{soln}

\footnotetext{\href{https://www.imo-official.org/problems/IMO2014SL.pdf}{IMO SL 2014 N5.}}

\end{document}