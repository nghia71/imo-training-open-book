\documentclass[../05-modular-arithmetic-a.tex]{subfiles}

\begin{document}

\begin{example*}[\gls{IRN 2015 MO}/N4]\label{example:IRN-2015-MO-N4}\textbf{[\nameref{definition:35M}]}
	Cho các số nguyên dương \( a, b, c, d, k, \ell \) sao cho với mọi số tự nhiên \( n \), tập các thừa số nguyên tố của hai số
	\[
		n^k + a^n + c \quad \text{và} \quad n^\ell + b^n + d
	\]
	là giống nhau. Chứng minh rằng \( a = b \), \( c = d \), và \( k = \ell \).
\end{example*}

\begin{story*}
    Ta cần chứng minh rằng nếu hai biểu thức \( n^k + a^n + c \) và \( n^\ell + b^n + d \) luôn có cùng tập thừa số nguyên tố với mọi \( n \), thì các tham số phải giống hệt nhau.

    Hướng tiếp cận:
    \begin{itemize}[topsep=0pt, partopsep=0pt, itemsep=0pt]
        \item Xét các giá trị lớn của \( n \), và sử dụng mô hình đồng dư modulo một số nguyên tố \( p \) thích hợp.
        \item Xây dựng một đa thức \( P(s) \) và chứng minh rằng nó phải đồng nhất bằng 0.
        \item Từ đó rút ra điều kiện bắt buộc giữa các hệ số.
    \end{itemize}
\end{story*}

\begin{soln}\footnotemark
	Giả sử tồn tại các bộ số \( (a, b, c, d, k, \ell) \) thỏa mãn điều kiện đề bài nhưng không đồng nhất.

	Xét \( n = (kp - t)(p - 1) + s \), với \( p \) là số nguyên tố lớn, \( t, s \in \mathbb{N} \) cố định. Khi đó:
	\[
		\begin{aligned}
			n &\equiv s \Mod{p - 1} \implies a^n \equiv a^s,\quad b^n \equiv b^s \Mod{p}, \\
			n &\equiv t + s \Mod{p} \implies n^k \equiv (t + s)^k,\quad n^\ell \equiv (t + s)^\ell \Mod{p}.
		\end{aligned}
	\]

	Ta có:
	\[
		\begin{aligned}
			&n^k + a^n + c \equiv (t + s)^k + a^s + c \Mod{p},\quad
			n^\ell + b^n + d \equiv (t + s)^\ell + b^s + d \Mod{p}.\\
			&p \mid n^k + a^n + c \implies (t + s)^k \equiv -a^s - c \Mod{p},\quad (t + s)^\ell \equiv -b^s - d \Mod{p}.
		\end{aligned}
	\]

	Nâng hai vế lên bội chung:
	\[
		(-(a^s + c))^\ell \equiv (-(b^s + d))^k \Mod{p} \implies p \mid (a^s + c)^\ell - (b^s + d)^k.
	\]

	Đặt:
	\[
	P(s) := (a^s + c)^\ell - (b^s + d)^k.
	\]

	Nếu \( P(s) \not\equiv 0 \), thì \( P(s) \) là đa thức khác hằng \( \Rightarrow \) có vô hạn thừa số nguyên tố.

	Nhưng với mỗi \( s \), có thể chọn \( t, p \) sao cho \( p \mid n^k + a^n + c \Rightarrow p \mid P(s) \), điều này mâu thuẫn trừ khi \( P(s) = 0 \) với mọi \( s \in \mathbb{N} \).

	Vậy:
	\[
		(a^s + c)^\ell = (b^s + d)^k\quad \text{với mọi } s \in \mathbb{N}.
	\]

	Đặt \( j = \gcd(k, \ell) \), viết \( k = j \cdot k' \), \( \ell = j \cdot \ell' \). Khi đó:
	\[
		(a^s + c)^{\ell'} = (b^s + d)^{k'}.
	\]

	Giả sử \( k' > 1 \), với \( s \) đủ lớn thì $a^s \gg c$, nên $a^s + c$ không thể là lũy thừa bậc \( k' \), mâu thuẫn.

	Tương tự với \( \ell' > 1 \). Vậy \( k' = \ell' = 1 \Rightarrow k = \ell \), và:
	\[
		a^s + c = b^s + d \implies a^s - b^s = d - c.
	\]

	Hiệu vế trái thay đổi theo \( s \) nếu \( a \ne b \), trong khi vế phải là hằng số suy ra mâu thuẫn cho nên \( a = b,\ c = d \).

	\textbf{Kết luận cuối cùng:}
	\[
		\boxed{a = b, \quad c = d, \quad k = \ell}
	\]
\end{soln}

\footnotetext{\href{https://artofproblemsolving.com/community/c6h1138949p5340611}{Lời giải của \textbf{mojyla222}.}}

\end{document}