\documentclass[../05-modular-arithmetic-a.tex]{subfiles}

\begin{document}

\begin{exercise*}[\gls{THA 2015 MO}/P8]\label{example:THA-2015-MO-P8}\textbf{[\nameref{definition:25M}]}
    Cho \( m, n \) là các số nguyên dương sao cho \( m - n \) là số lẻ. Chứng minh rằng biểu thức
    \[
        (m + 3n)(5m + 7n)
    \]
    không thể là số chính phương.
\end{exercise*}

\begin{remark*}
    Sử dụng lập luận modulo 4 hoặc modulo 8 để loại trừ khả năng biểu thức là một số chính phương.
    Ngoài ra, hãy xét tính chẵn/lẻ của hai thừa số, và tổng quát hoá bằng phân tích đồng dư.
\end{remark*}

% \begin{story*}
%     Bài toán yêu cầu chứng minh rằng biểu thức \( (m + 3n)(5m + 7n) \) không thể là số chính phương nếu \( m - n \) là số lẻ.

%     Ta xét biểu thức modulo 4:
%     \begin{itemize}[topsep=0pt, partopsep=0pt, itemsep=0pt]
%         \item Nếu \( m \equiv n \pmod{2} \), thì \( m - n \) chẵn — loại.
%         \item Nếu \( m - n \) lẻ, thì \( m \not\equiv n \pmod{2} \). Vậy \((m,n) \in \{(0,1), (1,0)\} \pmod{2}\).
%         \item Xét từng trường hợp theo modulo 4 và đánh giá đồng dư của biểu thức.
%     \end{itemize}

%     Ý tưởng chính là nếu biểu thức \( (m + 3n)(5m + 7n) \equiv r \pmod{4} \), mà \( r \notin \{0,1\} \), thì biểu thức không thể là số chính phương.
% \end{story*}

\end{document}