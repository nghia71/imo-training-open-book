\documentclass[../05-modular-arithmetic-a.tex]{subfiles}

\begin{document}

\begin{example*}[\gls{FRA 2015 RMM}/P3]\label{example:FRA-2015-RMM-P3}[\textbf{\nameref{definition:30M}}]
	Cho số nguyên tố \( p \geq 5 \). Chứng minh rằng tập hợp
	\[
		K = \left\{ a \in \mathbb{Z} \mid a^{p-1} \equiv 1 \Mod{p^2} \right\}
	\]
	chứa ít nhất hai số nguyên tố lẻ phân biệt nhỏ hơn \( p \).
\end{example*}

\begin{story*}
    Cần chứng minh rằng trong tập \( K \), có ít nhất hai số nguyên tố lẻ phân biệt nhỏ hơn \( p \).  
    Ta sử dụng ý tưởng sau:
    \begin{itemize}[topsep=0pt, partopsep=0pt, itemsep=0pt]
        \item Xét nhóm con \( K \) của \( \mathbb{Z}_p^* \) mở rộng modulo \( p^2 \).
        \item Dùng khai triển nhị thức để tìm điều kiện về \( (a + kp)^{p-1} \equiv 1 \pmod{p^2} \).
        \item Xác định các số nguyên tố nhỏ hơn \( p \) và chỉ ra mâu thuẫn khi giả sử có ít hơn hai số như vậy trong \( K \).
    \end{itemize}
    Kết hợp các yếu tố trên, ta chứng minh được yêu cầu bài toán.
\end{story*}

\bigbreak

\begin{soln}(Cách 1)\footnotemark
	Trước hết, bài toán là hiển nhiên nếu \( p = 5 \), vì khi đó
	\[
		2^{p-1} \equiv 16 \not\equiv 1 \pmod{p^2}
		\quad\text{và}\quad
		3^{p-1} \equiv 6 \not\equiv 1 \pmod{p^2}.
	\]  
	
	Giả sử từ đây \( p \geq 7 \). Đặt 
	\[
		K = \{\,n\in \mathbb{Z}\,\mid\,n^{p-1}\equiv 1 \pmod{p^2}\}.
	\]
	
	(a) \(K\) đóng dưới phép nhân.  
	
	(b) Nếu \(a\in K\) (tức \(a\not\equiv0\pmod{p}\)) và \(k\in\mathbb{Z}\), ta tính:
	\[
		(a + kp)^{p-1} \;\equiv\; a^{p-1} + (p-1)a^{p-2}\,k\,p \;\equiv\;1 - kp\,a^{p-2}\pmod{p^2}.
	\]
	Vậy \(a+kp\in K\iff k\in p\mathbb{Z}\).  

	Tiếp theo, vì \(\{-1,1\}\subset K\), phải có \(\{p-1,p+1\}\not\subset K\).
	Thế nên tồn tại hai số nguyên tố \(q \mid (p-1)\) và \(r \mid (p+1)\) với \(\{q,r\}\cap K=\emptyset\).
	Khi \(q\ne r\), ta được hai số nguyên tố lẻ phân biệt \(\le p\). Nếu \(q=r\), thì \(q=2\), cũng suy ra 2 \(\notin K\).  

	Trong trường hợp còn lại (tức mọi số nguyên tố lẻ \(\le p-1\) đều thuộc \(K\)) sẽ dẫn đến mâu thuẫn:
	ta chọn \(\theta=1\) hoặc 5 tùy \(p\mod3\), từ đó suy ra \(2p+\theta\in K\) trong khi \(\theta\in K\) và \(\theta\notin p\mathbb{Z}\), gây nghịch lý.  

	Vậy kết luận, ắt phải có hai số nguyên tố lẻ < \(p\) nằm trong \(K\).  
\end{soln}

\footnotetext{\href{http://maths-olympiques.fr/wp-content/uploads/2017/10/ofm-2014-2015-test-rmm-corrige.pdf}{Lời giải chính thức.}}

\end{document}