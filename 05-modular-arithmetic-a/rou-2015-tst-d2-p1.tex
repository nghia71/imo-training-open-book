\documentclass[../05-modular-arithmetic-a.tex]{subfiles}

\begin{document}

\begin{example*}[\nameref{example:ROU-2015-TST-D2-P1}][\textbf{\nameref{definition:25M}}]
    Cho \( a \in \mathbb{Z} \) và \( n \in \mathbb{N}_{>0} \). Chứng minh rằng:
    \[
        \sum_{k=1}^{n} a^{\gcd(k,n)}
    \]
    luôn chia hết cho \( n \), trong đó \( \gcd(k,n) \) là ước chung lớn nhất của \( k \) và \( n \).
\end{example*}

\begin{story*}
    Ta nhóm các chỉ số \( k \) theo giá trị \( d = \gcd(k, n) \). Mỗi giá trị \( a^d \) xuất hiện đúng \(\phi(n/d)\) lần trong tổng, nên ta có:
    \[
        \sum_{k=1}^{n} a^{\gcd(k,n)} = \sum_{d \mid n} \phi\left(\frac{n}{d}\right) a^d.
    \]
    Để chứng minh tổng này chia hết cho \(n\), ta xét các ước nguyên tố \(p\) của \(n\) và chứng minh tổng chia hết cho \(p^{v_p(n)}\) với mỗi \(p\). Từ đó suy ra tổng chia hết cho \(n\) theo định lý cơ bản của số học.
\end{story*}

\bigbreak

\begin{soln}
    Với mỗi ước \( d \mid n \), số lần \( a^d \) xuất hiện là \( \phi(n/d) \). Vậy:
    \[
        \sum_{k=1}^n a^{\gcd(k,n)} = \sum_{d \mid n} \phi\left( \frac{n}{d} \right)\, a^d.
    \]

    Xét một ước nguyên tố \( p \mid n \), gọi \( j = v_p(n) \).
    Mỗi ước \( d \mid n \) có thể viết dưới dạng \( d = p^i d' \), với \( 0 \le i \le j \) và \( (d', p) = 1 \). Khi đó:
    \[
        \sum_{d \mid n} \phi\left(\frac{n}{d}\right)\, a^d
        = \sum_{\substack{d' \mid n \\ (d',p)=1}} \sum_{i=0}^j \phi\left( \frac{n}{p^i d'} \right)\, a^{p^i d'}.
    \]

    Với \( x = a^{d'} \), \( \phi(n/(p^i d')) = \phi(n/(p^j d'))\, \phi(p^{j - i}) \). Áp dụng \nameref{lemma:euler-power-sum-mod}:
    \[
        \sum_{i = 0}^j \phi(p^{j - i})\, x^{p^i} \equiv 0 \pmod{p^j}.
    \]
    Nên toàn bộ tổng chia hết cho \( p^j \). Lặp lại với mọi \( p \mid n \), suy ra:
    \[
        \sum_{k=1}^n a^{\gcd(k,n)} \equiv 0 \pmod{n}.
    \]
\end{soln}

\end{document}