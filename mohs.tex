\documentclass[../imo-training-open-book.tex]{subfiles}
\begin{document}

\section*{Thang độ khó MOHS}

Trong \href{https://web.evanchen.cc/upload/MOHS-hardness.pdf}{tài liệu này},
Evan Chen cung cấp xếp hạng độ khó cá nhân cho các bài toán từ một số kỳ thi gần đây.
Điều này đòi hỏi phải xác định một tiêu chí đánh giá độ khó một cách cẩn thận.
Evan Chen gọi hệ thống này là \textbf{thang độ khó MOHS} (phát âm là “moez”);
đôi khi anh cũng sử dụng đơn vị “M” (viết tắt của “Mohs”).

Thang đo này tiến hành theo bước nhảy 5M, với mức thấp nhất là 0M và mức cao nhất là 60M.
Tuy nhiên, trên thực tế, rất ít bài toán được xếp hạng cao hơn 50M,
nên có thể coi nó chủ yếu là một thang đo từ 0M đến 50M, với một số bài toán thuộc dạng “vượt mức thông thường”.

Bên dưới là bản dịch tiếng Việt từ tài liêu trên.

\subsection*{Xếp hạng dựa theo ý kiến cá nhân của Evan Chen}

Mặc dù có rất nhiều điều đã được viết ra ở đây, nhưng cuối cùng, những xếp hạng này vẫn chỉ là ý kiến cá nhân của Evan Chen.
Evan Chen không khẳng định rằng các xếp hạng này là khách quan hoặc phản ánh một sự thật tuyệt đối nào đó.

\textbf{Lưu ý hài hước (Bảo hành xếp hạng).} Các xếp hạng được cung cấp “nguyên trạng”, không có bất kỳ bảo hành nào,
dù rõ ràng hay ngụ ý, bao gồm nhưng không giới hạn ở các bảo hành về khả năng thương mại, sự phù hợp với một mục đích cụ thể,
và việc không vi phạm quyền sở hữu trí tuệ. Trong mọi trường hợp, Evan không chịu trách nhiệm đối với bất kỳ khiếu nại,
thiệt hại hoặc trách nhiệm pháp lý nào phát sinh từ, liên quan đến, hoặc có liên quan đến những xếp hạng này.

\subsection*{Hướng dẫn sử dụng}

\textbf{Cảnh báo quan trọng:} Lạm dụng các xếp hạng này có thể gây hại cho bạn.

Ví dụ, nếu bạn quyết định không nghiêm túc thử sức với một số bài toán chỉ vì chúng được xếp hạng 40M trở lên,
bạn có thể tự làm khó mình bằng cách tước đi cơ hội tiếp xúc với những bài toán khó.
Nếu bạn không thường xuyên thử sức với các bài toán cấp độ IMO3 một cách nghiêm túc,
bạn sẽ không bao giờ đạt đến mức độ có thể thực sự giải được chúng. 

Vì lý do này, nghịch lý thay, đôi khi việc không biết bài toán khó đến mức nào lại tốt hơn,
để bạn không vô thức có thái độ bỏ cuộc ngay từ đầu.

Các xếp hạng này được thiết kế để làm tài liệu tham khảo. Một cách sử dụng hợp lý là không xem xếp hạng bài toán cho đến khi bạn đã giải xong;
điều này mô phỏng tốt nhất điều kiện thi đấu thực tế,
khi bạn không biết độ khó của bài toán cho đến khi bạn giải được nó hoặc hết giờ và thấy những ai khác đã giải được.
Bạn đã được cảnh báo. Chúc may mắn!

\section*{Ý nghĩa của các mức xếp hạng bài toán}

Dưới đây là ý nghĩa của từng mức độ xếp hạng bài toán theo thang đo MOHS.

\begin{definition*}[0M] \label{definition:0M}
    Bài toán có mức 0M quá dễ để xuất hiện trong IMO.
    Thông thường, một học sinh giỏi trong lớp toán nâng cao có thể giải được bài toán này mà không cần đào tạo chuyên sâu về toán olympic.
\end{definition*}

\begin{definition*}[5M] \label{definition:5M}
    Đây là mức dễ nhất có thể xuất hiện trong IMO nhưng vẫn đáp ứng tiêu chuẩn của kỳ thi. Những bài toán này có thể được giải quyết rất nhanh.

    \textbf{Ví dụ:}
    \begin{itemize}[topsep=0pt, partopsep=0pt, itemsep=0pt]
        \item IMO 2019/1 về phương trình $f(2a) + 2f(b) = f(f(a + b))$
        \item IMO 2017/1 về căn bậc hai $\sqrt{a_n}$ hoặc $a_n + 3$
    \end{itemize}
\end{definition*}

\begin{definition*}[10M] \label{definition:10M}
    Đây là mức độ dành cho các bài toán IMO số 1 hoặc 4 mà hầu hết các thí sinh không gặp khó khăn khi giải. Tuy nhiên, vẫn cần có một số công việc để hoàn thành lời giải.

    \textbf{Ví dụ:}
    \begin{itemize}[topsep=0pt, partopsep=0pt, itemsep=0pt]
        \item IMO 2019/4 về $k! = (2^n-1)\ldots$
        \item IMO 2018/1 về $DE \parallel FG$
    \end{itemize}
\end{definition*}

\begin{definition*}[15M] \label{definition:15M}
    Đây là mức thấp nhất của các bài toán có thể xuất hiện dưới dạng bài số 2 hoặc 5 của IMO, nhưng thường phù hợp hơn với bài số 1 hoặc 4.
    Những bài toán này thường có thể được giải quyết dễ dàng bởi các đội tuyển thuộc top 10 thế giới.

    \textbf{Ví dụ:}
    \begin{itemize}[topsep=0pt, partopsep=0pt, itemsep=0pt]
        \item IMO 2019/5 về bài toán ``Ngân hàng Bath''
        \item IMO 2018/4 về ``Amy/Ben và lưới $20 \times 20$''
        \item IMO 2017/4 về tiếp tuyến $KT$ của $\Gamma$
    \end{itemize}
\end{definition*}

\begin{definition*}[20M] \label{definition:20M}
    Những bài toán ở mức này có thể quá khó để xuất hiện dưới dạng IMO 1/4 nhưng vẫn chưa đạt đến độ khó trung bình của IMO 2/5.

    \textbf{Ví dụ:}
    \begin{itemize}[topsep=0pt, partopsep=0pt, itemsep=0pt]
        \item IMO 2018/5 về $a_1, a_2, \dots, a_n$ sao cho $\frac{a_1}{a_2} + \dots + \frac{a_n}{a_1} \in \mathbb{Z}$
    \end{itemize}
\end{definition*}

\begin{definition*}[25M] \label{definition:25M}
    Đây là mức độ phù hợp nhất với các bài toán IMO 2/5. Những bài toán này là thử thách thực sự ngay cả với các đội tuyển hàng đầu.

    \textbf{Ví dụ:}
    \begin{itemize}[topsep=0pt, partopsep=0pt, itemsep=0pt]
        \item IMO 2019/2 về ``$P_1, Q_1, P, Q$ đồng viên''
    \end{itemize}
\end{definition*}

\begin{definition*}[30M] \label{definition:30M}
    Những bài toán ở mức này khó hơn một chút so với mức trung bình của IMO 2/5, nhưng vẫn chưa đủ khó để được sử dụng làm bài số 3 hoặc 6.

    \textbf{Ví dụ:}
    \begin{itemize}[topsep=0pt, partopsep=0pt, itemsep=0pt]
        \item IMO 2018/2 về phương trình $a_i a_{i+1} + 1 = a_{i+2}$
    \end{itemize}
\end{definition*}

\begin{definition*}[35M] \label{definition:35M}
    Đây là mức độ khó cao nhất dành cho các bài toán IMO 2/5 và cũng là mức độ dễ nhất của các bài toán IMO 3/6.

    \textbf{Ví dụ:}
    \begin{itemize}[topsep=0pt, partopsep=0pt, itemsep=0pt]
        \item IMO 2019/6 về ``$DI \cap PQ$ trên phân giác góc ngoài $\angle A$''
        \item IMO 2017/5 về ``Ngài Alex và các cầu thủ bóng đá''
    \end{itemize}
\end{definition*}

\begin{definition*}[40M] \label{definition:40M}
    Những bài toán ở mức này quá khó để xuất hiện ở IMO 2/5. Ngay cả các đội tuyển hàng đầu cũng không thể đạt điểm tuyệt đối với bài toán ở mức này.

    \textbf{Ví dụ:}
    \begin{itemize}[topsep=0pt, partopsep=0pt, itemsep=0pt]
        \item IMO 2019/3 về ``mạng xã hội và xor tam giác''
        \item IMO 2017/2 về phương trình $f(f(x)f(y)) + f(x+y) = f(xy)$
        \item IMO 2017/3 về ``thợ săn và con thỏ''
        \item IMO 2017/6 về ``nội suy đa thức thuần nhất''
    \end{itemize}
\end{definition*}

\begin{definition*}[45M] \label{definition:45M}
    Bài toán thuộc hạng này thường chỉ có một số ít thí sinh giải được. Đây là mức độ của những bài toán IMO 3/6 khó hơn mức trung bình.

    \textbf{Ví dụ:}
    \begin{itemize}[topsep=0pt, partopsep=0pt, itemsep=0pt]
        \item IMO 2018/3 về ``tam giác phản Pascal''
        \item IMO 2018/6 về ``$\angle BXA + \angle DXC = 180^\circ$''
    \end{itemize}
\end{definition*}

\begin{definition*}[50M] \label{definition:50M}
    Đây là mức khó nhất mà một bài toán vẫn có thể xuất hiện trong kỳ thi IMO hoặc bài kiểm tra chọn đội tuyển của các quốc gia hàng đầu.
\end{definition*}

\begin{definition*}[55M] \label{definition:55M}
    Bài toán ở mức này quá dài dòng hoặc tốn nhiều thời gian để giải quyết trong một kỳ thi có giới hạn thời gian.
\end{definition*}

\begin{definition*}[60M] \label{definition:60M}
    Bài toán ở mức này không thể giải trong vòng 4,5 giờ bởi học sinh trung học, nhưng vẫn có thể được giải quyết trong điều kiện không giới hạn thời gian.
    Ví dụ, một kết quả từ một nghiên cứu tổ hợp với chứng minh dài 15 trang có thể rơi vào hạng này.
\end{definition*}

\textit{Lưu ý:} Evan Chen sử dụng bội số của 5 để tránh nhầm lẫn giữa số bài toán (ví dụ: bài toán số 6) với mức độ khó (ví dụ: 30M).

\end{document}