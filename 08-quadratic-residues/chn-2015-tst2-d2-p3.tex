\documentclass[../08-quadratic-residues.tex]{subfiles}

\begin{document}

\begin{example*}[\gls{CHN 2015 TST}2/D2/P3]\label{example:CHN-2015-TST2-D3}[\textbf{\nameref{definition:25M}}]
    Chứng minh rằng tồn tại vô hạn số nguyên \( n \) sao cho \( n^2 + 1 \) là số không có ước chính phương.
\end{example*}

\begin{story*}
    Ta cần chỉ ra rằng với vô hạn số nguyên \( n \), biểu thức \( n^2 + 1 \) không chia hết cho bình phương của một số nguyên tố.
    Hướng tiếp cận là đánh giá số nghiệm của phương trình \( x^2 \equiv -1 \Mod{p^2} \) và áp dụng định lý số nguyên tố về các \( p \equiv 1 \Mod{4} \).
    Dùng tính chặt của ước lượng mật độ nghiệm để chỉ ra rằng số lượng \( x \) khiến \( x^2 + 1 \) có ước chính phương là ít hơn tổng thể \( x \).
\end{story*}

\begin{soln}\footnotemark
    Trước hết ta chứng minh khẳng định sau với mọi số nguyên tố \( p \).
    \begin{theorem*}[giai-thua-mod]
        Phương trình \( x^2 \equiv -1 \Mod{p^2} \) có không quá 2 nghiệm trong tập \( \{0,1,\dots,p^2-1\} \).
    \end{theorem*}
    
    \begin{subproof}
        Trường hợp \( p = 2 \) có thể dễ dàng kiểm tra trực tiếp, nên ta giả sử \( p \) là số lẻ.
    
        Giả sử phản chứng rằng tồn tại ít nhất ba số nguyên phân biệt \( a, b, c \) sao cho:
        \[
            a^2 + 1 \equiv b^2 + 1 \equiv c^2 + 1 \equiv 0 \Mod{p^2} \implies a^2 \equiv b^2 \Mod{p^2} \implies p^2 \mid (a - b)(a + b). \quad (1)
        \]
        
        Xét hai trường hợp:
        
        \textit{Trường hợp 1:} \( p \mid a - b \). Khi đó, vì \( |a - b| < p^2 \), ta có \( p^2 \nmid a - b \). Từ (1), suy ra \( p \mid a + b \).  
        Xét tổng và hiệu:
        \[
            p \mid (a - b) + (a + b) = 2a, \quad p \mid (a + b) - (a - b) = 2b.
        \]

        Do \( p \neq 2 \), suy ra \( p \mid a \), nhưng điều này mâu thuẫn với giả thiết \( p^2 \mid a^2 + 1 \).
    
        \textit{Trường hợp 2:} \( p \nmid a - b \). Từ (1), ta suy ra \( p^2 \mid a + b \). Tương tự, có thể chứng minh \( p^2 \mid a + c \).

        Suy ra \( p^2 \mid (a + c) - (a + b) = b - c \). Vì \( |b - c| < p^2 \), ta có \( b = c \), mâu thuẫn với giả thiết ban đầu.
    \end{subproof}
    
    Gọi \( \text{X}(n) \) và \( \text{P}_{4,1}(n) \) là hai tập hợp sau:
    \[
        \begin{aligned}
            &\text{X}(n) = \{ x \mid x^2 + 1 \text{ có ước chính phương} \},\\
            &\text{P}_{4,1}(n) = \{ p \mid p \text{ là số nguyên tố},\ p \equiv 1 \Mod{4} \}.
        \end{aligned}
    \]

    Theo khẳng định trên, \( x^2 \equiv -1 \Mod{p^2} \) trong khoảng \( [0, n-1] \) có số nghiệm không vượt quá:
    \[
        2 \cdot \frac{n}{p^2} + 2.
    \]
    
    Do đó tổng số phần tử trong \( \text{X}(n) \):
    \[
        \begin{aligned}
            |\text{X}(n)| &\le \sum_{p \in \text{P}_{4,1}(n)} \left(2 + \frac{2n}{p^2} \right)
            \le 2|\text{P}_{4,1}(n)| + 2n \sum_{p \in \text{P}_{4,1}(n)} \frac{1}{p^2}.
        \end{aligned}
    \]

    Số lượng số nguyên tố \( p \leq n \) thỏa mãn \( p \equiv 1 \Mod{4} \) được ước lượng bởi:
    \[
        |\text{P}_{4,1}(n)| \leq 1 + \frac{n}{4}.
    \]

    Theo \nameref{theorem:alternating-series-inequality}, ta có:
    \[
        \sum_{p} \frac{2}{p^2} < \frac{1}{4} \implies 2n \sum_{p} \frac{1}{p^2} < \frac{n}{6} \implies |\text{X}(n)| \leq 2 + \frac{2n}{3}.
    \]
    
    Từ đó, số phần tử \( x \) nhỏ hơn \( n \) sao cho \( x^2 + 1 \) không có ước chính phương ít nhất là:
    \[
        n - |\text{X}(n)| \geq \frac{n}{3} - 2.
    \]

    Vì \( \frac{n}{3} - 2 \) có thể lớn tùy ý khi \( n \) tăng, nên có vô hạn số \( n \) sao cho \( n^2 + 1 \) không có ước chính phương.
\end{soln}

\footnotetext{\href{https://artofproblemsolving.com/community/c6h1064571p27346813}{Lời giải của \textbf{rafayaashary1}.}}

\end{document}