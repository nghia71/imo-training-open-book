\documentclass[../02-modular-arithmetic-b.tex]{subfiles}

\begin{document}

\begin{exercise*}[\gls{HUN 2015 TST}/KMA/643]\label{example:HUN-2015-TST-KMA-643}[\textbf{\nameref{definition:30M}}]
    Với mỗi số nguyên dương \( n \), ký hiệu \( P(n) \) là ước số nguyên tố lớn nhất của \( n^2 + 1 \).
    Hãy chứng minh rằng tồn tại vô hạn bộ bốn số nguyên dương \( (a, b, c, d) \) thỏa mãn \( a < b < c < d \) và \( P(a) = P(b) = P(c) = P(d) \).
\end{exercise*}

\begin{remark*}
    Xét các số \( n \) sao cho \( n^2 + 1 \) cùng chia hết cho một số nguyên tố lớn đặc biệt, ví dụ các số nguyên tố \( p \equiv 1 \Mod{4} \),
    rồi khai thác tính chất đồng dư và mật độ để tìm các bộ có cùng ước lớn nhất.
\end{remark*}

% \begin{story*}
%     Ý tưởng chính là cố định một số nguyên tố \( p \equiv 1 \Mod{4} \) và tìm các số \( n \) sao cho \( p \mid n^2 + 1 \),
%     tức là \( n^2 \equiv -1 \Mod{p} \). Vì \( -1 \) là phần dư bậc hai modulo \( p \), nên tồn tại hai nghiệm phân biệt modulo \( p \),
%     và ta có thể tạo ra vô hạn các số \( n \) với \( n \equiv r \Mod{p} \) sao cho \( p \mid n^2 + 1 \).
    
%     Nếu chọn \( p \) đủ lớn thì nó sẽ là ước nguyên tố lớn nhất của \( n^2 + 1 \).
%     Từ đó ta có thể chọn 4 giá trị khác nhau của \( n \) có cùng giá trị \( P(n) = p \).
% \end{story*}

\end{document}