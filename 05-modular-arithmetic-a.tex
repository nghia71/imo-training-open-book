\documentclass[../imo-training-open-book.tex]{subfiles}

\begin{document}

\newpage

\section{Các ví dụ}

\subfile{./05-modular-arithmetic-a/apmo-2015-mo-p3.tex} \newpage
\subfile{./05-modular-arithmetic-a/can-2015-mo-p5.tex} \newpage
\subfile{./05-modular-arithmetic-a/emc-2015-p1.tex} \newpage
\subfile{./05-modular-arithmetic-a/fra-2015-rmm-p3.tex} \newpage
\subfile{./05-modular-arithmetic-a/ger-2015-mo-p4.tex} \newpage
\subfile{./05-modular-arithmetic-a/imo-2015-n1.tex} \newpage
\subfile{./05-modular-arithmetic-a/ind-2015-tst3-p2.tex} \newpage
\subfile{./05-modular-arithmetic-a/ind-2015-tst4-p3.tex} \newpage
\subfile{./05-modular-arithmetic-a/irn-2015-mo-n4.tex} \newpage
\subfile{./05-modular-arithmetic-a/pol-2015-mo-p3.tex} \newpage
\subfile{./05-modular-arithmetic-a/rmm-2015-p5.tex} \newpage
\subfile{./05-modular-arithmetic-a/rou-2015-tst-d2-p1.tex} \newpage
\subfile{./05-modular-arithmetic-a/rus-2015-tst-d7-p5.tex} \newpage
\subfile{./05-modular-arithmetic-a/tha-2015-mo-p5.tex} \newpage

\section{Bài tập}

\subfile{./05-modular-arithmetic-a/hun-2015-tst-kma-640.tex} \bigbreak 
\subfile{./05-modular-arithmetic-a/jpn-2015-mo-p3.tex} \bigbreak
\subfile{./05-modular-arithmetic-a/tha-2015-mo-p8.tex} \bigbreak
\subfile{./05-modular-arithmetic-a/twn-2015-tst2-q2-p1.tex} \bigbreak

\newpage

\section{Định lý, bổ đề, và hằng đẳng thức}

\begin{lemma}[\href{https://w.wiki/9V6Y}{Tổng Euler lũy thừa theo modulo}]
    \label{lemma:euler-power-sum-mod}
    Với mọi số nguyên tố \( p \) và số nguyên \( j \ge 1 \), và với mọi \( x \in \mathbb{Z} \), ta có:
    \[
        \sum_{k = 0}^{j} \phi\left(p^k\right)\, x^{p^{j - k}} \equiv 0 \pmod{p^j}.
    \]
    \end{lemma}

\end{document}