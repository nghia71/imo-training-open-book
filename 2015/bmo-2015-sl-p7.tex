\ifshowproblem
\begin{problem}[\gls{BMO 2015 SL}/P7]\label{problem:BMO-2015-P7}
    Một số nguyên dương \( m \) được gọi là hoán vị chữ số (anagram) của một số nguyên dương \( n \)
    nếu mỗi chữ số \( a \) xuất hiện trong biểu diễn thập phân của \( m \) đúng bằng số lần nó xuất hiện trong biểu diễn thập phân của \( n \).

    Có thể tìm được bốn số nguyên dương phân biệt sao cho mỗi số trong bốn số đó là hoán vị chữ số của tổng ba số còn lại không?
\end{problem}
\fi

\ifshowinfo
Đánh giá [\textbf{\nameref{definition:20M}}]\footnotemark
\footnotetext{\href{https://artofproblemsolving.com/community/c6h1889231p12884001}{Thảo luận AoPS.}}
\fi