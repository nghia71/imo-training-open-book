\ifshowproblemandsoln
\ifshowproblem\begin{problem}[\gls{BW 2015}/P16]\label{problem:BW-2015-P16}\fi
\ifshowsoln\begin{problem}[\nameref{problem:BW-2015-P16}]\fi
    Ký hiệu \( P(n) \) là ước số nguyên tố lớn nhất của \( n \).  
    Tìm tất cả các số nguyên \( n \geq 2 \) sao cho:
    \[
        P(n) + \lfloor \sqrt{n} \rfloor = P(n+1) + \lfloor \sqrt{n+1} \rfloor.
    \]
\end{problem}
\fi

\ifshowsoln
\begin{soln}\footnotemark
    Đáp án: Chỉ có \( n = 3 \) thỏa mãn.

    Dễ thấy \( P(n) \ne P(n+1) \), nên để hai vế bằng nhau, cần \( \lfloor \sqrt{n} \rfloor \ne \lfloor \sqrt{n+1} \rfloor \), tức \( n + 1 \) là số chính phương.
    Khi đó:
    \[
        \lfloor \sqrt{n} \rfloor + 1 = \lfloor \sqrt{n+1} \rfloor, \quad \implies \quad P(n) = P(n+1) + 1.
    \]
    
    Do cả hai đều là số nguyên tố, ta có:
    \[
        P(n) = 3,\quad P(n+1) = 2 \quad \implies \quad n = 3^a,\quad n + 1 = 2^b.
    \]
    
    Khi đó:
    \[
        3^a = 2^b - 1 \implies 2^b = 3^a + 1.
    \]

    Xét modulo 3:
    \[
        2^b \equiv 1 \pmod{3} \implies b \text{ chẵn. Đặt } b = 2c.
    \]

    Khi đó:
    \[
        3^a = (2^c - 1)(2^c + 1).
    \]

    Trong hai thừa số liên tiếp, chỉ một chia hết cho 3, do đó \( 2^c - 1 = 1 \implies c = 1 \implies b = 2 \implies n + 1 = 4 \implies n = 3 \).

    Vậy nghiệm duy nhất là \( \boxed{n = 3} \).
\end{soln}
\footnotetext{\href{https://www.math.olympiaadid.ut.ee/eng/archive/bw/bw15sol.pdf}{Lời giải chính thức.}}
\fi

\ifshowhint
\begin{hint*}[\nameref{problem:BW-2015-P16}]
    Giả sử hai phần nguyên của căn bậc hai khác nhau, từ đó suy ra \(n+1\) là một số chính phương. Sau đó xét hiệu giữa hai ước nguyên tố lớn nhất.
\end{hint*}
\fi

\ifshowremark
\begin{remark*}
    Đánh giá [\textbf{\nameref{definition:10M}}]
\end{remark*}
\newpage
\fi