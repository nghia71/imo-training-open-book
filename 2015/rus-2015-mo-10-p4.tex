\ifshowproblem
\begin{problem}[\gls{RUS 2015 MO}/10/P4]\label{example:RUS-2015-MO-10-P4}
    Ký hiệu \( S(k) \) là tổng các chữ số của một số nguyên dương \( k \).  
    Ta gọi một số nguyên dương \( a \) là \textit{tốt với bậc \( n \)} (hay \( n \)-tốt) nếu tồn tại một dãy các số nguyên dương 
    \( a_0, a_1, \ldots, a_n \) sao cho \( a_n = a \) và 
    \[
        a_{i+1} = a_i - S(a_i), \quad \text{với mọi } i = 0, 1, \ldots, n-1.
    \]
    
    Có đúng là với mọi số nguyên dương \( n \), tồn tại một số nguyên dương \( b \) sao cho \( b \) là \( n \)-tốt nhưng không phải là \( (n+1) \)-tốt?
\end{problem}
\fi

\ifshowinfo
[\textbf{\nameref{definition:20M}}]\footnotemark
\footnotetext{\href{https://artofproblemsolving.com/community/c6h1126562p5209235}{Thảo luận AoPS.}}
\fi