\ifshowproblemandsoln
\ifshowproblem\begin{problem}[\gls{USA 2015 MO}/P5]\label{problem:USA-2015-MO-P5}\fi
\ifshowsoln\begin{problem}[\nameref{problem:USA-2015-MO-P5}]\fi
    Cho \( a, b, c, d, e \) là các số nguyên dương phân biệt thỏa mãn:
    \[
        a^4 + b^4 = c^4 + d^4 = e^5.
    \]

    Chứng minh rằng \( ac + bd \) là một hợp số.
\end{problem}
\fi

\ifshowsoln
\begin{soln}[Lời giải 1.]\footnotemark
    Xét đẳng thức \( a^4 + b^4 = e^5 \). Đây là một trường hợp đặc biệt của Giả thuyết Beal, 
    phát biểu rằng phương trình \( A^x + B^y = C^z \) không có nghiệm nguyên dương với \( \gcd(a,b,c) = 1 \) và \( x, y, z > 2 \).

    Trường hợp đặc biệt với bộ mũ \( (4, 4, 5) \) đã được chứng minh vào năm 2009 bởi Michael Bennett, Jordan Ellenberg và Nathan Ng. 
    Điều đó có nghĩa là \( a \) và \( b \) phải có ước chung lớn hơn 1.

    Gọi \( f = \gcd(a, b) > 1 \). Khi đó \( a = f \cdot a_1 \), \( b = f \cdot b_1 \) với \( a_1, b_1 \in \mathbb{Z}_{>0} \). Xét biểu thức:
    \[
        ac + bd = f a_1 c + f b_1 d = f(a_1 c + b_1 d).
    \]

    Vì \( c, d \) là các số dương, nên \( a_1 c + b_1 d > 1 \), và vì \( f > 1 \), nên tích trên lớn hơn 1 và chia hết cho \( f \).

    Vậy \( ac + bd \) là hợp số.
\end{soln}
\footnotetext{\href{https://w.wiki/E4kL}{Dựa trên Giả thuyết Beal.}}

\begin{soln}[Lời giải 2.]\footnotemark
    Ta sử dụng phản chứng.

    Không mất tính tổng quát, giả sử \( a > d \).

    Vì \( a^4 + b^4 = c^4 + d^4 \), nên rõ ràng \( b < c \).

    Ta xây dựng biểu thức:
    \[
        (a^4 + b^4)c^2d^2 - (c^4 + d^4)a^2b^2 = (a^2c^2 - b^2d^2)(a^2d^2 - b^2c^2),
    \]
    (vế phải là phân tích nhân tử của vế trái)

    Sử dụng giả thiết \( a^4 + b^4 = c^4 + d^4 = e^5 \), ta có thể phân tích thành:
    \[
        e^5(cd - ab)(cd + ab) = (ac - bd)(ac + bd)(ad - bc)(ad + bc).
    \]

    Nếu \( ac - bd \) hoặc \( ad - bc \) bằng 0, thì \( \frac{a}{b} = \frac{c}{d} \) hoặc \( \frac{a}{b} = \frac{d}{c} \) tương ứng, 
    điều này là không thể vì \( a^4 + b^4 = c^4 + d^4 \) và các số \( a, b, c, d, e \) là phân biệt.

    Do đó, vế phải của phương trình trên là khác 0, và vế trái phải chia hết cho \( ac + bd \).

    Giả sử \( ac + bd \) là một số nguyên tố.
    Khi đó, ta có:
    \[
        ac + bd - (cd + ab) = (a - d)(c - b) > 0,
    \]
    vì \( a > d \) và \( c > b \).

    Điều đó dẫn đến:
    \[
        ac + bd > cd + ab > cd - ab,
    \]
    và do đó, không thể có \( cd + ab \) hoặc \( cd - ab \) chia hết cho \( ac + bd \).
    
    Điều này có nghĩa là \( e \) phải chia hết cho \( ac + bd \), tức là tồn tại số nguyên \( k \) sao cho:
    \[
        e = k(ac + bd).
    \]

    Nhưng rõ ràng điều này là vô lý vì:
    \[
        (k(ac + bd))^5 > a^4 + b^4.
    \]

    Do đó, theo phản chứng, \( ac + bd \) không thể là số nguyên tố. Vậy nên nó là một hợp số.
\end{soln}
\footnotetext{\href{https://artofproblemsolving.com/wiki/index.php/2015_USAMO_Problems/Problem_5}{Lời giải từ AoPS.}}

\begin{soln}[Lời giải 3.]\footnotemark
    Giả sử ngược lại rằng \( p = ac + bd \). Khi đó:
    \[
        ac \equiv -bd \pmod{p}
        \implies a^4 c^4 \equiv b^4 d^4 \pmod{p}.
    \]

    Sử dụng \( a^4 + b^4 = c^4 + d^4 = e^5 \), ta thay thế \( c^4 = e^5 - d^4 \), \( b^4 = e^5 - a^4 \), được:
    \[
        a^4 (e^5 - d^4) \equiv (e^5 - a^4) d^4 \pmod{p}
        \implies a^4 e^5 \equiv d^4 e^5 \pmod{p}
        \implies e^5 (a^4 - d^4) \equiv 0 \pmod{p}.
    \]

    Suy ra:
    \[
        p \mid e^5 (a - d)(a + d)(a^2 + d^2).
    \]

    \begin{claim*}
        \( p > e \)
    \end{claim*}
    \begin{subproof}
        Ta có:
        \[
            e^5 = a^4 + b^4 \le a^5 + b^5 < (ac + bd)^5 = p^5 \implies p > e.
        \]
    \end{subproof}

    Từ phép chia ở trên, ta suy ra:
    \[
        p \le \max\{a - d,\ a + d,\ a^2 + d^2\} = a^2 + d^2.
    \]

    Tương tự, ta có:
    \[
        p \le b^2 + c^2.
    \]

    Do đó:
    \[
        p = ac + bd \le \min\{a^2 + d^2,\ b^2 + c^2\}.
    \]

    Trừ hai vế:
    \[
        0 \le \min\{a(a - c) + d(d - b),\ b(b - d) + c(c - a)\}.
    \]

    Nhưng vì \( a^4 + b^4 = c^4 + d^4 \), nên hiệu \( a - c \) và \( d - b \) phải cùng dấu.
    Khi đó tổng \( a(a - c) + d(d - b) \) có dấu dương, dẫn đến mâu thuẫn với bất đẳng thức trên.

    Vậy giả thiết \( p = ac + bd \) là sai, suy ra \( ac + bd \) không thể là số nguyên tố, suy ra nó là hợp số.
\end{soln}
\footnotetext{\href{https://web.evanchen.cc/exams/USAMO-2015-notes.pdf}{Từ lời giải do Evan Chen biên soạn.}}
\fi

\ifshowhint
\begin{hint*}[\nameref{problem:USA-2015-MO-P5}]
    Hãy giả sử ngược lại rằng \( ac + bd \) là một số nguyên tố. Từ giả thiết \( a^4 + b^4 = c^4 + d^4 \), 
    hãy thử xây dựng một biểu thức đối xứng chứa \( ac + bd \) và sử dụng phân tích nhân tử để dẫn đến mâu thuẫn với giả thiết số nguyên tố.
\end{hint*}
\fi

\ifshowremark
\begin{remark*}
    Đánh giá [\textbf{\nameref{definition:25M}}]
\end{remark*}
\newpage
\fi