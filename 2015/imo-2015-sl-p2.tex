\ifshowproblemandsoln
\ifshowproblem\begin{problem}[\gls{IMO 2015 SL}/P2]\label{problem:IMO-2015-SL-P2}\fi
\ifshowsoln\begin{problem}[\nameref{problem:IMO-2015-SL-P2}]\fi
    Cho \( a \) và \( b \) là các số nguyên dương sao cho \( a! + b! \mid a!b! \).  
    Chứng minh rằng:
    \[
        3a \ge 2b + 2.
    \]
\end{problem}
\fi

\ifshowsoln
\begin{soln}[Lời giải 1.]\footnotemark
    Nếu \( a > b \) thì bất đẳng thức \( 3a \ge 2b + 2 \) hiển nhiên. Nếu \( a = b \), thì điều này tương đương với \( a \ge 2 \).
    Ta dễ dàng kiểm tra rằng \( (a, b) = (1,1) \) không thỏa mãn \( a! + b! \mid a!b! \). Vậy ta giả sử \( a < b \), đặt \( c = b - a \).
    Khi đó, bất đẳng thức trở thành \( a \ge 2c + 2 \).

    Giả sử ngược lại rằng \( a \le 2c + 1 \). Đặt
    \[
        M = \frac{b!}{a!} = (a+1)(a+2)\cdots(a+c).
    \]

    Từ điều kiện \( a! + b! \mid a!b! \Rightarrow 1 + M \mid a!M \), ta suy ra \( 1 + M \mid a! \). Ta nhận thấy nếu \( c \ge a \), thì \( 1 + M > a! \), mâu thuẫn.
    Vậy \( c < a \).

    Vì \( M \) là tích của \( c \) số liên tiếp, nên \( c! \mid M \Rightarrow \gcd(1 + M, c!) = 1 \), từ đó suy ra:
    \[
        1 + M \mid \frac{a!}{c!} = (c+1)(c+2)\cdots a. \tag{1}
    \]

    Nếu \( a \le 2c \), thì vế phải là tích của nhiều nhất \( c \) số không vượt quá \( a \), còn \( M \) là tích của \( c \) số lớn hơn \( a \).
    Do đó, \( 1 + M > \frac{a!}{c!} \), mâu thuẫn.

    Xét trường hợp còn lại \( a = 2c + 1 \Rightarrow a + 1 = 2(c + 1) \Rightarrow c + 1 \mid M \). Khi đó, từ (1), ta có:
    \[
        1 + M \mid (c+2)(c+3)\cdots a.
    \]
    Vế phải là tích của \( c \) số không vượt quá \( a \), trong khi \( M \) là tích của \( c \) số lớn hơn \( a \),
    suy ra \( 1 + M > (c+2)(c+3)\cdots a \), mâu thuẫn.
\end{soln}
\footnotetext{\href{http://www.imo-official.org/problems/IMO2015SL.pdf}{Lời giải chính thức.}}

\begin{soln}[Lời giải 2.]
    Giống như lời giải 1, giả sử \( a < b \), đặt \( c = b - a \), và giả sử ngược lại \( a \le 2c + 1 \). Từ điều kiện \( a! + b! \mid a!b! \), ta có:
    \[
        N = 1 + (a+1)(a+2)\cdots(a+c) \mid (a+c)!.
    \]

    Điều đó nghĩa là mọi ước nguyên tố của \( N \) không lớn hơn \( a + c \).

    Gọi \( p \) là một ước nguyên tố của \( N \). Nếu \( p \le c \) hoặc \( p \ge a+1 \), thì \( p \mid (a+1)\cdots(a+c) \Rightarrow p \mid N - 1 \), mâu thuẫn.
    Do đó, \( c + 1 \le p \le a \). Hơn nữa, ta cần \( 2p > a + c \); nếu không, ta có:
    \[
        a + 1 \le 2c + 2 \le 2p \le a + c \Rightarrow p \mid N - 1,
    \]
    mâu thuẫn. Vậy \( p \in \left( \frac{a + c}{2}, a \right] \Rightarrow p^2 \mid (a + c)! \), vì \( 2p > a + c \).

    Nếu \( a \le c + 2 \), thì đoạn \( \left( \frac{a + c}{2}, a \right] \) chứa nhiều nhất một số nguyên suy ra nhiều nhất một số nguyên tố, phải là \( a \).
    Nhưng \( p^2 \mid N \Rightarrow N = a \) hoặc \( N = 1 \), đều vô lý vì \( N > a \).

    Suy ra \( a \ge c + 3 \Rightarrow \frac{a + c + 1}{2} \ge c + 2 \). Mọi ước nguyên tố \( p \) của \( N \) nằm trong đoạn \( [c+2, a] \)
    và mỗi \( p \) chỉ xuất hiện một lần. Vậy:
    \[
        N \mid (c+2)(c+3)\cdots a,
    \]
    nhưng tích này là của \( a - c - 1 \le c \) số nhỏ hơn hoặc bằng \( a \), trong khi \( N = 1 + (a+1)\cdots(a+c) > (c+2)\cdots a \), mâu thuẫn.

    \textbf{Kết luận:} luôn có \( 3a \ge 2b + 2 \). Đẳng thức xảy ra khi \( (a, b) = (2, 2), (4, 5) \).
\end{soln}
\fi

\ifshowhint
\begin{hint*}[\nameref{problem:IMO-2015-SL-P2}]
    Xét biểu thức \( \frac{b!}{a!} \) và áp dụng giả thiết chia hết để đưa ra các điều kiện về \( a \) và \( b \). Sử dụng ước nguyên tố để đạt đến mâu thuẫn.
\end{hint*}
\fi

\ifshowremark
\begin{remark*}
    Đánh giá [\textbf{\nameref{definition:20M}}]
\end{remark*}
\newpage
\fi