\ifshowproblemandsoln
\ifshowproblem\begin{problem}[\gls{IMO 2015 SL}/P8]\label{problem:IMO-2015-SL-P8}\fi
\ifshowsoln\begin{problem}[\nameref{problem:IMO-2015-SL-P8}]\fi
    Với mỗi số nguyên dương \( n \) có phân tích thành thừa số nguyên tố \( n = \prod_{i = 1}^{k} p_i^{\alpha_i} \), định nghĩa:
    \[
        \mho(n) = \sum_{\substack{i:\\ p_i > 10^{100}}} \alpha_i.
    \]
    Nói cách khác, \( \mho(n) \) là tổng số mũ của các thừa số nguyên tố lớn hơn \( 10^{100} \)
    trong phân tích thừa số nguyên tố của \( n \), tính cả bội số.
    
    Tìm tất cả các hàm \( f: \mathbb{Z} \to \mathbb{Z} \) \textit{tăng chặt} (tức là \( a > b \Rightarrow f(a) > f(b) \)) sao cho:
    \[
        \mho(f(a) - f(b)) \le \mho(a - b) \quad \text{với mọi } a > b \text{ trong } \mathbb{Z}.
    \]    
\end{problem}
\fi

\ifshowsoln
\begin{soln}\footnotemark
    \textbf{Đáp án:} Tất cả các hàm dạng \( f(x) = ax + b \), trong đó \( b \in \mathbb{Z} \), \( a \in \mathbb{Z}_{>0} \), và \( \mho(a) = 0 \).

    Rõ ràng mọi hàm dạng như trên đều thỏa mãn điều kiện. Ta cần chứng minh chiều ngược lại:
    nếu \( f \) thỏa mãn điều kiện (1), thì \( f(x) = ax + b \) với \( \mho(a) = 0 \).

    Gọi \( g(x) := f(x) - f(0) \). Khi đó \( g \) cũng thỏa mãn điều kiện (1), và \( g(0) = 0 \), nên \( f(n) = g(n) + f(0) \).
    Ta sẽ chứng minh rằng tồn tại số nguyên dương \( a \) với \( \mho(a) = 0 \) sao cho \( g(n) = an \) với mọi \( n \in \mathbb{Z} \).

    \textbf{Bước 1.} Với mọi số lớn \( k \) (tức \( k \) chỉ có thừa số nguyên tố lớn hơn \( 10^{100} \)), ta có:
    \[
        k \mid f(a) - f(b) \quad \Leftrightarrow \quad k \mid a - b.
    \]

    Ký hiệu \( L(n) \) là thừa số lớn nhất của \( n \) (chỉ gồm các thừa số nguyên tố lớn). Khi đó, cần chứng minh:
    \[
        L(f(a) - f(b)) = L(a - b) \quad \text{với mọi } a > b.
    \]

    Chứng minh bằng quy nạp theo \( k \). Cơ sở \( k = 1 \) là hiển nhiên. Giả sử mệnh đề đúng với mọi \( k < k_0 \), xét \( k_0 \):

    \begin{claim*}
        Với mọi \( x, y \) sao cho \( 0 < x - y < k_0 \), thì \( k_0 \nmid f(x) - f(y) \).
    \end{claim*}

    \begin{subproof}
        Giả sử ngược lại: \( k_0 \mid f(x) - f(y) \). Gọi \( \lambda = L(x - y) \), thì \( \lambda < k_0 \). 
        Theo giả thiết quy nạp, \( \lambda \mid f(x) - f(y) \), do đó:
        \[
            \text{lcm}(k_0, \lambda) \mid f(x) - f(y).
        \]

        Vì \( \text{lcm}(k_0, \lambda) \ge k_0 + 1 \), nên:
        $\mho(f(x) - f(y)) \ge \mho(\text{lcm}(k_0, \lambda)) > \mho(\lambda) = \mho(x - y)$, mâu thuẫn với (1).
    \end{subproof}

    Từ đó suy ra rằng dãy \( f(a), f(a+1), \ldots, f(a+k_0-1) \) là một hệ đại diện đầy đủ modulo \( k_0 \), nên \( f(a) \equiv f(a + k_0) \mod k_0 \). 
    Vì vậy \( f(a) \equiv f(b) \mod k_0 \) khi \( a \equiv b \mod k_0 \).

    Nếu \( a \not\equiv b \mod k_0 \), tồn tại \( b_1 \equiv b \mod k_0 \) sao cho \( |a - b_1| < k_0 \). 
    Khi đó \( f(b) \equiv f(b_1) \not\equiv f(a) \mod k_0 \) suy ra \( f(a) \not\equiv f(b) \mod k_0 \). Vậy:
    \[
        k_0 \mid f(a) - f(b) \Leftrightarrow k_0 \mid a - b \implies L(f(a) - f(b)) = L(a - b).
    \]

    \textbf{Bước 2.} Ta chứng minh rằng tồn tại số nguyên nhỏ \( a \) sao cho \( f(n) = an \) với vô hạn số nguyên \( n \). 
    Tức là \( f \) tuyến tính trên một tập vô hạn.

    \begin{claim*}
        Tồn tại hằng số \( c \) sao cho \( f(t) < ct \) với mọi \( t > 10^{100} \).
    \end{claim*}

    \begin{subproof}
        Gọi \( d \) là tích của tất cả các số nguyên tố nhỏ hơn hoặc bằng \( 10^{100} \), chọn \( \alpha \) sao cho \( 2^\alpha > f(N) \). 
        Với mọi \( p \) nhỏ, \( f(0), \ldots, f(N) \) đều khác nhau modulo \( p^\alpha \). Đặt \( P = d^\alpha \), \( c = P + f(N) \).

        Với mọi \( t > N \), có nhiều nhất một \( i \le N \) sao cho \( p^\alpha \mid f(t) - f(i) \) với mỗi \( p \in S \).
        Do \( |S| < N \), tồn tại \( j \le N \) sao cho với mọi \( p \in S \), \( p^\alpha \nmid f(t) - f(j) \). Khi đó \( S(f(t) - f(j)) < P \).

        Từ Bước 1, \( L(f(t) - f(j)) = L(t - j) \le t - j \). Nên:
        \[
            f(t) = f(j) + L(f(t) - f(j)) \cdot S(f(t) - f(j)) < f(N) + P \cdot (t - j) \le ct.
        \]
    \end{subproof}

    Xét tập \( T \) các số nguyên tố lớn. Với mọi \( t \in T \), theo Bước 1: \( L(f(t)) = L(t) = t \) suy ra \( f(t)/t \in \mathbb{Z} \).
    Do \( f(t) < ct \), tỉ số \( f(t)/t \) chỉ nhận hữu hạn giá trị suy ra tồn tại tập con vô hạn \( T' \subseteq T \) và \( a \in \mathbb{Z}_{>0} \) 
    sao cho \( f(t) = at \) với mọi \( t \in T' \).
    Khi đó \( L(at) = aL(t) \Rightarrow L(a) = 1 \) suy ra \( a \) là số nhỏ.

    \textbf{Bước 3.} Ta chứng minh rằng \( f(x) = ax \) với mọi \( x \in \mathbb{Z} \).
    Gọi \( R_i = \{ x \in \mathbb{Z} : x \equiv i \mod N! \} \).
    \begin{claim*}
        Nếu tồn tại \( r \) sao cho \( f(n) = an \) với vô hạn \( n \in R_r \), thì \( f(x) = ax \) với mọi \( x \in R_{r+1} \).
    \end{claim*}
    \begin{subproof}
        Chọn \( x \in R_{r+1} \). Chọn \( n \in R_r \) sao cho \( f(n) = an \) và \( |n - x| > |f(x) - ax| \). 
        Khi đó \( |n - x| \) là số lớn suy ra \( f(x) \equiv f(n) = an \equiv ax \mod n - x \) suy ra \( n - x \mid f(x) - ax \) suy ra hiệu bằng 0.
    \end{subproof}

    Tập \( T' \) chứa vô hạn phần tử thuộc một lớp \( R_i \). Lặp lại áp dụng bổ đề suy ra \( f(x) = ax \) với mọi \( x \in \mathbb{Z} \).
\end{soln}
\footnotetext{\href{http://www.imo-official.org/problems/IMO2015SL.pdf}{Lời giải chính thức.}}

\begin{remark*}
    Nếu thay điều kiện (1) bằng \( L(f(a) - f(b)) = L(a - b) \), ta có thể bỏ qua Bước 1.
\end{remark*}

\begin{remark*}
    Bước 2 là bước chính. Có thể chứng minh bằng nhiều cách khác nhau.

    \textit{Cách 1.} Xét \( f(di) \) với \( d \) là tích các số nguyên tố nhỏ. Khi đó \( L(f(di) - f(dj)) = di - dj \), tương tự Bước 1.
    Với \( \alpha \) đủ lớn và \( P = d^\alpha \), có thể chọn tập \( I \) vô hạn sao cho \( f(di) = a \cdot di \) với \( a \) nhỏ cố định.

    \textit{Cách 2.} Áp dụng bổ đề sau:
    \begin{lemma*}
        Tồn tại hằng số \( c > 0 \) sao cho:
        \[
            \prod_{i=1}^{3N} L(f(k) - f(i)) \ge c f(k)^{2N} \quad \text{với mọi } k > 3N.
        \]
    \end{lemma*}

    Từ đó suy ra:
    \[
        k^{3N} \ge \prod_{i=1}^{3N} L(k - i) = \prod_{i=1}^{3N} L(f(k) - f(i)) \ge c f(k)^{2N},
    \]
    nên \( f(k) \le Ck^{3/2} \). Sau đó, chọn \( a = f(1) \), và sử dụng đánh giá \( |f(n) - an| < \frac{n(n-1)}{2} \).
    Kết hợp với điều kiện đồng dư để suy ra \( f(n) = an \).
\end{remark*}

\begin{remark*}
    Trong Bước 3, chỉ cần \( f(n) = an \) trên một tập vô hạn. Nếu tập này có cấu trúc tốt (như cùng lớp modulo \( N! \)), 
    có thể suy ra dễ hơn bằng cách dùng định lý phần dư Trung Hoa.
\end{remark*}
\fi

\ifshowhint
\begin{hint*}[\nameref{problem:IMO-2015-SL-P8}]
    Sử dụng định nghĩa \( \mho(n) \) và phân biệt thừa số lớn/nhỏ. Áp dụng lập luận chia hết, định lý phần dư Trung Hoa,
    và lập luận đồng dư để kiểm tra tuyến tính trên tập vô hạn.
\end{hint*}
\fi

\ifshowremark
\begin{remark*}
    Đánh giá [\textbf{\nameref{definition:35M}}]
\end{remark*}
\newpage
\fi
