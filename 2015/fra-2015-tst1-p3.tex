\ifshowproblemandsoln
\ifshowproblem\begin{problem}[\gls{FRA 2015 TST}1/P3]\label{problem:FRA-2015-TST1-P3}\fi
\ifshowsoln\begin{problem}[\nameref{problem:FRA-2015-TST1-P3}]\fi
	Cho \( n \) là một số nguyên dương sao cho \( n(n + 2015) \) là một số chính phương.
    \begin{itemize}[topsep=0pt, partopsep=0pt, itemsep=0pt]
        \item Chứng minh rằng \( n \) không phải là số nguyên tố.
        \item Cho một ví dụ về số nguyên \( n \) như vậy.
    \end{itemize}
\end{problem}
\fi

\ifshowsoln
\begin{soln}\footnotemark
    \textbf{a)} Giả sử \( n \) là số nguyên tố và tồn tại một số nguyên \( m \) sao cho \( n(n + 2015) = m^2 \). Khi đó \( n \mid m^2 \), nên \( n \mid m \). Viết \( m = nr \), ta có:
    \[
        n(n + 2015) = n^2 r^2 \implies n + 2015 = nr^2 \implies 2015 = n(r^2 - 1).
    \]
    Suy ra \( n \mid 2015 \). Vì \( 2015 = 5 \times 13 \times 31 \), nên \( n \in \{5, 13, 31\} \).

    - Nếu \( n = 5 \) thì \( r^2 - 1 = 403 \implies r^2 = 404 \), không là số chính phương.

    - Nếu \( n = 13 \implies r^2 = \frac{2015}{13} + 1 = 156 \), không là số chính phương.

    - Nếu \( n = 31 \implies r^2 = \frac{2015}{31} + 1 = 66 \), cũng không là số chính phương.

    Vậy \( n \) không thể là số nguyên tố.

    \textbf{b)} Ta cần tìm các số nguyên \( n, m \) sao cho:
    \[
        (2m)^2 = 4n(n + 2015).
    \]
    Nhân hai vế:
    \[
        (2m)^2 = (2n + 2015)^2 - 2015^2.
    \]
    Do đó:
    \[
        2015^2 = (2n + 2015 + 2m)(2n + 2015 - 2m).
    \]

    Đặt \( a = 2015 \times 5 = 10075 \), \( b = \frac{2015}{5} = 403 \). Khi đó:
    \[
        2n + 2015 + 2m = a, \quad 2n + 2015 - 2m = b.
    \]
    Cộng hai phương trình:
    \[
        4n + 4030 = a + b = 10478 \implies n = \frac{10478 - 4030}{4} = 1612.
    \]
    Trừ hai phương trình:
    \[
        4m = a - b = 9672 \implies m = \frac{9672}{4} = 2418.
    \]

    Vậy \( n = 1612 \) là một ví dụ thỏa mãn yêu cầu đề bài.
\end{soln}
\footnotetext{\href{https://maths-olympiques.fr/wp-content/uploads/2017/10/ofm-2014-2015-test-janvier-corrige.pdf}{Lời giải chính thức.}}
\fi

\ifshowhint
\begin{hint*}[\nameref{problem:FRA-2015-TST1-P3}]
    Giả sử \( n \) là số nguyên tố và sử dụng điều kiện \( n(n + 2015) = m^2 \) để dẫn đến mâu thuẫn. Sau đó thử viết biểu thức dưới dạng hiệu bình phương.
\end{hint*}
\fi

\ifshowinfo
\begin{remark*}
    Đánh giá [\textbf{\nameref{definition:0M}}]
\end{remark*}
\newpage
\fi