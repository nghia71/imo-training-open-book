\ifshowproblemandsoln
\ifshowproblem\begin{problem}[\gls{NMC 2015}/P2]\label{problem:NMC-2015-P2}\fi
\ifshowsoln\begin{problem}[\nameref{problem:NMC-2015-P2}]\fi
    Tìm các số nguyên tố \( p, q, r \) sao cho một trong hai số \( pqr \) và \( p + q + r \) bằng \( 101 \) lần số còn lại.
\end{problem}
\fi

\ifshowsoln
\begin{soln}\footnotemark
    Ta có thể giả sử \( r = \max\{p, q, r\} \). Khi đó:
    \[
        p + q + r \le 3r, \quad pqr \ge 4r.
    \]

    Do đó, tổng ba số luôn nhỏ hơn tích của chúng, suy ra trường hợp có thể xảy ra duy nhất là:
    \[
        pqr = 101(p + q + r).
    \]

    Ta nhận thấy rằng 101 là một số nguyên tố, nên một trong ba số \( p, q, r \) phải bằng 101. Giả sử \( r = 101 \). Khi đó ta có:
    \[
        pqr = 101(p + q + r) \implies pq \cdot 101 = 101(p + q + 101).
    \]

    Chia hai vế cho 101:
    \[
        pq = p + q + 101 \implies pq - p - q = 101 \implies (p - 1)(q - 1) = 102.
    \]

    Ta phân tích \( 102 = 1 \cdot 102 = 2 \cdot 51 = 3 \cdot 34 = 6 \cdot 17 \), nên các khả năng cho tập \( \{p, q\} \) là:
    \[
        \{2, 103\}, \{3, 52\}, \{4, 35\}, \{7, 18\}.
    \]

    Trong các trường hợp này, chỉ có cặp \( \{2, 103\} \) là cả hai đều là số nguyên tố.

    Vậy nghiệm duy nhất là \( \{p, q, r\} = \{2, 101, 103\} \).
\end{soln}
\footnotetext{\href{https://www.georgmohr.dk/nmcperm/probl/2015/sol.pdf}{Lời giải chính thức.}}
\fi

\ifshowhint
\begin{hint*}[\nameref{problem:NMC-2015-P2}]
    Giả sử \( r \) là số lớn nhất trong ba số nguyên tố \( p, q, r \). So sánh \( p + q + r \) và \( pqr \), rồi thử thay một trong các số bằng 101.
    Sau đó tìm cách biến đổi phương trình thành tích hai biểu thức.
\end{hint*}
\fi

\ifshowremark
\begin{remark*}
    Đánh giá [\textbf{\nameref{definition:5M}}]
\end{remark*}
\newpage
\fi
