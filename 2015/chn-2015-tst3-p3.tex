\ifshowproblem
\begin{problem}[\gls{CHN 2015 TST}3/3]\label{problem:CHN-2015-TST3-P3}
	Cho $a, b$ là hai số nguyên sao cho ước số chung lớn nhất của chúng có ít nhất hai thừa số nguyên tố.
	Gọi $S = \{ x \mid x \in \mathbb{N},\ x \equiv a \pmod{b} \}$ và gọi $y \in S$ là không phân tích được
	nếu không thể biểu diễn $y$ dưới dạng tích của hai hoặc nhiều phần tử của $S$ (không nhất thiết khác nhau).
	Chứng minh rằng tồn tại một số nguyên $t$ sao cho mọi phần tử của $S$
	đều có thể được biểu diễn dưới dạng tích của nhiều nhất $t$ phần tử không phân tích được.
\end{problem}
\fi

\ifshowinfo
Đánh giá [\textbf{\nameref{definition:25M}}]\footnotemark
\footnotetext{\href{https://artofproblemsolving.com/community/c6h1069413p4644173}{Thảo luận AoPS.}}
\fi