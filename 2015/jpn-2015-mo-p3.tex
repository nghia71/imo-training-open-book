\ifshowproblem
\begin{problem}[\gls{JPN 2015 MO}/P3]\label{example:JPN-2015-MO-P3}
	Một dãy số nguyên dương \( \{a_n\}_{n=1}^{\infty} \) được gọi là \textit{tăng mạnh} nếu với mọi số nguyên dương \( n \), ta có:
	\[
		a_n < a_{n+1} < a_n + a_{n+1} < a_{n+2}.
	\]
	
	\begin{itemize}[topsep=0pt, partopsep=0pt, itemsep=0pt]
		\item[(a)] Chứng minh rằng nếu \( \{a_n\} \) là dãy tăng mạnh thì các số nguyên tố lớn hơn \( a_1 \) chỉ xuất hiện hữu hạn lần trong dãy.
		\item[(b)] Chứng minh rằng tồn tại dãy \( \{a_n\} \) tăng mạnh sao cho không có số nào chia hết cho bất kỳ số nguyên tố nào đã xuất hiện trong dãy.
	\end{itemize}
\end{problem}
\fi

\ifshowinfo
[\textbf{\nameref{definition:20M}}]
\fi