\ifshowproblem
\begin{problem}[\gls{ROU 2015 TST}/D3/P3]\label{example:ROU-2015-TST-D3-P3}
    Nếu \( k \) và \( n \) là các số nguyên dương, với \( k \leq n \), ký hiệu \( M(n,k) \) là bội chung nhỏ nhất của các số 
    \( n, n-1, \ldots, n - k + 1 \).  
    Gọi \( f(n) \) là số nguyên dương lớn nhất \( k \leq n \) sao cho:
    \[
        M(n,1) < M(n,2) < \ldots < M(n,k).
    \]
    
    Chứng minh rằng:
    \begin{itemize}[topsep=0pt, partopsep=0pt, itemsep=0pt]
        \item[(a)] Với mọi số nguyên dương \( n \), ta có \( f(n) < 3\sqrt{n} \).
        \item[(b)] Với mọi số nguyên dương \( N \), tồn tại hữu hạn số \( n \) sao cho \( f(n) \leq N \).
    \end{itemize}
\end{problem}
\fi

\ifshowinfo
[\textbf{\nameref{definition:25M}}]\footnotemark
\footnotetext{\href{https://artofproblemsolving.com/community/c6h1097389p4927074}{Thảo luận AoPS.}}
\fi