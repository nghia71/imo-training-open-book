\ifshowproblem
\begin{problem}[\gls{RUS 2015 MO}/11/P2]\label{problem:RUS-2015-MO-11-P2}
    Cho \( n > 1 \) là một số tự nhiên. Ta viết các phân số \( \frac{1}{n}, \frac{2}{n}, \ldots, \frac{n-1}{n} \)
    sao cho tất cả đều ở dạng tối giản. Gọi tổng các tử số trong những phân số này là \( f(n) \).
    
    Tìm tất cả các \( n > 1 \) sao cho một trong hai số \( f(n) \) và \( f(2015n) \) là số lẻ, còn số kia là số chẵn.
\end{problem}
\fi

\ifshowinfo
Đánh giá [\textbf{\nameref{definition:20M}}]\footnotemark
\footnotetext{\href{https://artofproblemsolving.com/community/c6h1172743p5641792}{Thảo luận AoPS.}}
\fi