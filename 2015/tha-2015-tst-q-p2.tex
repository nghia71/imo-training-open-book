\ifshowproblem
\begin{problem}[\gls{THA 2015 TSTST}/Q/P1]\label{example:THA-2015-TSTST-Q-P1}
    Xác định số nguyên nhỏ nhất \( n > 1 \) sao cho trung bình bình phương (quadratic mean) của \( n \) số nguyên dương đầu tiên là một số nguyên.

    \textit{Chú thích:} Trung bình bình phương của các số \( a_1, a_2, \ldots, a_n \) được định nghĩa là:
    \[
        \sqrt{\frac{a_1^2 + a_2^2 + \cdots + a_n^2}{n}}.
    \]
\end{problem}
\fi

\ifshowinfo
[\textbf{\nameref{definition:10M}}]\footnotemark
\footnotetext{\href{https://artofproblemsolving.com/community/c6h2794807p24596837}{Thảo luận AoPS.}}
\fi