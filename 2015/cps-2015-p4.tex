\ifshowproblem
\begin{problem}[\gls{CPS 2015}/P4]\label{problem:CPS-2015-P4}
    Một chiếc máy tính lạ chỉ có hai nút, mỗi nút mang một số nguyên dương gồm đúng hai chữ số. Ban đầu, máy hiển thị số \(1\).  
    Mỗi khi nhấn một nút có giá trị \(N\), máy sẽ thay số đang hiển thị \(X\) bằng \(X \cdot N\) hoặc \(X + N\).  
    Hai phép toán nhân và cộng sẽ xen kẽ nhau, bắt đầu bằng phép nhân.
    
    (Ví dụ: nếu nút 1 có giá trị 10, nút 2 có giá trị 20, và ta lần lượt nhấn nút 1, nút 2, nút 1, nút 1 thì ta sẽ thu được:
    \(1 \cdot 10 = 10\), \(10 + 20 = 30\), \(30 \cdot 10 = 300\), và \(300 + 10 = 310\)).
    
    Hỏi có tồn tại hai giá trị cụ thể cho hai nút (mỗi giá trị là số có hai chữ số) sao cho ta có thể tạo ra vô hạn số khác nhau  
    (bằng cách tiếp tục nhấn nút, không xóa màn hình) sao cho mỗi số thu được đều có tận cùng là:
    \begin{itemize}
        \item[(a)] \(2015\),
        \item[(b)] \(5813\)?
    \end{itemize} 
\end{problem}
\fi

\ifshowinfo
Đánh giá [\textbf{\nameref{definition:30M}}]
\fi