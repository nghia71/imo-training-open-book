\ifshowproblemandsoln
\ifshowproblem\begin{problem}[\gls{EGMO 2015}/P3]\label{problem:EGMO-2015-P3}\fi
\ifshowsoln\begin{problem}[\nameref{problem:EGMO-2015-P3}]\fi
    Cho \( n, m \) là các số nguyên lớn hơn \( 1 \), và \( a_1, a_2, \dots, a_m \) là các số nguyên dương không vượt quá \( n^m \).  
    Chứng minh rằng tồn tại các số nguyên dương \( b_1, b_2, \dots, b_m \) không vượt quá \( n \), sao cho:
    \[
        \gcd(a_1 + b_1, a_2 + b_2, \dots, a_m + b_m) < n,
    \]
    trong đó \( \gcd(x_1, x_2, \dots, x_m) \) là ước chung lớn nhất của các số \( x_1, x_2, \dots, x_m \).
\end{problem}
\fi

\ifshowsoln
\begin{soln}[Lời giải 1.]\footnotemark
    Giả sử không mất tính tổng quát rằng \( a_1 \) là nhỏ nhất trong các \( a_i \).
    Nếu \( a_1 \ge n^m - 1 \), thì bài toán trở nên đơn giản: hoặc tất cả \( a_i \) bằng nhau, hoặc \( a_1 = n^m - 1 \) và tồn tại \( a_j = n^m \).

    \begin{itemize}[topsep=0pt, partopsep=0pt, itemsep=0pt]
        \item Trong trường hợp đầu, ta chọn (ví dụ) \( b_1 = 1, b_2 = 2 \), các \( b_i \) còn lại tùy ý. Khi đó:
        \[
            \gcd(a_1 + b_1, a_2 + b_2, \ldots, a_m + b_m) \le \gcd(a_1 + b_1, a_2 + b_2) = 1.
        \]
        \item Trong trường hợp thứ hai, chọn \( b_1 = 1, b_j = 1 \), các \( b_i \) còn lại tùy ý, ta cũng có:
        \[
            \gcd(a_1 + b_1, a_2 + b_2, \ldots, a_m + b_m) \le \gcd(a_1 + b_1, a_j + b_j) = 1.
        \]
    \end{itemize}

    Do đó, ta chỉ cần xét trường hợp \( a_1 \le n^m - 2 \).

    Giả sử ngược lại rằng không tồn tại bộ \( b_1, \ldots, b_m \in \{1, \ldots, n\} \) sao cho:
    \[
        \gcd(a_1 + b_1, a_2 + b_2, \ldots, a_m + b_m) < n.
    \]

    Khi đó, với mọi lựa chọn \( b_1, \ldots, b_m \), ta có:
    \[
        \gcd(a_1 + b_1, \ldots, a_m + b_m) \ge n.
    \]

    Đồng thời:
    \[
        \gcd(a_1 + b_1, \ldots, a_m + b_m) \le a_1 + b_1 \le n^m + n - 2.
    \]

    Có tối đa \( n^m - 1 \) giá trị có thể cho ước số chung lớn nhất. Nhưng có \( n^m \) bộ giá trị \( (b_1, \ldots, b_m) \).
    Theo nguyên lý Dirichlet, tồn tại hai bộ \( (b_1, \ldots, b_m) \) cho cùng một giá trị gcd là \( d \ge n \).
    Tuy nhiên, với mỗi \( i \), chỉ tồn tại duy nhất một \( b_i \in \{1, \ldots, n\} \) sao cho \( a_i + b_i \equiv 0 \pmod{d} \).
    Suy ra chỉ có nhiều nhất một bộ \( (b_1, \ldots, b_m) \) cho giá trị gcd bằng \( d \), mâu thuẫn.
\end{soln}
\footnotetext{\href{https://www.egmo.org/egmos/egmo4/solutions.pdf}{Lời giải chính thức.}}

\begin{soln}[Lời giải 2.]
    Tương tự lời giải 1, giả sử \( a_1 \le n^m - 2 \). Ta xét:
    \[
        \gcd(a_1 + 1, a_2 + 1, a_3 + 1, \ldots, a_m + 1)
    \]
    và
    \[
        \gcd(a_1 + 1, a_2 + 2, a_3 + 1, \ldots, a_m + 1).
    \]

    Hai giá trị này nguyên tố cùng nhau, nên:
    \[
        a_1 + 1 \ge n^2.
    \]

    Nếu tiếp tục thay các số 1 thành 2 trong các vị trí khác, sau \( m - 1 \) bước, ta có:
    \[
        a_1 + 1 \ge n^m \implies a_1 \ge n^m - 1,
    \]
    mâu thuẫn với giả thiết ban đầu.
\end{soln}

\newpage

\begin{soln}[Lời giải 3.]
    Ta sẽ chứng minh một phiên bản mạnh hơn:
    \begin{claim*}
        Với \( m, n > 1 \), nếu tồn tại \( a_i \le n^{2^{m-1}} \), thì tồn tại \( b_1, \ldots, b_m \in \{1, 2\} \) sao cho:
        \[
            \gcd(a_1 + b_1, \ldots, a_m + b_m) < n.
        \]
    \end{claim*}
    \begin{subproof}
        Giả sử ngược lại. Khi đó, với mọi bộ \( (b_1, \ldots, b_m) \) có \( b_1 = 1 \), \( b_i \in \{1, 2\} \),
        ta được \( 2^{m-1} \) số khác nhau, và mọi cặp trong số đó nguyên tố cùng nhau vì tồn tại \( i > 1 \)
        sao cho \( a_i + 1 \) và \( a_i + 2 \) lần lượt xuất hiện.
        
        Vậy mọi số này đều chia hết \( a_1 + 1 \), và mỗi số \( \ge n \), với nhiều nhất một số bằng \( n \), suy ra:
        \[
            a_1 + 1 \ge n(n + 1)^{2^{m-1} - 1} \implies a_1 \ge n^{2^{m-1}},
        \]
        mâu thuẫn với giả thiết. Vậy tồn tại bộ \( b_i \in \{1, 2\} \) như yêu cầu.
    \end{subproof}

    \textit{Chú thích:} Giới hạn \( n^{2^{m-1}} \) có thể cải thiện.
\end{soln}
\fi

\ifshowhint
\begin{hint*}[\nameref{problem:EGMO-2015-P3}]
    Dùng nguyên lý Dirichlet để dẫn đến mâu thuẫn nếu giả sử không tồn tại bộ \( b_i \) phù hợp. Thử xét các lựa chọn \( b_i \in \{1, 2\} \).
\end{hint*}
\fi

\ifshowremark
\begin{remark*}
    Đánh giá [\textbf{\nameref{definition:25M}}]
\end{remark*}
\newpage
\fi