\ifshowproblem
\begin{problem}[\gls{IRN 2015 TST}1/P3]\label{problem:IRN-2015-TST1-P3}
    Gọi \( b_1 < b_2 < b_3 < \dots \) là dãy tất cả các số tự nhiên có thể viết được dưới dạng tổng hai bình phương của hai số tự nhiên.
    Chứng minh rằng tồn tại vô hạn số tự nhiên \( m \) sao cho \( b_{m+1} - b_m = 2015 \).
\end{problem}
\fi

\ifshowinfo
Đánh giá [\textbf{\nameref{definition:30M}}]\footnotemark
\footnotetext{\href{https://artofproblemsolving.com/community/c6h1087616p4817123}{Thảo luận AoPS.}}
\fi