\ifshowproblem
\begin{problem}[\gls{KOR 2015 MO}/P1]\label{example:KOR-2015-MO-P1}
    Với một số nguyên dương cố định \( k \), ta định nghĩa hai dãy số \( A_n \) và \( B_n \) theo quy luật sau:
    \[
        \begin{aligned}
            &A_1 = k,\quad A_2 = k,\quad A_{n+2} = A_n A_{n+1}, \\
            &B_1 = 1,\quad B_2 = k,\quad B_{n+2} = \frac{B_{n+1}^3 + 1}{B_n}.
        \end{aligned}
    \]
    
    Chứng minh rằng với mọi số nguyên dương \( n \), biểu thức \( A_{2n} B_{n+3} \) là một số nguyên.
\end{problem}
\fi

\ifshowinfo
[\textbf{\nameref{definition:25M}}]\footnotemark
\footnotetext{\href{https://artofproblemsolving.com/community/c6h1065491p4627516}{Thảo luận AoPS.}}
\fi