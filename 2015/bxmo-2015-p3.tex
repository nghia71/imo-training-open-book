\ifshowproblemandsoln
\ifshowproblem\begin{problem}[\gls{BxMO 2015}/P3]\label{problem:BxMO-2015-P3}\fi
\ifshowsoln\begin{problem}[\nameref{problem:BxMO-2015-P3}]\fi
    Có tồn tại số nguyên tố nào có dạng thập phân là \( 3811\cdots1 \) hay không,  
    tức là bắt đầu bằng các chữ số \( 3, 8, 1, 1 \), theo đúng thứ tự, và sau đó là một hoặc nhiều chữ số \( 1 \)?
\end{problem}
\fi

\ifshowsoln
\begin{soln}\footnotemark
    Gọi
    \[
        a(n) = 38\underbrace{11\ldots11}_{n \text{ chữ số } 1}.
    \]
    Xét ba trường hợp theo phần dư của \( n \mod 3 \):

    \textbf{Trường hợp 1:} \( n = 3k + 1 \equiv 1 \pmod{3} \). Khi đó, tổng chữ số của \( a(n) \) là:
    \[
        3 + 8 + n = 11 + 3k + 1 = 3(k + 4),
    \]
    chia hết cho 3, suy ra \( a(n) \) chia hết cho 3, suy ra \( a(n) \) không nguyên tố.

    \textbf{Trường hợp 2:} \( n = 3k + 2 \equiv 2 \pmod{3} \). Ta có:
    \[
        a(2) = 3811 = 3700 + 111,
    \]
    chia hết cho 37. Dễ thấy:
    \[
        a(3k + 2) = 1000 \cdot a(3k - 1) + 111,
    \]
    nên theo quy nạp \( a(3k + 2) \) chia hết cho 37, suy ra \( a(n) \) không nguyên tố.

    \textbf{Trường hợp 3:} \( n = 3k \equiv 0 \pmod{3} \). Ta có:
    \[
        9a(3k) = 342\underbrace{99\ldots99}_{n \text{ chữ số } 9} = 7 \cdot 10^k \cdot \frac{10^{3k} - 1}{9}.
    \]
    Suy ra:
    \[
        a(3k) = \frac{7 \cdot 10^k \cdot (10^{3k} - 1)}{81},
    \]
    chia hết cho một số lớn hơn 9, suy ra \( a(n) \) có ước số khác 1 và chính nó, suy ra không nguyên tố.

    \textbf{Kết luận:} Trong mọi trường hợp, \( a(n) \) không phải số nguyên tố.
\end{soln}
\footnotetext{\href{https://www.k12mathcontests.com/download/bxmo/2015}{Lời giải chính thức.}}
\fi

\ifshowhint
\begin{hint*}[\nameref{problem:BxMO-2015-P3}]
    Xét ba trường hợp theo modulo 3 của số chữ số \( n \). Trong mỗi trường hợp, phân tích tổng chữ số, chia hết, hoặc biểu thức đại số để suy ra tính chia hết.
\end{hint*}
\fi

\ifshowremark
\begin{remark*}
    Đánh giá [\textbf{\nameref{definition:10M}}]
\end{remark*}
\newpage
\fi
