\ifshowproblem
\begin{problem}[\gls{CAN 2015 QRC}/P3]\label{example:CAN-2015-QRC-P3}
	Cho \( N \) là một số có ba chữ số phân biệt và khác 0. Ta gọi \( N \) là một số \textit{tầm thường} nếu nó có tính chất sau:
	khi viết ra tất cả 6 hoán vị có ba chữ số từ các chữ số của \( N \), trung bình cộng của chúng bằng chính \( N \).
	Ví dụ: \( N = 481 \) là số \textit{tầm thường} vì trung bình cộng của các số \( \{418, 481, 148, 184, 814, 841\} \) bằng \( 481 \).
	Hãy xác định số \textit{tầm thường} lớn nhất.	
\end{problem}
\fi

\ifshowinfo
[\textbf{\nameref{definition:5M}}]\footnotemark
\footnotetext{\href{https://artofproblemsolving.com/community/c6h1259093p6527057}{Thảo luận AoPS.}}
\fi