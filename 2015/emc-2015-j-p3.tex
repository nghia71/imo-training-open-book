\ifshowproblemandsoln
\ifshowproblem\begin{problem}[\gls{EMC 2015}/J/P3]\label{problem:EMC-2015-J-P3}\fi
\ifshowsoln\begin{problem}[\nameref{problem:EMC-2015-J-P3}]\fi
    Ký hiệu \( d(n) \) là số lượng ước dương của \( n \).  
    Với số nguyên dương \( n \), ta định nghĩa:
    \[
        f(n) = d(k_1) + d(k_2) + \cdots + d(k_m),
    \]
    trong đó \( 1 = k_1 < k_2 < \cdots < k_m = n \) là tất cả các ước của số \( n \).
    
    Ta gọi một số nguyên \( n > 1 \) là \textit{gần hoàn hảo} (almost perfect) nếu \( f(n) = n \).
    
    Hãy tìm tất cả các số gần hoàn hảo.    
\end{problem}
\fi

\ifshowsoln
\begin{soln}[Lời giải 1.]\footnotemark

    Cách định nghĩa khác của hàm \( f(n) \) là:
    \[
        f(n) = \sum_{k \mid n,\, k \ge 1} d(k).
    \]
    
    Giả sử \( n = p_1^{a_1} p_2^{a_2} \cdots p_r^{a_r} \) là phân tích thành thừa số nguyên tố. Ta biết:
    \[
        d(n) = \prod_{i=1}^r (a_i + 1).
    \]

    Ta chứng minh rằng hàm \( f \) là hàm nhân tính, nghĩa là
    \begin{claim*}
        Nếu \( \gcd(n, m) = 1 \), thì: $f(nm) = f(n)f(m).$
    \end{claim*}
    \begin{subproof}
        Áp dụng tính nhân của hàm \( d \), ta có:
        \[
            f(nm) = \sum_{k \mid nm} d(k) = \sum_{k_1 \mid n,\, k_2 \mid m} d(k_1 k_2) = \sum_{k_1 \mid n,\, k_2 \mid m} d(k_1)d(k_2) = f(n)f(m).
        \]
    \end{subproof}

    Nếu \( r = 1 \) thì \( n = p^{a_1} \). Khi đó các ước của \( n \) là \( 1, p, p^2, \ldots, p^{a_1} \), nên:
    \[
        f(n) = \sum_{i = 0}^{a_1} (i + 1) = \frac{(a_1 + 1)(a_1 + 2)}{2}.
    \]

    Kết hợp với tính nhân, ta được:
    \[
        f(n) = \prod_{i = 1}^r \frac{(a_i + 1)(a_i + 2)}{2}.
    \]

    Ta sẽ chứng minh rằng với \( p \ge 5 \), và với \( p = 3 \) khi \( a \ge 3 \), thì:
    \begin{claim*}
        $f(p^a) = \frac{(a+1)(a+2)}{2} < \frac{2}{3}p^a.$
    \end{claim*}
    \begin{subproof}
        \textit{Cơ sở:} Dễ thấy \( 3 < \frac{2}{3}p \) với \( p \ge 5 \), và \( 6 < \frac{2}{3} \cdot 27 \).

        \textit{Bước quy nạp:} nhận thấy:
        \[
            \frac{a+3}{a+1} \le 2 < p.
        \]
    \end{subproof}

    Tương tự, ta cũng chứng minh được rằng với \( p = 2 \), thì \( f(2^a) < 2^a \) khi \( a \ge 4 \).
    Kiểm tra trực tiếp với \( p = 2 \), \( a = 1, 2, 3 \), và \( p = 3 \), \( a = 1, 2 \), ta có:
    \[
        f(p^a) \le \frac{3}{2}p^a, \quad f(p^a) \le p^a \text{ với } p \ge 3 \text{ hoặc } (p = 2, a \ge 4).
    \]

    Nếu \( f(n) = n \), thì:
    \[
        \prod_{i = 1}^r \frac{f(p_i^{a_i})}{p_i^{a_i}} = 1.
    \]
    Từ đó suy ra rằng các ước nguyên tố khả dĩ của \( n \) chỉ là 2 và 3.

    Nếu \( r = 1 \), thì nghiệm duy nhất là \( n = 3 \).

    Nếu \( r = 2 \), với \( p_1 = 2, p_2 = 3 \), và \( 1 \le a_1 \le 2, 1 \le a_2 \le 2 \), có 4 trường hợp để kiểm tra, trong đó \( n = 18, 36 \) thỏa mãn.

    Vậy:
    \[
        \boxed{\text{Tất cả các số gần hoàn hảo là } 3,\ 18,\ 36.}
    \]
\end{soln}
\footnotetext{\href{https://emc.mnm.hr/wp-content/uploads/2015/12/EMC_2015_Juniors_ENG_Solutions.pdf}{Lời giải chính thức.}}

\newpage

\begin{soln}[Lời giải 2.]\footnotemark
    Ta đưa ra lời giải khác không sử dụng quá nhiều tính chất của hàm \( f \).

    Trước tiên, ta chứng minh bổ đề sau:
    \begin{lemma*}
        Với mọi số nguyên \( n > 1 \) và số nguyên tố \( p \), ta có:
        \[
            f(pn) \le 3f(n),
        \]
        và dấu $=$ xảy ra khi và chỉ khi \( \gcd(p, n) = 1 \).
    \end{lemma*}
    \begin{subproof}
        Với mọi số nguyên \( m \), tập hợp các ước của \( pm \) là hợp của hai tập:
        \begin{itemize}[topsep=0pt, partopsep=0pt, itemsep=0pt]
            \item Tập các ước của \( m \),
            \item Tập các ước của \( m \) nhân với \( p \).
        \end{itemize}

        Hai tập này rời nhau khi và chỉ khi \( \gcd(p, m) = 1 \): nếu \( p \) và \( m \) nguyên tố cùng nhau, rõ ràng không có ước nào thuộc cả hai tập;
        nếu không cùng nhau, thì \( p \) thuộc cả hai.

        Do đó, ta có \( d(pm) \le 2d(m) \), và:
        \[
            f(pn) = \sum_{k \mid pn} d(k) \le \sum_{k \mid n} d(k) + \sum_{k \mid n} d(pk) \le f(n) + 2f(n) = 3f(n).
        \]

        Cả hai dấu $=$ xảy ra khi và chỉ khi hai tập là rời nhau, tức \( \gcd(p, n) = 1 \).
    \end{subproof}

    Ngoài ra, ta có:
    \[
        f(2^k) = d(1) + d(2) + \cdots + d(2^k) = 1 + 2 + \cdots + (k+1) = \frac{(k+1)(k+2)}{2}.
    \]

    Lưu ý rằng nếu tồn tại số nguyên dương \( n \) sao cho \( f(n) < n \), thì với mọi \( p > 3 \), ta có:
    \[
        f(pn) \le 3f(n) \le pf(n) < pn.
    \]

    Do đó, nếu \( f(n) < n \), thì với mọi số lẻ \( m \), ta có \( f(mn) < mn \).

    Vì điều này, ta đưa ra định nghĩa:
    \begin{definition*}
        Một số \( n \) là \textit{bội đẹp (nice multiple)} của \( m \) nếu \( m \mid n \) và \( \frac{n}{m} \) là số lẻ. 
        Tương tự, ta định nghĩa \textit{ước đẹp (nice divisor)}.
    \end{definition*}

    Khẳng định từ trên: nếu với một số \( n \) nào đó ta có \( f(n) < n \), thì không có bội đẹp nào của \( n \) là số gần hoàn hảo.

    \textbf{Chiến lược:} ta sẽ kiểm tra các số nhỏ và xét tỉ lệ \( f(n) / n \). Khi \( f(n) < n \), ta loại mọi bội đẹp của \( n \). 
    Với công thức \( f(2^k) \), ta kết luận rằng với \( k \) đủ lớn (khi \( f(2^k) < 2^k \)) thì không còn số gần hoàn hảo nào. 
    Dễ thấy rằng bằng quy nạp ta có \( f(2^k) < 2^k \) với \( k > 4 \).

    Do đó, không có số gần hoàn hảo nào có dạng \( 2^k \cdot m \) với \( k > 4 \) và \( m \) lẻ, vì tất cả những số này có \( 2^k \) là ước đẹp.

    Vậy ta chỉ cần kiểm tra các số có dạng \( 2^k \cdot m \) với \( k \le 3 \) và \( m \) lẻ.

    \textbf{Trường hợp 1: \( k = 0 \)}

    Với mọi số nguyên tố lẻ \( p \), ta có \( f(p) = d(1) + d(p) = 3 \le p \). Suy ra \( n = 3 \) là một nghiệm. 
    Không có nghiệm nào khác: nếu một số lẻ có ước nguyên tố khác 3 thì do \( f(p) < p \), nó không thể là số gần hoàn hảo; 
    nếu là lũy thừa của 3 lớn hơn 3 thì \( f(9) < 3f(3) = 9 \), nên cũng không thỏa. (9 là ước đẹp của mọi lũy thừa lớn hơn của 3.)

    \textbf{Trường hợp 2: \( k = 1 \)}

    Với mọi số nguyên tố lẻ \( p \), ta có \( f(2p) = 3f(2) = 9 \). Nếu \( p > 5 \), thì \( 2p > f(2p) \), 
    nên mọi số gần hoàn hảo dạng \( 2 \cdot m \) phải có các ước nguyên tố là 3 và/hoặc 5. Dễ thấy \( 6 \) và \( 10 \) đều không là số gần hoàn hảo. 
    Vậy ứng viên phải có ước đẹp là \( 2 \cdot 9,\ 2 \cdot 15,\ 2 \cdot 25 \). Với \( n = 18 \), ta có nghiệm. Hai trường hợp còn lại cho \( f(n) < n \).

    Nếu muốn tìm nghiệm mới, do chúng không thể là bội đẹp của 30 hay 50, khả năng duy nhất là có ước đẹp \( 2 \cdot 27 \). 
    Nhưng theo trường hợp đẳng thức của bổ đề:
    \[
        f(2 \cdot 27) < 3f(2 \cdot 9) = 2 \cdot 27.
    \]
    Không còn nghiệm nào trong trường hợp này.

    \textbf{Trường hợp 3: \( k = 2 \)}

    Với mọi số nguyên tố lẻ \( p \), ta có \( f(4p) = 3f(4) = 18 \). Nếu \( p > 5 \), thì \( 4p > f(4p) \), 
    nên mọi số gần hoàn hảo dạng này có các ước nguyên tố là 3 và/hoặc 5. Dễ thấy \( 12 \) và \( 20 \) không là số gần hoàn hảo. 
    Ứng viên phải có ước đẹp là \( 4 \cdot 9,\ 4 \cdot 15,\ 4 \cdot 25 \). Với \( n = 36 \), ta có nghiệm. Hai trường hợp còn lại cho \( f(n) < n \).

    Nếu muốn nghiệm mới, do chúng không thể là bội đẹp của 60 hay 100, khả năng duy nhất là có ước đẹp \( 4 \cdot 27 \). Nhưng:
    \[
        f(4 \cdot 27) < 3f(4 \cdot 9) = 4 \cdot 27.
    \]
    Không có nghiệm nào khác.

    \textbf{Trường hợp 4: \( k = 3 \)}

    Với mọi số nguyên tố lẻ \( p \), ta có \( f(8p) = 3f(8) = 30 \). Tương tự các trường hợp trên, chỉ xét các số dạng \( 8 \cdot 3^l \). 
    Ta có \( 8 \cdot 3 = 24 \) không là số gần hoàn hảo. Tất cả các ứng viên khác có ước đẹp là \( 8 \cdot 9 = 72 \), mà \( f(72) = 60 < 72 \).

    Như vậy:
    \[
        \boxed{3,\ 18,\ 36 \text{ là tất cả các số gần hoàn hảo.}}
    \]
\end{soln}
\footnotetext{\href{https://emc.mnm.hr/wp-content/uploads/2015/12/EMC_2015_Juniors_ENG_Solutions.pdf}{Lời giải chính thức.}}
\fi

\ifshowhint
\begin{hint*}[\nameref{problem:EMC-2015-J-P3}]
    Xét tính nhân của hàm \( f \), tìm giới hạn chặt với hàm \( f(p^a) \), và sử dụng loại trừ với các ước nguyên tố.
\end{hint*}
\fi

\ifshowremark
\begin{remark*}
    Đánh giá [\textbf{\nameref{definition:20M}}]
\end{remark*}
\newpage
\fi