\ifshowproblem
\begin{problem}[\gls{EMC 2015}/J/P3]\label{example:EMC-2015-J-P3}
    Ký hiệu \( d(n) \) là số lượng ước dương của \( n \).  
    Với số nguyên dương \( n \), ta định nghĩa:
    \[
        f(n) = d(k_1) + d(k_2) + \cdots + d(k_m),
    \]
    trong đó \( 1 = k_1 < k_2 < \cdots < k_m = n \) là tất cả các ước của số \( n \).
    
    Ta gọi một số nguyên \( n > 1 \) là \textit{gần hoàn hảo} (almost perfect) nếu \( f(n) = n \).
    
    Hãy tìm tất cả các số gần hoàn hảo.    
\end{problem}
\fi

\ifshowinfo
[\textbf{\nameref{definition:20M}}]
\footnotetext{\href{https://emc.mnm.hr/wp-content/uploads/2015/12/EMC_2015_Juniors_ENG_Solutions.pdf}{Lời giải chính thức.}}
\fi