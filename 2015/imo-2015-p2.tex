\ifshowproblemandsoln
\ifshowproblem\begin{problem}[\gls{IMO 2015}/P2]\label{problem:IMO-2015-P2}\fi
\ifshowsoln\begin{problem}[\nameref{problem:IMO-2015-P2}]\fi
    Find all positive integers $(a,b,c)$ such that
    $$ab-c,\quad bc-a,\quad ca-b$$are all powers of $2$.
\end{problem}
\fi

\ifshowsoln
\begin{soln}[Lời giải 1.]\footnotemark
    Dễ thấy rằng mười sáu bộ ba sau thỏa mãn yêu cầu: \( (2, 2, 2) \), ba hoán vị của \( (2, 2, 3) \), và sáu hoán vị của mỗi bộ \( (2, 6, 11) \) và \( (3, 5, 7) \).

    Giả sử \( (a, b, c) \) là bộ ba bất kỳ thỏa mãn điều kiện. Nếu \( a = 1 \), thì \( bc - a \) và \( cb - a \) đều là lũy thừa của 2,
    nên tổng của chúng là 0 suy ra mâu thuẫn suy ra \( a, b, c \ge 2 \).

    \textbf{Trường hợp 1.} Có ít nhất hai số bằng nhau. Giả sử \( a = b \), khi đó \( a^2 - c \) và \( a(c - 1) \) là lũy thừa của 2
    suy ra \( a \) và \( c - 1 \) là lũy thừa của 2 suy ra tồn tại \( \alpha, \gamma \) sao cho \( a = 2^\alpha \), \( c = 2^\gamma + 1 \). Khi đó:
    \[
        a^2 - c = 2^{2\alpha} - 2^\gamma - 1 \text{ là lũy thừa của 2} \implies \gamma \le 1.
    \]

    Do \( 2^{2\alpha} - 2 \) và \( 2^{2\alpha} - 3 \) chỉ là lũy thừa của 2 khi \( \alpha = 1 \) suy ra chỉ có thể là \( (2, 2, 2) \) hoặc \( (2, 2, 3) \).

    \textbf{Trường hợp 2.} Ba số phân biệt. Giả sử \( 2 \le a < b < c \). Khi đó tồn tại số nguyên không âm \( \alpha, \beta, \gamma \) sao cho:
    \[
        bc - a = 2^\alpha, \quad ac - b = 2^\beta, \quad ab - c = 2^\gamma.
    \]

    Từ điều kiện, dễ thấy \( \alpha > \beta > \gamma \).

    \textbf{Trường hợp 2.1.} \( a = 2 \). Giả sử \( \gamma > 0 \implies ab - c \) chẵn suy ra \( c \) chẵn
    suy ra \( b \) chẵn suy ra \( bc \equiv 2 \mod 4 \), mâu thuẫn suy ra \( \gamma = 0 \implies c = 2b - 1 \).

    Thế vào \( ac - b = 2\beta \implies 3b - 2 = 2^\beta \). Với \( b > 2 \), chỉ có thể là \( \beta = 4 \implies b = 6, c = 11 \)
    suy ra nghiệm \( (2, 6, 11) \). Với \( \beta \ge 5 \), vế phải không chia hết cho 32 suy ra mâu thuẫn với \( \alpha > \beta \).

    \textbf{Trường hợp 2.2.} \( a \ge 3 \). Chọn \( \vartheta \in \{-1, +1\} \) sao cho \( c - \vartheta \) không chia hết cho 4. Khi đó:
    \[
        2^\alpha + \vartheta 2^\beta = (b + a\vartheta)(c - \vartheta).
    \]
    
    Do \( b + a\vartheta \) chia hết cho \( 2^{\beta - 1} \), mà \( a, b < 2^{\beta - 1} \),
    điều này chỉ có thể nếu \( \vartheta = 1 \), và \( a + b = 2^{\beta - 1} \). Khi đó:
    \[
        ac - b = 2(a + b), \implies c = b + 2.
    \]

    Thế vào \( bc - a = (b - 1)(b + 3) \) là lũy thừa của 2 suy ra hai thừa số cách nhau 4
    suy ra \( b = 5 \implies c = 7 \implies a = 3 \). Vậy nghiệm \( (3, 5, 7) \).
\end{soln}
\footnotetext{\href{http://www.imo-official.org/problems/IMO2015SL.pdf}{Lời giải chính thức.}}

\begin{soln}[Lời giải 2.]
    Giống phần đầu lời giải 1, \( a, b, c \ge 2 \). Xét theo tính chẵn lẻ của các số.

    \textbf{Trường hợp 1.} Cả ba số chẵn. Gọi \( 2^A, 2^B, 2^C \) là lũy thừa 2 lớn nhất chia \( a, b, c \), giả sử \( A \le B \le C \).

    Khi đó \( ac - b = 2^B \le b \), và \( bc - a = 2^A \le a \). Cộng lại được \( (a + b)c \le 2(a + b) \implies c \le 2 \implies a = b = c = 2 \).

    \textbf{Trường hợp 2.} Cả ba số lẻ. Nếu có hai số bằng nhau, ví dụ \( a = b \implies ac - b = a(c - 1) \) có ước lẻ suy ra không là lũy thừa 2 suy ra loại.

    Giả sử \( a < b < c \). Với \( \alpha, \beta \) như trên:
    \[
        bc - a = 2^\alpha, \quad ac - b = 2^\beta \implies 2^\beta \mid b^2 - a^2 = (b + a)(b - a).
    \]

    Một trong hai thừa số chia hết cho \( 2^{\beta - 1} \), từ đó:
    \[
        ac - b \le 2(a + b) \implies a = 3, \quad c = b + 2.
    \]
    
    Thế vào \( bc - a = (b - 1)(b + 3) \implies b = 5, c = 7 \implies a = 3 \) suy ra nghiệm \( (3, 5, 7) \).

    \textbf{Trường hợp 3.} Có cả số chẵn và lẻ. Giả sử \( c \) lẻ, \( a \le b \). Khi đó \( ab - c \) lẻ suy ra phải bằng 1 suy ra
    \[
        ab - c = 1. \tag{1}
    \]

    Nếu \( a = b \implies c = a^2 - 1 \), và \( ac - b = a(a^2 - 2) \) là lũy thừa của 2
    suy ra \( a, a^2 - 2 \) là lũy thừa của 2 suy ra \( a = 2 \implies (2, 2, 3) \).

    Nếu \( a < b \), gọi \( \alpha, \beta \) như trên:
    \[
        bc - a = 2^\alpha, \quad ac - b = 2^\beta. \tag{2}
    \]

    Nếu \( \beta = 0 \implies ac - b = 1 \) và từ (1) suy ra \( b = c = 1 \), mâu thuẫn suy ra \( \beta > 0 \implies a, b \) chẵn.

    Thế \( c = ab - 1 \) vào (2), ta có:
    \[
        bc - a = ab^2 - (a + b) = 2^\alpha, \quad ac - b = a^2b - (a + b) = 2^\beta.
    \]

    Cộng hai phương trình:
    \[
        2^\alpha + 2^\beta = (ab - 2)(a + b).
    \]

    Phía trái là lũy thừa của 2 suy ra \( ab - 2 \) chẵn nhưng không chia hết 4 suy ra \( a + b \) chia hết \( 2^{\beta - 1} \).

    Từ hai biểu thức ta được:
    \[
        a = 2^\tau A, \quad b = 2^\tau B, \quad a + b = 2^{3\tau}C \implies \tau = 1, A = 1, B = 3, C = 1.
    \]

    Suy ra \( a = 2, b = 6, c = 11 \).
\end{soln}

\begin{remark*}
    Xét lại Trường hợp 3 trong Lời giải 2: đặt \( d = \gcd(a, b), a = dp, b = dq, \gcd(p, q) = 1 \). Khi đó:
    \[
        ab - c = 1 \implies c = d^2pq - 1.
    \]

    Thế vào:
    \[
        bc - a = dpd^2q^2 - (p + q), \quad ac - b = dpd^2p^2q - (p + q).
    \]

    Hiệu hai biểu thức:
    \[
        d^3pq(q - p) \implies \text{phân tích chia hết và rút gọn, được } p = 1, q = d^2 - 1 \implies d = 2.
    \]

    Suy ra \( a = 2, b = 6, c = 11 \).
\end{remark*}
\fi

\ifshowhint
\begin{hint*}[\nameref{problem:IMO-2015-P2}]
    Áp dụng điều kiện tích trừ đi một số bằng lũy thừa của 2. Bắt đầu xét các trường hợp: hai số bằng nhau, ba số phân biệt, các hoán vị chẵn lẻ khác nhau.
\end{hint*}
\fi

\ifshowproblemandsoln
\ifshowproblem\begin{problem}[\gls{IMO 2015}/P2]\label{problem:IMO-2015-P2}\fi
\ifshowsoln\begin{problem}[\nameref{problem:IMO-2015-P2}]\fi
    Find all positive integers $(a,b,c)$ such that
    $$ab-c,\quad bc-a,\quad ca-b$$are all powers of $2$.
\end{problem}
\fi

\ifshowsoln
\begin{soln}[Lời giải 1.]\footnotemark
    Dễ thấy rằng mười sáu bộ ba sau thỏa mãn yêu cầu: \( (2, 2, 2) \), ba hoán vị của \( (2, 2, 3) \), và sáu hoán vị của mỗi bộ \( (2, 6, 11) \) và \( (3, 5, 7) \).

    Giả sử \( (a, b, c) \) là bộ ba bất kỳ thỏa mãn điều kiện. Nếu \( a = 1 \), thì \( bc - a \) và \( cb - a \) đều là lũy thừa của 2,
    nên tổng của chúng là 0 suy ra mâu thuẫn suy ra \( a, b, c \ge 2 \).

    \textbf{Trường hợp 1.} Có ít nhất hai số bằng nhau. Giả sử \( a = b \), khi đó \( a^2 - c \) và \( a(c - 1) \) là lũy thừa của 2
    suy ra \( a \) và \( c - 1 \) là lũy thừa của 2 suy ra tồn tại \( \alpha, \gamma \) sao cho \( a = 2^\alpha \), \( c = 2^\gamma + 1 \). Khi đó:
    \[
        a^2 - c = 2^{2\alpha} - 2^\gamma - 1 \text{ là lũy thừa của 2} \implies \gamma \le 1.
    \]

    Do \( 2^{2\alpha} - 2 \) và \( 2^{2\alpha} - 3 \) chỉ là lũy thừa của 2 khi \( \alpha = 1 \) suy ra chỉ có thể là \( (2, 2, 2) \) hoặc \( (2, 2, 3) \).

    \textbf{Trường hợp 2.} Ba số phân biệt. Giả sử \( 2 \le a < b < c \). Khi đó tồn tại số nguyên không âm \( \alpha, \beta, \gamma \) sao cho:
    \[
        bc - a = 2^\alpha, \quad ac - b = 2^\beta, \quad ab - c = 2^\gamma.
    \]

    Từ điều kiện, dễ thấy \( \alpha > \beta > \gamma \).

    \textbf{Trường hợp 2.1.} \( a = 2 \). Giả sử \( \gamma > 0 \implies ab - c \) chẵn suy ra \( c \) chẵn
    suy ra \( b \) chẵn suy ra \( bc \equiv 2 \mod 4 \), mâu thuẫn suy ra \( \gamma = 0 \implies c = 2b - 1 \).

    Thế vào \( ac - b = 2\beta \implies 3b - 2 = 2^\beta \). Với \( b > 2 \), chỉ có thể là \( \beta = 4 \implies b = 6, c = 11 \)
    suy ra nghiệm \( (2, 6, 11) \). Với \( \beta \ge 5 \), vế phải không chia hết cho 32 suy ra mâu thuẫn với \( \alpha > \beta \).

    \textbf{Trường hợp 2.2.} \( a \ge 3 \). Chọn \( \vartheta \in \{-1, +1\} \) sao cho \( c - \vartheta \) không chia hết cho 4. Khi đó:
    \[
        2^\alpha + \vartheta 2^\beta = (b + a\vartheta)(c - \vartheta).
    \]
    
    Do \( b + a\vartheta \) chia hết cho \( 2^{\beta - 1} \), mà \( a, b < 2^{\beta - 1} \),
    điều này chỉ có thể nếu \( \vartheta = 1 \), và \( a + b = 2^{\beta - 1} \). Khi đó:
    \[
        ac - b = 2(a + b), \implies c = b + 2.
    \]

    Thế vào \( bc - a = (b - 1)(b + 3) \) là lũy thừa của 2 suy ra hai thừa số cách nhau 4
    suy ra \( b = 5 \implies c = 7 \implies a = 3 \). Vậy nghiệm \( (3, 5, 7) \).
\end{soln}
\footnotetext{\href{http://www.imo-official.org/problems/IMO2015SL.pdf}{Lời giải chính thức.}}

\begin{soln}[Lời giải 2.]
    Giống phần đầu lời giải 1, \( a, b, c \ge 2 \). Xét theo tính chẵn lẻ của các số.

    \textbf{Trường hợp 1.} Cả ba số chẵn. Gọi \( 2^A, 2^B, 2^C \) là lũy thừa 2 lớn nhất chia \( a, b, c \), giả sử \( A \le B \le C \).

    Khi đó \( ac - b = 2^B \le b \), và \( bc - a = 2^A \le a \). Cộng lại được \( (a + b)c \le 2(a + b) \implies c \le 2 \implies a = b = c = 2 \).

    \textbf{Trường hợp 2.} Cả ba số lẻ. Nếu có hai số bằng nhau, ví dụ \( a = b \implies ac - b = a(c - 1) \) có ước lẻ suy ra không là lũy thừa 2 suy ra loại.

    Giả sử \( a < b < c \). Với \( \alpha, \beta \) như trên:
    \[
        bc - a = 2^\alpha, \quad ac - b = 2^\beta \implies 2^\beta \mid b^2 - a^2 = (b + a)(b - a).
    \]

    Một trong hai thừa số chia hết cho \( 2^{\beta - 1} \), từ đó:
    \[
        ac - b \le 2(a + b) \implies a = 3, \quad c = b + 2.
    \]
    
    Thế vào \( bc - a = (b - 1)(b + 3) \implies b = 5, c = 7 \implies a = 3 \) suy ra nghiệm \( (3, 5, 7) \).

    \textbf{Trường hợp 3.} Có cả số chẵn và lẻ. Giả sử \( c \) lẻ, \( a \le b \). Khi đó \( ab - c \) lẻ suy ra phải bằng 1 suy ra
    \[
        ab - c = 1. \tag{1}
    \]

    Nếu \( a = b \implies c = a^2 - 1 \), và \( ac - b = a(a^2 - 2) \) là lũy thừa của 2
    suy ra \( a, a^2 - 2 \) là lũy thừa của 2 suy ra \( a = 2 \implies (2, 2, 3) \).

    Nếu \( a < b \), gọi \( \alpha, \beta \) như trên:
    \[
        bc - a = 2^\alpha, \quad ac - b = 2^\beta. \tag{2}
    \]

    Nếu \( \beta = 0 \implies ac - b = 1 \) và từ (1) suy ra \( b = c = 1 \), mâu thuẫn suy ra \( \beta > 0 \implies a, b \) chẵn.

    Thế \( c = ab - 1 \) vào (2), ta có:
    \[
        bc - a = ab^2 - (a + b) = 2^\alpha, \quad ac - b = a^2b - (a + b) = 2^\beta.
    \]

    Cộng hai phương trình:
    \[
        2^\alpha + 2^\beta = (ab - 2)(a + b).
    \]

    Phía trái là lũy thừa của 2 suy ra \( ab - 2 \) chẵn nhưng không chia hết 4 suy ra \( a + b \) chia hết \( 2^{\beta - 1} \).

    Từ hai biểu thức ta được:
    \[
        a = 2^\tau A, \quad b = 2^\tau B, \quad a + b = 2^{3\tau}C \implies \tau = 1, A = 1, B = 3, C = 1.
    \]

    Suy ra \( a = 2, b = 6, c = 11 \).
\end{soln}

\begin{remark*}
    Xét lại Trường hợp 3 trong Lời giải 2: đặt \( d = \gcd(a, b), a = dp, b = dq, \gcd(p, q) = 1 \). Khi đó:
    \[
        ab - c = 1 \implies c = d^2pq - 1.
    \]

    Thế vào:
    \[
        bc - a = dpd^2q^2 - (p + q), \quad ac - b = dpd^2p^2q - (p + q).
    \]

    Hiệu hai biểu thức:
    \[
        d^3pq(q - p) \implies \text{phân tích chia hết và rút gọn, được } p = 1, q = d^2 - 1 \implies d = 2.
    \]

    Suy ra \( a = 2, b = 6, c = 11 \).
\end{remark*}
\fi

\ifshowhint
\begin{hint*}[\nameref{problem:IMO-2015-P2}]
    Áp dụng điều kiện tích trừ đi một số bằng lũy thừa của 2. Bắt đầu xét các trường hợp: hai số bằng nhau, ba số phân biệt, các hoán vị chẵn lẻ khác nhau.
\end{hint*}
\fi

\ifshowremark
\begin{remark*}
    Đánh giá [\textbf{\nameref{definition:25M}}]
\end{remark*}
\newpage
\fi