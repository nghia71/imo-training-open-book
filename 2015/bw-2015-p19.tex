\ifshowproblemandsoln
\ifshowproblem\begin{problem}[\gls{BW 2015}/P19]\label{problem:BW-2015-P19}\fi
\ifshowsoln\begin{problem}[\nameref{problem:BW-2015-P19}]\fi
    Ba số nguyên dương phân biệt từng đôi một \( a, b, c \) thỏa mãn \( \gcd(a,b,c) = 1 \), đồng thời:
    \[
        a \mid (b - c)^2,\quad b \mid (a - c)^2,\quad c \mid (a - b)^2.
    \]
    Chứng minh rằng không tồn tại tam giác không suy biến nào có độ dài ba cạnh là \( a, b, c \).
\end{problem}
\fi

\ifshowsoln
\begin{soln}\footnotemark
    Trước hết, ta nhận thấy rằng ba số \( a, b, c \) nguyên dương phân biệt đôi một này phải nguyên tố cùng nhau từng đôi một.
    Thật vậy, giả sử \( a \) và \( b \) cùng chia hết cho một số nguyên tố \( p \), thì \( p \mid b \), mà theo giả thiết 
    \[ b \mid (a - c)^2 \Rightarrow p \mid (a - c)^2 \Rightarrow p \mid a - c \Rightarrow p \mid c. \]
    
    Vậy \( p \) là ước chung của \( a, b, c \), mâu thuẫn với giả thiết \( \gcd(a, b, c) = 1 \).

    Giờ ta xét biểu thức:
    \[
        M = 2ab + 2bc + 2ac - a^2 - b^2 - c^2.
    \]
    
    Từ điều kiện bài toán, ta thấy rằng \( a \mid (b - c)^2 \Rightarrow a \mid b^2 - 2bc + c^2 \). Tương tự, \( b \mid (c - a)^2 \), \( c \mid (a - b)^2 \).
    Cộng ba biểu thức lại, ta thu được rằng \( a \mid M \), \( b \mid M \), \( c \mid M \) suy ra \( abc \mid M \).

    Giả sử tồn tại tam giác có ba cạnh là \( a, b, c \), thì theo bất đẳng thức tam giác:
    \[
        a < b + c \Rightarrow a^2 < ab + ac.
    \]

    Tương tự:
    \[
        b^2 < bc + ba,\quad c^2 < ca + cb.
    \]

    Cộng ba bất đẳng thức trên, ta có:
    \[
        a^2 + b^2 + c^2 < ab + bc + ca \Rightarrow M > 0.
    \]
    Vì \( abc \mid M \), nên \( M \ge abc \).

    Mặt khác:
    \[
        a^2 + b^2 + c^2 > ab + bc + ac \Rightarrow M < ab + bc + ac.
    \]

    Giả sử không mất tính tổng quát \( a > b > c \), thì:
    \[
        M < 3ab.
    \]

    Kết hợp với \( M \ge abc \Rightarrow abc < 3ab \Rightarrow c < 3 \Rightarrow c = 1 \text{ hoặc } c = 2 \).

    - Nếu \( c = 1 \), thì \( b < a < b + 1 \Rightarrow a = b \), mâu thuẫn vì \( a \ne b \).

    - Nếu \( c = 2 \), thì \( b < a < b + 2 \Rightarrow a = b + 1 \). Khi đó \( c = 2 \not |\ 1 = (a - b)^2 = 1 \), mâu thuẫn.

    Vậy không tồn tại tam giác không suy biến nào có ba cạnh là \( a, b, c \). Điều phải chứng minh.
\end{soln}
\footnotetext{\href{https://www.math.olympiaadid.ut.ee/eng/archive/bw/bw15sol.pdf}{Lời giải chính thức.}}
\fi

\ifshowhint
\begin{hint*}[\nameref{problem:BW-2015-P19}]
    Xét đồng thời ba điều kiện chia hết trong giả thiết.
    Định nghĩa một biểu thức đối xứng \( M \) mà mỗi số trong bộ ba chia hết cho nó.
    Sau đó giả sử tồn tại tam giác với độ dài ba cạnh đó, và dùng bất đẳng thức trong tam giác để đưa ra mâu thuẫn.
\end{hint*}
\fi

\ifshowremark
\begin{remark*}
    Đánh giá [\textbf{\nameref{definition:25M}}]
\end{remark*}
\newpage
\fi
