\ifshowproblem
\begin{problem}[\gls{GBR 2015 TST}/N3/P3]\label{example:GBR-2015-TST-N3-P3}
    Cho \( a_1 < a_2 < \cdots < a_n \) là các số nguyên dương đôi một nguyên tố cùng nhau,
    với \( a_1 \) là số nguyên tố và \( a_1 \ge n + 2 \).  
    Xét đoạn thẳng \( I = [0, a_1 a_2 \cdots a_n] \) trên trục số thực.  
    Trên đoạn này, đánh dấu tất cả các số nguyên chia hết cho ít nhất một trong các số \( a_1, a_2, \ldots, a_n \).  
    Các điểm này chia đoạn \( I \) thành một số đoạn nhỏ hơn.  
    Chứng minh rằng tổng bình phương độ dài của các đoạn nhỏ này chia hết cho \( a_1 \).
\end{problem}
\fi

\ifshowinfo
[\textbf{\nameref{definition:30M}}]
\fi