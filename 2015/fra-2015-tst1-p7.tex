\ifshowproblem
\begin{problem}[\gls{FRA 2015 TST}1/P7]\label{example:FRA-2015-TST1-P7}
    Cho các số nguyên $a, b, c, n$ với $n \geq 2$.
    Gọi $p$ là một số nguyên tố chia hết cả hai biểu thức $a^2 + ab + b^2$ và $a^n + b^n + c^n$, nhưng không chia hết $a + b + c$.  
    Chứng minh rằng $n$ và $p - 1$ không nguyên tố cùng nhau.
\end{problem}
\fi

\ifshowinfo
[\textbf{\nameref{definition:20M}}]\footnotemark
\footnotetext{\href{https://maths-olympiques.fr/wp-content/uploads/2017/10/ofm-2014-2015-test-janvier-corrige.pdf}{Lời giải chính thức.}}
\fi