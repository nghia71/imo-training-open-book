\ifshowproblemandsoln
\ifshowproblem\begin{problem}[\gls{FRA 2015 TST}1/P7]\label{problem:FRA-2015-TST1-P7}\fi
\ifshowsoln\begin{problem}[\nameref{problem:FRA-2015-TST1-P7}]\fi
    Cho các số nguyên $a, b, c, n$ với $n \geq 2$.
    Gọi $p$ là một số nguyên tố chia hết cả hai biểu thức $a^2 + ab + b^2$ và $a^n + b^n + c^n$, nhưng không chia hết $a + b + c$.  
    Chứng minh rằng $n$ và $p - 1$ không nguyên tố cùng nhau.
\end{problem}
\fi

\ifshowsoln
\begin{soln}\footnotemark
    Nếu \( p \mid a \) và \( p \mid b \) thì \( p \mid a^n + b^n + c^n \implies p \mid c^n \implies p \mid c \),
    mâu thuẫn với giả thiết \( p \nmid a + b + c \). Vậy, giả sử \( p \nmid b \). Nhân với nghịch đảo của \( b \mod p \), ta có thể giả sử \( b = 1 \), nên:
    \[
        p \mid a^2 + a + 1, \quad p \mid a^n + 1 + c^n, \quad p \nmid a + 1 + c.
    \]

    Vì \( a^2 + a + 1 \) lẻ nên \( p \) cũng lẻ. Hơn nữa, do:
    \[
        a^3 \equiv 1 \pmod{p} \quad \text{(vì } (a^2 + a + 1)(a - 1) = a^3 - 1),
    \]
    nên bậc modulo của \( a \mod p \) là 1 hoặc 3.

    \textbf{Trường hợp 1:} \( a \equiv 1 \pmod{p} \implies p = 3 \). Khi đó:
    \[
        c^n \equiv -a^n - 1 \equiv -1 - 1 = -2 \equiv 1 \pmod{3} \implies c \equiv \pm 1 \pmod{3}.
    \]
    Do \( p \nmid a + 1 + c \implies 1 + 1 + c \not\equiv 0 \pmod{3} \implies c \equiv -1 \pmod{3} \).
    Vì \( c^n \equiv 1 \pmod{3} \) nên \( n \) chẵn, tức là \( \gcd(n, p - 1) = \gcd(n, 2) \ne 1 \).

    \textbf{Trường hợp 2:} bậc modulo của \( a \mod p \) là 3. Do định lý Fermat, bậc modulo chia hết \( p - 1 \implies 3 \mid p - 1 \).

    Giả sử \( \gcd(n, p - 1) = 1 \). Khi đó, ánh xạ \( x \mapsto x^n \) là một hoán vị trên \( \mathbb{Z}/p\mathbb{Z} \). Khi đó:

    \textbf{Nếu } \( n \equiv 1 \pmod{3} \) thì:
    \[
        c^n \equiv -a^n - 1 \equiv -a - 1 \equiv a^2 \pmod{p}.
    \]
    Vì ánh xạ là hoán vị nên \( c \equiv a^2 \equiv -a - 1 \implies a^2 + a + 1 \equiv 0 \pmod{p} \),
    mâu thuẫn vì \( p \mid a^2 + a + 1 \), nên không thể có \( c \equiv -a - 1 \).

    \textbf{Nếu } \( n \equiv 2 \pmod{3} \) thì:
    \[
        c^n \equiv -a^n - 1 \equiv -a^2 - 1 \equiv a \pmod{p} \implies c \equiv a \equiv -a - 1,
    \]
    dẫn đến \( 2a + 1 \equiv 0 \pmod{p} \), thay vào \( a^2 + a + 1 \), ta lại có mâu thuẫn.

    Vậy \( n \equiv 0 \pmod{3} \implies \gcd(n, p - 1) \) không thể là 1. Suy ra:
    \[
        \gcd(n, p - 1) \ne 1.
    \]
\end{soln}
\footnotetext{\href{https://maths-olympiques.fr/wp-content/uploads/2017/10/ofm-2014-2015-test-janvier-corrige.pdf}{Lời giải chính thức.}}
\fi

\ifshowhint
\begin{hint*}[\nameref{problem:FRA-2015-TST1-P7}]
    Đưa \( b \) về 1 bằng cách nhân với nghịch đảo, rồi dùng bậc modulo của phần tử trong \( \mathbb{Z}_p^\times \)
    để xét các khả năng của \( a \) theo mod \( p \), từ đó dẫn đến điều kiện về \( n \).
\end{hint*}
\fi

\ifshowremark
\begin{remark*}
    Đánh giá [\textbf{\nameref{definition:20M}}]
\end{remark*}
\newpage
\fi