\ifshowproblemandsoln
\ifshowproblem\begin{problem}[\gls{IMO 2015 SL}/P6]\label{problem:IMO-2015-SL-P6}\fi
\ifshowsoln\begin{problem}[\nameref{problem:IMO-2015-SL-P6}]\fi
    Ký hiệu \( \mathbb{Z}_{>0} \) là tập các số nguyên dương.  
    Xét một hàm \( f: \mathbb{Z}_{>0} \to \mathbb{Z}_{>0} \). Với mọi \( m, n \in \mathbb{Z}_{>0} \), ta định nghĩa:
    \[
        f^n(m) = \underbrace{f(f(\ldots f}_{n \text{ lần}}(m)\ldots)).
    \]

    Giả sử hàm \( f \) thỏa mãn hai tính chất sau:
    \begin{enumerate}[topsep=0pt, partopsep=0pt, itemsep=0pt, label=(\roman*)]
        \item Với mọi \( m, n \in \mathbb{Z}_{>0} \), ta có \( \frac{f^n(m) - m}{n} \in \mathbb{Z}_{>0} \);  
        \item Tập \( \mathbb{Z}_{>0} \setminus \{ f(n) \mid n \in \mathbb{Z}_{>0} \} \) là hữu hạn.
    \end{enumerate}

    Chứng minh rằng dãy số \( f(1) - 1, f(2) - 2, f(3) - 3, \ldots \) là tuần hoàn.
\end{problem}
\fi

\ifshowsoln
\begin{soln}\footnotemark
    \textbf{Bước 1.} Ta bắt đầu bằng việc chứng minh rằng \( f \) là đơn ánh. Giả sử tồn tại \( m, k \in \mathbb{Z}_{>0} \) sao cho \( f(m) = f(k) \).
    Theo giả thiết (i), với mọi \( n \in \mathbb{Z}_{>0} \), ta có:
    \[
        \frac{f^n(m) - m}{n} \in \mathbb{Z}, \quad \frac{f^n(k) - k}{n} \in \mathbb{Z}
        \implies
        \frac{k - m}{n} = \frac{f^n(m) - m}{n} - \frac{f^n(k) - k}{n} \in \mathbb{Z}.
    \]
    
    Lấy \( n = |k - m| + 1 \), mẫu số lớn hơn tử số suy ra \( \frac{k - m}{n} \notin \mathbb{Z} \), trừ khi \( k = m \). Vậy \( f \) là đơn ánh.

    Theo giả thiết (ii), tồn tại hữu hạn số nguyên dương \( a_1, \ldots, a_k \) sao cho:
    \[
        \mathbb{Z}_{>0} = \{ a_1, \ldots, a_k \} \cup \{ f(n) : n \in \mathbb{Z}_{>0} \}.
    \]
    
    Ngoài ra, thay \( n = 1 \) vào giả thiết (i), ta có \( f(m) > m \) với mọi \( m \in \mathbb{Z}_{>0} \).

    Ta chứng minh rằng mọi số nguyên dương \( n \) đều có dạng \( f^j(a_i) \) với một số \( j \ge 0 \) và \( i \in \{1, \ldots, k\} \).
    Do \( f \) đơn ánh, biểu diễn này là duy nhất.

    Ta xây dựng bảng sau (gọi là “Bảng”):
    \[
        \begin{array}{cccc}
            a_1 & f(a_1) & f^2(a_1) & \cdots \\
            a_2 & f(a_2) & f^2(a_2) & \cdots \\
            \vdots & \vdots & \vdots & \\
            a_k & f(a_k) & f^2(a_k) & \cdots \\
        \end{array}
    \]

    \textbf{Bước 2.} Ta sẽ chứng minh rằng mỗi hàng trong Bảng là một cấp số cộng.

    Giả sử ngược lại, chỉ có \( t < k \) hàng là cấp số cộng với công sai \( T_1, \ldots, T_t \).
    Đặt \( T = \text{lcm}(T_1, \ldots, T_t) \), \( A = \max(a_1, \ldots, a_t) \).
    Xét đoạn \( \Delta_n = [n+1, n+T] \), số phần tử từ các hàng đầu tiên thuộc đoạn này không đổi theo \( n \ge A \),
    nên số phần tử từ các hàng còn lại (từ hàng \( t+1 \) trở đi) trong \( \Delta_n \) là không đổi.

    Nhưng các hàng còn lại có vô hạn phần tử suy ra tồn tại hàng \( x > t \)
    sao cho đoạn \( [A+1, A + (d+1)(k - t)T] \) chứa ít nhất \( d+1 \) phần tử từ hàng thứ \( x \). Điều này dẫn đến:
    \[
        f^d(a_x) \le A + (d+1)(k - t)T.
    \]

    Do có hữu hạn giá trị \( x \), tồn tại một hàng \( x > t \) sao cho:
    \[
        X := \{ d \in \mathbb{Z}_{\ge 0} : f^d(a_x) \le A + (d+1)(k - t)T \}
    \]
    là vô hạn. Đặt:
    \[
        \omega_d = \frac{f^d(a_x) - a_x}{d}, \quad \text{với } d \in X.
    \]

    Ta có:
    \[
        \omega_d \le \frac{A + (d+1)(k - t)T - a_x}{d} \le A + 2(k - t)T.
    \]
    
    Do đó, \( \omega_d \) chỉ có thể nhận hữu hạn giá trị suy ra tồn tại \( T_x \) sao cho:
    \[
        f^d(a_x) = a_x + d \cdot T_x
    \]
    xảy ra với vô hạn \( d \in X \).

    Đặt \( j \) tùy ý. Chọn \( y \in X \) sao cho \( y - j > |f^j(a_x) - (a_x + jT_x)| \). Khi đó:
    \[
        f^y(a_x) - f^j(a_x) = f^{y-j}(f^j(a_x)) - f^j(a_x),\quad \text{và} \quad f^y(a_x) - (a_x + jT_x) = (y - j)T_x.
    \]
    
    Hiệu của hai số trên chia hết cho \( y - j \), nhưng trị tuyệt đối nhỏ hơn \( y - j \) suy ra hiệu bằng 0.

    Suy ra:
    \[
        f^j(a_x) = a_x + jT_x,
    \]
    tức hàng thứ \( x \) là cấp số cộng, mâu thuẫn giả thiết. Vậy tất cả các hàng đều là cấp số cộng.

    \textbf{Bước 3.} Gọi \( T_i \) là công sai của hàng thứ \( i \), và \( T = \text{lcm}(T_1, \ldots, T_k) \).
    Với mọi \( n \in \mathbb{Z}_{>0} \), nếu \( n \) nằm trong hàng \( i \), thì:
    \[
        f^j(n) = n + jT_i \implies f(n + T) = f(n) + T.
    \]
    
    Suy ra:
    \[
        f(n + T) - (n + T) = f(n) - n.
    \]
    
    Vậy dãy \( f(n) - n \) là tuần hoàn suy ra dãy \( f(n) - n \) với \( n = 1, 2, 3, \ldots \) là tuần hoàn.
\end{soln}
\footnotetext{\href{http://www.imo-official.org/problems/IMO2015SL.pdf}{Lời giải chính thức.}}

\begin{remark*}
    Có một số cách khác để hoàn tất phần thứ hai của lời giải sau khi chỉ số \( x \) ứng với một hàng \textit{dày đặc} đã được xác định.
    Ví dụ, người ta có thể chứng minh rằng tồn tại một số nguyên \( T_x^* \) sao cho tập
    \[
        Y^* := \left\{ j \in \mathbb{Z}_{\ge 0} \;\middle|\; f^{j+1}(a_x) - f^j(a_x) = T_x^* \right\}
    \]
    là vô hạn, và sau đó có thể kết luận bằng một lập luận chia hết tương tự như trước.
\end{remark*}

\begin{remark*}
    Ta có thể kiểm tra rằng, ngược lại, nếu điền các số nguyên dương vào Bảng bằng cách sử dụng hữu hạn nhiều dãy cấp số cộng
    sao cho mỗi số nguyên dương xuất hiện đúng một lần, thì sẽ thu được một hàm \( f \) thỏa mãn cả hai điều kiện được nêu trong bài toán.

    Ví dụ, ta có thể sắp xếp các số nguyên dương như sau:
    \[
    \begin{array}{cccccc}
    2 & 4 & 6 & 8 & 10 & \cdots \\
    1 & 5 & 9 & 13 & 17 & \cdots \\
    3 & 7 & 11 & 15 & 19 & \cdots \\
    \end{array}
    \]
    Điều này tương ứng với hàm:
    \[
        f(n) =
        \begin{cases}
            n + 2 & \text{nếu } n \text{ chẵn}, \\
            n + 4 & \text{nếu } n \text{ lẻ}.
        \end{cases}
    \]

    Như ví dụ này cho thấy, không đúng rằng hàm \( f(n) - n \) phải luôn là hằng số. Tuy nhiên, nó vẫn là một dãy tuần hoàn.
\end{remark*}
\fi

\ifshowhint
\begin{hint*}[\nameref{problem:IMO-2015-SL-P6}]
    Chứng minh \( f \) là đơn ánh và mọi số đều sinh ra từ hữu hạn phần tử khởi tạo.
    Dùng lập luận chia hết để chứng minh các hàng là cấp số cộng, từ đó suy ra dãy \( f(n) - n \) là tuần hoàn.
\end{hint*}
\fi

\ifshowremark
\begin{remark*}
    Đánh giá [\textbf{\nameref{definition:35M}}]
\end{remark*}
\newpage
\fi