\ifshowproblem
\begin{problem}[\gls{IMO 2015 SL}/P6]\label{example:IMO-2015-SL-P6}
    Ký hiệu \( \mathbb{Z}_{>0} \) là tập các số nguyên dương.  
    Xét một hàm \( f: \mathbb{Z}_{>0} \to \mathbb{Z}_{>0} \). Với mọi \( m, n \in \mathbb{Z}_{>0} \), ta định nghĩa:
    \[
        f^n(m) = \underbrace{f(f(\ldots f}_{n \text{ lần}}(m)\ldots)).
    \]

    Giả sử hàm \( f \) thỏa mãn hai tính chất sau:
    \begin{itemize}[topsep=0pt, partopsep=0pt, itemsep=0pt]
        \item Với mọi \( m, n \in \mathbb{Z}_{>0} \), ta có \( \frac{f^n(m) - m}{n} \in \mathbb{Z}_{>0} \);  
        \item Tập \( \mathbb{Z}_{>0} \setminus \{ f(n) \mid n \in \mathbb{Z}_{>0} \} \) là hữu hạn.
    \end{itemize}

    Chứng minh rằng dãy số \( f(1) - 1, f(2) - 2, f(3) - 3, \ldots \) là tuần hoàn.
\end{problem}
\fi

\ifshowinfo
[\textbf{\nameref{definition:35M}}]\footnotemark
\footnotetext{\href{http://www.imo-official.org/problems/IMO2015SL.pdf}{Lời giải chính thức.}}
\fi