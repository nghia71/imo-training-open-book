\ifshowproblem
\begin{problem}[\gls{USA 2015 TSTST}/P3]\label{example:USA-2015-TSTST-P3}
    Gọi \( P \) là tập hợp tất cả các số nguyên tố, và \( M \) là một tập con không rỗng của \( P \).  
    Giả sử rằng với mọi tập con không rỗng \( \{p_1, p_2, \ldots, p_k\} \) của \( M \),  
    tất cả các thừa số nguyên tố của \( p_1p_2\cdots p_k + 1 \) đều nằm trong \( M \).  
    Chứng minh rằng \( M = P \).
\end{problem}
\fi

\ifshowinfo
[\textbf{\nameref{definition:25M}}]\footnotemark
\footnotetext{\href{https://web.evanchen.cc/exams/sols-TSTST-2015.pdf}{Lời giải Evan Chen soạn.}}
\fi