\ifshowproblem
\begin{problem}[\gls{RUS 2015 SMO}/10/P6]\label{example:RUS-2015-SMO-10-P6}
    Một dãy số nguyên được định nghĩa như sau: \( a_1 = 1,\ a_2 = 2,\ a_3 = 3 \) và với \( n > 3 \),
    $a_n$ là số nguyên nhỏ nhất chưa xuất hiện trước đó, nguyên tố cùng nhau với $a_{n-1}$ nhưng không nguyên tố cùng nhau với $a_{n-2}$.
    Chứng minh rằng mọi số tự nhiên xuất hiện đúng một lần trong dãy này.
\end{problem}
\fi

\ifshowinfo
[\textbf{\nameref{definition:20M}}]\footnotemark
\footnotetext{\href{https://artofproblemsolving.com/community/c6h1310104p7016465}{Thảo luận AoPS.}}
\fi