\ifshowproblem
\begin{problem}[\gls{IRN 2015 TST}3/P2]\label{example:IRN-2015-TST3-P2}
	Giả sử \( a_1, a_2, a_3 \) là ba số nguyên dương cho trước. Xét dãy số được xác định bởi công thức:
	\[
		a_{n+1} = \text{lcm}[a_n, a_{n-1}] - \text{lcm}[a_{n-1}, a_{n-2}] \quad \text{với } n \geq 3,
	\]
	trong đó \( [a, b] \) ký hiệu bội chung nhỏ nhất của \( a \) và \( b \), và chỉ được áp dụng với các số nguyên dương.

	Chứng minh rằng tồn tại một số nguyên dương \( k \leq a_3 + 4 \) sao cho \( a_k \leq 0 \).
\end{problem}
\fi

\ifshowinfo
[\textbf{\nameref{definition:20M}}]\footnotemark
\footnotetext{\href{https://artofproblemsolving.com/community/c6h1100830p4959212}{Thảo luận AoPS.}}
\fi