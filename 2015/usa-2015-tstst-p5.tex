\ifshowproblemandsoln
\ifshowproblem\begin{problem}[\gls{USA 2015 TSTST}/P5]\label{problem:USA-2015-TSTST-P5}\fi
\ifshowsoln\begin{problem}[\nameref{problem:USA-2015-TSTST-P5}]\fi
    Ký hiệu \( \varphi(n) \) là số lượng các số nguyên dương nhỏ hơn \( n \) và nguyên tố cùng nhau với \( n \).  
    Chứng minh rằng tồn tại một số nguyên dương \( m \) sao cho phương trình \( \varphi(n) = m \) có ít nhất \( 2015 \) nghiệm \( n \).
\end{problem}
\fi

\ifshowsoln
\begin{soln}[Lời giải 1.]\footnotemark
    Ta xét 11 số nguyên tố sau:
    \[
        S = \{11, 13, 17, 19, 29, 31, 37, 41, 43, 61, 71\}.
    \]

    Tập này có tính chất rằng với mọi \( p \in S \), tất cả các thừa số nguyên tố của \( p - 1 \) đều là số có một chữ số.

    Đặt \( N = (210)^{\text{tỷ}} \), và xét:
    \[
        M = \varphi(N).
    \]

    Với mỗi tập con \( T \subseteq S \), ta có:
    \[
        \varphi\left( N \cdot \prod_{p \in T} (p - 1) \cdot \prod_{p \in T} p \right) = M.
    \]

    Vì số lượng tập con là \( 2^{|S|} = 2^{11} = 2048 > 2015 \), ta đã xây dựng được ít nhất 2015 số \( n \) khác nhau sao cho \( \varphi(n) = M \). Điều phải chứng minh được hoàn tất.

    \textit{Chú thích.} Lời giải này được gợi cảm hứng từ một kết quả sâu sắc rằng chẳng hạn như:
    \[
        \varphi(11 \cdot 1000) = \varphi(10 \cdot 1000).
    \]
\end{soln}
\footnotetext{\href{https://web.evanchen.cc/exams/sols-TSTST-2015.pdf}{Lời giải của Evan Chen.}}

\begin{soln}[Lời giải 2]\footnotemark
    Gọi \( 2 = p_1 < p_2 < \cdots < p_{2015} \) là 2015 số nguyên tố nhỏ nhất.

    Ta xây dựng 2015 số:
    \[
        n_1 = (p_1 - 1)p_2 \cdots p_{2015},\quad
        n_2 = p_1(p_2 - 1) \cdots p_{2015},\quad
        \ldots,\quad
        n_{2015} = p_1p_2 \cdots (p_{2015} - 1).
    \]

    Mỗi số \( n_i \) ở trên đều có cùng giá trị hàm Euler:
    \[
        \varphi(n_i) = \varphi(p_1p_2 \cdots p_{2015}) = \prod_{i = 1}^{2015} (p_i - 1).
    \]

    Như vậy, \( \varphi(n) = m \) có ít nhất 2015 nghiệm, với \( m = \prod_{i = 1}^{2015} (p_i - 1) \).
\end{soln}
\footnotetext{\href{https://web.evanchen.cc/exams/sols-TSTST-2015.pdf}{Lời giải của Yang Liu.}}
\fi

\ifshowhint
\begin{hint*}[\nameref{problem:USA-2015-TSTST-P5}]
    Hãy thử chọn một số lớn \( N \) và xét \( \varphi(N) \). Với các số nguyên tố \( p \) sao cho \( p - 1 \) có nhiều ước nhỏ, 
    nếu \( p \mid n \) và \( (p - 1) \mid n \), bạn có thể làm cho \( \varphi(n) \) bằng một giá trị cố định. 
    Thử kết hợp nhiều tập con của các số nguyên tố dạng đó.
\end{hint*}

\begin{hint*}[\nameref{problem:USA-2015-TSTST-P5}]
    Nhớ rằng \( \varphi(n) \) là tích \( \prod (p_i - 1)p_i^{k_i - 1} \) nếu \( n \) có phân tích nguyên tố \( p_i^{k_i} \). 
    Hãy thử lấy tích của nhiều số nguyên tố nhỏ và thay thế một trong số các \( p_i \) bằng \( p_i - 1 \) rồi nhân với các \( p_j \) còn lại. 
    Liệu \( \varphi \) có giữ nguyên không?
\end{hint*}
\fi

\ifshowremark
\begin{remark*}
    Đánh giá [\textbf{\nameref{definition:20M}}]
\end{remark*}
\newpage
\fi