\ifshowproblem
\begin{problem}[\gls{BW 2015}/P20]\label{example:BW-2015-P20}
    Với mỗi số nguyên \( n \ge 2 \), ta định nghĩa \( A_n \) là số lượng các số nguyên dương \( m \) thỏa mãn tính chất sau:

    Khoảng cách từ \( n \) đến bội gần nhất của \( m \) bằng khoảng cách từ \( n^3 \) đến bội gần nhất của \( m \).  
    (Khoảng cách giữa hai số nguyên \( a \) và \( b \) được định nghĩa là \( |a - b| \).)

    Tìm tất cả các số nguyên \( n \ge 2 \) sao cho \( A_n \) là số lẻ.
\end{problem}
\fi

\ifshowinfo
[\textbf{\nameref{definition:35M}}]\footnotemark
\footnotetext{\href{https://www.math.olympiaadid.ut.ee/eng/archive/bw/bw15sol.pdf}{Lời giải chính thức.}}
\fi