\ifshowproblemandsoln
\ifshowproblem\begin{problem}[\gls{BW 2015}/P20]\label{problem:BW-2015-P20}\fi
\ifshowsoln\begin{problem}[\nameref{problem:BW-2015-P20}]\fi
    Với mỗi số nguyên \( n \ge 2 \), ta định nghĩa \( A_n \) là số lượng các số nguyên dương \( m \) thỏa mãn tính chất sau:

    Khoảng cách từ \( n \) đến bội gần nhất của \( m \) bằng khoảng cách từ \( n^3 \) đến bội gần nhất của \( m \).  
    (Khoảng cách giữa hai số nguyên \( a \) và \( b \) được định nghĩa là \( |a - b| \).)

    Tìm tất cả các số nguyên \( n \ge 2 \) sao cho \( A_n \) là số lẻ.
\end{problem}
\fi

\ifshowsoln
\begin{soln}\footnotemark
    Với mỗi số nguyên \( n \ge 2 \), ta định nghĩa \( A_n \) là số lượng các số nguyên dương \( m \)
    sao cho khoảng cách từ \( n \) đến bội không âm gần nhất của \( m \) bằng với khoảng cách từ \( n^3 \) đến bội không âm gần nhất của \( m \).

    Với một số nguyên \( m \), khoảng cách từ \( n \) đến bội gần nhất của \( m \) là \( d \), khi đó:
    \[
        m \mid n \pm d \implies n \equiv \pm d \pmod{m}.
    \]

    Nếu khoảng cách từ \( n \) đến bội gần nhất của \( m \) bằng khoảng cách từ \( n^3 \) đến bội gần nhất của \( m \), thì:
    \[
        n \equiv \pm d \pmod{m},\quad n^3 \equiv \pm d \pmod{m} \implies n \equiv \pm n^3 \pmod{m}.
    \]

    Ngược lại, nếu \( n \equiv \pm n^3 \pmod{m} \), thì tồn tại \( 0 \le d \le \frac{m}{2} \) sao cho \( n \equiv \pm d \pmod{m} \),
    và đồng thời \( n^3 \equiv \pm d \pmod{m} \), nên khoảng cách từ \( n \) đến bội gần nhất của \( m \)
    bằng khoảng cách từ \( n^3 \) đến bội gần nhất của \( m \).

    Do đó, ta cần đếm số lượng \( m \in \mathbb{Z}_{>0} \) sao cho:
    \[
        n \equiv \pm n^3 \pmod{m} \iff m \mid n^3 \pm n.
    \]

    Vậy:
    \[
        A_n = \tau(n^3 - n) + \tau(n^3 + n) - \tau(\gcd(n^3 - n, n^3 + n)),
    \]
    với \( \tau(k) \) là số lượng ước dương của \( k \).

    Nhắc lại rằng \( \tau(k) \) lẻ khi và chỉ khi \( k \) là một số chính phương. Hơn nữa, ta có:
    \[
        \gcd(n, n^2 \pm 1) = 1.
    \]

    Nếu \( n^3 \pm n \) là chính phương thì \( n \) và \( n^2 \pm 1 \) đều là chính phương,
    điều này là không thể vì \( n^2 \pm 1 \) không là chính phương (vì chỉ có \( 0, 1 \) là các chính phương liên tiếp).

    Do đó, \( n^3 - n \) và \( n^3 + n \) không là chính phương suy ra \( \tau(n^3 - n) \), \( \tau(n^3 + n) \) đều chẵn.

    Vậy \( A_n \) lẻ khi và chỉ khi \( \gcd(n^3 - n, n^3 + n) \) là chính phương.
    Ta có:
    \[
        \gcd(n^2 - 1, n^2 + 1) = \gcd(n^2 - 1, 2) = 
        \begin{cases}
            1 & \text{n chẵn}, \\
            2 & \text{n lẻ}.
        \end{cases}
        \implies
        \gcd(n^3 - n, n^3 + n) = 
        \begin{cases}
            n & \text{n chẵn}, \\
            2n & \text{n lẻ}.
        \end{cases}
    \]

    Với \( n \) lẻ, \( 2n \) không bao giờ là chính phương vì chứa đúng một thừa số 2.

    Với \( n \) chẵn, \( A_n \) là lẻ khi và chỉ khi \( n \) là số chính phương.

    \textbf{Kết luận:} \( A_n \) là số lẻ khi và chỉ khi \( n \) là một số chính phương chẵn.
\end{soln}
\footnotetext{\href{https://www.math.olympiaadid.ut.ee/eng/archive/bw/bw15sol.pdf}{Lời giải chính thức.}}
\fi

\ifshowhint
\begin{hint*}[\nameref{problem:BW-2015-P20}]
    Với mỗi \( m \), điều kiện khoảng cách bằng nhau tương đương với \( m \mid n^3 \pm n \). Sau đó sử dụng số lượng ước nguyên dương và tính chẵn/lẻ của chúng.
\end{hint*}
\fi

\ifshowremark
\begin{remark*}
    Đánh giá [\textbf{\nameref{definition:30M}}]
\end{remark*}
\newpage
\fi