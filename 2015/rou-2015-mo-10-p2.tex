\ifshowproblem
\begin{problem}[\gls{ROU 2015 MO}/10/P2]\label{example:ROU-2015-MO-10-P2}
    Xét một số tự nhiên \( n \) sao cho tồn tại một số tự nhiên \( k \) và \( k \) số nguyên tố phân biệt thỏa mãn \( n = p_1 \cdot p_2 \cdots p_k \).

    \begin{itemize}[topsep=0pt, partopsep=0pt, itemsep=0pt]
        \item Tìm số lượng các hàm \( f:\{1, 2, \ldots, n\} \longrightarrow \{1, 2, \ldots, n\} \) 
        sao cho tích \( f(1) \cdot f(2) \cdots f(n) \) chia hết \( n \).
        \item Nếu \( n = 6 \), hãy tìm số lượng các hàm \( f:\{1, 2, 3, 4, 5, 6\} \longrightarrow \{1, 2, 3, 4, 5, 6\} \) 
        sao cho tích \( f(1)\cdot f(2)\cdot f(3)\cdot f(4)\cdot f(5)\cdot f(6) \) chia hết cho \( 36 \).
    \end{itemize}    
\end{problem}
\fi

\ifshowinfo
[\textbf{\nameref{definition:20M}}]\footnotemark
\footnotetext{\href{https://artofproblemsolving.com/community/c6h1902309p13008239}{Thảo luận AoPS.}}
\fi