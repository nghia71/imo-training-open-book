\ifshowproblemandsoln
\ifshowproblem\begin{problem}[\gls{MEMO 2015}/I/P4]\label{problem:MEMO-2015-I-P4}\fi
\ifshowsoln\begin{problem}[\nameref{problem:MEMO-2015-I-P4}]\fi
    Tìm tất cả các cặp số nguyên dương \( (m, n) \) sao cho tồn tại các số nguyên \( a, b > 1 \) nguyên tố cùng nhau, thỏa mãn:
    \[
        \frac{a^m + b^m}{a^n + b^n} \in \mathbb{Z}.
    \]
\end{problem}
\fi

\ifshowsoln
\begin{soln}[Lời giải 1.]\footnotemark
    \textbf{Đáp án.} Tất cả các cặp \( (m, n) = (qn, n) \), trong đó \( q \) là số nguyên dương lẻ và \( n \) là số nguyên dương tùy ý.

    Nếu \( \frac{m}{n} = q \) là một số nguyên lẻ, thì:
    \[
        \frac{a^m + b^m}{a^n + b^n} = \frac{a^{qn} + b^{qn}}{a^n + b^n}
    \]
    là một số nguyên, bởi vì tử số chia hết cho mẫu số theo công thức chia đa thức.

    Ta sẽ chứng minh rằng các cặp \( (qn, n) \), với \( q \) là một số nguyên lẻ, là các nghiệm duy nhất.

    Giả sử ngược lại, rằng tồn tại các cặp \( (m, n) \) là nghiệm cho bài toán nhưng \( \frac{m}{n} \) không phải là số nguyên lẻ. Trong các cặp đó, chọn cặp có tổng \( m + n \) nhỏ nhất.

    Rõ ràng \( m > n \). Gọi \( m = n + k \), với \( k \in \mathbb{Z}_{>0} \). Không mất tính tổng quát, ta giả sử \( a = b \). Khi đó:
    \[
        \frac{a^m + b^m}{a^n + b^n} = \frac{a^{n+k} + b^{n+k}}{a^n + b^n} = \frac{b^k(a^n + b^n)}{a^n + b^n} + t = b^k + t,\ \text{với}\ t \in \mathbb{Z}_{>0}.
    \]

    Do đó tồn tại số nguyên dương \( t \) sao cho:
    \[
        \frac{a^m + b^m}{a^n + b^n} = b^k + t.
    \]

    Phương trình tương đương:
    \[
        a^m + b^m = (b^k + t)(a^n + b^n).
    \]

    Vì \( a \) và \( b \) nguyên tố cùng nhau, nên \( a^n + b^n \) và \( a^n \) cũng nguyên tố cùng nhau.
    
    Từ đó suy ra \( t \) chia hết cho \( a^n \). Gọi \( c \in \mathbb{Z}_{>0} \) sao cho \( t = c a^n \). Khi đó:
    \[
        a^m + b^m = (b^k + c a^n)(a^n + b^n).
    \]

    Vế phải lớn hơn \( a^n \), nên ta kết luận rằng \( k > n \).

    Ta viết lại phương trình:
    \[
        a^n(a^{k - n} - c) = b^n(b^{k - n} + c).
    \]

    Gọi \( x \in \mathbb{Z}_{>0} \) sao cho:
    \[
        b^{k - n} + c = x a^n + c,\quad a^{k - n} - c = x b^n.
    \]

    Cộng hai phương trình trên:
    \[
        a^{k - n} + b^{k - n} = x(a^n + b^n) \implies \frac{a^{k - n} + b^{k - n}}{a^n + b^n} = x \in \mathbb{Z}.
    \]

    Vì \( k - n < m \), và \( (m, n) \) là cặp được chọn sao cho tổng \( m + n \) là nhỏ nhất (sao cho $\frac{m}{n}$ không phải là số nguyên lẻ),
    nên ta suy ra rằng \( \frac{k - n}{n} \) là một số nguyên lẻ. Gọi \( \frac{k - n}{n} = 2r + 1 \), khi đó:
    \[
        k = (2r + 2)n \implies m = n + k = (2r + 3)n,
    \]
    là bội số lẻ của \( n \), mâu thuẫn với giả thiết rằng \( \frac{m}{n} \) không phải là số nguyên lẻ.

    Do đó, các nghiệm duy nhất là các cặp \( (m, n) = (qn, n) \), với \( q \) là số nguyên dương lẻ.
\end{soln}
\footnotetext{\href{https://www.k12mathcontests.com/download/memo/2015/individual}{Lời giải chính thức.}}

\begin{soln}[Lời giải 2.]
    Rõ ràng \( m > n \). Viết \( m = kn + r \), với \( k \ge 1 \), \( 0 \le r < n \). Khi đó:
    \[
        \frac{a^m + b^m}{a^n + b^n} = a^{(k - 1)n + r} + \frac{b^m - a^{(k - 1)n + r} b^n}{a^n + b^n}.
    \]

    Từ giả thiết, biểu thức trên là số nguyên, nên:
    \[
        \frac{b^m - a^{(k - 1)n + r} b^n}{a^n + b^n} \in \mathbb{Z}.
    \]

    Vì \( a \) và \( b \) nguyên tố cùng nhau:
    \[
        \frac{b^{m-n} - a^{(k - 1)n + r}}{a^n + b^n} = -a^{(k - 2)n + r} + \frac{a^{(k - 2)n + r}b^n + b^{m-n}}{a^n+b^n}
    \]
    cũng là một số nguyên.

    Tiếp tục quy trình ta nhận được: $a^n+b^n$ chia hết $b^r + (-1)^k a^r.$

    Vì \( |b^{r} + (-1)^k a^r| < a^n + b^n \), nên điều này chỉ có thể xảy ra khi $b^{r} + (-1)^k a^r = 0$, tức là \( r = 0 \) và \( k \) lẻ.
    Vậy các nghiệm duy nhất là \( (m, n) = (kn, n) \) với \( k \) lẻ.
\end{soln}
\fi

\ifshowhint
\begin{hint*}[\nameref{problem:MEMO-2015-I-P4}]
    Thử chia \( m \) cho \( n \), đặt \( m = kn + r \). Với \( a = b \), xét biểu thức \(\frac{a^m + b^m}{a^n + b^n}\), tách thành \( b^k + t \). 
    Suy luận chia hết và giả sử ngược lại để tìm mâu thuẫn với giả thiết về \( \frac{m}{n} \).
\end{hint*}
\fi

\ifshowremark
\begin{remark*}
    Đánh giá [\textbf{\nameref{definition:25M}}]
\end{remark*}
\newpage
\fi