\ifshowproblem
\begin{problem}[\gls{IRN 2015 TST}1/P4]\label{problem:IRN-2015-TST1-P4}
	Cho trước số tự nhiên \( n \). Tìm giá trị nhỏ nhất của \( k \) sao cho với mọi tập \( A \) gồm \( k \) số tự nhiên,
	luôn tồn tại một tập con của \( A \) có số phần tử chẵn và tổng các phần tử chia hết cho \( n \).
\end{problem}
\fi

\ifshowinfo
Đánh giá [\textbf{\nameref{definition:20M}}]\footnotemark
\footnotetext{\href{https://artofproblemsolving.com/community/c6h1113216p5083604}{Thảo luận AoPS.}}
\fi