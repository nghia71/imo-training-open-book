\ifshowproblemandsoln
\ifshowproblem\begin{problem}[\gls{BMO 2015}/P4]\label{problem:BMO-2015-P4}\fi
\ifshowsoln\begin{problem}[\nameref{problem:BMO-2015-P4}]\fi
    Chứng minh rằng trong bất kỳ \( 20 \) số nguyên dương liên tiếp nào cũng tồn tại một số nguyên \( d \)
    sao cho với mọi số nguyên dương \( n \), bất đẳng thức sau luôn đúng:
    \[
        n \sqrt{d} \left\{ n \sqrt{d} \right\} > \frac{5}{2},
    \]
    trong đó \( \left\{ x \right\} \) ký hiệu phần thập phân (phần lẻ) của số thực \( x \),
    được định nghĩa là hiệu giữa số \( x \) và phần nguyên lớn nhất không vượt quá \( x \).    
\end{problem}
\fi

\ifshowsoln
\begin{soln}\footnotemark
    Trong 20 số nguyên liên tiếp luôn tồn tại một số có dạng \( 20k + 15 = 5(4k + 3) \). Ta sẽ chứng minh rằng với \( d = 5(4k + 3) \), bất đẳng thức
    \[
        n \sqrt{d} \cdot \{n \sqrt{d}\} > \frac{5}{2}
    \]
    luôn đúng với mọi \( n \in \mathbb{Z}_{>0} \). Lưu ý rằng số có dạng \( 5(4k + 3) \) vừa chia hết cho 5,
    vừa có ít nhất một ước nguyên tố dạng \( 4s + 3 \), là điều kiện cần cho lập luận chia hết phía dưới.

    Vì \( d \equiv -1 \pmod{4} \), nên \( d \) không phải là một số chính phương. Do đó với mỗi \( n \in \mathbb{N} \), tồn tại \( a \in \mathbb{N} \):
    \[
        a < \sqrt{n^2 d} < a + 1 \implies a^2 < n^2 d < (a + 1)^2.
    \]

    Ta sẽ chứng minh rằng \( n^2 d \ge a^2 + 5 \).

    Mỗi số nguyên dương dạng \( 4s + 3 \) đều có một ước nguyên tố cũng thuộc dạng \( 4s + 3 \).
    Gọi \( p \mid (4k + 3) \) là một ước nguyên tố như vậy. Vì \( p \equiv 3 \pmod{4} \),

    \begin{lemma*}
        Một số nguyên tố \( p \equiv 3 \pmod{4} \) không thể là ước của bất kỳ số nào có dạng \( a^2 + 1 \), với \( a \in \mathbb{Z} \).
    \end{lemma*}
    
    \begin{subproof}
        Giả sử ngược lại rằng tồn tại \( a \in \mathbb{Z} \) sao cho \( p \mid a^2 + 1 \). Khi đó:
        \[
            a^2 \equiv -1 \pmod{p}.
        \]

        Nâng hai vế lên lũy thừa \( \frac{p - 1}{2} \), ta được:
        \[
            (a^2)^{\frac{p-1}{2}} \equiv (-1)^{\frac{p-1}{2}} \pmod{p} \implies a^{p-1} \equiv (-1)^{\frac{p-1}{2}} \pmod{p}.
        \]

        Mặt khác, theo định lý Fermat nhỏ (vì \( p \) là số nguyên tố và \( p \nmid a \)):
        \[
            a^{p-1} \equiv 1 \pmod{p}.
        \]

        Suy ra:
        \[
            1 \equiv (-1)^{\frac{p-1}{2}} \pmod{p}.
        \]

        Do \( p = 4k + 3 \implies \frac{p - 1}{2} = 2k + 1 \) là số lẻ, nên:
        \[
            (-1)^{2k + 1} = -1 \implies 1 \equiv -1 \pmod{p} \implies 2 \equiv 0 \pmod{p}.
        \]

        Điều này vô lý vì \( p \ge 7 \). Vậy giả thiết ban đầu là sai, tức là \( p \nmid a^2 + 1 \).
    
        Do đó, không tồn tại số nguyên \( a \) sao cho \( a^2 + 1 \equiv 0 \pmod{p} \) nếu \( p \equiv 3 \pmod{4} \), như cần chứng minh.
    \end{subproof}

    Từ bổ đề suy ra các số \( a^2 + 1^2 = a^2 + 1 \) và \( a^2 + 2^2 = a^2 + 4 \) không chia hết cho \( p \), nên \( n^2 d \ne a^2 + 1, a^2 + 4 \).

    Mặt khác, vì \( 5 \mid d \implies 5 \mid n^2 d \), mà \( 5 \nmid a^2 + 2, a^2 + 3 \), nên \( n^2 d \ne a^2 + 2, a^2 + 3 \).

    Do \( n^2 d > a^2 \) và đã loại trừ các giá trị \( a^2 + 1, \ldots, a^2 + 4 \), ta suy ra:
    \[
        n^2 d \ge a^2 + 5.
    \]

    Khi đó:
    \[
        n \sqrt{d} \cdot \{n \sqrt{d}\} = n \sqrt{d} \cdot (\sqrt{n^2 d} - a) \ge \sqrt{a^2 + 5} \cdot (\sqrt{a^2 + 5} - a).
    \]

    Xét biểu thức phía phải:
    \[
        \sqrt{a^2 + 5} \cdot (\sqrt{a^2 + 5} - a) = (a^2 + 5 - a^2) + \frac{(a^2 + 5 - a^2)^2}{4(a^2 + 5)} = \frac{5}{2} + \frac{25}{4(a^2 + 5)} > \frac{5}{2}.
    \]

    Vậy bất đẳng thức đã cho luôn đúng với \( d = 5(4k + 3) \), và lời giải hoàn tất.
\end{soln}
\footnotetext{\href{https://www.massee-org.eu/mathematical/bmo/item/download/54_36222e39c3de7fbc544bf7b1eed42001}{Lời giải chính thức.}}
\fi

\ifshowhint
\begin{hint*}[\nameref{problem:BMO-2015-P4}]
    Trong 20 số nguyên liên tiếp luôn tồn tại một số chia hết cho $5$ và có dạng $5(4k+3)$.
    Xét số đó và chứng minh rằng \( n^2 d \) không thể là một bình phương gần với \( a^2, a^2+1, \ldots, a^2+4 \).
\end{hint*}
\fi

\ifshowremark
\begin{remark*}
    Đánh giá [\textbf{\nameref{definition:30M}}]
\end{remark*}
\newpage
\fi