\ifshowproblem
\begin{problem}[\gls{BMO 2015}/P4]\label{example:BMO-2015-P4}
    Chứng minh rằng trong bất kỳ \( 20 \) số nguyên dương liên tiếp nào cũng tồn tại một số nguyên \( d \)
    sao cho với mọi số nguyên dương \( n \), bất đẳng thức sau luôn đúng:
    \[
        n \sqrt{d} \left\{ n \sqrt{d} \right\} > \frac{5}{2},
    \]
    trong đó \( \left\{ x \right\} \) ký hiệu phần thập phân (phần lẻ) của số thực \( x \),
    được định nghĩa là hiệu giữa số \( x \) và phần nguyên lớn nhất không vượt quá \( x \).    
\end{problem}
\fi

\ifshowinfo
[\textbf{\nameref{definition:35M}}]\footnotemark
\footnotetext{\href{https://www.massee-org.eu/mathematical/bmo/item/download/54_36222e39c3de7fbc544bf7b1eed42001}{Lời giải chính thức.}}
\fi