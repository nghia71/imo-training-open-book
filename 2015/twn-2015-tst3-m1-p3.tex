\ifshowproblem
\begin{problem}[\gls{TWN 2015 TST}3/M2/P3]\label{example:TWN-2015-TST3-M1-P3}
    Cho \( c \ge 1 \) là một số nguyên. Xét dãy các số nguyên dương được định nghĩa bởi:
    \[
        a_1 = c,\quad a_{n+1} = a_n^3 - 4c\cdot a_n^2 + 5c^2\cdot a_n + c \quad \text{với mọi } n \ge 1.
    \]
    Chứng minh rằng với mỗi số nguyên \( n \ge 2 \), tồn tại một số nguyên tố \( p \) chia hết \( a_n \)
    nhưng không chia hết bất kỳ số nào trong các số \( a_1, a_2, \ldots, a_{n-1} \).
\end{problem}
\fi

\ifshowinfo
[\textbf{\nameref{definition:25M}}]\footnotemark
\footnotetext{\href{https://artofproblemsolving.com/community/c6h1113202p5083576}{Thảo luận AoPS.}}
\fi