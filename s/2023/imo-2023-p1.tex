\documentclass[../2023-n-s.tex]{subfiles}

\begin{document}

\begin{soln}[\autoref{problem:IMO-2023-P1}][\gls{IMO 2023}/P1](Cách 1)[\footnotemark]

    \textit{Trường hợp 1:} Giả sử \( n = p^r \). Khi đó \( d_i = p^{i-1} \), và
    \[
        d_i \mid d_{i+1} + d_{i+2} 
        \Leftrightarrow p^{i-1} \mid p^i + p^{i+1} = p^i(1 + p),
    \]
    luôn đúng với mọi \( i \). Vậy mọi lũy thừa của số nguyên tố đều thỏa mãn.

    \medskip

    \textit{Trường hợp 2:} Giả sử \( n \) có ít nhất hai thừa số nguyên tố. Gọi \( p < q \) là hai thừa số nguyên tố nhỏ nhất.

    Khi đó tồn tại các ước số liên tiếp:
    \[
        d_j = p^{j-1},\quad d_{j+1} = p^j,\quad d_{j+2} = q.
    \]
    Suy ra các ước số đối xứng:
    \[
        d_{k-j-1} = \frac{n}{q},\quad d_{k-j} = \frac{n}{p^j},\quad d_{k-j+1} = \frac{n}{p^{j-1}}.
    \]

    Từ giả thiết \( d_{k-j-1} \mid d_{k-j} + d_{k-j+1} \), ta có:
    \[
        \frac{n}{q}\ \bigg| \left( \frac{n}{p^j} + \frac{n}{p^{j-1}} \right) 
        \implies p^j \mid q(p + 1) 
        \implies p \mid q,\quad \text{mâu thuẫn vì } p \ne q.
    \]

    \textbf{Kết luận:} \( n \) phải là lũy thừa của một số nguyên tố.

    \vspace{1em}
    \textbf{[\nameref{definition:5M}]}\index{5M}.
    \textbf{Tham chiếu:} \nameref{theorem:prime-divides-product}.
\end{soln}

\footnotetext{\samepage \href{https://www.imo-official.org/problems/IMO2023SL.pdf}{Shortlist 2023 with solutions.}}

\newpage

\begin{soln}[\autoref{problem:IMO-2023-P1}][\gls{IMO 2023}/P1](Cách 2)[\footnotemark]

    \begin{claim*}
        \( d_i \mid d_{i+1} \) với mọi \( 1 \leq i \leq k - 1 \).
    \end{claim*}
    \begin{subproof}
        Chứng minh bằng quy nạp.

        \textit{Bước cơ sở:} \( d_1 = 1 \Rightarrow d_1 \mid d_2 \).

        \textit{Bước quy nạp:} Giả sử \( d_{i-1} \mid d_i \). Từ giả thiết:
        \[
            d_{i-1} \mid d_i + d_{i+1} \implies d_{i-1} \mid d_{i+1}. \tag{1}
        \]

        Xét các ước đối xứng:
        \[
            \frac{d_{k-i} + d_{k-i+1}}{d_{k-i-1}} 
            = \frac{\frac{n}{d_i} + \frac{n}{d_{i-1}}}{\frac{n}{d_{i+1}}}
            = \frac{d_{i+1}}{d_i} + \frac{d_{i+1}}{d_{i-1}}. \tag{2}
        \]

        Do (2) là số nguyên, nên từ (1) suy ra \( d_i \mid d_{i+1} \).
    \end{subproof}

    Do mọi ước \( d_i \) là bội của \( d_2 \) – ước nguyên tố nhỏ nhất của \( n \) – suy ra \( n \) là lũy thừa của một số nguyên tố.

    \vspace{1em}
    \textbf{[\nameref{definition:5M}]}\index{5M}.
    \textbf{Tham chiếu:} \nameref{theorem:prime-divides-product} \nameref{theorem:induction-principle}.
\end{soln}

\footnotetext{\samepage \href{https://www.imo-official.org/problems/IMO2023SL.pdf}{Shortlist 2023 with solutions.}}

\newpage

\begin{soln}[\autoref{problem:IMO-2023-P1}][\gls{IMO 2023}/P1](Cách 3)[\footnotemark]

    Sử dụng đối xứng \( d_i d_{k+1-i} = n \). Từ giả thiết:
    \[
        d_{k-i-1} \mid d_{k-i} + d_{k-i+1} 
        \implies \frac{n}{d_{i+2}} \ \bigg| \left( \frac{n}{d_{i+1}} + \frac{n}{d_i} \right).
    \]

    Nhân hai vế với \( d_i d_{i+1} d_{i+2} \), ta có:
    \[
        d_i d_{i+1} \mid d_i d_{i+2} + d_{i+1} d_{i+2} 
        \implies d_i \mid d_{i+1} d_{i+2}. \tag{1}
    \]

    Mặt khác, từ đề bài:
    \[
        d_i \mid d_{i+1} + d_{i+2} 
        \implies d_i \mid d_{i+1}^2 + d_{i+1} d_{i+2}. \tag{2}
    \]

    Kết hợp (1) và (2) suy ra \( d_i \mid d_{i+1}^2 \).

    Gọi \( p = d_2 \) là ước nguyên tố nhỏ nhất. Dùng quy nạp dễ dàng chứng minh được \( p \mid d_i \) với mọi \( i \ge 2 \).  
    Nếu tồn tại ước nguyên tố khác \( q \ne p \), thì \( p \mid q \), mâu thuẫn.

    \textbf{Kết luận:} \( n \) là lũy thừa của một số nguyên tố.

    \vspace{1em}
    \textbf{[\nameref{definition:5M}]}\index{5M}.
    \textbf{Tham chiếu:} \nameref{theorem:prime-divides-product} \nameref{theorem:induction-principle}.
\end{soln}

\footnotetext{\samepage \href{https://www.imo-official.org/problems/IMO2023SL.pdf}{Shortlist 2023 with solutions.}}

\end{document}