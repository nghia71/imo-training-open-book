\documentclass[../2015-n-s.tex]{subfiles}

\begin{document}

\begin{soln}[\autoref{problem:MEMO-2015-I-P4}][\gls{MEMO 2015}/I/P4][\footnotemark]

	Giả sử \( \gcd(a, b) = 1 \), \( a, b > 1 \), và phân số:
	\[
		\frac{a^m + b^m}{a^n + b^n}
	\]
	là một số nguyên.

	Ta có \( a^n + b^n \mid a^m + b^m \), nên \( m \ge n \). Đặt \( m = kn + r \), với \( k \ge 1,\ 0 \le r < n \).

	Ta phân tích:
	\[
		\frac{a^m + b^m}{a^n + b^n} = \frac{(a^n)^k a^r + (b^n)^k b^r}{a^n + b^n}
	\]

	Tử số là tổ hợp của \( a^n + b^n \) nhân với hệ số nào đó cộng với phần dư. Quá trình này lặp lại đến khi còn lại:
	\[
		\frac{a^r + b^r}{a^n + b^n}
	\]

	Nếu \( r > 0 \), thì \( a^r + b^r < a^n + b^n \), nên không thể chia hết, mâu thuẫn. Vậy \( r = 0 \implies m = kn \).

	Ta kiểm tra điều kiện cần. Nếu \( k \) lẻ:
	\[
		a^m + b^m = (a^n)^k + (b^n)^k \equiv - (a^n + b^n) \Mod{a^n + b^n} \implies \text{chia hết}.
	\]

	Nếu \( k \) chẵn, thử với \( a = -b \), thì:
	\[
		a^m + b^m = 0,\quad a^n + b^n = 0 \implies \text{phân số không xác định}.
	\]

	\textbf{Kết luận:} Tất cả các cặp thỏa mãn là:
	\[
		\boxed{(m, n) = (kn, n)\quad \text{với } k \text{ lẻ}}.
	\]

	\vspace{1em}
	\textbf{[\nameref{definition:10M}]}\index{10M}.
	\textbf{Tham chiếu:} \nameref{theorem:euclidean-division}.
\end{soln}

\footnotetext{\samepage \href{http://memo2015.dmfa.si/files/solutions.pdf}{Lời giải chính thức.}}

\end{document}