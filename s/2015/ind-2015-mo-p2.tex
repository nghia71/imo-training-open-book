\documentclass[../2015-n-s.tex]{subfiles}

\begin{document}

\begin{soln}[\autoref{problem:IND-2015-N2}][\gls{IND 2015 MO}/P2][\footnotemark]
	
	Gọi biểu diễn thập phân của \( \frac{1}{n} \) là:
	\[
		\frac{1}{n} = 0.a_1a_2 \cdots a_{x_n} \overline{b_1b_2 \cdots b_{\ell_n}},
	\]
	trong đó \( x_n \): độ dài phần không tuần hoàn, \( \ell_n \): độ dài phần tuần hoàn.

	Ta có:
	\[
		\frac{10^{x_n + \ell_n} - 10^{x_n}}{n} \in \mathbb{Z}^+
		\implies n \mid \left(10^{x_n + \ell_n} - 10^{x_n}\right)
		\implies n \mid 10^{x_n}(10^{\ell_n} - 1).
	\]

	Giả sử \( n = 2^a \cdot 5^b \cdot q \), với \( \gcd(q,10)=1 \).
	Để \( \frac{1}{n} \) có phần không tuần hoàn dài \( x_n \), thì:
	\[
		2^a 5^b \mid 10^{x_n}
		\implies x_n = \min\{x:\ 2^a 5^b \mid 10^x\} = \max(a, b).
	\]

	\textbf{Kết luận:} Độ dài phần không tuần hoàn trong biểu diễn thập phân vô hạn của \( \frac{1}{n} \) là:
	\[
		x_n = \max(a, b) \quad \text{với } n = 2^a \cdot 5^b \cdot q,\ \gcd(q, 10) = 1.
	\]
	
	\vspace{1em}
	\textbf{[\nameref{definition:5M}]}\index{5M}.
	\textbf{Tham chiếu:} \nameref{lemma:repunit-form}.
\end{soln}

\footnotetext{\href{https://artofproblemsolving.com/community/c6h623454p3730817}{Lời giải của \textbf{utkarshgupta}.}}

\end{document}