\documentclass[../2015-n-s.tex]{subfiles}

\begin{document}

\begin{soln}[\autoref{problem:IND-2015-N6}][\gls{IND 2015 MO}/P6][\footnotemark]
	
	Các phần dư khả dĩ của một số chính phương modulo 12 là:
	\[
		x^2 \equiv 0, 1, 4, 9 \Mod{12}.
	\]

	Gọi \(S\) là tập gồm 11 số chính phương. Mỗi phần tử của \(S\) thuộc một trong 4 lớp dư trên.
	
	Ta cần tìm hai tập rời nhau \(A, B \subset S\), mỗi tập gồm 3 phần tử, sao cho
	\[
		\sum_{x \in A} x \equiv \sum_{y \in B} y \Mod{12}.
	\]
	
	\textit{Trường hợp 1:} Có ít nhất 6 phần tử trong \(S\) có cùng phần dư modulo 12 (giả sử là \(r\)).  
	Ta có thể chọn hai bộ ba rời nhau gồm các phần tử đều có phần dư \(r\), nên tổng mỗi bộ ba là
	\[
		3r \Mod{12}\ \text{Hai tổng này bằng nhau modulo}\ 12.
	\]

	\vspace{1em}
	\textit{Trường hợp 2:} Có một phần dư xuất hiện 4 hoặc 5 lần.  
	Vì tổng cộng có 11 phần tử và chỉ 4 phần dư khả dĩ, theo nguyên lý Dirichlet sẽ có ít nhất một phần dư khác xuất hiện từ 2 lần trở lên.  
	Từ đó ta có thể chọn hai bộ ba từ các phần dư khác nhau. Vì số phần dư là hữu hạn, nên số tổng modulo 12 có thể tạo ra từ các bộ ba cũng hữu hạn.
	Do đó vẫn tồn tại hai bộ ba có tổng bằng nhau modulo 12.
	
	\vspace{1em}
	\textit{Trường hợp 3:} Mỗi phần dư trong \(\{0,1,4,9\}\) xuất hiện tối đa 3 lần.  
	Vì \( |S| = 11\), trường hợp này tồn tại.
	
	Số bộ ba phân biệt có thể chọn từ \(S\) là:
	\[
		\binom{11}{3} = 165.
	\]
	
	Xét các bộ ba (cho phép trùng lặp phần tử, không phân biệt thứ tự) từ tập \(\{0,1,4,9\}\), là số tổ hợp có lặp:
	\[
		\binom{4 + 3 - 1}{3} = \binom{6}{3} = 20.
	\]

	Tức là chỉ có tối đa 20 tổng khác nhau modulo 12 có thể thu được từ một bộ ba.
	
	Vì có 165 bộ ba trong \(S\) nhưng chỉ có 20 khả năng tổng modulo 12, theo nguyên lý Dirichlet tồn tại hai bộ ba khác nhau có tổng đồng dư modulo 12.
	
	\textbf{Kết luận:} Trong cả ba trường hợp, luôn tồn tại hai tập rời nhau gồm ba số chính phương có tổng bằng nhau modulo 12.

	\vspace{1em}
	\textbf{[\nameref{definition:15M}]}\index{15M}.
	\textbf{Tham chiếu:} \nameref{theorem:pigeonhole-principle}.
\end{soln}

\footnotetext{\href{https://artofproblemsolving.com/community/c6h623456p3730860}{Dựa theo lời giải của \textbf{Sahil}.}}

\end{document}