\documentclass[../2015-n-s.tex]{subfiles}

\begin{document}

\begin{soln}[\autoref{problem:EGMO-2015-P3}][\gls{EGMO 2015}/P3][\footnotemark]
	
	Giả sử không mất tính tổng quát rằng \(a_1\) là số nhỏ nhất trong các số \(a_i\).
	
	Nếu \(a_1 \ge n^m\), thì tất cả các số \(a_i\) đều bằng nhau,
	hoặc \(a_1 = n^m - 1\) và tồn tại chỉ số \(j\) sao cho \(a_j = n^m\).

	Trong trường hợp đầu tiên, ta có thể chọn (ví dụ) \(b_1 = 1\), \(b_2 = 2\), và các số \(b_i\) còn lại tuỳ ý. Khi đó,
	\[
		\gcd(a_1 + b_1, a_2 + b_2, \ldots, a_m + b_m) \le \gcd(a_1 + b_1, a_2 + b_2) = 1.
	\]
	
	Trong trường hợp thứ hai, ta chọn \(b_1 = 1\), \(b_j = 1\), và các \(b_i\) còn lại tuỳ ý, ta lại có
	\[
		\gcd(a_1 + b_1, a_2 + b_2, \ldots, a_m + b_m) \le \gcd(a_1 + b_1, a_j + b_j) = 1.
	\]
	
	Vì vậy, từ đây trở đi, ta giả sử rằng \(a_1 \le nm - 2\).
	
	Giả sử rằng không tồn tại bộ số \(b_1, \ldots, b_m\) như mong muốn và ta sẽ tìm mâu thuẫn.

	Khi này, với mọi bộ số \(b_1, \ldots, b_m \in \{1, \ldots, n\}\), ta luôn có
	\[
		\gcd(a_1 + b_1, a_2 + b_2, \ldots, a_m + b_m) \ge n.
	\]
	
	Mặt khác, ta cũng có
	\[
		\gcd(a_1 + b_1, a_2 + b_2, \ldots, a_m + b_m) \le a_1 + b_1 \le n^m + n - 2.
	\]
	
	Do đó, chỉ có tối đa \(n^m - 1\) giá trị khả dĩ cho ước chung lớn nhất. Tuy nhiên, có \(n^m\) cách chọn bộ \(m\) số \((b_1, \ldots, b_m)\).
	Khi đó, theo \nameref{theorem:pigeonhole-principle}, tồn tại hai bộ khác nhau cho cùng một giá trị ước chung lớn nhất, gọi là \(d\).
	
	Nhưng vì \(d \ge n\), nên với mỗi \(i\), chỉ có nhiều nhất một giá trị \(b_i \in \{1, 2, \ldots, n\}\) sao cho \(a_i + b_i\) chia hết cho \(d\).
	Do đó, chỉ có nhiều nhất một bộ \(m\)-tuple \((b_1, b_2, \ldots, b_m)\) sao cho ước chung lớn nhất là \(d\). Điều này dẫn đến mâu thuẫn như mong muốn.

	\textbf{Kết luận:} Phản chứng sai, do đó tồn tại bộ \( (b_1, \dots, b_m) \in \{1, \dots, n\}^m \) sao cho
	\[
		\gcd(a_1 + b_1, \dots, a_m + b_m) < n.
	\]

	\vspace{1em}
	\textbf{[\nameref{definition:10M}]}\index{10M}.
	\textbf{Tham chiếu:} \nameref{theorem:proof-by-contradiction}. \nameref{theorem:pigeonhole-principle}. \nameref{theorem:gcd-lcm-properties}.
\end{soln}

\footnotetext{\href{https://www.egmo.org/egmos/egmo4/solutions.pdf}{Lời giải chính thức.}}

\end{document}