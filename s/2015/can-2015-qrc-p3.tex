\documentclass[../2015-n-s.tex]{subfiles}

\begin{document}

\begin{soln}[\autoref{problem:CAN-2015-QRC-P3}][\gls{CAN 2015 QRC}/P3][\footnotemark]

	Giả sử \( abc \) là một số \textit{tầm thường}. Sáu hoán vị gồm \( \{abc, acb, bac, bca, cab, cba\} \). Tổng của chúng là:
	\[
		222(a + b + c) \implies \frac{222(a + b + c)}{6} = 37(a + b + c).
	\]
	
	Vì \( abc \) là \textit{tầm thường}, nên:
	\[
		100a + 10b + c = 37(a + b + c) \implies 63a = 27b + 36c \implies 7a = 3b + 4c.
		\implies 3(a-b) = 4(c-a).
	\]

	Như vậy $3 \mid c-a$, do đó $c \ge a + 3.$ Tương tự $a \ge b + 4$. Nên $c \ge b+7.$
	$c \ge 9$ nên $b \le 2,$ vì $b > 0$ nên $b$ chỉ có thể là $1$ hoặc $2.$ Do đó $abc$ là $518$ hoặc $629.$

	\textbf{Kết luận:} Số \textit{tầm thường} lớn nhất là \( \boxed{629} \).

	\vspace{1em}
	\textbf{[\nameref{definition:0M}]}\index{0M}.
	\textbf{Tham chiếu:} \nameref{lemma:repunit-form} \nameref{theorem:euclidean-division}.
\end{soln}

\footnotetext{\href{https://cms.math.ca/wp-content/uploads/2019/07/2015official_solutions.pdf}{Lời giải chính thức.}}

\end{document}