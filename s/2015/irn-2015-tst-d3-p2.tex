\documentclass[../2015-n-s.tex]{subfiles}

\begin{document}

\begin{soln}[\autoref{problem:IRN-2015-TST-D3-P2}][\gls{IRN 2015 TST}/D3-P2][\footnotemark]

	Ta chứng minh điều mạnh hơn: tồn tại \( k \leq a_3 + 3 \) sao cho \( a_k \leq 0 \).

	\textbf{Bước 1:} Đặt
	\[
		b_n = \frac{a_n}{\text{lcm}(a_{n-2}, a_{n-3})} \quad \text{với mọi } n \geq 5.
	\]
	
	Ta chứng minh rằng \( b_n \in \mathbb{N} \) và giảm dần.

	Với \( n \geq 4 \), \( a_{n-2} \mid a_n \) và \( a_{n-3} \mid a_n \), nên \( \text{lcm}(a_{n-2}, a_{n-3}) \mid a_n \), suy ra \( b_n \in \mathbb{N} \).

	\begin{claim*}
		Với mọi \( n \geq 5 \), ta có \( b_{n+1} < b_n \).
	\end{claim*}
	\begin{subproof}
		Ta có
		\[
			a_{n+1} = \text{lcm}(a_n, a_{n-1}) - \text{lcm}(a_{n-1}, a_{n-2}).
		\]
		
		Thay \( a_n = b_n \cdot \text{lcm}(a_{n-2}, a_{n-3}) \), ta suy ra:
		\[
			a_{n+1} = \text{lcm}(b_n, a_{n-2}, a_{n-3}, a_{n-1}) - \text{lcm}(a_{n-1}, a_{n-2}).
		\]
		
		Vì \( a_{n-3} \mid a_{n-1} \), nên
		\[
			a_{n+1} = \text{lcm}(b_n, a_{n-2}, a_{n-1}) - \text{lcm}(a_{n-1}, a_{n-2}) \implies b_{n+1} < b_n.
		\]
	\end{subproof}

	\textbf{Bước 2:} Ước lượng \( b_5 \).

	Ta có
	\[
		a_4 = \text{lcm}(a_3, a_2) - \text{lcm}(a_2, a_1) = c \cdot a_2,\quad \text{với } c \leq a_3 - 1,
	\]
	và
	\[
		a_5 = \text{lcm}(a_4, a_3) - \text{lcm}(a_3, a_2) = \text{lcm}(c a_2, a_3) - \text{lcm}(a_3, a_2).
	\]
	
	Suy ra:
	\[
		b_5 = \frac{a_5}{\text{lcm}(a_3, a_2)} \leq c - 1 \leq a_3 - 2.
	\]

	\textbf{Kết luận:} Dãy \( b_n \) nguyên, giảm dần, bắt đầu từ \( b_5 \leq a_3 - 2 \), nên sau tối đa \( a_3 - 2 \) bước sẽ đạt giá trị không dương.
	Do đó, tồn tại \( k \leq a_3 + 3 \) sao cho \( a_k = 0 \leq 0 \).

	\vspace{1em}
	\textbf{[\nameref{definition:25M}]}\index{25M}.
	\textbf{Tham chiếu:} \nameref{theorem:gcd-lcm-properties}. \nameref{theorem:extremal-principle}.
\end{soln}

\footnotetext{\href{https://artofproblemsolving.com/community/c6h1100830p25301244}{Lời giải của \textbf{guptaamitu1}.}}

\end{document}