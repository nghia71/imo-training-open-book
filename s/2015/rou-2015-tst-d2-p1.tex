\documentclass[../03-arithmetic-functions.tex]{subfiles}

\begin{document}

\begin{soln}[\autoref{problem:ROU-2015-TST-D2-P1}][\gls{ROU 2015 TST}/D2/P1][\footnotemark]

    Ta chứng minh khẳng định sau:

    \begin{claim*}
        Nếu \( p \) là số nguyên tố và \( \gcd(a,p) = 1 \), thì:
        \[
            a^{p^k} \equiv a^{p^{k-1}} \Mod{p^k}.
        \]
    \end{claim*}

    \begin{subproof}
        Theo định lý Euler, \( a^{\phi(p^k)} \equiv 1 \Mod{p^k} \), với \( \phi(p^k) = p^{k-1}(p - 1) \).  
        Suy ra:
        \[
            a^{p^k} = a^{p \cdot p^{k-1}} \equiv a^{p^{k-1}} \Mod{p^k}.
        \]
    \end{subproof}

    \textit{Trường hợp 1: \( n = p^s \) là lũy thừa của số nguyên tố.}  

    Theo \nameref{theorem:gcd-power-sum}:
    \[
        \sum_{k = 1}^{p^s} a^{\gcd(k, p^s)} = \sum_{d \mid p^s} \varphi(d) \cdot a^{p^s/d}.
    \]

    Các ước \( d \) là \( p^0, p^1, \ldots, p^s \), do đó:
    \[
        S = a^{p^s} + (p - 1) \cdot a^{p^{s - 1}} + (p^2 - p) \cdot a^{p^{s - 2}} + \dots + (p^s - p^{s - 1}) \cdot a.
    \]

    Ta chứng minh \( p^s \mid S \) bằng quy nạp theo \( s \):
    \begin{itemize}[topsep=0pt, itemsep=0pt]
        \item Cơ sở \( s = 1 \): \( a^p + (p - 1)a = pa \equiv 0 \Mod{p} \).
        \item Bước quy nạp: Áp dụng bổ đề trên cho từng số hạng, suy ra \( p^s \mid S \).
    \end{itemize}

    \textit{Trường hợp 2: \( n \) có ít nhất hai thừa số nguyên tố.}  

    Giả sử \( n = p^s m \) với \( \gcd(p, m) = 1 \). Khi đó:
    \[
        \sum_{k=1}^{n} a^{\gcd(k,n)} = \sum_{d \mid n} \varphi(d) \cdot a^{n/d}
        = \sum_{\substack{d \mid n \\ p \nmid d}} \varphi(d) \cdot a^{n/d}
        + \sum_{\substack{d \mid n \\ p \mid d}} \varphi(d) \cdot a^{n/d}.
    \]

    \begin{itemize}[topsep=0pt, itemsep=0pt]
        \item Phần đầu chia hết cho \( m \) nhờ giả thiết quy nạp.
        \item Phần sau chia hết cho \( p^s \) theo kết quả ở Trường hợp 1.
    \end{itemize}

    Vì \( \gcd(m, p^s) = 1 \), theo \nameref{theorem:chinese-remainder-theorem}, tổng chia hết cho \( n = p^s m \).

    \[
        \boxed{n \mid \sum_{k = 1}^{n} a^{\gcd(k,n)}}
    \]

    \vspace{1em}
    \textbf{[\nameref{definition:25M}]}  
    \textbf{Tham chiếu:} \nameref{theorem:gcd-power-sum}. \nameref{theorem:euler}. \nameref{theorem:chinese-remainder-theorem}.
\end{soln}

\footnotetext{\href{https://artofproblemsolving.com/community/c6h1097351p4930928}{Dựa theo lời giải của \textbf{andria}.}}

\end{document}