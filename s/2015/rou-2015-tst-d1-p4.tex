\documentclass[../03-arithmetic-functions.tex]{subfiles}

\begin{document}

\begin{soln}[\autoref{problem:ROU-2015-TST-D1-P4}][\gls{ROU 2015 TST}/D1/P4][\footnotemark]
    
    Đặt \( a = \frac{1 + \sqrt{k}}{2} \), với \( k \equiv 1 \Mod{4} \) và \( k \) không phải là số chính phương. Khi đó
    \[
        a^2 = a + \frac{k - 1}{4} = a + t, \quad \text{với } t \in \mathbb{Z}_{>0}.
    \]
    
    Đặt \( \varepsilon_n = a n - \lfloor a n \rfloor \in (0,1) \). Khi đó
    \[
        a^2 n = a n + t n = \lfloor a n \rfloor + \varepsilon_n + t n.
    \]
    
    Vậy
    \[
        \lfloor a^2 n \rfloor = \lfloor a n \rfloor + t n + \delta_n, \quad \delta_n \in \{0,1\}.
    \]

    Đặt \( m_n = \lfloor a n \rfloor \). Khi đó
    \[
        \lfloor a m_n \rfloor = \lfloor a (\lfloor a n \rfloor) \rfloor.
    \]
    
    Vậy
    \[
        \lfloor a^2 n \rfloor \;-\; \lfloor a \lfloor a n \rfloor \rfloor 
        = t n + \lfloor a n \rfloor - \lfloor a m_n \rfloor + \delta_n.
    \]
    
    Vì \( a \) là vô tỷ, \( \varepsilon_n \) phân bố đều trong khoảng \( (0,1) \), và \((a - 1)\varepsilon_n\) cũng phân bố đều trong \( (0, a - 1) \).  
    Do đó
    \[
        \left\{ 
            \lfloor a^2 n \rfloor - \lfloor a \lfloor a n \rfloor \rfloor 
            \;:\; n \in \mathbb{N}_{>0} 
        \right\}
    \]
    là tập hợp chính xác \(\{1, 2, \ldots, \lfloor a \rfloor\}\), vì mỗi giá trị nguyên như vậy xuất hiện chính xác một lần nhờ vào sự phân bố đồng đều của các phần thập phân.

    \vspace{1em}
    \textbf{[\nameref{definition:25M}]}\index{25M}.
    \textbf{Tham chiếu:} \nameref{lemma:irrational-uniform-distribution}.
\end{soln}

\end{document}