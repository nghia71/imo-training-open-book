\documentclass[../03-arithmetic-functions.tex]{subfiles}

\begin{document}

\begin{soln}[\autoref{problem:ROU-2015-TST-D3-P3}][\gls{ROU 2015 TST}/D3/P3][\footnotemark]

    \textbf{Phần (a):} Giả sử ngược lại rằng \( f(n) \ge 3\sqrt{n} \). Đặt \( s = \lfloor \sqrt{n} \rfloor \), và \( a = s(s - 1) \). Khi đó:
    \[
        a = s(s - 1) < s(s + 1) \le n \implies a < n.
    \]

    Xét tập các số nguyên \( \{a+1, a+2, \dots, n\} \). Tập này chứa cả:
    \[
        s^2 = s \cdot s > s(s - 1) = a \Rightarrow s^2 \in [a+1, n],
    \]
    và
    \[
        (s + 1)(s - 1) = s^2 - 1 \Rightarrow s^2 - 1 \in [a+1, n].
    \]

    Do đó, \( M(n, n - a) = \mathrm{lcm}(n, n-1, \dots, a+1) \) chia hết cho \( s^2 \) và \( s^2 - 1 \), nên cũng chia hết cho \( a = s(s - 1) \).  

    Suy ra:
    \[
        M(n, n - a + 1) = \mathrm{lcm}(M(n, n - a), a) = M(n, n - a),
    \]
    mâu thuẫn với giả thiết rằng chuỗi \( M(n,1) < M(n,2) < \dots \). Vậy:
    \[
        \boxed{f(n) < 3\sqrt{n}}.
    \]

    \medskip

    \textbf{Phần (b):} Gọi \( N \in \mathbb{N}_{>0} \). Ta chứng minh rằng nếu \( n > N! + N \) thì \( f(n) > N \).

    Giả sử tồn tại \( k \le N \) sao cho:
    \[
        M(n, k) = M(n, k + 1).
    \]
    
    Khi đó:
    \[
        M(n, k + 1) = \mathrm{lcm}(M(n, k), n - k) = M(n, k)
        \Rightarrow (n - k) \mid M(n, k).
    \]

    Mà \( M(n, k) \mid n(n-1)\cdots(n - k + 1) \), nên:
    \[
        (n - k) \mid k!.
    \]

    Tuy nhiên, nếu \( n - k > N! \ge k! \), thì mâu thuẫn xảy ra.

    Vậy với mọi \( k \le N \), ta đều có \( M(n, k) < M(n, k + 1) \Rightarrow f(n) > N \).  

    Từ đó suy ra:
    \[
        \boxed{\text{Chỉ có hữu hạn } n \text{ sao cho } f(n) \le N.}
    \]

    \vspace{1em}
    [\textbf{\nameref{definition:30M}}]  
    \textbf{Tham chiếu:} \nameref{definition:tau-function}. \nameref{definition:sigma-function}. \nameref{theorem:basic-divisibility-properties}.
\end{soln}

\footnotetext{\href{https://artofproblemsolving.com/community/c6h1097389p6337541}{Dựa theo lời giải của \textbf{Aiscrim}.}}

\end{document}