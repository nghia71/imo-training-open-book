\documentclass[../2015-n-s.tex]{subfiles}

\begin{document}

\begin{soln}[\autoref{problem:CHN-2015-TST1-D1-P2}][\gls{CHN 2015 TST}1/D1/P2][\footnotemark]

	Giả sử ngược lại: từ một chỉ số \( K \) trở đi, các giá trị \( \operatorname{lcm}(a_k, a_{k+1}) \) đều không vượt quá \( ck \).
	\begin{claim*}
		Với mọi \(k \ge K\), ta có
		\[
			\frac{1}{a_k} + \frac{1}{a_{k+1}} \ge \frac{3}{ck}.
		\]
	\end{claim*}
	\begin{subproof}
		Do \(a_k \ne a_{k+1}\), và vì \(\gcd(a_k, a_{k+1})\) chia hết \(a_k + a_{k+1}\), cho nên
		\[
			a_k + a_{k+1} > 2\min\{a_k, a_{k+1}\} \ge 2\gcd(a_k, a_{k+1}) \implies a_k + a_{k+1} \ge 3\gcd(a_k, a_{k+1}).
		\]
		
		Do đó,
		\[
			\frac{1}{a_k} + \frac{1}{a_{k+1}} = \frac{a_k + a_{k+1}}{\gcd(a_k, a_{k+1}) \cdot \operatorname{lcm}(a_k, a_{k+1})} \ge \frac{3}{\operatorname{lcm}(a_k, a_{k+1})} \ge \frac{3}{ck}.
		\]
	\end{subproof}
	
	Lấy tổng từ \(k = K\) đến \(k = N\), với \(N\) tùy ý, ta có
	\[
		\sum_{i=K}^N\left( \frac{1}{a_i} + \frac{1}{a_{i+1}} \right) \ge \frac{3}{c}\left( \frac{1}{K} + \cdots + \frac{1}{N} \right) \tag{1}
	\]
	
	Tuy nhiên, do các số \(a_K, \ldots, a_{N+1}\) là đôi một phân biệt, nên trong tổng bên trái,
	mỗi số hạng \( \frac{1}{j} \) chỉ xuất hiện tối đa hai lần với mỗi số nguyên dương \(j\); do đó,
	\[
		\sum_{i=K}^N\left( \frac{1}{a_i} + \frac{1}{a_{i+1}} \right) \le \frac{2}{1} + \cdots + \frac{2}{N}  \tag{2}
	\]
	
	Kết hợp (1) và (2), rồi sắp xếp lại, ta được
	\[
		\frac{2}{1} + \cdots + \frac{2}{K} \ge \left( \frac{3}{c} - 2 \right)\left( \frac{1}{K} + \cdots + \frac{1}{N} \right).
	\]
	
	Vì \(3/c > 2\), nên vế trái là cố định còn vế phải lại không bị chặn khi \(N \to \infty\), vô lý.

	\textbf{Kết luận:} tồn tại vô hạn số nguyên dương \( k \) sao cho: $\operatorname{lcm}(a_k, a_{k+1}) > c k.$

	\vspace{1em}
	\textbf{[\nameref{definition:10M}]}\index{0M}.
	\textbf{Tham chiếu:} \nameref{lemma:harmonic-growth}. \nameref{theorem:gcd-lcm-properties}.
\end{soln}

\footnotetext{\href{https://artofproblemsolving.com/community/c6h1062614p18497986}{Lời giải của \textbf{TheUltimate123}.}}

\end{document}