\documentclass[../02-modular-arithmetic-b.tex]{subfiles}

\begin{document}

\begin{exercise*}[\gls{ROU 2014 MO}/G9/P2]\label{example:ROU-2014-MO-G9-P2}[\textbf{\nameref{definition:15M}}]
    Cho \( a \) là một số tự nhiên lẻ không phải là một số chính phương, và \( m, n \in \mathbb{N} \). Khi đó:

    \begin{itemize}[topsep=0pt, partopsep=0pt, itemsep=0pt]
        \item \( \left\{ m\left( a+\sqrt{a} \right) \right\} \neq \left\{ n\left( a-\sqrt{a} \right) \right\} \)
        \item \( \left[ m\left( a+\sqrt{a} \right) \right] \neq \left[ n\left( a-\sqrt{a} \right) \right] \)
    \end{itemize}

    Trong đó, \( \{\cdot\} \) ký hiệu phần thập phân (phần lẻ), và \( [\cdot] \) ký hiệu phần nguyên.
\end{exercise*}

\begin{remark*}
    Vì \( \sqrt{a} \) là số vô tỉ (do \( a \) không phải là số chính phương), nên các biểu thức \( a+\sqrt{a} \) và \( a-\sqrt{a} \) là liên hợp, có tổng nguyên nhưng đối xứng về trục. Tuy nhiên, \( a+\sqrt{a} > a \) và \( a-\sqrt{a} < a \), nên phần thập phân của chúng không thể trùng nhau khi nhân với số nguyên. Đồng thời, hiệu \( m(a+\sqrt{a}) - n(a-\sqrt{a}) \) không thể là số nguyên, vì vậy phần nguyên của chúng cũng khác nhau.
\end{remark*}

\end{document}