\documentclass[../06-largest-exponent.tex]{subfiles}

\begin{document}

\begin{example*}[\gls{CHN 2015 MO}/P4]\label{example:CHN-2015-MO-P4}[\textbf{\nameref{definition:25M}}]
	Xác định tất cả các số nguyên \( k \) sao cho tồn tại vô hạn số nguyên dương \( n \) không thỏa mãn:
	\[
		n + k \mid \binom{2n}{n}.
	\]
\end{example*}

\begin{story*}
	Dùng định lý Kummer về số mũ của một số nguyên tố trong nhị thức, ta xét giá trị \( \nu_2\left( \binom{2n}{n} \right) \).  
	Chọn \( n = 2^\alpha - k \), khi đó \( n + k = 2^\alpha \), có \( \nu_2(n + k) = \alpha \).

	Mặt khác, định lý Kummer nói rằng:
	\[
		\nu_2\left( \binom{2n}{n} \right) = \text{số chữ số “nhớ” khi cộng } n + n \text{ trong hệ nhị phân} < \alpha.
	\]
	Từ đó, với \( k \ne 1 \), ta luôn có thể chọn \( \alpha \) lớn để \( \nu_2(n + k) > \nu_2\left( \binom{2n}{n} \right) \), suy ra không chia hết.

	Trường hợp duy nhất không xảy ra điều này là \( k = 1 \), khi đó biểu thức trở thành \( (n+1) \mid \binom{2n}{n} \), luôn đúng vì là số Catalan.
\end{story*}

\begin{soln}\footnotemark
	Ta xét ba trường hợp:

	\textbf{Trường hợp 1:} \(k = 0\).  
	Chọn \(n = 2^\alpha\), với \( \alpha \ge 2 \). Khi đó:
	\[
		\nu_2(n + k) = \alpha,\quad
		\nu_2\left( \binom{2n}{n} \right) = 1 \quad (\text{do Kummer})
	\Rightarrow n + k \nmid \binom{2n}{n}.
	\]
	Có vô hạn \( \alpha \), nên tồn tại vô hạn \( n \) vi phạm chia hết.

	\textbf{Trường hợp 2:} \(k \ne 0, 1\).  
	Chọn \(n = 2^\alpha - k\) với \( \alpha \ge \log_2(|k|) + 3 \), đủ lớn để:
	\[
		n + k = 2^\alpha \Rightarrow \nu_2(n + k) = \alpha.
	\]
	Dùng định lý Kummer:
	\[
		\nu_2\left( \binom{2n}{n} \right) \le \alpha - 1 \Rightarrow \text{không chia hết}.
	\]
	Vậy có vô hạn \( n \) sao cho \( n + k \nmid \binom{2n}{n} \).

	\textbf{Trường hợp 3:} \(k = 1\).  
	Khi đó:
	\[
		\frac{1}{n + 1} \binom{2n}{n} = \binom{2n}{n} - \binom{2n}{n + 1}
	\]
	là số nguyên, tức \( n + 1 \mid \binom{2n}{n} \) với mọi \( n \). Không tồn tại vô hạn \( n \) vi phạm chia hết.

	\textbf{Kết luận:}
	\[
		\boxed{\text{Tồn tại vô hạn } n \text{ sao cho } n + k \nmid \binom{2n}{n} \iff k \ne 1.}
	\]
\end{soln}

\footnotetext{\href{https://artofproblemsolving.com/community/c6h618236p3687002}{Lời giải của yunxiu.}}

\end{document}