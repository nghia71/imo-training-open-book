\documentclass[../imo-training-open-book.tex]{subfiles}
\begin{document}

\chapter{Khái niệm và Định nghĩa}

\begin{definition*}[Quan hệ thứ tự]
    \label{definition:order-relation}
    Một quan hệ thứ tự trên một tập hợp là một quan hệ \( \leq \) thỏa mãn ba tính chất:
    \begin{enumerate}[topsep=0pt, partopsep=0pt, itemsep=0pt]
        \item \( a \leq a \).
        \item Nếu \( a \leq b \) và \( b \leq a \) thì \( a = b \).
        \item Nếu \( a \leq b \) và \( b \leq c \) thì \( a \leq c \).
    \end{enumerate}
\end{definition*}

\begin{definition*}[Quan hệ thứ tự toàn phần]
    \label{definition:total-order-relation}
    Một quan hệ thứ tự được gọi là \textit{toàn phần} nếu mọi cặp phần tử đều có thể so sánh được, tức là với mọi \( a \) và \( b \),
    ta luôn có hoặc \( a \leq b \) hoặc \( b \leq a \).
\end{definition*}

\begin{definition*}[Dãy số]
    \label{definition:sequence}
    Một \textbf{dãy số} là một hàm được xác định cho mọi số nguyên không âm \( n \), với \( x_n = f(n) \).
    Số hạng thứ \( n \), \( x_n \), thường có thể được tính từ các số hạng trước đó theo một công thức xác đijnh bởi một hàm số $F$:
    \[
        x_n = F(x_{n-1}, x_{n-2}, \ldots)
    \]
\end{definition*}

\begin{definition*}[Dãy đơn điệu]
    \label{definition:monotonic-sequence}
    Một dãy số \( \{ a_n \}_{n=1}^\infty \) được gọi là:
    \begin{enumerate}[topsep=0pt, partopsep=0pt, itemsep=0pt]
        \item \textbf{tăng đơn điệu} nếu \( a_k \leq a_{k+1} \) với mọi \( k \geq 1 \).
        \item \textbf{giảm đơn điệu} nếu \( a_k \geq a_{k+1} \) với mọi \( k \geq 1 \).
        \item \textbf{tăng nghiêm ngặt} nếu \( a_k < a_{k+1} \) với mọi \( k \geq 1 \).
        \item \textbf{giảm nghiêm ngặt} nếu \( a_k > a_{k+1} \) với mọi \( k \geq 1 \).
    \end{enumerate}
    Trong bất kỳ trường hợp nào ở trên, dãy số \( \{ a_n \} \) được gọi là \textbf{đơn điệu}.
\end{definition*}

\begin{definition*}[\href{https://en.wikipedia.org/wiki/P-adic_valuation}{Chuẩn $p$-adic}] 
    \label{theorem:p-adic-valuation}
    Trong số học, chuẩn \( p \)-adic (hoặc bậc \( p \)-adic) của một số nguyên \( n \) là số mũ của lũy thừa lớn nhất của số nguyên tố \( p \) mà \( n \) chia hết.
    \[
        \nu_p(n) = 
        \begin{cases}
            \max \{ k \in \mathbb{N}_0 : p^k \mid n \} & \text{if } n \neq 0, \\
            \infty & \text{if } n = 0,
        \end{cases}
    \]

    Nói một cách khác, chuẩn \( p \)-adic là số mũ của \( p \) trong phân tích thừa số nguyên tố của \( n \):
    \[
        n={p_{1}}^{\alpha _{1}}{p_{2}}^{\alpha _{2}}{\dots }{p_{k}}^{\alpha _{k}} \implies \nu_{p_i}(n) = \alpha_i,\ \forall\ i=1,2,\ldots,k.
    \]
    trong đó $p_{1},p_{2},\ldots,p_{k}$ là các số nguyên tố và $\alpha _{1},\alpha _{2},\dots ,\alpha _{k}$ là các số nguyên dương.
\end{definition*}

\begin{definition*}[Hàm số tự nghịch đảo]
    \label{definition:involution-function}
    Trong toán học, một \textbf{hàm số tự nghịch đảo} là một hàm số \( f \) sao cho:
    \[
        f(f(x)) = x
    \]
    với mọi \( x \) thuộc tập xác định của \( f \).  
    Nói cách khác, \( f \) là hàm ngược của chính nó.
\end{definition*}

\newpage

\chapter{Định lý và bổ đề}

\begin{theorem*}[Nguyên lý Chuồng Bồ Câu]
    \label{theorem:pigeonhole-principle}
    Nguyên lý Chuồng Bồ Câu (còn được gọi là nguyên lý hộp Dirichlet, nguyên lý Dirichlet hoặc nguyên lý hộp)
    phát biểu rằng nếu có nhiều hơn \( n \) con bồ câu được đặt vào \( n \) chuồng, thì ít nhất một chuồng phải chứa hai con bồ câu trở lên.
    
    Một cách phát biểu khác là: trong một tập hợp gồm \( n \) số nguyên bất kỳ, luôn tồn tại hai số có cùng phần dư khi chia cho \( n-1 \).

    Phiên bản mở rộng của nguyên lý Chuồng Bồ Câu phát biểu rằng nếu \( k \) đối tượng được đặt vào \( n \) hộp,
    thì ít nhất một hộp phải chứa tối thiểu \( \left\lceil \frac{k}{n} \right\rceil \) đối tượng.
    Ở đây, \( \lceil \cdot \rceil \) ký hiệu cho hàm trần.
\end{theorem*}

\begin{theorem*}[Nguyên lý Bất biến]
    \label{theorem:invariant-principle}
    Xét một tập hợp trạng thái \( S = (s_1, s_2, \ldots, s_n) \) và một tập hợp các phép chuyển đổi \( T \subseteq S \times S \).
    Một \textbf{bất biến} đối với \( T \) là một hàm \( f:\ S \mapsto \RR \)
    sao cho nếu \( (s_i, s_j) \in T \) thì \( f(s_i) = f(s_j) \).
\end{theorem*}

\begin{theorem*}[Nguyên lý Quy nạp]
    \label{theorem:induction-principle}
    Cho \( a \) là một số nguyên, và cho \( P(n) \) là một mệnh đề (hoặc phát biểu) về \( n \) với mọi số nguyên \( n \geq a \).
    \textbf{Nguyên lý quy nạp} là một phương pháp chứng minh rằng \( P(n) \) đúng với mọi số nguyên \( n \geq a \) thông qua hai bước:
    \begin{enumerate}[topsep=0pt, partopsep=0pt, itemsep=0pt]
        \item \textit{Cơ sở quy nạp:} Chứng minh rằng \( P(a) \) đúng.
        \item \textit{Bước quy nạp:} Giả sử rằng \( P(k) \) đúng với một số nguyên \( k \geq a \),
        và sử dụng giả thiết này để chứng minh rằng \( P(k+1) \) cũng đúng.
    \end{enumerate}
    Khi đó, ta có thể kết luận rằng \( P(n) \) đúng với mọi số nguyên \( n \geq a \).
\end{theorem*}

\begin{theorem*}[Nguyên lý sắp thứ tự tốt trên tập hữu hạn]
    \label{theorem:well-ordering-principle-on-finite-set}
    \textit{Nguyên lý sắp thứ tự tốt} phát biểu rằng mọi tập hợp có thứ tự toàn phần,
    hữu hạn và khác rỗng đều chứa \textit{một phần tử lớn nhất} và \textit{một phần tử nhỏ nhất}.
\end{theorem*}

\begin{theorem*}[Nguyên lý sắp thứ tự tốt trên tập hợp số nguyên dương]
    \label{theorem:well-ordering-principle}
    \textit{Nguyên lý sắp thứ tự tốt} phát biểu rằng mọi tập hợp khác rỗng của các số nguyên dương đều chứa \textit{một phần tử nhỏ nhất}.
\end{theorem*}

\begin{theorem*}[Nguyên lý cực hạn]
    \label{theorem:extremal-principle}
    Chúng ta sử dụng thuật ngữ \textit{Nguyên lý cực hạn} để chỉ sự tồn tại của một phần tử nhỏ nhất hoặc lớn nhất trong một tập hợp.

    Chúng ta thường kết hợp
    \begin{enumerate}[topsep=0pt, partopsep=0pt, itemsep=0pt]
        \item \textit{Nguyên lý cực hạn} với \textit{Chứng minh phản chứng} để chứng minh sự tồn tại hoặc không tồn tại của một đối tượng, hoặc
        \item \textit{Nguyên lý cực hạn} với \textit{Nguyên lý Chuồng Bồ Câu} để thiết lập điều kiện tồn tại hoặc để tăng hoặc giảm một đại lượng, ngoài ra
        \item \textit{Nguyên lý cực hạn} có thể được kết hợp với \textit{Nguyên lý Quy nạp} để đơn giản hóa một chứng minh.  
    \end{enumerate}
\end{theorem*}

\begin{theorem}[Bất đẳng thức Chuỗi Xen Kẽ]
    \label{theorem:alternating-series-inequality}
    Cho một dãy số giảm dần \( a_n \) với \( a_n > 0 \) và giới hạn \(\lim_{n \to \infty} a_n = 0 \), chuỗi xen kẽ:
    \[
        S = a_1 - a_2 + a_3 - a_4 + \dots
    \]
    hội tụ và sai số của tổng vô hạn được ước lượng bởi:
    \[
        \left| S - \sum_{k=1}^{N} (-1)^{k+1} a_k \right| \leq a_{N+1}.
    \]
\end{theorem}

\newpage

\chapter{Hằng đẳng thức}

\newpage

\end{document}

