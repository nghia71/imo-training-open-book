\documentclass[../imo-training-open-book.tex]{subfiles}
\begin{document}

\begin{theorem*}[\href{https://w.wiki/_pZXo}{Định lý cơ bản của số học}] 
    \label{theorem:Fundamental Theorem of Arithmetic}
    Mọi số tự nhiên lớn hơn 1 có thể viết một cách duy nhất (không kể sự sai khác về thứ tự các thừa số) thành tích các thừa số nguyên tố.
    
    Mọi số tự nhiên $n$ lớn hơn 1, có thể viết duy nhất dưới dạng:
    \[
        n={p_{1}}^{\alpha _{1}}{p_{2}}^{\alpha _{2}}{\dots }{p_{k}}^{\alpha _{k}}
    \]
    trong đó $p_{1},p_{2},\ldots,p_{k}$ là các số nguyên tố và $\alpha _{1},\alpha _{2},\dots ,\alpha _{k}$ là các số nguyên dương.
\end{theorem*}

\begin{theorem*}[\href{https://w.wiki/_pZXn}{Quy nạp toán học}] 
    \label{theorem:Induction Principle}
    Hình thức đơn giản và phổ biến nhất của phương pháp quy nạp toán học suy luận rằng một mệnh đề liên quan đến một số tự nhiên $n$ cũng đúng với tất cả các giá trị của $n$.
    Cách chứng minh bao gồm hai bước sau:

    Bước cơ sở: chứng minh rằng mệnh đề đúng với số tự nhiên đầu tiên $n$. Thông thường, $n = 0$ hoặc $n = 1$, hiếm khi có $n = -1$
    (mặc dù không phải là một số tự nhiên, phần mở rộng của các số tự nhiên đến $-1$ vẫn áp dụng được)
    
    Bước quy nạp: chứng minh rằng, nếu mệnh đề được dùng cho một số số tự nhiên $n$, sau đó cũng đúng với $n + 1$.
    Giả thiết ở bước quy nạp rằng mệnh đề đúng với các số n được gọi là giả thiết quy nạp.
    Để thực hiện bước quy nạp, phải giả sử giả thiết quy nạp là đúng và sau đó sử dụng giả thiết này để chứng minh mệnh đề với $n + 1$.
    
    Việc $n = 0$ hay $n = 1$ phụ thuộc vào định nghĩa của số tự nhiên. Nếu $0$ được coi là một số tự nhiên, bước cơ sở được đưa ra bởi $n = 0$.
    Nếu, mặt khác, $1$ được xem như là số tự nhiên đầu tiên, bước hợp cơ sở được đưa ra với $n = 1$.
\end{theorem*}

\begin{definition*}[\href{https://en.wikipedia.org/wiki/P-adic_valuation}{Chuẩn p-adic}] 
    \label{theorem:p-adic valuation}
    Trong số học, chuẩn \( p \)-adic (hoặc bậc \( p \)-adic) của một số nguyên \( n \) là số mũ của lũy thừa lớn nhất của số nguyên tố \( p \) mà \( n \) chia hết.
    Nó được ký hiệu là \( \nu_p(n) \). Tương đương, \( \nu_p(n) \) là số mũ của \( p \) trong phân tích thừa số nguyên tố của \( n \).
    \[
        \nu_p(n) = 
        \begin{cases}
            \max \{ k \in \mathbb{N}_0 : p^k \mid n \} & \text{if } n \neq 0, \\
            \infty & \text{if } n = 0,
        \end{cases}
    \]
\end{definition*}

\end{document}

