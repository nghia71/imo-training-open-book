\documentclass[../01-divisibility.tex]{subfiles}

\begin{document}

\begin{example*}[\gls{RUS 2015 TST}/D9/P1]\label{example:RUS-2015-TST-D9-P1}[\textbf{unrated}]
	Tìm tất cả các cặp số tự nhiên $(a,b)$ sao cho:
	\begin{itemize}[topsep=0pt, partopsep=0pt, itemsep=0pt]
		\item $b - 1$ chia hết cho $a + 1$, và
		\item $a^2 + a + 2$ chia hết cho $b$.
	\end{itemize}
\end{example*}

\begin{soln}(Cách 1)\footnotemark
	Đặt $c = a + 1 \ge 2$, khi đó các điều kiện tương đương:
	\begin{itemize}[topsep=0pt, partopsep=0pt, itemsep=0pt]
		\item $c \mid b - 1$,
		\item $b \mid c^2 - c + 2$.
	\end{itemize}

	Điều này cho thấy $b$ và $c$ là hai số nguyên tố cùng nhau và đều chia hết cho biểu thức
	\[
		c^2 - c + 2 - 2b \implies bc \mid c^2 - c + 2 - 2b.
	\]
	
	Dễ thấy rằng $\boxed{(a, 1)}$ là nghiệm với mọi $a \in \mathbb{N}$. Bây giờ ta xét trường hợp $b \ge 2$. Khi đó:
	\[
		c < b \implies bc > b^2 > c^2 \implies c^2 - c + 2 - 2b < bc.
	\]

	Đồng thời:
	\[
		c^2 - c + 2 - 2b > -2b \ge -bc \implies -bc < c^2 - c + 2 - 2b < bc \implies \text{giá trị này phải bằng 0}.
	\]

	Khi đó:
	\[
		c^2 - c + 2 = 2b \implies a^2 + a + 2 = 2b.
	\]

	Vậy điều kiện thứ hai trở thành:
	\[
		b \mid 2b \implies \text{luôn đúng}.
	\]

	Kiểm tra lại điều kiện thứ nhất:
	\[
		b = \frac{a^2 + a + 2}{2} \implies b - 1 = \frac{a^2 + a}{2} + 1.
	\]

	Ta cần:
	\[
		a + 1 \mid \frac{a^2 + a}{2} + 1.
	\]

	Biểu thức này chỉ là một số nguyên khi $a$ chẵn. Đặt $a = 2k$ thì:
	\[
		b = \frac{(2k)^2 + 2k + 2}{2} = \frac{4k^2 + 2k + 2}{2} = 2k^2 + k + 1.
	\]

	Do đó, thu được họ nghiệm thứ hai:
	\[
		\boxed{(a, b) = (2k, 2k^2 + k + 1), \quad k \in \mathbb{N}}.
	\]
\end{soln}

\footnotetext{\href{https://artofproblemsolving.com/community/c6h3057473p27559072}{Dựa theo lời giải của Tintam.}}

\begin{soln}(Cách 2)\footnotemark
	Đặt $t = a + 1$ ($\implies a = t - 1$). Điều kiện thứ nhất trở thành:
	\[
		t \mid b - 1 \implies b = tk + 1 \text{ với } k \in \mathbb{N}.
	\]

	Điều kiện thứ hai trở thành:
	\[
		b \mid a^2 + a + 2 = (t - 1)^2 + (t - 1) + 2 = t^2 - t + 2.
	\]
	
	Do đó:
	\[
		tk + 1 \mid t^2 - t + 2.
	\]
	
	Vì $tk + 1 \mid t^2 - t + 2$, và $tk + 1 \mid 2tk + 2$, suy ra:
	\[
		tk + 1 \mid (t^2 - t + 2) - (2tk + 2) = t^2 - t - 2tk = t(t - 2k - 1).
	\]
	
	Mặt khác, vì $\gcd(t, tk + 1) = 1$, nên:
	\[
		tk + 1 \mid t - 2k - 1.
	\]

	\textit{Trường hợp 1:} $k = 0$.
	Khi đó:
	\[
		b = 1, \quad a^2 + a + 2 \mid 1 \implies 1 \mid a^2 + a + 2 \text{ luôn đúng}.
	\]

	Vậy mọi $a \in \mathbb{N}$ đều thỏa mãn. Tập nghiệm:
	\[
		\boxed{(a, 1) \text{ với } a \in \mathbb{N}}.
	\]

	\textit{Trường hợp 2:} $k \ge 1$.
	Từ điều kiện $tk + 1 \mid t - 2k - 1$, giả sử $t > 2k + 1$. Khi đó:
	\[
		t - 2k - 1 \ge tk + 1 \implies t \ge tk + 2k + 2 > t,\ \text{mâu thuẫn.}
	\]
	
	Nếu $t < 2k + 1$, thì:
	\[
		2k + 1 \ge tk + t + 1 \implies t \le 1,\ \text{điều này vô lý vì}\ t = a + 1 \ge 2.
	\]
	
	Do đó, chỉ còn lại:
	\[
		t = 2k + 1 \implies a = t - 1 = 2k, \quad b = tk + 1 = (2k + 1)k + 1 = 2k^2 + 2k + 1.
	\]

	Vậy ta thu được họ nghiệm thứ hai:
	\[
		\boxed{(a, b) = (2k, 2k^2 + 2k + 1), \quad k \in \mathbb{N}}.
	\]
\end{soln}

\footnotetext{\href{https://artofproblemsolving.com/community/c6h3057473p27562946}{Dựa theo lời giải của grupyorum.}}

\end{document}