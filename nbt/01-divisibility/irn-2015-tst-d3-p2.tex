\documentclass[../01-divisibility.tex]{subfiles}

\begin{document}

\begin{example*}[\gls{IRN 2015 TST}/D3-P2]\label{example:IRN-2015-TST-D3-P2}\textbf{[\nameref{definition:30M}]}
	Giả sử \( a_1, a_2, a_3 \) là ba số nguyên dương cho trước. Xét dãy số được xác định bởi công thức:
	\[
		a_{n+1} = \text{lcm}[a_n, a_{n-1}] - \text{lcm}[a_{n-1}, a_{n-2}] \quad \text{với } n \geq 3
	\]
	(Ở đây, \( [a, b] \) ký hiệu bội chung nhỏ nhất của \( a \) và \( b \), và chỉ được áp dụng với các số nguyên dương.)

	Chứng minh rằng tồn tại một số nguyên dương \( k \leq a_3 + 4 \) sao cho \( a_k \leq 0 \).
\end{example*}

\begin{story*}
	Bài toán yêu cầu chứng minh rằng chuỗi được xây dựng theo công thức liên quan đến LCM sẽ đạt giá trị không dương trong thời gian hữu hạn.

	Chiến lược:
	\begin{itemize}[topsep=0pt, partopsep=0pt, itemsep=0pt]
		\item Xây dựng chuỗi phụ \( b_n = \dfrac{a_n}{\text{lcm}(a_{n-2}, a_{n-3})} \) để theo dõi sự suy giảm.
		\item Chứng minh \( b_n \) nguyên và giảm dần bằng cách phân tích dạng của \( a_{n+1} \).
		\item Dùng điểm dừng của chuỗi \( b_n \) để suy ra điểm dừng của \( a_n \).
	\end{itemize}
\end{story*}

\bigbreak

\begin{soln}\footnotemark
	Ta chứng minh điều mạnh hơn: tồn tại \( k \leq a_3 + 3 \) sao cho \( a_k \leq 0 \).

	\textbf{Bước 1:} Đặt
	\[
		b_n = \frac{a_n}{\text{lcm}(a_{n-2}, a_{n-3})} \quad \text{với mọi } n \geq 5
	\]
	Ta sẽ chứng minh rằng \( b_n \) là dãy số nguyên và giảm dần.

	Dễ thấy với mọi \( n \geq 4 \), \( a_{n-2} \mid a_n \implies a_{n-3} \mid a_n \), nên \( b_n \in \mathbb{N} \).

	\begin{claim*}
		Với mọi \( n \geq 5 \), ta có:
		\[
			b_{n+1} < b_n
		\]
	\end{claim*}
	\begin{subproof}
		Xét \( a_{n+1} = \text{lcm}(a_n, a_{n-1}) - \text{lcm}(a_{n-1}, a_{n-2}) \). Ta thay \( a_n = b_n \cdot \text{lcm}(a_{n-2}, a_{n-3}) \), rồi khai triển:
		\[
			a_{n+1} = \text{lcm}(b_n, a_{n-2}, a_{n-3}, a_{n-1}) - \text{lcm}(a_{n-1}, a_{n-2})
		\]

		Vì \( a_{n-3} \mid a_{n-1} \), nên:
		\[
			a_{n+1} = \text{lcm}(b_n, a_{n-2}, a_{n-1}) - \text{lcm}(a_{n-1}, a_{n-2}) \implies b_{n+1} < b_n
		\]
	\end{subproof}

	\textbf{Bước 2:} Tính giới hạn cho \( b_5 \). Ta có:
	\[
		a_4 = \text{lcm}(a_3, a_2) - \text{lcm}(a_2, a_1) = c \cdot a_2 \quad \text{với } c \leq a_3 - 1
	\]
	\[
		a_5 = \text{lcm}(a_4, a_3) - \text{lcm}(a_3, a_2) = \text{lcm}(c a_2, a_3) - \text{lcm}(a_3, a_2)
	\implies b_5 = \frac{a_5}{\text{lcm}(a_3, a_2)} \leq c - 1 \leq a_3 - 2
	\]

	\textbf{Kết luận:} Dãy \( b_n \) nguyên, giảm dần, bắt đầu từ \( b_5 \leq a_3 - 2 \), nên sẽ đạt giá trị không vượt quá 0 trong tối đa \( a_3 - 2 \) bước. Suy ra tồn tại \( k \leq a_3 + 3 \) sao cho \( a_k = 0 \leq 0 \).

	\textit{Ví dụ:} Với \( a_1 = 1,\ a_2 = 2,\ a_3 = 3 \), ta có:
	\[
		a_4 = 6 - 2 = 4,\quad a_5 = 12 - 6 = 6,\quad a_6 = 12 - 12 = 0
	\]
	suy ra \( k = 6 = a_3 + 3 \) là giá trị nhỏ nhất.
\end{soln}

\footnotetext{\href{https://artofproblemsolving.com/community/c6h1100830p25301244}{Lời giải của \textbf{guptaamitu1}.}}

\end{document}