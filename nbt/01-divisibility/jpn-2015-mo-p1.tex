\documentclass[../01-divisibility.tex]{subfiles}

\begin{document}

\begin{exercise*}[\gls{JPN 2015 MO}1/P1]\label{example:JPN-2015-MO-P1}\textbf{[\nameref{definition:15M}]}
	Tìm tất cả các số nguyên dương \( n \) sao cho 
	\[
		\frac{10^n}{n^3 + n^2 + n + 1}
	\]
	là một số nguyên.
\end{exercise*}

\begin{remark*}
    Hãy phân tích mẫu số \( n^3 + n^2 + n + 1 \) thành nhân tử, sau đó kiểm tra xem khi nào nó chia hết \( 10^n \). Có thể thử với các giá trị nhỏ của \( n \).
\end{remark*}

\begin{story*}
    Bài toán yêu cầu tìm tất cả \( n \) sao cho một biểu thức dạng \( \dfrac{10^n}{f(n)} \) là số nguyên.
    Ta nên bắt đầu bằng:
    \begin{itemize}[topsep=0pt, partopsep=0pt, itemsep=0pt]
        \item Phân tích \( n^3 + n^2 + n + 1 = (n + 1)(n^2 + 1) \).
        \item Quan sát xem khi nào \( (n + 1)(n^2 + 1) \mid 10^n \), tức là mẫu số phải là ước của \( 10^n \).
        \item Dùng thử giá trị nhỏ của \( n \), kết hợp ước lượng cấp số và xét tính chia hết để loại trừ hoặc chứng minh.
    \end{itemize}
    Đây là một bài toán điển hình về khử nghiệm và kiểm tra giới hạn tăng trưởng.
\end{story*}

\end{document}