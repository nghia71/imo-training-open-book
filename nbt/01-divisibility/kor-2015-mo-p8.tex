\documentclass[../01-divisibility.tex]{subfiles}

\begin{document}

\begin{example*}[\gls{KOR 2015 MO}/P8]\label{example:KOR-2015-MO-P8}[\textbf{unrated}]
	Cho \( n \) là một số nguyên dương. Các số \( a_1, a_2, \dots, a_k \) là các số nguyên dương không lặp lại, không lớn hơn \( n \),
	và nguyên tố cùng nhau với \( n \). Nếu \( k > 8 \), hãy chứng minh rằng:
	\[
		\sum_{i=1}^k \left| a_i - \frac{n}{2} \right| < \frac{n(k - 4)}{2}.
	\]
\end{example*}

\begin{soln}(Cách 1)\footnotemark
	Với các giá trị đặc biệt như \( n = p \) nguyên tố, \( n = p^2 \), hoặc \( n = pq \) với \( p, q \) là các số nguyên tố, ta có thể kiểm tra trực tiếp.
	
	Vì vậy, ta giả sử từ đây rằng \( n \) không thuộc các dạng này. $(\bigstar)$

	Với mỗi số \( a \) sao cho \( \gcd(a, n) = 1 \), thì \( \gcd(n - a, n) = 1 \). Trong cặp \( (a, n - a) \), một số nhỏ hơn \( \frac{n}{2} \) và số còn lại lớn hơn.
	Giả sử \( a < \frac{n}{2} \), ta có:
	\[
		\left| \frac{n}{2} - a \right| + \left| \frac{n}{2} - (n - a) \right| = \left( \frac{n}{2} - a \right) + \left( a - \frac{n}{2} \right) = n - 2a.
	\]

	Vậy nên:
	\[
		\sum_{i=1}^k \left| a_i - \frac{n}{2} \right| = \frac{n\varphi(n)}{2} - 2S, \tag{1}
	\]
	trong đó \( S \) là tổng các số nguyên dương nhỏ hơn \( \frac{n}{2} \) và nguyên tố cùng nhau với \( n \).

	Gọi \( p \) là ước nguyên tố nhỏ nhất của \( n \), và đặt \( m = \frac{n}{p} \).
	Gọi \( T \) là tổng các số nguyên dương nhỏ hơn \( m \) và nguyên tố cùng nhau với \( m \). Khi đó:
	\[
		T = \sum_{\substack{1 \le a < m \\ \gcd(a, m) = 1}} a = \frac{m\varphi(m)}{2}.
	\]

	Theo giả thiết \( (\bigstar) \), nên \( \varphi(m) \ge 2p \). Từ đó, ta suy ra:
	\[
		S \ge T = \frac{m\varphi(m)}{2} \ge p m = n, \tag{2}
	\]


	Thay (2) vào (1), ta được:
	\[
		\sum_{i=1}^k \left| a_i - \frac{n}{2} \right| \le \frac{n\varphi(n)}{2} - 2n = \frac{n(\varphi(n) - 4)}{2}.
	\]
	
	Vì \( \varphi(n) = k > 8 \), ta có:
	\[
		\frac{n(k - 4)}{2} > \sum_{i=1}^k \left| a_i - \frac{n}{2} \right|,
	\]
\end{soln}

\footnotetext{\href{https://artofproblemsolving.com/community/c6h1158071p5504225}{Lời giải của andria.}}


\begin{soln}(Cách 2)\footnotemark
	Ta dễ dàng nhận thấy rằng nếu \( a_i \) là một phần tử, thì \( n - a_i \) cũng thuộc tập và tổng của hai số này là \( n \). Do đó $k$ chẵn và
	\[
		a_{k/2} < \frac{n}{2} < a_{k/2+1}\quad \text{và}\quad \sum_{i=1}^k a_i = \frac{nk}{2}. \tag{1}
	\]
	
	Ta có:
	\[
		\sum_{i=1}^{k} \left| a_i - \frac{n}{2} \right| = \sum_{i=1}^{k/2} \left(\frac{n}{2} - a_i\right) + \sum_{i=k/2+1}^{k} \left(a_i - \frac{n}{2}\right).
	\]

	Kết hợp với (1), bất đẳng thức ban đầu tương đương với:
	\[
		S = \sum_{i=1}^{k/2} a_i > n.
	\]
	
	Ta cần chứng minh tổng các số nhỏ hơn \( \frac{n}{2} \) và nguyên tố cùng nhau với \( n \) phải lớn hơn \( n \).
	
	Gọi \( f(n) \) là một số nguyên gần nhất với \( \frac{n}{3} \), nhưng không chia hết cho 3:
	\[
		f(n) = 
		\begin{cases}
			\frac{n \pm 1}{3}, & n \equiv \mp 1 \Mod{3} \\
			\frac{n}{3} \pm 1, & n \equiv 0 \Mod{3}
		\end{cases}
		\implies 
		\begin{cases}
			&\frac{n}{3} - 1 \le f(n) \le \frac{n}{3} + 1,\ \text{và}\\
			&\gcd(f(n), n) = 1.
		\end{cases}
	\]
	
	\textit{Trường hợp 1: \( n \) lẻ.} Với \( n < 30 \), ta kiểm tra bằng tính toán trực tiếp.
	Giả sử \( n > 30 \). Đặt \( n = 2^k + m \), với \( 1 \le m \le 2^k - 1 \). Khi đó:
	\[
		2^{k-2} < f(n) < 2^k, \quad f(n) < \frac{n - 1}{2} \text{ với } n > 9.
	\]
	
	Suy ra:
	\[
		S \ge 1 + 2 + \cdots + 2^{k-2} + f(n) + \frac{n - 1}{2} \ge 2^{k-1} - 1 + \frac{n}{3} - 1 + \frac{n - 1}{2} = \frac{13}{12}n - \frac{5}{2} > n.
	\]
	
	\textit{Trường hợp 2: \( n \equiv 0 \mod 4 \).}
	Nếu \( \gcd(x,n) = 1 \), thì \( \gcd\left(\frac{n}{2} - x, n\right) \). Do đó:
	\[
		S = \frac{1}{2} \cdot \frac{n}{2} \cdot \frac{k}{2} = \frac{nk}{8} > n,\ \text{vì } k > 8.
	\]
	
	\textit{Trường hợp 3: \( n = 4m + 2 \).} Với \( m \le 7 \), ta kiểm tra bằng tính toán trực tiếp.
	Với \( m > 7 \), ta có:
	\[
		\gcd(2m - 1, 4m + 2) = 1, \quad m + 1 \le f(n) \le 2m - 1.
	\]

	Một trong hai số \( m \) hoặc \( m + 1 \) là lẻ, và nguyên tố cùng nhau với \( n \). Gọi số đó là \( t \). Khi đó:
	\[
		S \ge 1 + t + f(n) + 2m - 1 \ge 1 + m + \frac{4m + 2}{3} - 1 + 2m - 1 = \frac{13}{3}m - \frac{1}{3} > 4m + 2 = n.
	\]
	
	\paragraph{Kết luận.} Sau khi xét đầy đủ các trường hợp, ta có:
	\[
		\sum_{i=1}^{k/2} a_i > n \implies \sum_{i=1}^k \left| a_i - \frac{n}{2} \right| < \frac{n(k - 4)}{2}.
	\]
\end{soln}

\footnotetext{\href{https://artofproblemsolving.com/community/c6h1158071p5909876}{Lời giải của rkm0959.}}

\end{document}