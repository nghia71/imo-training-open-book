\documentclass[../01-divisibility.tex]{subfiles}

\begin{document}

\begin{example*}[\gls{IND 2015 MO}/P2]\label{example:IND-2015-N2}\textbf{[\nameref{definition:10M}]}
	Với mọi số tự nhiên \( n > 1 \), viết phân số \( \frac{1}{n} \) dưới dạng thập phân vô hạn (không viết dạng rút gọn hữu hạn,
	ví dụ: \( \frac{1}{2} = 0.4\overline{9} \), chứ không phải \( 0.5 \)).
	Hãy xác định độ dài phần \textbf{không tuần hoàn} trong biểu diễn thập phân vô hạn của \( \frac{1}{n} \).
\end{example*}

\begin{story*}
	Biểu diễn thập phân vô hạn tuần hoàn của \( \frac{1}{n} \) có thể chia làm hai phần: phần không tuần hoàn và phần tuần hoàn.
	Phần không tuần hoàn xuất hiện do các thừa số nguyên tố 2 và 5 trong mẫu số.

	Cách tiếp cận là:
	\begin{itemize}[topsep=0pt, partopsep=0pt, itemsep=0pt]
	    \item Phân tích \( n = 2^a \cdot 5^b \cdot q \), với \( \gcd(q, 10) = 1 \).
	    \item Phần không tuần hoàn tương ứng với số lượng chữ số cần để \( 10^k \) chia hết cho \( 2^a 5^b \).
	    \item Phần tuần hoàn đến từ phần còn lại \( q \), vốn sinh ra chu kỳ tuần hoàn thực sự.
	\end{itemize}
\end{story*}

\bigbreak

\begin{soln}\footnotemark
	Gọi biểu diễn thập phân của \( \frac{1}{n} \) là:
	\[
		\frac{1}{n} = 0.a_1a_2 \cdots a_{x_n} \overline{b_1b_2 \cdots b_{\ell_n}}
	\]
	trong đó \( x_n \): độ dài phần không tuần hoàn, \( \ell_n \): độ dài phần tuần hoàn.
	
	Ta có:
	\[
		\frac{10^{x_n + \ell_n} - 10^{x_n}}{n} \in \mathbb{Z}^+
		\implies n \mid \left(10^{x_n + \ell_n} - 10^{x_n}\right)
		\implies n \mid 10^{x_n}(10^{\ell_n} - 1)
	\]
	
	Giả sử \( n = 2^a \cdot 5^b \cdot q \), với \( \gcd(q,10)=1 \).

	Để \( \frac{1}{n} \) có phần không tuần hoàn dài \( x_n \), thì:
	\[
		2^a 5^b \mid 10^{x_n}
		\implies x_n \text{ là số nhỏ nhất sao cho } 2^a 5^b \mid 10^{x_n}
		\implies x_n = \max(a, b)
	\]

	Vì \( 10^{x_n} \) chia hết cho cả \( 2^a \) và \( 5^b \) khi \( x_n \geq \max(a, b) \), và đây là giá trị nhỏ nhất như vậy.

	\textbf{Kết luận:} Độ dài phần không tuần hoàn trong biểu diễn thập phân vô hạn của \( \frac{1}{n} \) chính là:
	\[
		x_n = \max(a, b)\quad \text{với } n = 2^a \cdot 5^b \cdot q,\ \gcd(q, 10) = 1
	\]	
\end{soln}

\footnotetext{\href{https://artofproblemsolving.com/community/c6h623454p3730817}{Lời giải của \textbf{utkarshgupta}.}}

\end{document}