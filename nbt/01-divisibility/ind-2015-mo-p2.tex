\documentclass[../01-divisibility.tex]{subfiles}

\begin{document}

\begin{example*}[\gls{IND 2015 MO}/P2]\label{example:IND-2015-N2}[\textbf{unrated}]
	Với mọi số tự nhiên \( n > 1 \), viết phân số \( \frac{1}{n} \) dưới dạng thập phân vô hạn (không viết dạng rút gọn hữu hạn,
	ví dụ: \( \frac{1}{2} = 0.4\overline{9} \), chứ không phải \( 0.5 \)).
	Hãy xác định độ dài phần \textbf{không tuần hoàn} trong biểu diễn thập phân vô hạn của \( \frac{1}{n} \).
\end{example*}

\begin{soln}(Cách 1)\footnotemark
	Gọi biểu diễn thập phân của \( \frac{1}{n} \) là:
	\[
		\frac{1}{n} = 0.a_1a_2 \cdots a_{x_n} \overline{b_1b_2 \cdots b_{\ell_n}}
	\]
	trong đó \( x_n \): độ dài phần không tuần hoàn, \( \ell_n \): độ dài phần tuần hoàn.
	
	Khi đó:
	\[
		\frac{10^{x_n + \ell_n} - 10^{x_n}}{n} \in \mathbb{Z}^+
		\implies n \mid \left(10^{x_n + \ell_n} - 10^{x_n}\right)
		\implies n \mid 10^{x_n}(10^{\ell_n} - 1)
	\]
	
	Giả sử \( n = 2^a \cdot 5^b \cdot q \), với \( q \) nguyên tố cùng nhau với 10 (tức \( \gcd(q,10)=1 \)).
	
	Để \( \frac{1}{n} \) có biểu diễn thập phân vô hạn tuần hoàn với phần không tuần hoàn dài \( x_n \), thì:
	\[
		2^a 5^b \mid 10^{x_n}
		\implies x_n \text{ là số nhỏ nhất sao cho } 2^a 5^b \mid 10^{x_n}
		\implies x_n = \max(a, b)
	\]
	
	Vì \( 10^{x_n} \) chia hết cho cả \( 2^a \) và \( 5^b \) khi \( x_n \geq \max(a, b) \), và đây là giá trị nhỏ nhất như vậy.
	
	\textbf{Kết luận:} Độ dài phần không tuần hoàn trong biểu diễn thập phân vô hạn của \( \frac{1}{n} \) chính là:
	\[
		x_n = \max(a, b)\ \text{với } n = 2^a \cdot 5^b \cdot q,~ \gcd(q, 10) = 1
	\]	
\end{soln}

\footnotetext{\href{https://artofproblemsolving.com/community/c6h623454p3730817}{Lời giải của utkarshgupta.}}

\end{document}