\documentclass[../01-divisibility.tex]{subfiles}

\begin{document}

\begin{example*}[\gls{IND 2015 TST}2/P1]\label{example:IND-2015-TST2-P1}\textbf{[\nameref{definition:25M}]}\footnotemark
	Cho số nguyên \( n \geq 2 \), và đặt:
	\[
		A_n = \{2^n - 2^k \mid k \in \mathbb{Z},\, 0 \leq k < n \}.
	\]
	Tìm số nguyên dương lớn nhất không thể biểu diễn được dưới dạng tổng của một hay nhiều (không nhất thiết khác nhau) phần tử trong tập \( A_n \).
\end{example*}

\begin{story*}
	Bài toán này liên quan đến cấu trúc của các số có dạng \( 2^n - 2^k \), với kỹ thuật chính là:
	\begin{itemize}[topsep=0pt, partopsep=0pt, itemsep=0pt]
	    \item \textbf{Chứng minh bao phủ}: dùng quy nạp để bao phủ mọi số lớn hơn một ngưỡng.
	    \item \textbf{Tìm phần tử cực đại chưa biểu diễn được}: khai thác biểu diễn duy nhất với phần dư modulo \( 2^n \), từ đó tìm số nhỏ nhất không biểu diễn được.
	    \item Sử dụng \textbf{tính duy nhất của biểu diễn nhị phân} để đảm bảo mâu thuẫn khi có lặp lại.
	\end{itemize}
\end{story*}

\bigbreak

\begin{soln}
	\textbf{Bước 1:} Chứng minh rằng mọi số lớn hơn \( (n - 2) \cdot 2^n + 1 \) đều có thể biểu diễn được bằng tổng các phần tử của \( A_n \).
	Ta dùng quy nạp theo \( n \).

	\textit{Trường hợp cơ sở:} \( n = 2 \implies A_2 = \{3, 2\} \).  
	Mọi số nguyên dương \( m \neq 1 \) đều biểu diễn được:  
	nếu \( m \) chẵn thì dùng 2, nếu \( m \) lẻ thì dùng 3 rồi 2.

	\textit{Giả sử mệnh đề đúng với \( n - 1 \)}, xét \( n > 2 \) và số \( m > (n - 2) \cdot 2^n + 1 \).

	\textit{Trường hợp 1:} \( m \) chẵn. Khi đó, theo giả thiết quy nạp:
	\[
		\frac{m}{2} > (n - 3) \cdot 2^{n - 1} + 1
		\implies \frac{m}{2} = \sum (2^{n-1} - 2^{k_i}) \implies m = \sum (2^n - 2^{k_i + 1}) \in A_n.
	\]

	\textit{Trường hợp 2:} \( m \) lẻ. Tương tự:
	\[
		\frac{m - (2^n - 1)}{2} > (n - 3) \cdot 2^{n - 1} + 1
	\]
	suy ra biểu diễn như trước, và thêm \( 2^n - 1 \in A_n \) để có \( m \).

	\textbf{Bước 2:} Chứng minh rằng số \( (n - 2) \cdot 2^n + 1 \) không thể biểu diễn được.

	Gọi \( N \) là số nguyên dương nhỏ nhất \( \equiv 1 \Mod{2^n} \) và biểu diễn được thành tổng phần tử từ \( A_n \). Khi đó:
	\[
		N = \sum (2^n - 2^{k_i}) \tag{1}
	\]

	Giả sử tồn tại \( k_i = k_j \), ta có thể giảm biểu diễn bằng cách thay: $2 \cdot (2^n - 2^k) \to 2^n - 2^{k + 1}$
	và suy ra \( N - 2^n \) cũng biểu diễn được, mâu thuẫn với tính nhỏ nhất của \( N \).
	Vậy các \( k_i \) phân biệt, nên:
	\[
		\sum 2^{k_i} \leq 2^0 + 2^1 + \dots + 2^{n-1} = 2^n - 1
	\]
	
	Nhưng từ (1) ta có:
	\[
		\sum 2^{k_i} \equiv -N \equiv -1 \Mod{2^n} \implies \sum 2^{k_i} = 2^n - 1
	\]

	Xảy ra khi \( \{k_i\} = \{0, 1, \dots, n-1\} \), và khi đó:
	\[
		N = n \cdot 2^n - (2^0 + \dots + 2^{n-1}) = (n - 1) \cdot 2^n + 1
	\]

	\textbf{Kết luận:} Vậy số nguyên dương lớn nhất không thể biểu diễn được là:
	\[
		(n - 2) \cdot 2^n + 1
	\]
\end{soln}

\footnotetext{\href{https://www.imo-official.org/problems/IMO2014SL.pdf}{IMO SL 2014 N1.}}

\end{document}