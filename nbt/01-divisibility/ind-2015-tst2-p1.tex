\documentclass[../01-divisibility.tex]{subfiles}

\begin{document}

\begin{example*}[\gls{IND 2015 TST}2/P1]\label{example:IND-2015-TST2-P1}[\textbf{unrated}]\footnotemark
	Cho số nguyên \( n \geq 2 \), và đặt:
	\[
		A_n = \{2^n - 2^k \mid k \in \mathbb{Z},\, 0 \leq k < n \}.
	\]
	Tìm số nguyên dương lớn nhất không thể biểu diễn được dưới dạng tổng của một hay nhiều (không nhất thiết khác nhau) phần tử trong tập \( A_n \).
\end{example*}

\begin{soln}(Cách 1)
	Trước hết, ta chứng minh rằng mọi số nguyên lớn hơn \( (n - 2) \cdot 2^n + 1 \) đều có thể biểu diễn dưới dạng tổng như yêu cầu.
	Ta sẽ sử dụng quy nạp theo \( n \).

	Với \( n = 2 \), ta có \( A_2 = \{2^2 - 2^0, 2^2 - 2^1\} = \{3, 2\} \).
	Khi đó, mọi số nguyên dương \( m \neq 1 \) đều có thể viết dưới dạng tổng các phần tử của \( A_2 \):
	nếu \( m \) chẵn thì \( m = 2 + 2 + \dots + 2 \); nếu \( m \) lẻ thì \( m = 3 + 2 + \dots + 2 \).

	Giả sử mệnh đề đúng với mọi số nhỏ hơn \( n \), ta xét \( n > 2 \), và một số nguyên \( m > (n - 2) \cdot 2^n + 1 \).

	Nếu \( m \) là chẵn, xét:
	\[
		\frac{m}{2} \geq \frac{(n - 2) \cdot 2^n + 2}{2} = (n - 2) \cdot 2^{n-1} + 1 > (n - 3) \cdot 2^{n-1} + 1.
	\]
	
	Theo giả thiết quy nạp, tồn tại cách biểu diễn:
	\[
		\frac{m}{2} = (2^{n-1} - 2^{k_1}) + (2^{n-1} - 2^{k_2}) + \dots + (2^{n-1} - 2^{k_r})
	\]
	với \( 0 \leq k_i < n - 1 \). Suy ra:
	\[
		m = (2^n - 2^{k_1 + 1}) + (2^n - 2^{k_2 + 1}) + \dots + (2^n - 2^{k_r + 1}),
	\]
	tức là tổng các phần tử trong \( A_n \).

	Nếu \( m \) là lẻ, xét:
	\[
		\frac{m - (2^n - 1)}{2} > \frac{(n - 2) \cdot 2^n + 1 - (2^n - 1)}{2} = (n - 3) \cdot 2^{n-1} + 1.
	\]

	Theo giả thiết quy nạp, tồn tại biểu diễn:
	\[
		\frac{m - (2^n - 1)}{2} = (2^{n-1} - 2^{k_1}) + \dots + (2^{n-1} - 2^{k_r}),
	\]
	với \( k_i < n - 1 \). Suy ra:
	\[
		m = (2^n - 2^{k_1 + 1}) + \dots + (2^n - 2^{k_r + 1}) + (2^n - 1),
	\]
	là tổng các phần tử thuộc \( A_n \).


	Bây giờ ta chứng minh rằng số \( (n - 2) \cdot 2^n + 1 \) không thể biểu diễn được.

	Gọi \( N \) là số nguyên dương nhỏ nhất sao cho \( N \equiv 1 \pmod{2^n} \) và \( N \) có thể biểu diễn được thành tổng các phần tử từ \( A_n \). Xét biểu diễn:
	\[
		N = (2^n - 2^{k_1}) + (2^n - 2^{k_2}) + \dots + (2^n - 2^{k_r}) \tag{1}
	\]
	với \( 0 \leq k_i < n \).

	Giả sử tồn tại hai chỉ số \( i, j \) sao cho \( k_i = k_j \).

	Nếu \( k_i = k_j = n - 1 \), thì:
	\[
		N - 2 \cdot (2^n - 2^{n-1}) = N - 2^n
	\]
	cũng có thể biểu diễn được, mâu thuẫn với tính nhỏ nhất của \( N \).

	Nếu \( k_i = k_j = k < n - 1 \), thì:
	\[
		N - 2 \cdot (2^n - 2^k) + (2^n - 2^{k+1}) = N - 2^n
	\]
	cũng mâu thuẫn.

	Vậy các \( k_i \) đều phân biệt, nên:
	\[
		2^{k_1} + \dots + 2^{k_r} \leq 2^0 + 2^1 + \dots + 2^{n-1} = 2^n - 1
	\]

	Mặt khác, từ (1) xét modulo \( 2^n \), ta có:
	\[
		2^{k_1} + \dots + 2^{k_r} \equiv -N \equiv -1 \pmod{2^n} \implies 2^{k_1} + \dots + 2^{k_r} = 2^n - 1
	\]

	Điều này xảy ra khi \( \{k_1, \dots, k_r\} = \{0, 1, \dots, n-1\} \), và khi đó:
	\[
		N = n \cdot 2^n - (2^0 + \dots + 2^{n-1}) = (n - 1) \cdot 2^n + 1.
	\]

	Do đó, số \( (n - 2) \cdot 2^n + 1 \) không thể biểu diễn được như yêu cầu.
\end{soln}

\footnotetext{\href{https://www.imo-official.org/problems/IMO2014SL.pdf}{IMO SL 2014 N1.}}

\end{document}