\documentclass[../01-divisibility.tex]{subfiles}

\begin{document}

\begin{example*}[\gls{EGMO 2015}/P3]\label{example:EGMO-2015-P3}\textbf{[\nameref{definition:20M}]}
	Cho \( n, m \) là các số nguyên lớn hơn \( 1 \), và \( a_1, a_2, \dots, a_m \) là các số nguyên dương không vượt quá \( n^m \).
	Hãy chứng minh rằng tồn tại các số nguyên dương \( b_1, b_2, \dots, b_m \leq n \) sao cho:
	\[
		\gcd(a_1 + b_1, a_2 + b_2, \dots, a_m + b_m) < n.
	\]
\end{example*}

\begin{story*}
	Bài toán yêu cầu tìm bộ \( (b_1, \dots, b_m) \in \{1, \dots, n\}^m \) sao cho \textbf{ước chung lớn nhất} của \( a_i + b_i \) nhỏ hơn \( n \). 
	Hướng giải dùng hai ý tưởng chính:
	\begin{itemize}[topsep=0pt, partopsep=0pt, itemsep=0pt]
		\item Nếu tồn tại hai số \( a_i \) bằng nhau hoặc chênh lệch 1, ta có thể dùng các giá trị \( b_i \) nhỏ để tạo ra hai số liên tiếp và suy ra GCD bằng 1.
		\item Nếu không, ta xét toàn bộ \( n^m \) tổ hợp các \( b_i \), trong khi số lượng giá trị GCD lớn hơn hoặc bằng \( n \) bị giới hạn. 
		      \textbf{Nguyên lý Dirichlet} buộc phải có hai tổ hợp dẫn đến cùng một GCD \( d \geq n \), từ đó dẫn tới mâu thuẫn vì mỗi \( a_i \) chỉ cho phép nhiều nhất một giá trị \( b_i \) để chia hết cho \( d \).
	\end{itemize}
\end{story*}

\bigbreak

\begin{soln}\footnotemark
	\textbf{Bước 1:} Giả sử không mất tính tổng quát rằng \( a_1 \) là nhỏ nhất trong các \( a_i \).

	\textit{Trường hợp 1:} \( a_1 \geq n^m - 1 \)

	Nếu tất cả \( a_i = a_1 \), chọn \( b_1 = 1, b_2 = 2 \), các \( b_i \) còn lại tùy ý. Khi đó:
	\[
		\gcd(a_1 + 1, a_1 + 2) = 1.
	\]

	Nếu \( a_1 = n^m - 1 \) và tồn tại \( a_j = n^m \), lấy \( b_1 = b_j = 1 \), các \( b_i \) khác tùy ý. Khi đó:
	\[
		\gcd(n^m, n^m + 1) = 1.
	\]

	Vậy ta chỉ cần xét trường hợp \( a_1 \leq n^m - 2 \).

	\textbf{Bước 2:} Giả sử ngược lại rằng với mọi bộ \( (b_1, \dots, b_m) \in \{1, \dots, n\}^m \), ta có:
	\[
		\gcd(a_1 + b_1, \dots, a_m + b_m) \geq n.
	\]

	Ta xét các bộ có \( b_1 = 1 \), và \( b_i \in \{1, 2, \dots, n\} \) với \( i > 1 \). Có \( n^{m-1} \) bộ như vậy. Gọi các GCD tương ứng là \( d_1, \dots, d_{n^{m-1}} \).

	Mỗi \( d_j \geq n \) và đều chia \( a_1 + 1 \). Ta chứng minh rằng các \( d_j \) đôi một nguyên tố cùng nhau.

	Thật vậy, với hai bộ khác nhau, tồn tại \( i > 1 \) sao cho \( b_i \) khác nhau trong hai bộ. Khi đó \( d_j \mid a_i + b_i \) và \( d_k \mid a_i + b_i' \) với \( |b_i - b_i'| = 1 \), nên:
	\[
		d_j, d_k \mid (a_i + b_i') - (a_i + b_i) = \pm 1 \implies \gcd(d_j, d_k) = 1.
	\]

	Vậy \( d_1, \dots, d_{n^{m-1}} \) là các ước số nguyên tố cùng nhau của \( a_1 + 1 \), và mỗi \( d_j \geq n \), với nhiều nhất một \( d_j = n \). 

	Suy ra:
	\[
		a_1 + 1 \geq n(n+1)^{n^{m-1} - 1} \implies a_1 \geq n^m,
	\]
	mâu thuẫn với giả thiết \( a_1 \leq n^m \).

	\textbf{Kết luận:} Phản chứng sai, do đó tồn tại bộ \( (b_1, \dots, b_m) \in \{1, \dots, n\}^m \) sao cho:
	\[
		\gcd(a_1 + b_1, \dots, a_m + b_m) < n.
	\]
\end{soln}

\footnotetext{\href{https://www.egmo.org/egmos/egmo4/solutions.pdf}{Lời giải chính thức.}}

\newpage

\begin{example*}[Bản nâng cao \gls{EGMO 2015}/P3]\label{example:EGMO-2015-P3-strong}\textbf{[\nameref{definition:25M}]}
	Với \( m, n > 1 \), giả sử \( a_1, \dots, a_m \) là các số nguyên dương sao cho có ít nhất một \( a_i \leq n^{2^{m-1}} \). 
	Chứng minh rằng tồn tại các số nguyên \( b_1, \dots, b_m \in \{1, 2\} \) sao cho:
	\[
		\gcd(a_1 + b_1, \dots, a_m + b_m) < n.
	\]
\end{example*}

\begin{story*}
	So với bài toán gốc \nameref{example:EGMO-2015-P3}, bài toán này giới hạn các \( b_i \in \{1, 2\} \) thay vì \( \leq n \),
	nhưng giả thiết về \( a_i \) được nới lỏng: chỉ cần một trong số chúng nhỏ hơn hoặc bằng \( n^{2^{m-1}} \).

	Hướng giải sử dụng:
	\begin{itemize}[topsep=0pt, partopsep=0pt, itemsep=0pt]
		\item \textbf{Phản chứng}: giả sử mọi tổ hợp \( (b_1, \dots, b_m) \in \{1,2\}^m \) đều cho GCD \( \geq n \).
		\item Với \( 2^{m-1} \) tổ hợp \( (b_1, \dots, b_m) \) cố định \( b_1 = 1 \), các GCD thu được là các ước của \( a_1 + 1 \) và \textit{đôi một nguyên tố cùng nhau}.
		\item Từ đó suy ra rằng \( a_1 + 1 \) phải có ít nhất \( 2^{m-1} \) ước số nguyên tố lớn hơn \( n \), dẫn đến bất đẳng thức mâu thuẫn.
	\end{itemize}
\end{story*}

\bigbreak

\begin{soln}\footnotemark
	\textbf{Bước 1:} Giả sử ngược lại rằng với mọi bộ \( (b_1, \dots, b_m) \in \{1, 2\}^m \), ta có:
	\[
		\gcd(a_1 + b_1, \dots, a_m + b_m) \geq n.
	\]

	Xét các bộ có \( b_1 = 1 \), còn \( b_i \in \{1, 2\} \) với \( i > 1 \). Có \( 2^{m-1} \) bộ như vậy. Gọi các GCD tương ứng là \( d_1, \dots, d_{2^{m-1}} \).

	Mỗi \( d_j \geq n \) và đều chia \( a_1 + 1 \).

	\textbf{Bước 2:} Ta chứng minh rằng các \( d_j \) đôi một nguyên tố cùng nhau.

	Thật vậy, với hai bộ khác nhau, tồn tại \( i > 1 \) sao cho một có \( b_i = 1 \), bộ kia có \( b_i = 2 \). Khi đó:
	\[
		d_j \mid a_i + 1,\quad d_k \mid a_i + 2 \implies d_j, d_k \mid 1 \implies \gcd(d_j, d_k) = 1.
	\]

	\textbf{Bước 3:} Vậy \( d_1, \dots, d_{2^{m-1}} \) là các ước số nguyên tố cùng nhau của \( a_1 + 1 \), và mỗi \( d_j \geq n \), với nhiều nhất một \( d_j = n \). 

	Suy ra:
	\[
		a_1 + 1 \geq n(n+1)^{2^{m-1} - 1} \implies a_1 \geq n^{2^{m-1}},
	\]
	mâu thuẫn với giả thiết \( a_i \leq n^{2^{m-1}} \) với một \( i \), chẳng hạn \( i = 1 \).

	\textbf{Kết luận:} Tồn tại bộ \( (b_1, \dots, b_m) \in \{1, 2\}^m \) sao cho:
	\[
		\gcd(a_1 + b_1, \dots, a_m + b_m) < n.
	\]
\end{soln}

\begin{remark*}
	Giới hạn \( n^{2^{m-1}} \) trong giả thiết có thể được cải thiện thêm.
\end{remark*}

\footnotetext{\samepage \href{https://www.egmo.org/egmos/egmo4/solutions.pdf}{Lời giải chính thức.}}

\end{document}