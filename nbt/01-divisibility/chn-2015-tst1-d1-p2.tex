\documentclass[../01-divisibility.tex]{subfiles}

\begin{document}

\begin{example*}[\gls{CHN 2015 TST}1/D1/P2]\label{example:CHN-2015-TST1-D1-P2}\textbf{[\nameref{definition:10M}]}
	Cho dãy các số nguyên dương phân biệt \( a_1, a_2, a_3, \ldots \), và một hằng số thực \( 0 < c < \dfrac{3}{2} \).  
	Chứng minh rằng tồn tại vô hạn số nguyên dương \( k \) sao cho:
	\[
		\operatorname{lcm}(a_k, a_{k+1}) > c k.
	\]
\end{example*}

\begin{story*}
	Ta giả sử phản chứng rằng từ một thời điểm \( K \) trở đi, tất cả các bội chung nhỏ nhất \( \operatorname{lcm}(a_k, a_{k+1}) \le ck \).
	Sau đó, bằng một bất đẳng thức đơn giản liên quan đến tổng nghịch đảo và tính chất của \( \gcd \) và \( \operatorname{lcm} \),
	ta thu được tổng vô hạn bị chặn, mâu thuẫn. Mấu chốt là:  
    \[
    	\frac{1}{a_k} + \frac{1}{a_{k+1}} \ge \frac{3}{\operatorname{lcm}(a_k, a_{k+1})} \ge \frac{3}{ck}.
    \]
    Và mỗi số \( 1/a_j \) chỉ xuất hiện tối đa hai lần, dẫn đến mâu thuẫn khi tổng phát triển không giới hạn.
\end{story*}

\bigbreak

\begin{soln}\footnotemark
	Giả sử phản chứng rằng tồn tại số nguyên \( K \) sao cho:
	\[
		\operatorname{lcm}(a_k, a_{k+1}) \le ck \quad \text{với mọi } k \ge K.
	\]
	
	\textbf{Khẳng định.} Với mọi \( k \ge K \), ta có:
	\[
		\frac1{a_k}+\frac1{a_{k+1}} \ge \frac3{ck}.
	\]
	
	\textit{Chứng minh.} Ta dùng:
	\[
		\frac1{a_k}+\frac1{a_{k+1}} = \frac{a_k + a_{k+1}}{\gcd(a_k, a_{k+1}) \cdot \operatorname{lcm}(a_k, a_{k+1})}.
	\]
	Vì \( a_k \ne a_{k+1} \), ta có:
	\[
		a_k + a_{k+1} > 2\min\{a_k, a_{k+1}\} \ge 2\gcd(a_k, a_{k+1}).
	\]
	Vì \( a_k + a_{k+1} \) chia hết cho \( \gcd(a_k, a_{k+1}) \), ta suy ra:
	\[
		a_k + a_{k+1} \ge 3\gcd(a_k, a_{k+1}),
	\]
	dẫn đến:
	\[
		\frac1{a_k}+\frac1{a_{k+1}} \ge \frac3{\operatorname{lcm}(a_k, a_{k+1})} \ge \frac3{ck}.
	\quad \blacksquare
	\]

	Cộng từ \( k = K \) đến \( k = N \), ta được:
	\[
		\sum_{k=K}^N \left( \frac1{a_k}+\frac1{a_{k+1}} \right) \ge \frac{3}{c} \sum_{k=K}^N \frac1{k}.
	\]
	Mặt khác, vì các số \( a_K, \ldots, a_{N+1} \) là phân biệt nên mỗi số \( 1/a_j \) chỉ xuất hiện tối đa hai lần trong tổng bên trái.  
	Suy ra:
	\[
		\sum_{k=K}^N \left( \frac1{a_k}+\frac1{a_{k+1}} \right) \le \sum_{j=1}^{\max a_j} \frac{2}{j}.
	\]
	Tổng bên phải bị chặn, còn tổng bên trái có giới hạn dưới là \( \left( \frac{3}{c} - 2 \right) \sum_{k=K}^N \frac1{k} \), sẽ tiến ra vô hạn vì \( \frac{3}{c} - 2 > 0 \).  
	Mâu thuẫn.

	Vậy giả thiết là sai. Do đó, tồn tại vô hạn chỉ số \( k \) sao cho:
	\[
		\operatorname{lcm}(a_k, a_{k+1}) > ck.
	\]
\end{soln}

\footnotetext{\href{https://artofproblemsolving.com/community/c6h1062614p18497986}{Lời giải của \textbf{TheUltimate123}.}}

\end{document}