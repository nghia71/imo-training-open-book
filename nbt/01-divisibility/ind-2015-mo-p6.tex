\documentclass[../01-divisibility.tex]{subfiles}

\begin{document}

\begin{example*}[\gls{IND 2015 MO}/P6]\label{example:IND-2015-N6}\textbf{[\nameref{definition:15M}]}
	Chứng minh rằng từ một tập gồm \( 11 \) số chính phương, ta luôn có thể chọn ra sáu số \( a^2, b^2, c^2, d^2, e^2, f^2 \) sao cho:
	\[
		a^2 + b^2 + c^2 \equiv d^2 + e^2 + f^2 \Mod{12}
	\]
\end{example*}

\begin{story*}
	Bài toán yêu cầu chứng minh tồn tại hai bộ ba số chính phương sao cho tổng của chúng đồng dư modulo 12.

	Kỹ thuật chính:
	\begin{itemize}[topsep=0pt, partopsep=0pt, itemsep=0pt]
	    \item \textbf{Phân tích phần dư} của bình phương modulo 12: chỉ có thể là \( 0, 1, 4, 9 \).
	    \item Áp dụng \textbf{nguyên lý Dirichlet} để buộc tồn tại ít nhất một phần dư xuất hiện nhiều lần.
	    \item Xét các tổ hợp có thể từ các phần dư và so sánh tổng modulo 12 để tạo hai tập có tổng bằng nhau.
	\end{itemize}
\end{story*}

\bigbreak

\begin{soln}
	Phần dư khả dĩ của một số chính phương modulo 12 là:
	\[
		x^2 \equiv 0, 1, 4, 9 \Mod{12}
	\]

	Gọi \( S \) là tập gồm 11 số chính phương. Xét tập dư \( S_r \), mỗi phần tử của nó thuộc \( \{0, 1, 4, 9\} \).

	Ta cần chứng minh tồn tại hai tập con rời nhau \( A, B \subset S \), mỗi tập gồm 3 phần tử, sao cho:
	\[
		\sum_{x \in A} x \equiv \sum_{y \in B} y \Mod{12}
	\]

	\textit{Trường hợp 1:} Có ít nhất 6 phần tử trong \( S_r \) có cùng phần dư.

	Khi đó ta chia thành 2 bộ ba giống nhau, có tổng bằng nhau nên đồng dư modulo 12.

	\textit{Trường hợp 2:} Có một phần dư xuất hiện 4 hoặc 5 lần.

	Vì \( S \) có 11 phần tử và chỉ có 4 loại phần dư, theo Dirichlet, tồn tại phần dư khác xuất hiện ít nhất 2 lần.

	Ta chọn 3 phần tử từ nhóm lớn cho tập A, và 3 phần tử khác từ phần dư còn lại cho tập B. Vì chỉ có vài tổng khả dĩ, nên tồn tại 2 bộ có tổng giống nhau modulo 12.

	\textit{Trường hợp 3:} Mỗi phần dư xuất hiện tối đa 3 lần.

	Tổng số phần tử tối đa khi phân bổ như vậy là: $3 + 3 + 3 + 2 = 11$.
	Trường hợp này đúng nhưng giới hạn cực đoan, vẫn đảm bảo tồn tại nhiều cách chọn bộ ba có tổng giống nhau.

	\textbf{Kết luận:} Trong mọi trường hợp, luôn tồn tại hai tập rời nhau gồm ba số chính phương sao cho tổng của chúng đồng dư modulo 12.
\end{soln}

\footnotetext{\href{https://artofproblemsolving.com/community/c6h623456p3730860}{Dựa theo lời giải của \textbf{Sahil}.}}

\end{document}