\documentclass[../01-divisibility.tex]{subfiles}

\begin{document}

\begin{example*}[\gls{CHN 2015 TST}1/D2/P2]\label{example:CHN-2015-TST1-D2-P2}\textbf{[\nameref{definition:10M}]}
	Cho trước một số nguyên dương \( n \). Chứng minh rằng: Với mọi số nguyên dương \( a, b, c \) không vượt quá \( 3n^2 + 4n \),
	tồn tại các số nguyên \( x, y, z \) có giá trị tuyệt đối không vượt quá \( 2n \) và không đồng thời bằng 0, sao cho
	\[
		ax + by + cz = 0.
	\]
\end{example*}

\begin{remark*}
    Hãy xét tập tất cả giá trị \( ax + by + cz \) với \( x, y, z \in [-n, n] \). Dùng nguyên lý Dirichlet để chỉ ra rằng phải có hai bộ khác nhau cho cùng một giá trị.
\end{remark*}

\begin{story*}
    Bài toán yêu cầu tìm một bộ số nguyên \( (x, y, z) \) nhỏ sao cho tổ hợp tuyến tính \( ax + by + cz = 0 \). 
    Ý tưởng then chốt là:
    \begin{itemize}[topsep=0pt, partopsep=0pt, itemsep=0pt]
		\item \textbf{Xét tập các giá trị} có thể đạt được của \( ax + by + cz \) trong một khối lập phương rời rạc giới hạn bởi \( [-n, n]^3 \).
		\item Đếm số phần tử của tập đó và so sánh với số lượng tổ hợp \( (x, y, z) \in [-n, n]^3 \).
		\item Áp dụng \textbf{nguyên lý Dirichlet} để đảm bảo tồn tại hai tổ hợp khác nhau cho cùng một giá trị.
	\end{itemize}
	Từ sự trùng giá trị, ta trừ vế với vế và thu được một tổ hợp không tầm thường \( (x, y, z) \) thoả mãn đẳng thức.
\end{story*}

\bigbreak

\begin{soln}\footnotemark
	\textbf{Bước 1:} Gọi  
	\[
		A = \{ ax + by + cz \mid x, y, z \in \mathbb{Z} \cap [-n, n] \}.
	\]
	
	Khi đó,
	\[
		\min A = -3n^3 - 4n^2,\quad \max A = 3n^3 + 4n^2 
		\implies |A| \leq 6n^3 + 8n^2 + 1.
	\]

	\textbf{Bước 2:} Có tổng cộng \( (2n + 1)^3 \) bộ \( (x, y, z) \in [-n, n]^3 \), và vì
	\[
		(2n+1)^3 > 6n^3 + 8n^2 + 1,
	\]
	theo \nameref{theorem:pigeonhole-principle}, tồn tại hai bộ khác nhau
	\[
		(x, y, z), (x', y', z') \in ([-n, n] \cap \mathbb{Z})^3 \quad \text{với} \quad ax + by + cz = ax' + by' + cz'.
	\]

	\textbf{Bước 3:} Trừ hai vế ta được:
	\[
		a(x - x') + b(y - y') + c(z - z') = 0,
	\]
	với \( x - x', y - y', z - z' \in [-2n, 2n] \) và không đồng thời bằng 0.

	\textbf{Kết luận:} Đã tồn tại một bộ \( (x, y, z) \) thoả mãn điều kiện đề bài.
\end{soln}

\footnotetext{\href{https://artofproblemsolving.com/community/c6h1063058p4617378}{Lời giải của \textbf{nayel}.}}

\end{document}