\documentclass[../01-divisibility.tex]{subfiles}

\begin{document}

\begin{example*}[\gls{RUS 2015 TST}/D10/P2]\label{example:RUS-2015-TST-D10-P2}[\textbf{\nameref{definition:25M}}]
	Cho số nguyên tố \( p \ge 5 \). Chứng minh rằng tập \( \{1,2,\ldots,p - 1\} \) có thể được chia thành hai tập con không rỗng
	sao cho tổng các phần tử của một tập con và tích các phần tử của tập con còn lại cho cùng một phần dư modulo \( p \).	
\end{example*}

\begin{story*}
    Bài toán yêu cầu tìm một cách phân chia tập \( \{1, 2, \dots, p-1\} \) thành hai tập con không rỗng \( S \) và \( S^c \) sao cho \( \sum_{S} \equiv \prod_{S^c} \Mod{p} \). Một cách tiếp cận là dùng các định lý cổ điển: tổng các số từ 1 đến \( p-1 \) là \( \frac{(p-1)p}{2} \equiv 1 \Mod{p} \), còn tích là \( (p-1)! \equiv -1 \Mod{p} \) theo Wilson. Sử dụng các biểu thức bổ sung này, ta đặt điều kiện tương đương với \( AB \equiv -1 \Mod{p} \). Khi đó, việc xây dựng một tập \( S \) có tổng và tích phù hợp được chia thành hai trường hợp: \( p \equiv 1 \Mod{4} \) và \( p \equiv 3 \Mod{4} \), với mỗi trường hợp dùng các kỹ thuật khác nhau — khai thác căn bậc hai của \( -1 \), phần tử sinh, và tổng cấp số nhân.
\end{story*}

\bigbreak

\begin{soln}\footnotemark
	Ta biết rằng:
    \[
        \sum_{i=1}^{p-1} i = \frac{(p-1)p}{2} \equiv 1 \Mod{p}, \quad \prod_{i=1}^{p-1} i \equiv -1 \Mod{p} \quad \text{(Định lý Wilson)}
    \]
    
	Gọi \( S \) là một tập con không rỗng của \( \{1, 2, \ldots, p-1\} \). Khi đó, phần bù của \( S \) là \( S^c = \{1, \ldots, p-1\} \setminus S \).
	Bài toán tương đương với việc tìm \( S \) sao cho:
    \[
        \sum_{i \in S} i \equiv \prod_{j \in S^c} j \Mod{p}.
    \]
    
	Đặt \( A = \sum_{i \in S} i, \quad B = \prod_{i \in S} i \), khi đó:
    \[
		\sum_{i \in S^c} i \equiv 1 - A \Mod{p}, \quad
		\prod_{i \in S^c} i \equiv \frac{-1}{B} \Mod{p}.
    \]
	
	Ta cần:
    \[
        A \equiv \frac{-1}{B} \Mod{p} \quad \Longleftrightarrow \quad AB \equiv -1 \Mod{p}.
    \]

    Ta sẽ xây dựng \( S \) thỏa mãn điều này.

    \textit{Trường hợp 1:} Nếu \( p \equiv 1 \Mod{4} \). Khi đó tồn tại \( a \in \mathbb{F}_p \) sao cho \( a^2 \equiv -1 \Mod{p} \). Lấy \( S = \{a\} \), ta có:
    \[
        A = a, \quad B = a \Rightarrow AB = a^2 \equiv -1 \Mod{p}.
    \]

    \textit{Trường hợp 2:} Nếu \( p \equiv 3 \Mod{4} \), thì \( p - 1 \equiv 2 \Mod{4} \), nên tồn tại số nguyên tố lẻ \( q \mid (p - 1) \) (do \( p \ge 5 \)).

    Khi đó, tồn tại phần tử sinh \( a \in \mathbb{F}_p \) sao cho \( \mathrm{ord}_p(a) = q \). Xét tập:
    \[
        S = \left\{a^{\frac{q-1}{2}}, a^{\frac{q-3}{2}}, \ldots, a, a^{-1}, a^{-3}, \ldots, a^{-\frac{q-1}{2}}\right\},\quad |S| = q - 1.
    \]

    Các phần tử đi thành cặp nghịch đảo, nên tích của \( S \) là:
	\[
		\prod_{i \in S} i \equiv 1 \Mod{p}.
	\]

	Ta nhân cả tổng với \( a^{\frac{q-1}{2}} \), thu được:
	\[
		a^{\frac{q-1}{2}} \sum_{i \in S} i = a^{q-1} + a^{q-2} + \cdots + 1 \equiv 0 \Mod{p},
	\]
	vì đây là tổng cấp số nhân với công bội \( a \), số hạng \( q \), nên tổng bằng \( \frac{a^q - 1}{a - 1} \equiv 0 \Mod{p} \).

    Do đó:
    \[
        \sum_{i \in S} i \equiv -1 \Mod{p}, \quad \prod_{i \in S} i \equiv 1 \Mod{p} \Rightarrow AB \equiv -1 \Mod{p}.
    \]
\end{soln}

\footnotetext{\href{https://artofproblemsolving.com/community/c6h3057483p28254744}{Dựa theo lời giải của \textbf{IAmTheHazard}.}}

\end{document}