\documentclass[../01-divisibility.tex]{subfiles}

\begin{document}

\begin{exercise*}[\gls{HUN 2015 TST}/KMA/633]\label{example:HUN-2015-TST-KMA-633}[\textbf{\nameref{definition:35M}}]
    Chứng minh rằng nếu \( n \) là một số nguyên dương đủ lớn,
    thì trong bất kỳ tập hợp gồm \( n \) số nguyên dương khác nhau nào cũng tồn tại bốn số sao cho bội chung nhỏ nhất của chúng lớn hơn \( n^{3{,}99} \).

    \begin{remark*}
        Gợi ý: Hãy sắp xếp các số đã cho theo thứ tự tăng dần, ước lượng kích thước của bội chung nhỏ nhất,
        và tìm cách chọn bốn số sở hữu lũy thừa nguyên tố lớn vượt mức \(n^{3.99}\).
    \end{remark*}

    \begin{story*}
        Phân tích thêm: một cách tiếp cận là xét các số có độ lớn gần nhau để tối ưu giá trị LCM.
        Sử dụng việc nếu các số có nhiều ước nguyên tố lớn hoặc lũy thừa cao, khi kết hợp chúng, LCM tăng rất nhanh.
        Từ đó suy ra khả năng bội chung nhỏ nhất vượt quá \(n^{3.99}\) khi \(n\) đủ lớn.
    \end{story*}
\end{exercise*}

\end{document}