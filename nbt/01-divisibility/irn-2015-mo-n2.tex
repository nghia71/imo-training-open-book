\documentclass[../01-divisibility.tex]{subfiles}

\begin{document}

\begin{example*}[\gls{IRN 2015 MO}/N2]\label{example:IRN-2015-N2}\textbf{[\nameref{definition:25M}]}
	Gọi \( M_0 \subset \mathbb{N} \) là một tập hợp hữu hạn, không rỗng các số tự nhiên. Ali tạo ra các tập \( M_1, M_2, \dots, M_n \) theo quy trình sau:
	Tại bước \( n \), Ali chọn một phần tử \( b_n \in M_{n-1} \), sau đó định nghĩa tập:
	\[
		M_n = \left\{ b_n m + 1 \mid m \in M_{n-1} \right\}
	\]

	Chứng minh rằng tồn tại một bước nào đó mà trong tập tạo ra, không có phần tử nào chia hết cho phần tử nào khác trong cùng tập.
\end{example*}

\begin{story*}
	Bài toán yêu cầu chứng minh rằng sau một số bước đủ lớn, trong tập \( M_n \) không tồn tại quan hệ chia hết giữa hai phần tử.

	Ý tưởng chính gồm:
	\begin{itemize}[topsep=0pt, partopsep=0pt, itemsep=0pt]
		\item Nếu có quan hệ chia hết giữa các phần tử, điều này dẫn đến một bất đẳng thức giữa độ rộng tập trước đó và giá trị nhỏ nhất ở bước hiện tại.
		\item Sử dụng chuỗi sinh nhanh của các phần tử trong \( M_n \) để đưa ra mâu thuẫn.
		\item Khi \( n \) đủ lớn, bất đẳng thức sẽ không còn thỏa mãn và không thể có quan hệ chia hết nào xảy ra.
	\end{itemize}
\end{story*}

\bigbreak

\begin{soln}\footnotemark
	Giả sử tại bước \( n \), tồn tại hai phần tử \( k, t \in M_{n-1} \) sao cho phần tử tương ứng trong \( M_n \) có quan hệ chia hết:
	\[
		b_n k + 1 \mid b_n t + 1
	\]
	Suy ra:
	\[
		b_n k + 1 \mid b_n(t - k) \implies b_n k + 1 \mid k - t
	\]

	Vậy nếu tồn tại quan hệ chia hết trong \( M_n \), thì hiệu giữa phần tử lớn nhất và nhỏ nhất của \( M_{n-1} \) phải lớn hơn hoặc bằng phần tử nhỏ nhất trong \( M_n \):
	\[
		\max(M_{n-1}) - \min(M_{n-1}) \geq \min(M_n) \tag{1}
	\]

	Gọi \( M = \max(M_1),\ m = \min(M_1) \). Ta biết rằng:
	\[
		M_n = \{ b_n m + 1 \mid m \in M_{n-1} \}
	\implies \max(M_n) - \min(M_n) = b_n b_{n-1} \cdots b_2 (M - m)
	\]
	và:
	\[
		\min(M_n) \geq b_n b_{n-1} \cdots b_2 m + b_{n-1} \cdots b_2
	\]

	Thế vào (1), ta có:
	\[
		b_2 \cdots b_{n-1} (M - m - 1) \geq b_2 \cdots b_n m
	\]
	Rút gọn cả hai vế:
	\[
		\frac{M - m - 1}{m} \geq b_n
	\]

	Mặt khác, vì \( b_n \in M_{n-1} \), và các phần tử tăng rất nhanh qua các bước, ta có:
	\[
		b_n \geq n - 2
	\]

	Như vậy, nếu bất đẳng thức sau thỏa mãn:
	\[
		\frac{M - m - 1}{m} < n - 2
	\]
	thì không thể xảy ra quan hệ chia hết giữa các phần tử trong \( M_n \).

	\textbf{Kết luận:} Khi \( n \) đủ lớn, điều kiện trên luôn xảy ra, nên tồn tại một bước mà trong tập \( M_n \), không có phần tử nào chia hết cho phần tử nào khác.
\end{soln}

\footnotetext{\href{https://artofproblemsolving.com/community/c6h1139106p16859237}{Dựa theo lời giải của \textbf{Arefe}.}}

\end{document}