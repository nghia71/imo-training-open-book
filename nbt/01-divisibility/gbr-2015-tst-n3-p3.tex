\documentclass[../03-arithmetic-functions.tex]{subfiles}

\begin{document}

\begin{exercise*}[\gls{GBR 2015 TST}/N3/P3]\label{example:GBR-2015-TST-N3-P3}\textbf{[\nameref{definition:25M}]}
    Cho các số nguyên dương phân biệt đôi một \( a_1 < a_2 < \cdots < a_n \), trong đó \( a_1 \) là số nguyên tố và \( a_1 \geq n + 2 \).

    Trên đoạn thẳng \( I = \left[0, \prod_{i=1}^n a_i \right] \) trên trục số thực,
    đánh dấu tất cả các số nguyên chia hết cho ít nhất một trong các số \( a_1, a_2, \ldots, a_n \).
    
    Các điểm này chia đoạn \( I \) thành nhiều đoạn con.
    
    Chứng minh rằng tổng bình phương độ dài các đoạn con đó chia hết cho \( a_1 \).
\end{exercise*}

\begin{remark*}
    Hãy xét các đoạn con nằm giữa hai số liên tiếp không bị đánh dấu. Gọi độ dài mỗi đoạn là \( \ell_i \), ta cần chứng minh \( \sum \ell_i^2 \equiv 0 \Mod{a_1} \).
\end{remark*}

\begin{story*}
    Bài toán này gợi ý sử dụng \textbf{hàm đặc trưng chia hết} để xác định vị trí các điểm chia đoạn.
    Từ đó, ta có thể mô hình hóa mỗi đoạn con bằng độ dài giữa hai bội liên tiếp (hoặc không liên tiếp) và khảo sát các \textbf{khoảng không bị đánh dấu}.

    Ý tưởng then chốt là:
    \begin{itemize}[topsep=0pt, partopsep=0pt, itemsep=0pt]
        \item Mỗi đoạn con là một khoảng giữa hai số nguyên liên tiếp không chia hết cho bất kỳ \( a_i \), hay nói cách khác, thuộc phần bù của hợp các tập bội.
        \item Tổng bình phương độ dài có thể được xử lý bằng \textbf{hàm Dirichlet} và công thức đếm dựa trên hợp của các cấp số cộng.
        \item Sử dụng \textbf{đối xứng theo modulo \( a_1 \)} hoặc phân tích tính chu kỳ để đưa biểu thức tổng về dạng chia hết cho \( a_1 \).
    \end{itemize}
\end{story*}

\end{document}