\documentclass[../01-divisibility.tex]{subfiles}

\begin{document}

\begin{example*}[\gls{BxMO 2015}/P3]\label{example:BxMO-2015-P3}\textbf{[\nameref{definition:10M}]}
	Có tồn tại số nguyên tố nào có biểu diễn thập phân dưới dạng
	\[
		3811\underbrace{11\ldots1}_{\text{n chữ số }1}?
	\]
	Tức là, bắt đầu bằng các chữ số \( 3, 8 \), sau đó là một hoặc nhiều chữ số \( 1 \).
\end{example*}

\begin{story*}
	Bài toán yêu cầu xét tính nguyên tố của các số có dạng đặc biệt, và lời giải sử dụng \textbf{kỹ thuật phân tích chuỗi chữ số theo modulo 3}. 
	Mỗi trường hợp ứng với số chữ số \( 1 \) chia cho \( 3 \) được \textit{diễn lại thành tích các số nhỏ hơn}, từ đó suy ra rằng số gốc luôn là hợp số.
\end{story*}

\begin{soln}\footnotemark
	Ta phân tích theo số lượng chữ số \( 1 \) theo modulo 3.

	\textit{Trường hợp 1:} Số chữ số \( 1 \) là \( 3k \)
	\[
		38\underbrace{11\ldots1}_{3k\text{ chữ số }1}
		= 2\underbrace{33\ldots3}_{k\text{ chữ số }3}
		\cdot 16\underbrace{33\ldots3}_{k-1\text{ chữ số }3}
		5\underbrace{66\ldots6}_{k-1\text{ chữ số }6}
		7
	\]
	
	\textit{Trường hợp 2:} Số chữ số \( 1 \) là \( 3k + 1 \)
	\[
		38\underbrace{11\ldots1}_{3k+1\text{ chữ số }1}
		= 3 \cdot 127\underbrace{037037\ldots037}_{k\text{ chuỗi }037}
	\]
	
	\textit{Trường hợp 3:} Số chữ số \( 1 \) là \( 3k + 2 \)
	\[
		38\underbrace{11\ldots1}_{3k+2\text{ chữ số }1}
		= 37 \cdot 103\underbrace{003003\ldots003}_{k\text{ chuỗi }003}
	\]

	\textbf{Kết luận:} Trong cả ba trường hợp, số đều chia hết cho một số nhỏ hơn, nên luôn là hợp số. Do đó, không tồn tại số nguyên tố nào có dạng \( 3811\ldots1 \).
\end{soln}

\footnotetext{\href{https://artofproblemsolving.com/community/c6h1087189p4837937}{Lời giải của \textbf{codyj}.}}

\end{document}