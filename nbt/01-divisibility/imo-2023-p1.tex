\documentclass[../01-divisibility.tex]{subfiles}

\begin{document}

\begin{example*}[\gls{IMO 2023}/P1]\label{example:IMO-2023-P1}\textbf{[\nameref{definition:5M}]}
    Xác định tất cả các số nguyên hợp dương \( n \) thỏa mãn tính chất sau: nếu các ước số dương của \( n \) là \( 1 = d_1 < d_2 < \dots < d_k = n \),
    thì \( d_i \mid (d_{i+1} + d_{i+2}) \quad \text{với mọi } 1 \leq i \leq k - 2. \)
\end{example*}

\begin{story*}
    Bài toán yêu cầu khảo sát cấu trúc các ước của một số nguyên hợp \( n \) sao cho mỗi ước \( d_i \) chia hết tổng của hai ước lớn hơn kế tiếp.
    Một số kỹ thuật nổi bật trong các lời giải bao gồm:
    \begin{itemize}[topsep=0pt, partopsep=0pt, itemsep=0pt]
        \item Khảo sát các số có dạng \( p^r \) và chứng minh rằng chúng luôn thỏa điều kiện.
        \item Giả sử tồn tại hai thừa số nguyên tố khác nhau và sử dụng các ước đối xứng để dẫn đến mâu thuẫn.
        \item Sử dụng quy nạp và các phép chia để suy ra rằng tất cả ước đều chia hết một số nguyên tố nhỏ nhất.
    \end{itemize}
\end{story*}

\bigbreak

\begin{soln}(Cách 1)\footnotemark
    \textbf{Bước 1:} Giả sử \( n = p^r \). Khi đó \( d_i = p^{i-1} \), và:
    \[
        d_i \mid d_{i+1} + d_{i+2} \Leftrightarrow p^{i-1} \mid p^i + p^{i+1} = p^i(1 + p)
    \]
    rõ ràng đúng với mọi \( i \). Vậy mọi lũy thừa của số nguyên tố đều thỏa mãn.

    \textbf{Bước 2:} Giả sử \( n \) có ít nhất hai thừa số nguyên tố, gọi \( p < q \) là hai thừa số nguyên tố nhỏ nhất.

    Khi đó tồn tại đoạn gồm các ước:
    \[
        d_j = p^{j-1},\quad d_{j+1} = p^j,\quad d_{j+2} = q
    \]
    và ở cuối:
    \[
        d_{k-j-1} = \frac{n}{q},\quad d_{k-j} = \frac{n}{p^j},\quad d_{k-j+1} = \frac{n}{p^{j-1}}
    \]

    Từ đề bài:
    \[
        \frac{n}{q} \mid \left( \frac{n}{p^j} + \frac{n}{p^{j-1}} \right) \implies p^j \mid q(p + 1) \implies p \mid q
    \]
    điều này mâu thuẫn vì \( p \neq q \).

    \textbf{Kết luận:} \( n \) phải là lũy thừa của một số nguyên tố.
\end{soln}

\bigbreak

\begin{soln}(Cách 2)\footnotemark[\value{footnote}]
    \begin{claim*}
        \( d_i \mid d_{i+1} \) với mọi \( 1 \leq i \leq k - 1 \).
    \end{claim*}
    \begin{subproof}
        Ta chứng minh bằng quy nạp.

        Cơ sở: \( d_1 = 1 \implies d_1 \mid d_2 \) đúng.

        Giả sử \( d_{i-1} \mid d_i \), từ đề bài:
        \[
            d_{i-1} \mid d_i + d_{i+1} \implies d_{i-1} \mid d_{i+1}
        \]
        Kết hợp với giả thiết: \( d_i \mid d_{i+1} \).
    \end{subproof}

    \textbf{Bước cuối:} Dựa vào mệnh đề trên, mọi ước sau chia hết cho ước trước → mọi ước \( d_i \) là bội của \( d_2 \), suy ra \( n \) là lũy thừa của một số nguyên tố.

    \textbf{Kết luận:} Mọi số thỏa mãn là các lũy thừa của một số nguyên tố.
\end{soln}

\newpage

\begin{soln}(Cách 3)\footnotemark[\value{footnote}]
    \textbf{Bước 1:} Sử dụng tính chất đối xứng \( d_i d_{k+1-i} = n \). Đặt:
    \[
        d_{k-i-1} \mid d_{k-i} + d_{k-i+1} \Leftrightarrow \frac{n}{d_{i+2}} \mid \left( \frac{n}{d_{i+1}} + \frac{n}{d_i} \right)
    \]

    Nhân hai vế với \( d_i d_{i+1} d_{i+2} \) và rút gọn:
    \[
        d_i d_{i+1} \mid d_i d_{i+2} + d_{i+1} d_{i+2} \implies d_i \mid d_{i+1} d_{i+2} \quad (1)
    \]

    Mặt khác, từ đề bài:
    \[
        d_i \mid d_{i+1}^2 + d_{i+1} d_{i+2}
    \]

    Kết hợp với (1) suy ra:
    \[
        d_i \mid d_{i+1}^2
    \]

    \textbf{Bước 2:} Gọi \( p = d_2 \) là ước nguyên tố nhỏ nhất. Ta dùng quy nạp để chứng minh:

    \begin{claim*}
        \( p \mid d_i,\ \forall\ 2 \leq i \leq k - 1. \)
    \end{claim*}
    \begin{subproof}
        Cơ sở đúng. Giả sử \( p \mid d_j \) thì \( d_j \mid d_{j+1}^2 \implies p \mid d_{j+1} \) (vì \( p \) nguyên tố).
    \end{subproof}

    \textbf{Kết luận:} Nếu tồn tại ước nguyên tố \( q \ne p \) thì \( p \mid q \), vô lý. Do đó \( n \) là lũy thừa của một số nguyên tố.
\end{soln}

\footnotetext{\samepage \href{https://www.imo-official.org/problems/IMO2023SL.pdf}{Shortlist 2023 with solutions.}}

\end{document}