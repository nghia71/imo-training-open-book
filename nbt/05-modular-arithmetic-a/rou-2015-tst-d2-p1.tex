\documentclass[../05-modular-arithmetic-a.tex]{subfiles}

\begin{document}

\begin{example*}[\nameref{example:ROU-2015-TST-D2-P1}][\textbf{\nameref{definition:25M}}]
    Cho \( a \in \mathbb{Z} \) và \( n \in \mathbb{N}_{>0} \). Chứng minh rằng:
    \[
        \sum_{k=1}^{n} a^{\gcd(k,n)}
    \]
    luôn chia hết cho \( n \), trong đó \( \gcd(k,n) \) là ước chung lớn nhất của \( k \) và \( n \).
\end{example*}

\begin{story*}
    Phương pháp chính: nhóm các chỉ số \(k\) theo giá trị \(d = \gcd(k, n)\). Khi đó, mỗi giá trị \( a^d \) xuất hiện đúng \(\phi(n/d)\) lần trong tổng, nên ta có:
    \[
        \sum_{k=1}^{n} a^{\gcd(k,n)} = \sum_{d \mid n} \phi\left(\frac{n}{d}\right) a^d.
    \]
    
    Để chứng minh tổng này chia hết cho \(n\), ta xét các ước nguyên tố \(p\) của \(n\) và chứng minh tổng chia hết cho \(p^{v_p(n)}\) với mỗi \(p\). Từ đó suy ra tổng chia hết cho \(n\) theo định lý cơ bản của số học.
\end{story*}

\bigbreak

\begin{soln}
    \begin{lemma*}
        Với mọi số nguyên tố \( p \) và số nguyên \( j \ge 1 \), và với mọi \( x \in \mathbb{Z} \), ta có:
        \[
            \sum_{k = 0}^{j} \phi\left(p^k\right)\, x^{p^{j - k}} \equiv 0 \pmod{p^j}.
        \]
    \end{lemma*}

    \begin{subproof}
        Ta chứng minh bằng quy nạp theo \( j \).
        
        \textbf{Cơ sở \( j = 1 \):} Khi đó:
        \[
            \phi(1)\,x^p + \phi(p)\,x = x^p + (p - 1)x.
        \]
        Áp dụng định lý Fermat nhỏ: \( x^p \equiv x \pmod{p} \Rightarrow x^p + (p - 1)x \equiv px \equiv 0 \pmod{p} \).

        \textbf{Bước quy nạp:} Giả sử mệnh đề đúng với \( j \), chứng minh đúng với \( j + 1 \).  
        Dựa vào công thức \( \phi(p^k) = p^{k-1}(p - 1) \) với \( k \ge 1 \), các bước biến đổi sẽ cho ta:
        \[
            \sum_{k=0}^{j+1} \phi(p^k)\, x^{p^{j + 1 - k}} \equiv 0 \pmod{p^{j+1}}.
        \]
        (Chi tiết chứng minh có thể viết rõ thêm nếu cần, hoặc tham khảo sách Số học nâng cao).
    \end{subproof}

    Quay lại bài toán, với mỗi ước \( d \mid n \), số lần \( a^d \) xuất hiện là \( \phi(n/d) \). Vậy:
    \[
        \sum_{k=1}^n a^{\gcd(k,n)} = \sum_{d \mid n} \phi\left( \frac{n}{d} \right)\, a^d.
    \]

    Xét một ước nguyên tố \( p \mid n \), gọi \( j = v_p(n) \). Mỗi ước \( d \mid n \) có thể viết dưới dạng \( d = p^i d' \), với \( 0 \le i \le j \) và \( (d', p) = 1 \). Khi đó:
    \[
        \sum_{d \mid n} \phi\left(\frac{n}{d}\right)\, a^d
        = \sum_{\substack{d' \mid n \\ (d',p)=1}} \sum_{i=0}^j \phi\left( \frac{n}{p^i d'} \right)\, a^{p^i d'}.
    \]

    Với \( x = a^{d'} \), \( \phi(n/(p^i d')) = \phi(n/(p^j d'))\, \phi(p^{j - i}) \). Áp dụng bổ đề:
    \[
        \sum_{i = 0}^j \phi(p^{j - i})\, x^{p^i} \equiv 0 \pmod{p^j}.
    \]
    Nên toàn bộ tổng chia hết cho \( p^j \). Lặp lại với mọi \( p \mid n \), suy ra:
    \[
        \sum_{k=1}^n a^{\gcd(k,n)} \equiv 0 \pmod{n}.
    \]
\end{soln}

\end{document}