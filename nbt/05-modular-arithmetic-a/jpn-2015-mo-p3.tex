\documentclass[../05-modular-arithmetic-a.tex]{subfiles}

\begin{document}

\begin{exercise*}[\gls{JPN 2015 MO}1/P3]\label{example:JPN-2015-MO-P3}\textbf{[\nameref{definition:30M}]}
	Một dãy số nguyên dương \( \{a_n\}_{n=1}^{\infty} \) được gọi là \textit{tăng mạnh} nếu với mọi số nguyên dương \( n \), ta có:
	\[
		a_n < a_{n+1} < a_n + a_{n+1} < a_{n+2}.
	\]
	
	\begin{itemize}[topsep=0pt, partopsep=0pt, itemsep=0pt]
		\item[(a)] Chứng minh rằng nếu \( \{a_n\} \) là dãy tăng mạnh thì các số nguyên tố lớn hơn \( a_1 \) chỉ xuất hiện hữu hạn lần trong dãy.
		\item[(b)] Chứng minh rằng tồn tại dãy \( \{a_n\} \) tăng mạnh sao cho không có số nào chia hết cho bất kỳ số nguyên tố nào đã xuất hiện trong dãy.
	\end{itemize}
\end{exercise*}

\begin{remark*}
    (a) Quan sát rằng điều kiện \( a_n + a_{n+1} < a_{n+2} \) dẫn đến tốc độ tăng nhanh của dãy.
    Từ đó, có thể áp dụng định lý số nguyên tố và ước lượng mật độ số nguyên tố để loại trừ vô hạn trường hợp.

    (b) Ta có thể xây dựng dãy bằng phương pháp quy nạp. Tại mỗi bước, chọn \( a_{n+1} \) đủ lớn để tránh chia hết cho tất cả các ước nguyên tố của các số đã có.
    Hãy cân nhắc lựa chọn số nguyên tố mới, và kiểm tra xem điều kiện tăng mạnh có được giữ không.
\end{remark*}

\begin{story*}
	\textbf{Phần (a):} Điều kiện tăng mạnh \( a_n < a_{n+1} < a_n + a_{n+1} < a_{n+2} \) dẫn đến \( a_{n+2} > 2a_{n+1} \), nghĩa là dãy tăng ít nhất theo cấp số nhân với công bội lớn hơn 2.

	Vì vậy, \( a_n \) tăng rất nhanh, trong khi mật độ số nguyên tố giảm dần theo định lý số nguyên tố. Từ đó, số nguyên tố lớn hơn \( a_1 \) chỉ có thể xuất hiện trong số lượng hữu hạn phần tử của dãy.

	\textbf{Phần (b):} Có thể xây dựng dãy \( \{a_n\} \) bằng quy nạp sao cho mỗi số \( a_n \) không chia hết cho bất kỳ số nguyên tố nào đã xuất hiện trong dãy.

	Tại mỗi bước, ta chọn một số nguyên tố mới lớn hơn tổng của hai số cuối cùng trong dãy để đảm bảo điều kiện tăng mạnh. Nhờ vậy, mọi ước nguyên tố của \( a_n \) là số mới chưa từng xuất hiện.
\end{story*}

\end{document}