\documentclass[../05-modular-arithmetic-a.tex]{subfiles}

\begin{document}

\begin{example*}[\gls{EMC 2015}/P1]\label{example:EGMO-2015-P1}\textbf{[\nameref{definition:20M}]}
	Cho tập \( A = \{a, b, c\} \) gồm ba số nguyên dương.
	Chứng minh rằng tồn tại tập con \( B = \{x, y\} \subset A \) sao cho với mọi số nguyên dương lẻ \( m, n \), ta có:
	\[
		10 \mid x^m y^n - x^n y^m.
	\]
\end{example*}

\begin{story*}
    Bài toán yêu cầu chứng minh rằng với mọi bộ ba số nguyên dương \( A = \{a, b, c\} \), luôn tồn tại hai phần tử \( x, y \) sao cho biểu thức
    \[
        x^m y^n - x^n y^m
    \]
    luôn chia hết cho 10 với mọi số lẻ \( m, n \).

    Hướng giải gồm hai phần:
    \begin{itemize}[topsep=0pt, partopsep=0pt, itemsep=0pt]
        \item Chứng minh \( 2 \mid x^m y^n - x^n y^m \) bằng cách xét chẵn–lẻ của \( x \) và \( y \). Dù \( x, y \) đều lẻ hay có một số chẵn, đều cho kết quả chẵn.
        \item Với modulo 5, áp dụng nguyên lý Dirichlet:
        \begin{itemize}[topsep=0pt, partopsep=0pt, itemsep=0pt]
            \item Nếu có hai phần tử cùng dư modulo 5 hoặc có phần tử chia hết cho 5, thì dễ chọn cặp thỏa mãn.
            \item Nếu ba phần tử có dư khác nhau, thì tồn tại cặp đối nhau mod 5 như \( (1, 4) \) hoặc \( (2, 3) \), sao cho \( x + y \equiv 0 \Mod{5} \Rightarrow 5 \mid f(x, y) \).
        \end{itemize}
    \end{itemize}

    Kết hợp cả chia hết cho 2 và 5, suy ra \( 10 \mid x^m y^n - x^n y^m \).
\end{story*}

\bigbreak

\begin{soln}(Cách 1)\footnotemark
	Xét \( f(x, y) = x^m y^n - x^n y^m \). Nếu \( m = n \), ta có \( f(x, y) = 0 \), chia hết cho 10 với mọi \( x, y \), nên ta giả sử \( n > m \).

	Vì \( m, n \) lẻ \( \implies n - m \) chẵn. Khi đó:
	\[
		f(x, y) = x^m y^m (y^{n - m} - x^{n - m})
	= x^m y^m (y^2 - x^2) Q(x, y)
= x^m y^m (y - x)(y + x) Q(x, y),
	\]
	với \( Q(x, y) \in \mathbb{Z}[x, y] \).

	\textbf{Xét chia hết cho 2:} Nếu \( x \) hoặc \( y \) chẵn \( \implies 2 \mid f(x, y) \).  
	Nếu \( x, y \) đều lẻ, thì \( x \pm y \) chẵn \( \implies 2 \mid f(x, y) \) cho nên Luôn có \( 2 \mid f(x, y) \).

	\textbf{Xét chia hết cho 5:}

	\textit{Trường hợp 1:} Tồn tại phần tử trong \( A \) chia hết cho 5.  
	Chọn phần tử đó và một phần tử bất kỳ khác, khi đó \( 5 \mid x \) hoặc \( y \) cho nên \( 5 \mid f(x, y) \).

	\textit{Trường hợp 2:} Không phần tử nào chia hết cho 5 cho nên mỗi phần tử \( \not\equiv 0 \Mod{5} \).  
	Do \( A \) có 3 phần tử, mỗi phần dư \( \Mod{5} \in \{1, 2, 3, 4\} \). Áp dụng nguyên lý Dirichlet:

	\begin{itemize}[topsep=0pt, partopsep=0pt, itemsep=0pt]
		\item Nếu tồn tại hai phần tử cùng dư modulo 5: chọn cặp đó cho nên \( x \equiv y \Mod{5} \implies x - y \equiv 0 \Mod{5} \implies 5 \mid f(x, y) \).
		\item Nếu cả ba phần tử khác nhau modulo 5 cho nên chắc chắn tồn tại một trong hai cặp \( (1, 4) \) hoặc \( (2, 3) \) thuộc \( A \).  
		Vì \( 1 + 4 \equiv 0 \Mod{5} \), \( 2 + 3 \equiv 0 \Mod{5} \) cho nên chọn cặp đó: \( x + y \equiv 0 \Mod{5} \implies 5 \mid f(x, y) \).
	\end{itemize}

	\textbf{Kết luận:} Trong mọi trường hợp, tồn tại \( \{x, y\} \subset A \) sao cho:
	\[
		\boxed{10 \mid x^m y^n - x^n y^m \quad \text{với mọi số lẻ } m, n}
	\]
\end{soln}

\footnotetext{\href{https://emc.mnm.hr/wp-content/uploads/2015/12/EMC_2015_Seniors_ENG_Solutions.pdf}{Lời giải chính thức.}}

\end{document}