\documentclass[../05-modular-arithmetic-a.tex]{subfiles}

\begin{document}

\begin{exercise*}[\gls{HUN 2015 TST}/KMA/640]\label{example:HUN-2015-TST-KMA-640}\textbf{[\nameref{definition:30M}]}
    Hãy xác định tất cả các số nguyên tố \( p \) và các số nguyên dương \( n \) sao cho các số có dạng \( (k+1)^n - 2k^n \)
    với \( k = 1, 2, \ldots, p \) tạo thành một hệ đầy đủ các số dư modulo \( p \).
\end{exercise*}

\begin{remark*}
    Hãy thử khảo sát với các giá trị nhỏ của \( p \), chẳng hạn \( p = 2, 3, 5 \), và xem biểu thức \( (k + 1)^n - 2k^n \) có cho ra đủ \( p \) phần dư khác nhau modulo \( p \) hay không.
    Nếu không, ta loại trừ. Nếu có, cần kiểm tra xem tính chất đó có đúng với các giá trị lớn hơn hay không.
\end{remark*}

\begin{story*}
    Bài toán yêu cầu tìm tất cả các cặp \( (p, n) \) sao cho dãy:
    \[
        a_k = (k + 1)^n - 2k^n \quad \text{với } k = 1, 2, \dots, p,
    \]
    tạo thành một hệ đầy đủ các số dư modulo \( p \), tức là \( \{0, 1, \dots, p - 1\} \).

    Một số hướng tiếp cận:
    \begin{itemize}[topsep=0pt, partopsep=0pt, itemsep=0pt]
        \item Với \( k = p \), ta có \( (p + 1)^n \equiv 1^n \equiv 1 \pmod{p} \), còn \( p^n \equiv 0 \pmod{p} \), nên:
        \[
            a_p \equiv 1 \pmod{p}.
        \]
        Do đó, phần dư 1 luôn xuất hiện trong dãy.

        \item Để dãy là hệ đầy đủ, các phần tử \( a_k \mod{p} \) phải phân biệt. Do đó, ta cần kiểm tra injectivity (tính đơn ánh) của hàm \( f(k) \) modulo \( p \).

        \item Dễ thấy với \( p = 2 \), ta có duy nhất \( k = 1 \), và:
        \[
            a_1 = 2^n - 2 \equiv 0 \pmod{2},
        \]
        nên dãy \( \{0\} \) chính là hệ đầy đủ modulo 2. Vậy \( (p,n) = (2,n) \) luôn thỏa mãn.

        \item Với \( p = 3 \), ta có thể kiểm tra trực tiếp với vài giá trị \( n \) nhỏ.

        \item Với \( p > 2 \), cần kết hợp định lý Fermat nhỏ và phân tích hàm \( f(k) \) trên \( \mathbb{F}_p \), đặc biệt xem liệu có tồn tại \( k_1 \ne k_2 \) sao cho \( a_{k_1} \equiv a_{k_2} \pmod{p} \).
    \end{itemize}
\end{story*}

\end{document}