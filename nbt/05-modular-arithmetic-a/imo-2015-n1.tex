\documentclass[../05-modular-arithmetic-a.tex]{subfiles}

\begin{document}

\begin{example*}[\gls{IMO 2015}/N1]\label{example:IMO-2015-N1}\textbf{[\nameref{definition:30M}]}
    Xác định tất cả các số nguyên dương \( M \) sao cho dãy số \( a_0, a_1, a_2, \dots \) được xác định bởi  
    \[
        a_0 = M + \frac{1}{2}\quad \text{và}\quad a_{k+1} = a_k \lfloor a_k \rfloor \quad \text{với } k = 0, 1, 2, \dots
    \]
    chứa ít nhất một số nguyên.
\end{example*}

\begin{story*}
    Để dãy \( a_k \) chứa một số nguyên, cần có \( a_k \in \mathbb{Z} \) với \( a_k = a_{k-1} \lfloor a_{k-1} \rfloor \).

    Đặt \( b_k = 2a_k \), ta có:
    \[
        b_0 = 2M + 1, \quad b_{k+1} = b_k \left\lfloor \frac{b_k}{2} \right\rfloor.
    \]
    Nếu \( b_0 \) là số lẻ, thì tất cả các \( b_k \) đều là số lẻ, và:
    \[
        b_{k+1} = \frac{b_k(b_k - 1)}{2}.
    \]

    Khi đó, ta dùng quy nạp để chứng minh rằng với \( M = 1 \), ta có \( b_k \equiv 3 \Mod{2^m} \) với mọi \( m \), nên tồn tại \( k \) sao cho \( b_k = 3 \Rightarrow a_k = \frac{3}{2} \). Điều này xảy ra duy nhất khi \( M = 1 \), vì các giá trị khác sẽ không giữ được tính chất đồng dư này.

    Do đó, giá trị duy nhất thỏa mãn điều kiện là \( \boxed{M = 1} \).
\end{story*}

\bigbreak

\begin{soln}(Cách 1)\footnotemark
	Đặt \( b_k = 2a_k \implies b_{k+1} = b_k \left\lfloor \frac{b_k}{2} \right\rfloor \).

	Vì \( b_0 = 2a_0 = 2M + 1 \) là số nguyên lẻ, suy ra mọi \( b_k \) là số nguyên, và nếu giả sử dãy \( a_k \) không bao giờ là số nguyên, thì tất cả \( b_k \) phải là số lẻ.

	Khi đó:
	\[
		\left\lfloor \frac{b_k}{2} \right\rfloor = \frac{b_k - 1}{2}
	\implies b_{k+1} = b_k \cdot \frac{b_k - 1}{2} = \frac{b_k(b_k - 1)}{2}. \tag{1}
	\]

	Ta sẽ chứng minh:
	\begin{claim*}
		Với mọi \( k \ge 0 \) và \( m \ge 1 \), ta có \( b_k \equiv 3 \Mod{2^m} \).
	\end{claim*}
	\begin{subproof}
		Chứng minh bằng \nameref{theorem:induction-principle} theo \( m \).

		\textbf{Bước cơ sở:} \( m = 1 \implies b_k \) là số lẻ, nên \( b_k \equiv 1 \) hoặc \( 3 \Mod{2} \), nhưng vì \( b_0 = 2M + 1 \equiv 3 \Mod{2} \) nếu \( M = 1 \), nên đúng.

		\textbf{Bước quy nạp:} Giả sử \( b_k \equiv 3 \Mod{2^m} \implies b_k = 2^m d_k + 3 \) với \( d_k \in \mathbb{Z} \).

		Khi đó:
		\[
		b_{k+1} = \frac{b_k(b_k - 1)}{2} = \frac{(2^m d_k + 3)(2^m d_k + 2)}{2}
		\equiv 3 \cdot 2^m d_k + 3 \Mod{2^{m+1}}.
		\]

		Suy ra \( d_k \) phải chẵn \( \implies b_{k+1} \equiv 3 \Mod{2^{m+1}} \). Khẳng định được chứng minh.
	\end{subproof}

	Vì \( b_k \equiv 3 \Mod{2^m} \) với mọi \( m \), nên \( b_k = 3 \implies a_k = \frac{3}{2} \implies M = 1 \).

	\textbf{Kết luận:}
	\[
		\boxed{M = 1}
	\]
	là giá trị duy nhất sao cho dãy \( a_k \) chứa một số nguyên.
\end{soln}

\footnotetext{\samepage \href{https://www.imo-official.org/problems/IMO2015SL.pdf}{Shortlist 2015 with solutions.}}

\end{document}