\documentclass[../05-modular-arithmetic-a.tex]{subfiles}

\begin{document}

\begin{example*}[\gls{GER 2015 MO}/P4]\label{example:GER-2015-MO-P4}[\textbf{\nameref{definition:20M}}]
	Cho số nguyên dương \( k \). Định nghĩa \( n_k \) là số có dạng thập phân \( 70\underbrace{00\ldots0}_{k \text{ chữ số } 0}1 \).
	Chứng minh rằng:
    \begin{itemize}[topsep=0pt, partopsep=0pt, itemsep=0pt]
        \item Không có số nào trong các số \( n_k \) chia hết cho 13.
        \item Có vô số số \( n_k \) chia hết cho 17.
    \end{itemize}
\end{example*}

\begin{story*}
    Ta viết lại:
    \[
        n_k = 70\underbrace{00\ldots0}_{k\text{ chữ số }0}1 = 7 \cdot 10^{k+1} + 1.
    \]

    \begin{itemize}[topsep=0pt, partopsep=0pt, itemsep=0pt]
        \item Với modulo \(13\): Ta cần \(7 \cdot 10^{k+1} \equiv -1 \pmod{13} \Rightarrow 10^{k+1} \equiv 11 \pmod{13}\).  
        Nhưng dãy \( 10^t \bmod{13} \) có chu kỳ 6 với giá trị: \(10, 9, 12, 3, 4, 1\), không bao giờ bằng 11.  
        Vậy không có \( k \) nào thỏa mãn \( n_k \equiv 0 \pmod{13} \).

        \item Với modulo \(17\): Ta cần \(7 \cdot 10^{k+1} \equiv -1 \pmod{17} \Rightarrow 10^{k+1} \equiv -7^{-1} \pmod{17}\).  
        Vì \( 7^{-1} \equiv 5 \pmod{17} \), nên \( -7^{-1} \equiv 12 \pmod{17} \).  
        Dãy \( 10^t \bmod{17} \) có chu kỳ 16, và \(10^{15} \equiv 12 \pmod{17}\).  

        Vậy \( k + 1 \equiv 15 \pmod{16} \Rightarrow k \equiv 14 \pmod{16} \).  
        Suy ra có vô hạn \( k \) sao cho \( n_k \equiv 0 \pmod{17} \).
    \end{itemize}
\end{story*}

\bigbreak

\begin{soln}\footnotemark
	\[
		n_k = 7 \cdot 10^{k+1} + 1.
	\]
	
	\textbf{(a)} Ta cần xét \(\;n_k\equiv 0\pmod{13}\iff 7\cdot 10^{k+1}\equiv -1\pmod{13}\).  
	
	Khi \(\;7^{-1}\equiv 2\pmod{13}\;\), điều này tương đương \(10^{k+1}\equiv 11\pmod{13}\).  

	Dãy \(\{10^t\bmod13\}\) có chu kỳ 6: 
	\[
		10^1\equiv10,\;10^2\equiv9,\;10^3\equiv12,\;10^4\equiv3,\;10^5\equiv4,\;10^6\equiv1,\dots
	\]
	không bao giờ bằng 11. Vậy không tồn tại \(k\) để \(n_k\equiv0\pmod{13}\).  

	\textbf{(b)} Tương tự, \(\;n_k\equiv0\pmod{17}\iff 7\cdot 10^{k+1}\equiv -1\pmod{17}\iff 10^{k+1}\equiv -7^{-1}\pmod{17}\).  
	
	Bởi \(7^{-1}\equiv5\pmod{17}\implies -7^{-1}\equiv -5\equiv12\pmod{17}\).  
	
	Xem \(\{10^t\bmod17\}\) có chu kỳ 16, tìm \(10^{k+1}\equiv12\). Quả thật \(10^{15}\equiv12\pmod{17}\).
	
	Vậy \(k+1\equiv15\pmod{16}\implies k\equiv14\pmod{16}\). Suy ra vô hạn \(k\) thoả mãn, nên vô hạn \(n_k\) chia hết cho 17.
\end{soln}

\footnotetext{\href{https://artofproblemsolving.com/community/c6h1442524p8218443}{Dựa theo giải của \textbf{RagvaloD}.}}
\end{document}