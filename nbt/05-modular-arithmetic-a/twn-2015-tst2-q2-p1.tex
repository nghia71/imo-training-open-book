\documentclass[../05-modular-arithmetic-a.tex]{subfiles}

\begin{document}

\begin{exercise*}[\gls{TWN 2015 TST}2/Q2/P1]\label{example:TWN-2015-TST2-Q2-P1}\textbf{[\nameref{definition:30M}]}
    Cho dãy số \( \{a_n\} \) xác định bởi:
    \[
        a_{n+1} = a_n^3 + 103,\quad \text{với } n = 1, 2, 3, \dots
    \]
    
    Chứng minh rằng có nhiều nhất một số hạng \( a_n \) là số chính phương.    
\end{exercise*}

\begin{remark*}
    Hãy thử kiểm tra các giá trị nhỏ của \( a_1 \), rồi khảo sát tốc độ tăng trưởng của dãy. Sử dụng mâu thuẫn: nếu hai số chính phương xuất hiện trong dãy, có thể dẫn đến bất khả về dạng số học.
\end{remark*}

\begin{story*}
    Dãy được định nghĩa bằng công thức đệ quy \( a_{n+1} = a_n^3 + 103 \). Vì khối lập phương tăng rất nhanh, nên các số hạng trong dãy sẽ phát triển rất lớn. Mục tiêu là chứng minh không thể có nhiều hơn một số chính phương trong dãy.

    Ta xét:
    \begin{itemize}[topsep=0pt, partopsep=0pt, itemsep=0pt]
        \item Nếu \( a_n \) là số chính phương, thì \( a_{n+1} = a_n^3 + 103 \) có cấu trúc cụ thể.
        \item Nếu \( a_n = x^2 \), thì \( a_{n+1} = x^6 + 103 \). Ta muốn biết khi nào biểu thức này lại là chính phương.
        \item Nếu \( x^6 + 103 = y^2 \), thì \( y^2 - x^6 = 103 \). Đây là phương trình Diophantine quan trọng và chỉ có nghiệm hữu hạn.
    \end{itemize}
\end{story*}

\end{document}