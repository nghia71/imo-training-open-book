\documentclass[../04-diophantine-equations.tex]{subfiles}

\begin{document}

\begin{exercise*}[\gls{FRA 2015 TST}/3/P6]\label{example:FRA-2015-TST3-P6}[\textbf{\nameref{definition:25M}}]
    Tìm tất cả các cặp số nguyên \( (x, y) \) thỏa mãn:
    \[
        x^2 = y^2(y^4 + 2y^2 + x).
    \]
\end{exercise*}

\begin{remark*}
    Gợi ý: Thử cô lập \(x\) để bộc lộ tính chất của \(x\) liên quan đến đa thức bậc cao của \(y\).  
    Có thể suy luận \(y\) không quá lớn hoặc dùng cách phân tích trường hợp \((y = 0)\), \((y \neq 0)\) và khống chế cỡ của \(x\) để tìm nghiệm hữu hạn.
\end{remark*}

\begin{story*}
    Ta bắt đầu bằng cách đưa phương trình về dạng quen thuộc hơn để tách biến và phân tích điều kiện.  
    Đặt lại phương trình:
    \[
        x^2 = y^6 + 2y^4 + xy^2.
    \]
    Chuyển vế:
    \[
        x^2 - xy^2 = y^6 + 2y^4 \Rightarrow x(x - y^2) = y^4(y^2 + 2).
    \]
    Từ đây, ta nhận được phương trình dạng tích \(x(x - y^2) = \text{đa thức theo } y\).  
    Do vế phải không âm, ta xét giá trị \(y = 0\), sau đó giả sử \(y \ne 0\), phân tích từng giá trị nhỏ của \(y\), vì vế phải tăng nhanh theo \(y\).  
    Đồng thời, vì phương trình chỉ chứa lũy thừa chẵn hoặc số dương, số nghiệm là hữu hạn và có thể liệt kê bằng phương pháp thử.
\end{story*}

\end{document}