\documentclass[../04-diophantine-equations.tex]{subfiles}

\begin{document}

\begin{exercise*}[\gls{FRA 2015 TST}/3/P9]\label{example:FRA-2015-TST3-P9}[\textbf{\nameref{definition:35M}}]
    Tìm tất cả các bộ ba \( (p, x, y) \), trong đó \( p \) là số nguyên tố và \( x, y \) là hai số nguyên dương, sao cho:
    \[
        x^{p-1} + y \quad \text{và} \quad x + y^{p-1}
    \]
    đều là các luỹ thừa của \( p \).
\end{exercise*}

\begin{remark*}
    Gợi ý: Khảo sát trước các trường hợp nhỏ cho \(x\) hay \(y\) (chẳng hạn \(1\) hoặc \(p\)) và kiểm tra tính chất đồng thời của hai biểu thức trở thành lũy thừa của \(p\).  
    Có thể dùng định lý LTE hoặc cân nhắc modulo \(p\) và modulo \(p^2\) để khống chế số mũ và tính chia hết.
\end{remark*}

\begin{story*}
    Ta có hai biểu thức: \( x^{p-1} + y \) và \( x + y^{p-1} \), cả hai đều là lũy thừa của \( p \). Vì \( p \) là số nguyên tố nên các lũy thừa của \( p \) có dạng \( p^k \).  
    Một hướng tiếp cận là đặt:
    \[
        x^{p-1} + y = p^a, \quad x + y^{p-1} = p^b,
    \]
    và khai thác các tính chất đồng dư:
    \[
        x^{p-1} \equiv 1 \pmod{p} \quad \text{(theo định lý Fermat nếu } p \nmid x).
    \]
    Khi đó, \( x^{p-1} + y \equiv 1 + y \equiv 0 \pmod{p} \Rightarrow y \equiv -1 \pmod{p} \), và tương tự với vế còn lại.  
    Ta thử các giá trị nhỏ như \( x = 1 \), \( y = 1 \), hoặc \( x = y \), từ đó kiểm tra tính chất đồng thời và rút ra trường hợp duy nhất.  
    Với các kỹ thuật như LTE (nếu áp dụng được), ta cũng có thể kiểm soát số mũ \( a \), \( b \), dẫn đến nghiệm duy nhất hoặc hữu hạn nghiệm.
\end{story*}

\end{document}