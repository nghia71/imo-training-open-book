\documentclass[../04-diophantine-equations.tex]{subfiles}

\begin{document}

\begin{exercise*}[\gls{RUS 2015 TST}/D10/P3]\label{example:RUS-2015-TST-D10-P3}[\textbf{\nameref{definition:30M}}]
	Tìm tất cả các số nguyên \( k \) sao cho tồn tại vô số bộ ba số nguyên \( (a, b, c) \) thỏa mãn:
	\[
		(a^2 - k)(b^2 - k) = c^2 - k.
	\]
\end{exercise*}

\begin{remark*}
	Gợi ý: Hãy xét một cách xây dựng (hoặc tham số hoá) các nghiệm \((a,b,c)\) - thí dụ, giả sử \((a^2-k)=\alpha\), \((b^2-k)=\beta\), thì \(\alpha\beta=c^2-k\).
	Cần tìm \(\alpha,\beta\) để vô hạn \((a,b,c)\) xuất hiện. Kiểm tra cách \((\alpha,\beta)=(t^2,\dots)\) hoặc \(\alpha=\beta\).
\end{remark*}

\begin{story*}
    Ta đặt \(\alpha = a^2 - k\), \(\beta = b^2 - k\), khi đó phương trình trở thành:
    \[
        \alpha \beta = c^2 - k.
    \]
    Nếu chọn \( \alpha = \beta \Rightarrow a = \pm b \), ta có:
    \[
        (a^2 - k)^2 = c^2 - k.
    \]
    Dễ thấy với \( k = 0 \), phương trình trở thành \( a^2 b^2 = c^2 \), có vô hạn nghiệm.  
    Nếu \( k \ne 0 \), biểu thức \( c^2 = (a^2 - k)^2 + k \) hiếm khi là số chính phương.  
    Do đó, giá trị duy nhất thoả mãn là \( \boxed{k = 0} \).
\end{story*}

\end{document}