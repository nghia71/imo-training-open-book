\documentclass[../04-diophantine-equations.tex]{subfiles}

\begin{document}

\begin{exercise*}[\gls{ROU 2014 MO}/G7/P1]\label{example:ROU-2014-MO-G7-P1}[\textbf{\nameref{definition:25M}}]
    Tìm tất cả các số nguyên tố \( p \) và \( q \), với \( p \le q \), sao cho
    \[
        p(2q + 1) + q(2p + 1) = 2(p^2 + q^2).
    \]
\end{exercise*}

\begin{remark*}
    Gợi ý: Thử xét phương trình theo modulo \(p\) hoặc \(q\), rồi phân tích trường hợp nhỏ cho các cặp số nguyên tố.
\end{remark*}

\begin{story*}
    Ta bắt đầu bằng cách phân tích và rút gọn biểu thức trong phương trình:
    \[
        p(2q + 1) + q(2p + 1) = 2pq + p + 2pq + q = 4pq + p + q,
    \]
    và vế phải là:
    \[
        2(p^2 + q^2).
    \]
    Do đó, phương trình tương đương với:
    \[
        4pq + p + q = 2(p^2 + q^2).
    \]

    Chuyển vế, ta có:
    \[
        2(p^2 + q^2 - 2pq) = p + q.
    \]
    Nhận ra rằng biểu thức bên trái là \(2(p - q)^2\), nên:
    \[
        2(p - q)^2 = p + q.
    \]
    Đây là phương trình Diophantine rất mạnh: vế trái là bội số của bình phương, vế phải là tổng hai số nguyên tố.

    Ta nhận ra chỉ có rất ít giá trị nhỏ của \(p, q\) thoả mãn phương trình này. Thử các cặp nhỏ \((p, q)\) với \(p \le q\), như \((2, 2), (2, 3), (3, 5), (5, 7)\), v.v. sẽ giúp ta tìm được nghiệm.

    Kết luận: bài toán có thể giải quyết bằng cách rút gọn đại số và thử hữu hạn trường hợp cho các số nguyên tố nhỏ.
\end{story*}

\end{document}