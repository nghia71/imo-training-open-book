\documentclass[../04-diophantine-equations.tex]{subfiles}

\begin{document}

\begin{example*}[\gls{CAN 2015 QRC}/P1]\label{example:CAN-2015-QRC-P1}[\textbf{\nameref{definition:25M}}]
	Tìm tất cả nghiệm nguyên của phương trình:
	\[
		7x^2y^2 + 4x^2 = 77y^2 + 1260.
	\]	
\end{example*}

\begin{story*}
    Ta phân tích phương trình theo hướng đưa về tích hai biểu thức. Nhận thấy rằng tất cả các hệ số đều chia hết cho 7 ngoại trừ hạng tử \(4x^2\), ta suy ra \(x\) phải chia hết cho 7 để phương trình có thể chia hết cho 7.  
    Sau khi chuyển vế và nhóm các hạng tử hợp lý, ta đưa được phương trình về dạng:
    \[
        (x^2 - 11)(7y^2 + 4) = 1216.
    \]
    Từ đây, vì vế phải là một số cố định, nên số lượng giá trị \(x\) và \(y\) nguyên dương khả dĩ là hữu hạn. Ta thử tất cả các ước của 1216 và tìm được các nghiệm cụ thể \((x, y) = (\pm 7, \pm 2)\).
\end{story*}

\bigbreak

\begin{soln}\footnotemark
	Nhận thấy rằng tất cả các hệ số đều chia hết cho 7 ngoại trừ 4, nên \( x \) phải chia hết cho 7.

	Biến đổi phương trình:
	\[
		7x^2y^2 + 4x^2 = 77y^2 + 1260
		\quad \Rightarrow \quad
		(x^2 - 11)(7y^2 + 4) = 1216.
	\]
	
	Vì vế phải là hằng số, ta chỉ cần xét một số hữu hạn giá trị \( x \). Giả sử \( x = 7t \), thì:
	\[
		x^2 = 49t^2 \leq 1216 + 77y^2 + 4x^2 \Rightarrow x^2 < 122 \Rightarrow |x| < 11 \Rightarrow x \in \{0, \pm7\}.
	\]

	\begin{itemize}[topsep=0pt, partopsep=0pt, itemsep=0pt]
	    \item Với \( x = 0 \Rightarrow 0 = 77y^2 + 1260 \), vô lý.
	    \item Với \( x = \pm7 \):
	    \[
	        (x^2 - 11)(7y^2 + 4) = 1216 \Rightarrow (49 - 11)(7y^2 + 4) = 1216 \Rightarrow 38(7y^2 + 4) = 1216.
	    \]
	    \[
	        7y^2 + 4 = \frac{1216}{38} = 32 \Rightarrow 7y^2 = 28 \Rightarrow y^2 = 4 \Rightarrow y = \pm2.
	    \]
	\end{itemize}

	Kết luận, các nghiệm nguyên là \( (x, y) = (\pm 7, \pm 2) \).
\end{soln}

\footnotetext{\href{https://cms.math.ca/wp-content/uploads/2019/07/2015official_solutions.pdf}{Lời giải chính thức.}}

\end{document}