\documentclass[../04-diophantine-equations.tex]{subfiles}

\begin{document}

\begin{example*}[\gls{MEMO 2015}/T/P7]\label{example:MEMO-2015-T-P7}\textbf{[\nameref{definition:25M}]}
	Tìm tất cả các cặp số nguyên dương \( (a, b) \) sao cho:
	\[
		a! + b! = a^b + b^a.
	\]	
\end{example*}

\begin{story*}
    Ta bắt đầu bằng cách xét các trường hợp nhỏ, chẳng hạn \( a = b \), hoặc một trong hai bằng 1.  
    Khi đó phương trình trở nên đơn giản và có thể kiểm tra trực tiếp.

    Với các trường hợp còn lại — đặc biệt khi \( 1 < a < b \) — ta so sánh tốc độ tăng của \( a! \) và \( a^b \) (giai thừa và lũy thừa).  
    Do \( a^b \) tăng nhanh hơn \( a! \), tương tự với \( b \), ta kỳ vọng vế phải lớn hơn vế trái.

    Để phản bác các trường hợp không tầm thường, ta có thể sử dụng phân tích p-adic:  
    Nếu tồn tại số nguyên tố \( p \mid a \), thì \( p \mid b! \Rightarrow p \mid b^a \), và do đó \( p \mid a^b + b^a \).  
    Ta tính số mũ \( p \) trong từng vế và chỉ ra rằng không thể có sự bằng nhau — dẫn đến mâu thuẫn.

    Kết luận, chỉ có vài nghiệm nhỏ duy nhất.
\end{story*}

\bigbreak

\begin{soln}\footnotemark
	\textbf{Trường hợp \(a = b\):} Khi đó:
	\[
		2a! = 2a^a \quad \Rightarrow \quad a! = a^a.
	\]
	Với \( a \ge 2 \), ta có \( a! < a^a \), nên không thỏa mãn.  
	Chỉ có \( a = 1 \Rightarrow (a, b) = (1, 1) \) là nghiệm.

	\textbf{Trường hợp \(a = 1\):}
	\[
		1! + b! = 1 + b! = 1^b + b^1 = 1 + b \Rightarrow b! = b \Rightarrow b = 2.
	\]
	Suy ra \( (a, b) = (1, 2) \) là nghiệm. Tương tự, \( b = 1 \Rightarrow (a, b) = (2, 1) \) cũng là nghiệm.

	\textbf{Giả sử \(1 < a < b\):} Khi đó:
	\[
		a! + b! < a^b + b^a,
	\]
	do \( a! < a^b \) và \( b! < b^a \), nên phương trình không thể đúng.

	Một cách khác: giả sử \( (a, b) \) là nghiệm với \( 1 < a < b \), xét một số nguyên tố \( p \mid a \). Khi đó:
	\[
		p \mid b! \Rightarrow p \mid b^a \Rightarrow p \mid a^b + b^a.
	\]
	Xét số mũ \( p \) hai vế:
	\begin{itemize}[topsep=0pt, partopsep=0pt, itemsep=0pt]
	    \item Vế phải: \( \nu_p(a^b + b^a) \ge a \).
	    \item Vế trái: \( \nu_p(a! + b!) = \nu_p(a!) \) (vì \( b! \) chia hết cho \( a! \)).  
	    Nhưng \( \nu_p(a!) = \sum_{k=1}^\infty \left\lfloor \frac{a}{p^k} \right\rfloor < a \).
	\end{itemize}
	Mâu thuẫn, nên loại trường hợp \( 1 < a < b \).

	\textbf{Kết luận:} Các nghiệm duy nhất là:
	\[
		\boxed{(a, b) \in \{(1, 1), (1, 2), (2, 1)\}}.
	\]
\end{soln}

\footnotetext{\href{http://memo2015.dmfa.si/files/solutions.pdf}{Lời giải chính thức.}}

\end{document}