\documentclass[../04-diophantine-equations.tex]{subfiles}

\begin{document}

\begin{exercise*}[\gls{GBR 2015 TST}/F1/P2]\label{example:GBR-2015-TST-F1-P2}[\textbf{\nameref{definition:20M}}]
    Cho dãy số nguyên \( (a_n)_{n \ge 0} \) thỏa mãn:
    \[
        a_0 = 1, \quad a_1 = 3, \quad \text{và} \quad a_{n+2} = 1 + \left\lfloor \frac{a_{n+1}^2}{a_n} \right\rfloor \text{ với mọi } n \ge 0.
    \]
    
    Chứng minh rằng với mọi \(n \ge 0\), ta có:
    \[
        a_n a_{n+2} - a_{n+1}^2 = 2^n.
    \]
\end{exercise*}

\begin{remark*}
    Gợi ý: Thử áp dụng quy nạp theo \(n\), và phân tích biểu thức \(a_{n+2} - 1 = \left\lfloor \dfrac{a_{n+1}^2}{a_n} \right\rfloor\) để liên hệ với \(a_n\),
    từ đó tính sai khác \(a_n a_{n+2} - a_{n+1}^2\) và chứng minh bằng lũy thừa của 2.
\end{remark*}

\begin{story*}
    Đặt mục tiêu chứng minh công thức truy hồi:
    \[
        a_n a_{n+2} - a_{n+1}^2 = 2^n \quad \text{với mọi } n \ge 0.
    \]
    Phương pháp tự nhiên là sử dụng quy nạp theo \(n\). Giả sử công thức đúng với một chỉ số \(n\), ta cần chứng minh rằng nó đúng với \(n+1\).

    Biểu thức \(a_{n+2} = 1 + \left\lfloor \dfrac{a_{n+1}^2}{a_n} \right\rfloor\) cho phép ta viết lại:
    \[
        a_n a_{n+2} = a_n + \left\lfloor \dfrac{a_{n+1}^2}{a_n} \right\rfloor \cdot a_n.
    \]
    Trừ đi \(a_{n+1}^2\) ở hai vế sẽ triệt tiêu phần nguyên \( \left\lfloor \cdots \right\rfloor \), còn lại là phần dư.

    Khi đó hiệu \(a_n a_{n+2} - a_{n+1}^2\) chính là số dư của phép chia \(a_{n+1}^2 \bmod a_n\), và với cơ sở ban đầu là:
    \[
        a_0 = 1, \quad a_1 = 3 \Rightarrow a_2 = 1 + \left\lfloor \dfrac{9}{1} \right\rfloor = 10 \Rightarrow a_0 a_2 - a_1^2 = 1 \cdot 10 - 9 = 1 = 2^0,
    \]
    ta có thể tiến hành quy nạp để chứng minh công thức đúng với mọi \(n\).
\end{story*}

\end{document}