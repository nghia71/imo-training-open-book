\documentclass[../04-diophantine-equations.tex]{subfiles}

\begin{document}

\begin{example*}[\gls{KOR 2015 MO}/P1]\label{example:KOR-2015-MO-P1}\textbf{[unrated]}
	Với mỗi số nguyên dương \( m \), \( (x, y) \) là một cặp số nguyên dương thỏa mãn hai điều kiện:
	\begin{enumerate}[topsep=0pt, partopsep=0pt, itemsep=0pt]
		\item[(i)] \( x^2 - 3y^2 + 2 = 16m \),
		\item[(ii)] \( 2y \le x - 1 \).
	\end{enumerate}
	Chứng minh rằng số lượng các cặp như vậy là số chẵn hoặc bằng 0.
\end{example*}

\begin{soln}(Cách 1)\footnotemark
	Nếu không tồn tại nghiệm nào, ta có số nghiệm là \( 0 \), thỏa yêu cầu.  
	Giả sử tồn tại một nghiệm \( (u, v) \in \mathbb{Z}_{>0}^2 \) thỏa mãn hai điều kiện.
	
	Định nghĩa ánh xạ:
	\[
		(u, v) \longmapsto (u', v') = (2u - 3v,\ u - 2v).
	\]
	
	\textit{Bước 1.} \( (u', v') \in \mathbb{Z}_{>0}^2 \).  
	Từ điều kiện \( u - 1 \ge 2v \implies u \ge 2v + 1 \), ta có:
	\[
		u' = 2u - 3v \ge 2(2v + 1) - 3v = v + 2 \ge 3,\quad v' = u - 2v \ge 1.
	\]
	
	\textit{Bước 2.} Ánh xạ bảo toàn phương trình:
	\begin{align*}
		(u')^2 - 3(v')^2 + 2 &= (2u - 3v)^2 - 3(u - 2v)^2 + 2
		&= u^2 - 3v^2 + 2 = 16m.
	\end{align*}
	
	\textit{Bước 3.} Ánh xạ bảo toàn bất đẳng thức:
	\[
		2v' \le u' - 1 \Leftrightarrow 2(u - 2v) \le (2u - 3v) - 1 \Leftrightarrow v \ge 1.
	\]
	
	\textit{Bước 4.} Ánh xạ là một \nameref{definition:involution-function}:
	\[
		T(T(u, v)) = (u, v).
	\]
	
	\textit{Bước 5.} Không có điểm bất biến (tức nghiệm cố định).
	Nếu \( (u, v) = (2u - 3v,\ u - 2v) \implies u = 3v \), thay vào:
	\[
		x = 3v \implies x^2 - 3y^2 + 2 = 6v^2 + 2 = 16m
		\implies v^2 = \frac{16m - 2}{6} \implies v^2 \equiv 5 \pmod{8},\ \text{vô lý.}
	\]
	
	Do đó, ánh xạ chia tập nghiệm thành các cặp phân biệt, nên số nghiệm là số chẵn hoặc 0.
\end{soln}

\footnotetext{\href{https://artofproblemsolving.com/community/c6h1158057p5501392}{Lời giải chính thức.}}

\end{document}