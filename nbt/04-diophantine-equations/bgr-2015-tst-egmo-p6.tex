\documentclass[../04-diophantine-equations.tex]{subfiles}

\begin{document}

\begin{exercise*}[\gls{BGR 2015 EGMO TST}/P6]\label{example:BGR-2015-EGMO-TST-P6}[\textbf{\nameref{definition:30M}}]
	Chứng minh rằng với mọi số nguyên dương \( n \geq 3 \),
	tồn tại \( n \) số nguyên dương phân biệt sao cho tổng các lập phương của chúng cũng là một lập phương hoàn hảo.
\end{exercise*}

\begin{remark*}
	Gợi ý: Thử xây dựng một họ các bộ \( n \) số (có thể đồng dư hoặc biến thiên theo tham số) sao cho tổng lập phương thu được là \( (\text{một giá trị tham số})^3 \).  
	Các ví dụ quen thuộc là các “nhóm” số mà tổng lập phương khéo léo triệt tiêu và cộng dồn thành một khối lập phương.
\end{remark*}

\begin{story*}
    Ý tưởng chính là xây dựng một tập \( n \) số sao cho tổng lập phương của chúng bằng \( a^3 \) với một số nguyên \( a \).  
    Ta có thể bắt đầu từ các ví dụ cổ điển như \( 1^3 + 2^3 + \cdots + k^3 = \left( \frac{k(k+1)}{2} \right)^2 \), mặc dù kết quả đó là một bình phương, không phải lập phương.  
    Một hướng tiếp cận hiệu quả hơn là tìm một bộ ba \( (a,b,c) \) sao cho \( a^3 + b^3 + c^3 = d^3 \) với \( d \) nguyên, rồi xây dựng các bộ lớn hơn bằng cách thêm các số có tổng lập phương bằng 0 — ví dụ \( x+y+z=0 \Rightarrow x^3 + y^3 + z^3 = 3xyz \), từ đó điều chỉnh lại để tổng toàn bộ là lập phương.  
    Kỹ thuật parametric hoặc nhân thêm hằng số để đảm bảo tích số cuối cùng là lập phương sẽ giúp xây dựng nghiệm cho mọi \( n \geq 3 \).
\end{story*}

\end{document}