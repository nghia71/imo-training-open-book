\documentclass[../../imo-training-open-book.tex]{subfiles}

\begin{document}

\section{Lý thuyết}

\begin{theorem*}[Định lý phần dư Trung Hoa]
    \label{theorem:chinese-remainder-theorem}
    Giả sử các số \( n_1, n_2, \dots, n_k \) đôi một nguyên tố cùng nhau, và \( a_1, a_2, \dots, a_k \) là các số nguyên tùy ý. Khi đó hệ phương trình đồng dư sau:
    \[
        \begin{aligned}
            x &\equiv a_1 \Mod{n_1} \\
            &\vdots \\
            x &\equiv a_k \Mod{n_k}
        \end{aligned}
    \]
    luôn có nghiệm.

    Hơn nữa, nếu \( x_1 \) và \( x_2 \) đều là nghiệm của hệ, thì chúng đồng dư modulo
    \[
        N = \prod_{i=1}^k n_i, \quad \text{tức là } x_1 \equiv x_2 \Mod{N}.
    \]
\end{theorem*}

\newpage

\section{Các ví dụ}

\subfile{./09-contruction-methods/egmo-2015-p5.tex} \newpage
\subfile{./09-contruction-methods/ger-2015-tst-p2.tex} \newpage
\subfile{./09-contruction-methods/irn-2015-mo-n1.tex} \newpage
\subfile{./09-contruction-methods/irn-2015-mo-n5.tex} \newpage
\subfile{./09-contruction-methods/irn-2015-tst-d1-p3.tex} \newpage
\subfile{./09-contruction-methods/irn-2015-tst-d2-p1.tex} \newpage
\subfile{./09-contruction-methods/pol-2015-mo-p6.tex} \newpage
\subfile{./09-contruction-methods/rou-2015-tst-d1-p3.tex} \newpage
\subfile{./09-contruction-methods/usa-2015-tstst-p3.tex} \newpage
\subfile{./09-contruction-methods/usa-2015-tstst-p5.tex} \newpage
\subfile{./09-contruction-methods/usa-2015-tst-p2.tex} \newpage

\section{Bài tập}

\subfile{./09-contruction-methods/gbr-2015-mo-p2.tex} \bigbreak
\subfile{./09-contruction-methods/kor-2015-fr-p5.tex} \bigbreak
\subfile{./09-contruction-methods/rou-2015-tst-d4-p2.tex} \bigbreak
\subfile{./09-contruction-methods/rus-2015-tst-d10-p1.tex} \bigbreak

\newpage

\end{document}