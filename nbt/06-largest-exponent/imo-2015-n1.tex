\documentclass[../03-arithmetic-functions.tex]{subfiles}

\begin{document}

\begin{example*}[\nameref{example:IMO-2015-N1}][\textbf{\nameref{definition:20M}}]
    Xác định tất cả các số nguyên dương \( M \) sao cho dãy số \( a_0, a_1, a_2, \dots \) được xác định bởi  
    \[
        a_0 = M + \frac{1}{2} \quad \text{và} \quad a_{k+1} = a_k \lfloor a_k \rfloor \quad \text{với } k = 0, 1, 2, \dots
    \]
    chứa ít nhất một số nguyên.
\end{example*}

\begin{story*}
    Đặt \( b_k = 2a_k \). Khi đó \( b_0 = 2a_0 = 2M + 1 \), là số nguyên lẻ, suy ra mọi \( b_k \) đều là số nguyên. 

    Nếu dãy \( (a_k) \) không bao giờ là số nguyên, thì các \( b_k \) đều là số lẻ. Khi đó công thức truy hồi trở thành:
    \[
        b_{k+1} = \frac{b_k(b_k - 1)}{2}.
    \]
    Ta chứng minh bằng phương pháp mâu thuẫn rằng không thể xảy ra điều này mãi mãi, vì hiệu \( b_k - 3 \) giảm nhanh hơn cấp số nhân, dẫn đến mâu thuẫn với nguyên lý cực hạn.

    Do đó \( b_0 - 3 \le 0 \Rightarrow b_0 = 3 \Rightarrow M = 1 \) là giá trị nhỏ nhất thỏa mãn. Với \( M = 1 \), \( a_0 = \tfrac{3}{2} \), và ta có \( a_1 = \tfrac{9}{2} \), \( a_2 = \tfrac{405}{2} \), \( a_3 = \tfrac{82005}{2} \), v.v... không có số nguyên nào xuất hiện. Từ đó suy ra chỉ khi \( M = 1 \), dãy không chứa số nguyên.

    Vậy để dãy chứa ít nhất một số nguyên thì cần và đủ \( \boxed{M \ne 1} \).
\end{story*}

\bigbreak

\begin{soln}(Cách 2)\footnotemark
    Xét \( b_k = 2a_k \). Khi đó,    
    \[
        b_{k+1} \;=\; 2a_{k+1} = b_k \Bigl\lfloor \frac{b_k}{2} \Bigr\rfloor.
    \]
    Nếu dãy \( (a_k) \) không chứa số nguyên, thì mọi \( b_k \) là số lẻ và ta có:
    \[
        b_{k+1} = \frac{b_k (b_k - 1)}{2}.
    \]
    Xét hiệu \( b_k - 3 \), ta thấy:
    \[
        b_0 - 3 = 2M + 1 - 3 = 2(M - 1),
    \]
    mà theo truy hồi trên, hiệu \( b_k - 3 \) giảm nhanh theo bậc số mũ. Điều này mâu thuẫn với nguyên lý cực hạn, vì một dãy số nguyên dương không thể giảm mãi mà vẫn dương.

    Suy ra, chỉ có thể xảy ra khi \( b_0 - 3 \le 0 \Rightarrow b_0 \le 3 \Rightarrow M = 1 \). Khi đó \( a_0 = \tfrac{3}{2} \), và dãy tiếp theo không bao giờ đạt giá trị nguyên.

    Vậy điều kiện để dãy chứa ít nhất một số nguyên là:
    \[
        \boxed{M \ne 1}.
    \]
\end{soln}

\footnotetext{\samepage \href{https://www.imo-official.org/problems/IMO2015SL.pdf}{Lời giải chính thức – IMO Shortlist 2015.}}

\end{document}