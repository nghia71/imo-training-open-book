\documentclass[../06-largest-exponent.tex]{subfiles}

\begin{document}

\begin{example*}[\gls{IMO 2023}/P1]\label{example:IMO-2023-P1}\textbf{[\nameref{definition:5M}]}
    Xác định tất cả các số nguyên hợp dương \( n \) thỏa mãn tính chất sau: nếu các ước số dương của \( n \) là \(1 = d_1 < d_2 < \dots < d_k = n,\)
    thì:
    \[
        d_i \mid (d_{i+1} + d_{i+2}) \quad \text{với mọi } 1 \leq i \leq k - 2.
    \]
\end{example*}

\begin{story*}
    Bài toán yêu cầu tìm các số nguyên hợp dương \( n \) sao cho mọi bộ ba ước liên tiếp \( (d_i, d_{i+1}, d_{i+2}) \) trong dãy các ước số dương tăng dần của \( n \), đều thỏa mãn \( d_i \mid (d_{i+1} + d_{i+2}) \).

    Hướng tiếp cận:
    \begin{itemize}[topsep=0pt, partopsep=0pt, itemsep=0pt]
        \item Kiểm tra trước các số dạng \( n = p^r \) với \( p \) là số nguyên tố, thấy rằng chúng đều thỏa mãn điều kiện.
        \item Giả sử \( n \) có ít nhất hai thừa số nguyên tố phân biệt, xét các ước liên tiếp và đưa về đánh giá chuẩn \( p \)-adic.
        \item Sử dụng mâu thuẫn về chỉ số \( p \)-adic để loại trừ trường hợp \( n \) có nhiều hơn một thừa số nguyên tố.
    \end{itemize}
\end{story*}

\bigbreak

\begin{soln}\footnotemark
    Trước tiên, ta chứng minh rằng mọi số dạng \( n = p^r \) với \( r \ge 2 \) (tức là lũy thừa bậc cao của một số nguyên tố) đều thỏa mãn điều kiện.

    Thật vậy, các ước số dương của \( p^r \) là:
    \[
        1 = p^0 < p^1 < p^2 < \dots < p^r.
    \]
    Ta có:
    \[
        p^i \mid (p^{i+1} + p^{i+2}) = p^i(p + p^2),
    \]
    nên điều kiện \( d_i \mid (d_{i+1} + d_{i+2}) \) được thỏa mãn với mọi \( i \).

    Bây giờ giả sử \( n \) là hợp số có ít nhất hai thừa số nguyên tố phân biệt. Gọi \( p < q \) là hai thừa số nguyên tố nhỏ nhất của \( n \). Khi đó \( n \) chia hết cho \( pq \), nên có các ước:
    \[
        d = \frac{n}{q}, \quad d' = \frac{n}{p^j} \text{ với một số } j, \quad d'' = \frac{n}{p^{j-1}}.
    \]

    Điều kiện bài toán yêu cầu:
    \[
        \frac{n}{q} \mid \left( \frac{n}{p^j} + \frac{n}{p^{j-1}} \right) = \frac{n}{p^j}(p + 1).
    \]

    Xét chuẩn \( p \)-adic hai vế:
    \[
        \nu_p\left( \frac{n}{q} \right) = \nu_p(n) \quad \text{vì } p \nmid q,
    \]
    trong khi:
    \[
        \nu_p\left( \frac{n}{p^j}(p + 1) \right) = \nu_p(n) - j.
    \]

    Vì \( p \nmid (p + 1) \), nên \( \nu_p(p + 1) = 0 \).  
    Suy ra:
    \[
        \nu_p\left( \frac{n}{q} \right) > \nu_p\left( \frac{n}{p^j}(p + 1) \right),
    \]
    điều này mâu thuẫn với giả thiết rằng \( \frac{n}{q} \mid \left( \frac{n}{p^j} + \frac{n}{p^{j-1}} \right) \).  

    Do đó, \( n \) không thể có nhiều hơn một thừa số nguyên tố.  

    \textbf{Kết luận:} Mọi số \( n \) thỏa mãn bài toán là các số có dạng:
    \[
        \boxed{n = p^r,\ \text{với } p\ \text{nguyên tố},\ r \ge 2.}
    \]
\end{soln}

\footnotetext{\samepage \href{https://www.imo-official.org/problems/IMO2023SL.pdf}{Shortlist 2023 with solutions.}}

\end{document}