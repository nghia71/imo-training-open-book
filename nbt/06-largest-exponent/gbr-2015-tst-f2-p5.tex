\documentclass[../03-arithmetic-functions.tex]{subfiles}

\begin{document}

\begin{exercise*}[\gls{GBR 2015 TST}/F2/P5]\label{example:GBR-2015-TST-F2-P5}[\textbf{unrated}]
    Cho dãy số nguyên $(a_n)_{n \ge 0}$ thỏa mãn:
    \[
        a_0 = 1, \quad a_1 = 3, \quad \text{và} \quad a_{n+2} = 1 + \left\lfloor \frac{a_{n+1}^2}{a_n} \right\rfloor \text{ với mọi } n \ge 0.
    \]
    
    Chứng minh rằng với mọi \(n \ge 0\), ta có:
    \[
        a_n a_{n+2} - a_{n+1}^2 = 2^n.
    \]
\end{exercise*}

\begin{remark*}
    Thử tính vài giá trị đầu của dãy để quan sát quy luật. Sau đó, đặt mục tiêu chứng minh \( a_n a_{n+2} - a_{n+1}^2 = 2^n \), có thể dùng phương pháp quy nạp. Để ý rằng công thức đệ quy dùng phần nguyên nên cần biến đổi phù hợp để tính toán chính xác.
\end{remark*}

\begin{story*}
    Dãy được cho bởi:
    \[
        a_0 = 1,\quad a_1 = 3,\quad a_{n+2} = 1 + \left\lfloor \frac{a_{n+1}^2}{a_n} \right\rfloor.
    \]
    Ta đặt mục tiêu chứng minh:
    \[
        a_n a_{n+2} - a_{n+1}^2 = 2^n \quad \text{với mọi } n \ge 0.
    \]
    Phương pháp hợp lý là quy nạp theo \( n \). Ở bước quy nạp, cần khai triển biểu thức \( a_{n+2} \) bằng công thức truy hồi, rồi thay vào \( a_n a_{n+2} - a_{n+1}^2 \) để kiểm tra xem có bằng \( 2^n \) hay không. Để đơn giản, có thể đặt \( \left\lfloor \frac{a_{n+1}^2}{a_n} \right\rfloor = q \) để viết lại biểu thức một cách rõ ràng hơn.
\end{story*}

\end{document}