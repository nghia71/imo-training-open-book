\documentclass[../06-largest-exponent.tex]{subfiles}

\begin{document}

\begin{example*}[\gls{IMO 2015}/P2]\label{example:IMO-2015-P2}\textbf{[\nameref{definition:30M}]}
	Hãy tìm tất cả các bộ số nguyên dương \( (a, b, c) \) sao cho mỗi số trong các số:
	\[
		ab - c,\quad bc - a,\quad ca - b
	\]
	đều là lũy thừa của 2.
\end{example*}

\begin{story*}
    Bài toán yêu cầu tìm tất cả các bộ ba \( (a, b, c) \in \mathbb{Z}_{>0}^3 \) sao cho ba biểu thức đối xứng \( ab - c \), \( bc - a \), và \( ca - b \) đều là lũy thừa của 2.

    Hướng giải sử dụng:
    \begin{itemize}[topsep=0pt, partopsep=0pt, itemsep=0pt]
        \item Giả sử \( a \le b \le c \) để tránh xét các hoán vị trùng lặp.
        \item Thử trực tiếp các giá trị nhỏ của \( a \) như \( 2, 3, 4 \), rồi giới hạn khả năng của \( b, c \).
        \item Dùng các bất đẳng thức và điều kiện số học (chia hết, chẵn/lẻ) để loại trừ trường hợp.
    \end{itemize}
\end{story*}

\bigbreak

\begin{soln}\footnotemark
	Các nghiệm thỏa mãn là: \( (2,2,2) \), \( (2,2,3) \), \( (2,6,11) \), \( (3,5,7) \), và các hoán vị của chúng.

	Giả sử \( a \le b \le c \), đặt:
	\[
		ab - c = 2^m,\quad ca - b = 2^n,\quad bc - a = 2^p,\quad m \le n \le p.
	\]

	Nếu \( a = 1 \Rightarrow b - c = 2^m \), điều này không thể xảy ra vì vế trái có thể âm. Vậy \( a \ge 2 \).

	Khi đó:
	\[
		ca - b \ge (a - 1)c \ge 2 \Rightarrow n, p \ge 1.
	\]

	\textbf{Trường hợp 1: \( a = b \ge 3 \)}

	Lúc này:
	\[
		ac - b = a(c - 1) \Rightarrow a \mid 2^n.
	\]
	Vì \( a \ge 3 \), điều này là mâu thuẫn. Do đó \( a \ne b \).

	\textbf{Trường hợp 2: \( a = 2 \)}

	Ta có:
	\[
		2b - c = 2^m,\quad 2c - b = 2^n,\quad bc - 2 = 2^p.
	\]

	Nếu \( p = 1 \Rightarrow bc = 4 \Rightarrow b = c = 2 \Rightarrow (2,2,2) \) là nghiệm.

	Nếu \( p > 1 \Rightarrow bc \) chẵn, và do đó \( c \) lẻ (vì \( b \) chẵn). Điều này buộc \( m = 0 \).

	Thử các giá trị nhỏ của \( n \) để giải hệ phương trình:
	\[
		2b - c = 1,\quad 2c - b = 2^n.
	\]

	Giải được nghiệm:
	\[
		(b,c) = (2,3),\quad (6,11) \Rightarrow (2,2,3),\ (2,6,11).
	\]

	\textbf{Trường hợp 3: \( a = 3 \)}

	Ta có:
	\[
		3b - c = 2^m,\quad 3c - b = 2^n,\quad bc - 3 = 2^p.
	\]

	Thử \( b = 5,\ c = 7 \Rightarrow ab - c = 15 - 7 = 8,\ bc - a = 35 - 3 = 32,\ ca - b = 21 - 5 = 16 \Rightarrow \) đều là lũy thừa của 2.

	Nên \( (3,5,7) \) là nghiệm.

	\textbf{Trường hợp 4: \( a \ge 4 \)}

	Trong trường hợp này, \( ca - b \ge (a - 1)c \) là rất lớn nên dễ vượt quá lũy thừa của 2 gần nhất.  
	Hơn nữa, giới hạn từ \( ab - c = 2^m \) sẽ mâu thuẫn với tốc độ tăng của \( ab \), do đó không thể xảy ra.

	\textbf{Kết luận:} Các bộ ba thỏa mãn là:
	\[
		\boxed{(a, b, c) \in \{ (2,2,2),\ (2,2,3),\ (2,6,11),\ (3,5,7) \} \text{ và các hoán vị.}}
	\]
\end{soln}

\footnotetext{\samepage \href{https://artofproblemsolving.com/wiki/index.php/2015_IMO_Problems/Problem_2}{Lời giải chính thức.}}

\end{document}