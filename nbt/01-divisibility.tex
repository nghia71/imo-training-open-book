\documentclass[../../imo-training-open-book.tex]{subfiles}

\begin{document}

\section{Lý thuyết}

\begin{theorem*}[\href{https://w.wiki/_pZXo}{Định lý cơ bản của số học}] 
    \label{theorem:fundamental-theorem-of-arithmetic}
    Mọi số tự nhiên lớn hơn 1 có thể viết một cách duy nhất (không kể sự sai khác về thứ tự các thừa số) thành tích các thừa số nguyên tố.
    
    Mọi số tự nhiên $n$ lớn hơn 1, có thể viết duy nhất dưới dạng:
    \[
        n={p_{1}}^{\alpha _{1}}{p_{2}}^{\alpha _{2}}{\dots }{p_{k}}^{\alpha _{k}}
    \]
    trong đó $p_{1},p_{2},\ldots,p_{k}$ là các số nguyên tố và $\alpha _{1},\alpha _{2},\dots ,\alpha _{k}$ là các số nguyên dương.
\end{theorem*}

\newpage

\section{Các ví dụ}

\subfile{./01-divisibility/chn-2015-tst1-d1-p2.tex}
\subfile{./01-divisibility/chn-2015-tst1-d2-p2.tex}
\subfile{./01-divisibility/imo-2023-p1.tex}

\newpage

\section{Bài tập}

\newpage

\end{document}