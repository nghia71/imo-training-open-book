\documentclass[../09-contruction-methods.tex]{subfiles}

\begin{document}

\begin{example*}[\gls{USA 2015 TSTST}/P5]\label{example:USA-2015-TSTST-P5}[\textbf{\nameref{definition:10M}}]
	Cho $\varphi(n)$ là số các số nguyên dương nhỏ hơn $n$ và nguyên tố cùng nhau với $n$.  
	Chứng minh rằng tồn tại một số nguyên dương $m$ sao cho phương trình
	\[
		\varphi(n) = m
	\]
	có ít nhất $2015$ nghiệm nguyên dương $n$.
\end{example*}

\begin{soln}(Cách 1)\footnotemark
	Xét tập các số nguyên tố:
	\[
		S = \{ 11, 13, 17, 19, 29, 31, 37, 41, 43, 61, 71 \},
	\]
	với tính chất rằng với mọi \( p \in S \), tất cả ước số nguyên tố của \( p - 1 \) đều là số có một chữ số.
	
	Gọi \( N = 210^{\text{tỷ}} \), và đặt \( M = \varphi(N) \).
	Với mỗi tập con \( T \subseteq S \), định nghĩa:
	\[
		n_T = \frac{N}{\prod_{p \in T}(p-1)} \cdot \prod_{p \in T} p.
	\]
	
	Khi đó:
	\[
		\varphi(n_T) = \varphi\left(\frac{N}{\prod_{p \in T}(p-1)} \cdot \prod_{p \in T} p\right) = \varphi(N) = M,
	\]
	vì thay mỗi thừa số \( p-1 \mid N \) bằng \( p \) không làm thay đổi giá trị \nameref{definition:euler-totient-function}.
	
	Do \( |S| = 11 \), ta có \( 2^{11} = 2048 > 2015 \) tập con, nên ta thu được ít nhất 2015 số nguyên dương phân biệt có cùng giá trị \( \varphi \).
\end{soln}

\footnotetext{\href{https://web.evanchen.cc/exams/sols-TSTST-2015.pdf}{Lời giải của Evan Chen.}}

\begin{remark*}
	Mẹo này bắt nguồn từ thực tế như:  
	\( \varphi(11 \cdot 1000) = \varphi(10 \cdot 1000) \), vì \( \varphi(11) = 10 = \varphi(10) \).
\end{remark*}
	
\begin{soln}(Cách 2)\footnotemark
	Gọi \( p_1 = 2 < p_2 < \cdots < p_{2015} \) là 2015 số nguyên tố nhỏ nhất.  
	Xét 2015 số \( n_1, \dots, n_{2015} \) được định nghĩa như sau:
	\begin{align*}
		n_1 &= (p_1 - 1)\cdot p_2 \cdots p_{2015}, \\
		n_2 &= p_1 \cdot (p_2 - 1) \cdot p_3 \cdots p_{2015}, \\
		&\vdotswithin{=} \\
		n_{2015} &= p_1 \cdots p_{2014} \cdot (p_{2015} - 1).
	\end{align*}
	Lưu ý rằng trong mỗi \( n_j \), một số nguyên tố \( p_j \) được thay thế bằng \( p_j - 1 \).  

	Vì \( \varphi(p_j - 1) = \varphi(p_j) = p_j - 1 \) nếu \( p_j \) là nguyên tố, nên:
	\[
		\varphi(n_j) = \prod_{i=1}^{2015} (p_i - 1) = \varphi(p_1 p_2 \cdots p_{2015}).
	\]

	Do đó \( n_1, \dots, n_{2015} \) là 2015 số nguyên dương đôi một phân biệt có cùng giá trị \nameref{definition:euler-totient-function}.

	Các số này chỉ có ước số nguyên tố trong \( \{p_1, \dots, p_{2015}\} \).
\end{soln}

\footnotetext{\href{https://web.evanchen.cc/exams/sols-TSTST-2015.pdf}{Lời giải của Yang Liu.}}

\newpage

\begin{soln}(Cách 3)\footnotemark
	Ta chứng minh bài toán tổng quát.
	\begin{claim*}
		Cho \( p_1 = 2 < p_2 < \dots < p_k \) là \( k \) số nguyên tố đầu tiên. Khi đó tồn tại ít nhất \( k \) số nguyên dương phân biệt \( n \) sao cho:
		\[
			\varphi(n) = \varphi(p_1p_2\cdots p_k)
		\]
		và mọi ước số nguyên tố của \( n \) đều thuộc tập \( \{p_1, p_2, \dots, p_k\} \).
	\end{claim*}
	\begin{subproof}
		Ta dùng phương pháp quy nạp theo \( k \).
	
		\textit{Bước cơ sở:} Với \( k = 1 \), ta có \( p_1 = 2 \), nên \( \varphi(2) = 1 \).  
		Số duy nhất \( n \) sao cho \( \varphi(n) = 1 \) là \( n = 2 \), thỏa mãn điều kiện. Mệnh đề đúng với \( k = 1 \).
		
		\textit{Bước quy nạp:} Giả sử với một số \( k \geq 1 \), tồn tại ít nhất \( k \) số nguyên dương phân biệt \( n_1, n_2, \dots, n_k \) sao cho:
		\[
			\varphi(n_j) = \varphi(P_k), \quad \text{với } P_k = \prod_{i=1}^k p_i,
		\]
		và mọi ước số nguyên tố của \( n_j \) đều thuộc \( \{p_1, \dots, p_k\} \).
		
		Ta chứng minh mệnh đề đúng với \( k + 1 \).
		Xét \( P_{k+1} = P_k \cdot p_{k+1} \). Khi đó:
		\[
			\varphi(P_{k+1}) = \varphi(P_k) \cdot (p_{k+1} - 1).
		\]
		
		Với mỗi \( j = 1, \dots, k \), xét số \( m_j = n_j \cdot p_{k+1} \).  
		Vì \( \gcd(n_j, p_{k+1}) = 1 \), ta có:
		\[
			\varphi(m_j) = \varphi(n_j) \cdot \varphi(p_{k+1}) = \varphi(P_k) \cdot (p_{k+1} - 1) = \varphi(P_{k+1}).
		\]
		
		Các \( m_j \) đều có ước số nguyên tố thuộc \( \{p_1, \dots, p_{k+1}\} \), và phân biệt vì \( n_j \) phân biệt.
		Như vậy ta đã có \( k \) số thỏa mãn. Ta cần thêm một số nữa.
		
		Vì \( p_{k+1} - 1 \) là số chẵn và nhỏ hơn \( p_{k+1} \), nên mọi ước số nguyên tố của nó đều thuộc \( \{p_1, \dots, p_k\} \).  
		Viết:
		\[
			p_{k+1} - 1 = \prod_{i=1}^k p_i^{e_i}.
		\]
		
		Đặt:
		\[
			m_{k+1} = \left( \prod_{i=1}^k p_i^{e_i + 1} \right).
		\]
		
		Vì \( \gcd(p_i, p_j) = 1 \) khi \( i \ne j \), ta có:
		\[
			\varphi(m_{k+1}) = \prod_{i=1}^k p_i^{e_i}(p_i - 1) = (p_{k+1} - 1) \cdot \varphi(P_k) = \varphi(P_{k+1}).
		\]
		
		Hơn nữa, \( m_{k+1} \) có dạng mũ khác với các số \( m_j \) trước đó (vì chúng đều chỉ có mũ \( \leq 1 \)), nên \( m_{k+1} \) là số thứ \( (k+1) \) phân biệt.
		Do đó, tồn tại ít nhất \( k+1 \) số nguyên dương phân biệt thỏa mãn yêu cầu với \( k+1 \).
	\end{subproof}
\end{soln}

\footnotetext{\href{https://artofproblemsolving.com/community/c6h1106802p9529278}{Dựa theo lời giải của mathocean97.}}

\end{document}