\documentclass[../09-contruction-methods.tex]{subfiles}

\begin{document}

\begin{example*}[\gls{GER 2015 TST}/P2]\label{example:GER-2015-TST-P2}[\textbf{\nameref{definition:15M}}]
    Một số nguyên dương \( n \) được gọi là \textbf{nghịch ngợm} nếu có thể viết dưới dạng
    \[
        n = ab + b
    \]
    với các số nguyên \( a, b \geq 2 \).

    Hỏi có tồn tại một dãy gồm 102 số nguyên dương liên tiếp sao cho chính xác 100 trong số đó là các số nghịch ngợm hay không?
\end{example*}

\begin{soln}\footnotemark
	Giả sử tồn tại một dãy gồm 102 số nguyên dương liên tiếp sao cho đúng 100 số trong đó là số nghịch ngợm, tức tồn tại hai số không nghịch ngợm.

	Nhắc lại định nghĩa: \( n \) là số nghịch ngợm nếu tồn tại \( a, b \ge 2 \) sao cho \( n = ab + b = b(a + 1) \), tức là \( b \mid n \) và \( \tfrac{n}{b} - 1 \ge 1 \).

	Do đó, mọi số nghịch ngợm đều là bội của một số \( b \ge 2 \), tức có ít nhất một ước số nguyên \( \ge 2 \).  
	Ngược lại, số không nghịch ngợm chỉ có thể là:
	\begin{itemize}[topsep=0pt, partopsep=0pt, itemsep=0pt]
		\item Số nguyên tố, vì không chia hết cho số nào \( < n \) và \( \ge 2 \),
		\item Số có dạng \( p^k \) mà không thỏa điều kiện \( \tfrac{n}{b} - 1 \ge 1 \),
		\item Các số nhỏ như \( 1 \) hoặc \( 2 \).
	\end{itemize}

	Nhưng trong 102 số liên tiếp, theo định lý số nguyên tố Bertrand và mật độ nguyên tố, ta không thể có tới 3 số không nghịch ngợm liên tiếp, và cũng không thể đảm bảo chỉ có đúng 2 số không nghịch ngợm.

	Mặt khác, ta chứng minh rằng với bất kỳ dãy 102 số nguyên dương liên tiếp, tồn tại ít nhất 3 số không nghịch ngợm.

	Thật vậy, xét \( n = 1 \) và \( n = 2 \):  
	\begin{itemize}[topsep=0pt, partopsep=0pt, itemsep=0pt]
		\item Với \( n = 1 \), không có \( b \ge 2 \) nào chia hết \( n \), nên không nghịch ngợm.  
		\item Với \( n = 2 \), tương tự, vì \( b \ge 2 \) thì \( b \nmid 2 \), cũng không nghịch ngợm.  
		\item Với \( n = 3 \): chỉ có \( b = 3 \), thì \( \tfrac{3}{3} - 1 = 0 \), không thỏa.
	\end{itemize}

	Vậy các số \( 1, 2, 3 \) không phải là số nghịch ngợm.  
	Nên trong dãy bất kỳ gồm 102 số liên tiếp, luôn có thể chứa tới 3 số như vậy.

	Do đó, không tồn tại dãy 102 số liên tiếp mà chỉ có đúng 2 số không nghịch ngợm.  
	
	\textbf{Kết luận:} \textit{Không tồn tại dãy gồm 102 số nguyên dương liên tiếp sao cho đúng 100 số trong đó là số nghịch ngợm.}
\end{soln}

\footnotetext{\href{https://artofproblemsolving.com/community/c6h1085942p4806253}{Dựa theo lời giải của \textbf{v\_Enhance} và \textbf{Stella Y.}}}

\begin{example*}[\gls{SRB 2014 MO}/P4]\label{example:SRB-2014-MO-P4}[\textbf{\nameref{definition:25M}}]
    Ta gọi một số tự nhiên \( n \) là \textbf{điên rồ} (crazy) nếu tồn tại các số tự nhiên \( a, b > 1 \) sao cho:
    \[
        n = a^b + b.
    \]
    Hỏi có tồn tại dãy gồm 2014 số tự nhiên liên tiếp sao cho chính xác 2012 trong số đó là số điên rồ hay không?
\end{example*}

\begin{story*}
    Bài toán yêu cầu tìm một đoạn gồm nhiều số liên tiếp trong đó phần lớn thỏa mãn một tính chất “tồn tại biểu diễn” dưới dạng \( ab + b \) hoặc \( a^b + b \).
    Hướng giải sử dụng kỹ thuật chọn đoạn lớn có cấu trúc đặc biệt để đảm bảo mọi phần tử trong đoạn đều thỏa mãn,
    sau đó áp dụng tính chất liên tục rời rạc (mỗi lần trượt đoạn, số lượng phần tử thỏa mãn chỉ thay đổi nhiều nhất 1)
    để đảm bảo tồn tại đoạn có đúng số lượng mong muốn.
\end{story*}

\bigbreak

\begin{soln}\footnotemark
    Xét tập hợp các số:
    \[
        S = \{2^{2014!}+1,\ 2^{2014!}+2,\ \dotsc,\ 2^{2014!}+2014\}.
    \]
    
    Với mỗi \( d \in \{1, 2, \dotsc, 2014\} \), ta có:
    \[
        2^{\frac{2014!}{d}} + d
    \]
    là một số thuộc \( S \), vì \( \frac{2014!}{d} \) là số nguyên.
    
    Chọn \( a = 2^{\frac{2014!}{d}} \), \( b = d \), thì:
    \[
        a^b + b = \left(2^{\frac{2014!}{d}}\right)^d + d = 2^{2014!} + d.
    \]
    
    Do đó, mỗi phần tử của \( S \) đều có dạng \( a^b + b \) với \( b > 1 \), tức là đều là số điên rồ.
    
    Vậy, \( S \) gồm 2014 số liên tiếp đều là số điên rồ.
    
    Bây giờ, định nghĩa hàm:
    \[
        f(n) = \text{số lượng số điên rồ trong đoạn } [n, n+2013].
    \]
    
    Ta có:
    \begin{itemize}[topsep=0pt, partopsep=0pt, itemsep=0pt]
        \item \( f(2^{2014!}+1) = 2014 \) (toàn bộ đoạn là crazy).
        \item Trong đoạn \( [1, 2014] \), có thể kiểm tra rằng \( f(1) < 2012 \) (vì với \( b \ge 2 \), \( a^b + b \) tăng nhanh, số lượng biểu diễn dạng đó là hạn chế).
    \end{itemize}
    
    Ngoài ra, khi dịch đoạn đi 1 đơn vị, giá trị của \( f(n) \) thay đổi nhiều nhất là 1:
    \[
        |f(n+1) - f(n)| \le 1.
    \]
    
    Do đó, theo tính chất “liên tục rời rạc”, tồn tại \( n \) sao cho \( f(n) = 2012 \), tức là có đúng 2012 số điên rồ trong đoạn gồm 2014 số tự nhiên liên tiếp.
    
    \textbf{Kết luận:} \textit{Tồn tại dãy gồm 2014 số tự nhiên liên tiếp sao cho chính xác 2012 số trong đó là số điên rồ.}
\end{soln}

\footnotetext{\href{https://artofproblemsolving.com/community/c6h1092597p22425505}{Lời giải của \textbf{Alan Bu, Alex Zhao, Christopher Qiu, Edward Yu, Eric Shen, Isaac Zhu, Jeffrey Chen, Kevin Wu, and Ryan Yang}.}}

\end{document}