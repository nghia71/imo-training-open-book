\documentclass[../01-divisibility.tex]{subfiles}

\begin{document}

\begin{example*}[\gls{RUS 2015 TST}/D10/P2]\label{example:RUS-2015-TST-D10-P2}[\textbf{unrated}]
	Cho số nguyên tố $p \ge 5$. Chứng minh rằng tập $\{1,2,\ldots,p - 1\}$ có thể được chia thành hai tập con không rỗng
	sao cho tổng các phần tử của một tập con và tích các phần tử của tập con còn lại cho cùng một phần dư modulo $p$.	
\end{example*}

\begin{soln}(Cách 1)\footnotemark
	Ta biết rằng:
    \[
        \sum_{i=1}^{p-1} i = \frac{(p-1)p}{2} \equiv 1 \pmod{p}\ \text{và}\ \prod_{i=1}^{p-1} i \equiv -1 \pmod{p} \quad \text{(Định lý Wilson)}
    \]
    
	Gọi $S$ là một tập con không rỗng của $\{1, 2, \ldots, p-1\}$. Khi đó, phần bù của $S$ là $S^c = \{1, \ldots, p-1\} \setminus S$.
	Bài toán tương đương với việc tìm $S$ sao cho:
    \[
        \sum_{i \in S} i \equiv \prod_{j \in S^c} j \pmod{p}.
    \]
    
	Đặt $A = \sum_{i \in S} i, \quad B = \prod_{i \in S} i,$ khi đó
    \[
		\sum_{i \in S^c} i = \sum_{i=1}^{p-1} i - \sum_{i \in S} i = 1 - A \pmod{p},\
		\prod_{i \in S^c} i = \frac{\prod_{i=1}^{p-1} i}{\prod_{i \in S} i} = \frac{-1}{B} \pmod{p}.
    \]

	
	Do đó, ta cần:
    \[
        A \equiv \frac{-1}{B} \pmod{p} \quad \Longleftrightarrow \quad AB \equiv -1 \pmod{p}.
    \]

    Ta sẽ xây dựng $S$ thỏa mãn điều này.

    \textit{Trường hợp 1:} Nếu $p \equiv 1 \pmod{4}$. Khi đó $\exists a \in \mathbb{F}_p$ sao cho $a^2 \equiv -1 \pmod{p}$. Khi đó, lấy $S = \{a\}$ thì:
    \[
        A = a, \quad B = a \implies AB = a^2 \equiv -1 \pmod{p}.
    \]

    \textit{Trường hợp 2:} Nếu $p \equiv 3 \pmod{4}$. Khi đó $p-1$ là số chẵn không chia hết cho 4, nên tồn tại số nguyên tố lẻ $q$ chia $p-1$ (do $p \ge 5$).
	Khi đó, tồn tại phần tử sinh $a \in \mathbb{F}_p$ sao cho $\mathrm{ord}_p(a) = q$. Xét tập:
    \[
        S = \left\{a^{\frac{q-1}{2}}, a^{\frac{q-3}{2}}, \ldots, a, a^{-1}, a^{-3}, \ldots, a^{-\frac{q-1}{2}}\right\},\ |S| = q-1.
    \]

    Do các phần tử đi thành cặp nghịch đảo, tích của $S$ là 1 modulo $p$:
	\[
		\prod_{i \in S} i \equiv 1 \pmod{p}.
	\]

	Ta nhân cả tổng với $a^{\frac{q-1}{2}}$:
	\[
		a^{\frac{q-1}{2}} \sum_{i \in S} i = a^{q-1} + a^{q-2} + \cdots + 1 \equiv 0 \pmod{p},
	\]
	vì đây là tổng cấp số nhân có công bội $a$ bậc $q$.

    Vậy:
    \[
        \sum_{i \in S} i \equiv -1 \pmod{p}, \quad \prod_{i \in S} i \equiv 1 \pmod{p} \implies AB \equiv -1 \pmod{p}
    \]
\end{soln}

\footnotetext{\href{https://artofproblemsolving.com/community/c6h3057483p28254744}{Dựa theo lời giải của IAmTheHazard.}}

\end{document}