\documentclass[../03-arithmetic-functions.tex]{subfiles}

\begin{document}

\begin{exercise*}[\gls{ROU 2015 TST}/D4/P2]\label{example:ROU-2015-TST-D4-P2}\textbf{[\nameref{definition:30M}]}
    Cho \( n \) là một số nguyên dương. Nếu \( \sigma \) là một hoán vị của \( n \) số nguyên dương đầu tiên,
    định nghĩa \( S(\sigma) \) là tập hợp tất cả các tổng phân đoạn khác nhau có dạng:
    \[
        \sum_{i=k}^{l} \sigma(i)
    \]
    với \( 1 \le k \le l \le n \).
    
    \begin{itemize}[topsep=0pt, partopsep=0pt, itemsep=0pt]
        \item[(a)] Hãy chỉ ra một hoán vị \( \sigma \) của \( \{1, 2, \ldots, n\} \) sao cho:
        \[
            |S(\sigma)| \geq \left\lfloor \frac{(n+1)^2}{4} \right\rfloor.
        \]
        \item[(b)] Chứng minh rằng với mọi hoán vị \( \sigma \), ta có:
        \[
            |S(\sigma)| > \frac{n\sqrt{n}}{4\sqrt{2}}.
        \]
    \end{itemize}
\end{exercise*}

\begin{remark*}
    \begin{itemize}[topsep=0pt, partopsep=0pt, itemsep=0pt]
        \item Phần (a) là một bài toán xây dựng hoán vị \(\sigma\) để tối đa hóa số tổng phân đoạn khác nhau.
        \item Phần (b) yêu cầu chứng minh một bất đẳng thức tổng quát áp dụng cho mọi hoán vị, với ý tưởng sử dụng bất đẳng thức Cauchy–Schwarz để ước lượng số lượng phần tử của tập \( S(\sigma) \).
    \end{itemize}
\end{remark*}

\begin{story*}
    \begin{itemize}[topsep=0pt, partopsep=0pt, itemsep=0pt]
        \item Phần (a): Đây là một bài toán xây dựng. Ta có thể lấy hoán vị \(\sigma(i) = i\) với mọi \( i \in \{1, \dots, n\} \).
        Khi đó, tổng đoạn từ \( i \) đến \( j \) là \( \frac{(j - i + 1)(i + j)}{2} \), và dễ kiểm tra rằng số lượng tổng đoạn phân biệt là ít nhất \( \left\lfloor \frac{(n+1)^2}{4} \right\rfloor \).

        \item Phần (b): Với mọi hoán vị \(\sigma\), ta cần chứng minh rằng \( |S(\sigma)| \) luôn lớn.
        Ý tưởng là sử dụng bất đẳng thức Cauchy–Schwarz để giới hạn số lần một tổng đoạn có thể lặp lại,
        rồi từ đó suy ra rằng \( |S(\sigma)| > \frac{n\sqrt{n}}{4\sqrt{2}} \).
    \end{itemize}
\end{story*}

\end{document}