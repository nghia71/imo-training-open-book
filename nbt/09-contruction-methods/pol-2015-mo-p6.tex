\documentclass[../09-contruction-methods.tex]{subfiles}

\begin{document}

\begin{example*}[\gls{POL 2015 MO}/P6]\label{example:POL-2015-MO-P6}[\textbf{\nameref{definition:30M}}]
	Chứng minh rằng với mọi số nguyên dương \( a \), tồn tại một số nguyên \( b > a \) sao cho:
	\[
		1 + 2^a + 3^a \mid 1 + 2^b + 3^b.
	\]	
\end{example*}

\begin{story*}
    Bài toán yêu cầu xây dựng \( b > a \) sao cho \( 1 + 2^a + 3^a \mid 1 + 2^b + 3^b \).
    
    Hướng tiếp cận:
    \begin{itemize}[topsep=0pt, partopsep=0pt, itemsep=0pt]
        \item Gọi \( N = 1 + 2^a + 3^a \), xét phân tích thừa số nguyên tố của \( N \).
        \item Đối với mỗi thừa số \( p^e \mid N \), cố gắng chọn \( b \equiv a \pmod{\varphi(p^e)} \) để đảm bảo đồng dư \( 2^b \equiv 2^a \), \( 3^b \equiv 3^a \) modulo \( p^e \).
        \item Đối với các thừa số nhỏ đặc biệt như \( p = 2 \) và \( p = 3 \), cần dùng đánh giá p-adic (LTE hoặc kiểm tra trực tiếp) để xử lý phần chia hết.
        \item Áp dụng định lý CRT để hợp nhất các điều kiện đồng dư, xây dựng \( b \) phù hợp và lớn hơn \( a \).
    \end{itemize}
\end{story*}

\bigbreak

\begin{soln}\footnotemark
	Gọi \( N = 1 + 2^a + 3^a = 2^{e_2} \cdot 3^{e_3} \cdot p_4^{e_4} \cdots p_n^{e_n} \) là phân tích thừa số nguyên tố của \( N \).
	Ta sẽ xây dựng một số \( b > a \) sao cho \( N \mid 1 + 2^b + 3^b \).

	\textit{Bước 1: Với các số nguyên tố \( p \ge 5 \)}

	Chọn \( b \equiv a \Mod{\varphi(p_k^{e_k})} \) cho mỗi \( k \ge 4 \), ta có:
	\[
		2^b \equiv 2^a,\quad 3^b \equiv 3^a \Mod{p_k^{e_k}} \implies 1 + 2^b + 3^b \equiv 0 \Mod{p_k^{e_k}}.
	\]

	\textit{Bước 2: Với số nguyên tố \( p = 2 \)}

	Xét modulo 8: \( 3^a \equiv 1 \) hoặc \( 3 \Mod{8} \), nên:
	\[
		1 + 2^a + 3^a \equiv 2 \text{ hoặc } 4 \Mod{8} \implies v_2(N) \le 2.
	\]
	
	Chọn \( b \equiv 3 \Mod{4} \) và đủ lớn, khi đó \( v_2(1 + 2^b + 3^b) = 2 \), đạt giá trị tối đa.

	\textit{Bước 3: Với số nguyên tố \( p = 3 \)}

	Sử dụng định lý LTE với \( b \) lẻ:
	\[
		v_3(1 + 2^b) = v_3(3) + v_3(b) = 1 + v_3(b).
	\]

	Chọn \[ b \equiv 0 \Mod{3^{e_3}} \Rightarrow v_3(b) \ge e_3 \Rightarrow v_3(1 + 2^b + 3^b) \ge 1 + e_3 \ge e_3. \]
	
	Tóm lại, \( v_p(1 + 2^b + 3^b) \ge v_p(N) \) với mọi \( p \mid N \), nên:
	\[
		1 + 2^a + 3^a \mid 1 + 2^b + 3^b.
	\]
\end{soln}

\footnotetext{\href{https://artofproblemsolving.com/community/c6h1279442p6726506}{Lời giải của \textbf{va2010}.}}

\newpage

\begin{soln}\footnotemark
	Với \( a = 1 \), lấy \( b = 3 \). Giả sử \( a > 1 \). Phân tích:
	\[
		1 + 2^a + 3^a = 2^n \cdot 3^m \cdot p_1^{\alpha_1} \cdots p_k^{\alpha_k}, \quad \text{với } n \ge 1,\ m \ge 0,\ \alpha_i \ge 1.
	\]

	Các \( p_i \) là các số nguyên tố lẻ khác \( 3 \). Theo định lý Fermat–Euler:
	\[
		b \equiv a \Mod{\varphi(p_i^{\alpha_i})} \implies 1 + 2^b + 3^b \equiv 1 + 2^a + 3^a \equiv 0 \Mod{p_i^{\alpha_i}}.
	\]

	Gọi \( N = \text{lcm}(\varphi(p_1^{\alpha_1}), \dots, \varphi(p_k^{\alpha_k})) \), chọn:
	\[
		b = a + 2 \cdot 3^t \cdot N,
	\]
	với \( t \) đủ lớn (xác định cụ thể sau). Khi đó \( P = p_1^{\alpha_1} \cdots p_k^{\alpha_k} \mid 1 + 2^b + 3^b \).

	Xét modulo 8:
	\begin{itemize}[topsep=0pt, partopsep=0pt, itemsep=0pt]
		\item Nếu \( a \) lẻ, thì \( v_2(1 + 2^a + 3^a) = 2 \).
		\item Nếu \( a \) chẵn, thì \( v_2(1 + 2^a + 3^a) = 1 \).
	\end{itemize}

	Vì \( b \equiv a \Mod{2} \), nên \( v_2(1 + 2^b + 3^b) = v_2(1 + 2^a + 3^a) \), do đó \( 2^n \mid 1 + 2^b + 3^b \).

	Tiếp theo, nếu \( a \) chẵn thì \( m = 0 \). Nếu \( a \) lẻ, thì theo LTE:
	\[
		v_3(1 + 2^a) = 1 + v_3(a) \Rightarrow m = 1 + v_3(a).
	\]

	Chọn \( t > 1 + v_3(a) \Rightarrow b \equiv a \Mod{3^{v_3(a)+1}} \Rightarrow v_3(b) = v_3(a) \), và \( b \) lẻ:
	\[
		v_3(1 + 2^b + 3^b) = v_3(1 + 2^b) = 1 + v_3(b) = 1 + v_3(a) = m.
	\]

	Suy ra:
	\[
		2^n \cdot 3^m \cdot P \mid 1 + 2^b + 3^b.
	\]
\end{soln}

\footnotetext{\href{https://artofproblemsolving.com/community/c6h1279442p6726507}{Lời giải của \textbf{elVerde}.}}

\end{document}