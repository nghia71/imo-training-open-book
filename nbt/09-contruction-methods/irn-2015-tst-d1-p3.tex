\documentclass[../09-contruction-methods.tex]{subfiles}

\documentclass[../09-contruction-methods.tex]{subfiles}

\begin{document}

\begin{example*}[\gls{IRN 2015 TST}/D1-P3]\label{example:IRN-2015-TST-D1-P3}[\textbf{\nameref{definition:45M}}]
	Gọi \( b_1 < b_2 < b_3 < \dots \) là dãy tất cả các số tự nhiên có thể viết được dưới dạng tổng hai bình phương của hai số tự nhiên.
	Chứng minh rằng tồn tại vô hạn số tự nhiên \( m \) sao cho \( b_{m+1} - b_m = 2015 \).
\end{example*}

\begin{story*}
	Bài toán yêu cầu tìm các khoảng trống đúng bằng \( 2015 \) giữa hai số liên tiếp có dạng tổng hai bình phương.
	Hướng tiếp cận gồm ba bước:
	\begin{itemize}[topsep=0pt, partopsep=0pt, itemsep=0pt]
		\item Chọn điểm đầu là một số có dạng \( a^2 + 1007^2 \).
		\item Dùng định lý tương hỗ bậc hai để loại trừ các số \( x + i \) (\(1 \le i \le 2014\)) không thể là tổng hai bình phương, bằng cách kiểm soát thừa số nguyên tố \( \equiv 3 \pmod{4} \) xuất hiện với số mũ lẻ.
		\item Áp dụng định lý đồng dư Trung Hoa và phép nâng Hensel để xây dựng nghiệm đồng thời modulo \( p_i^2 \), với mỗi \( p_i \equiv 3 \pmod{8} \).
	\end{itemize}
	Ý tưởng là dựng khoảng rỗng trong dãy các tổng hai bình phương mà hai đầu vẫn là tổng hai bình phương.
\end{story*}

\bigbreak

\begin{soln}\footnotemark
	\begin{theorem*}[quadratic-symbols]
		\textit{Với mỗi số nguyên \( 1 \leq i \leq 2014 \), tồn tại vô hạn số nguyên tố \( p \equiv 3 \Mod{8} \) sao cho:}
		\[
			\left( \frac{1007^2 + i}{p} \right) = -1
		\]	
	\end{theorem*}
	\begin{subproof}
		Viết:
		\[
			1007^2 + i = x^2 p_1 p_2 \dots p_k
		\]
		trong đó \( \mathcal{P} = \{ p_1, \dots, p_k \} \) là tập các thừa số nguyên tố của \( 1007^2 + i \) có số mũ lẻ.
		Ta có \( \mathcal{P} \ne \emptyset \) vì \( 1007^2 + i \) không thể là một số chính phương khi \( i \in [1, 2014] \).
		
		\textit{Trường hợp 1: Nếu \( \mathcal{P} = \{2\} \)}, thì:
		\[
			\left( \frac{1007^2 + i}{p} \right) = \left( \frac{2x^2}{p} \right) = \left( \frac{2}{p} \right) = -1 \quad \text{với } p \equiv 3 \Mod{8}
		\]
		
		\textit{Trường hợp 2: Nếu \( 2 \notin \mathcal{P} \)}, thì chọn \( p \) sao cho:
		\[
			\left( \frac{\prod_{q \in \mathcal{P}} q}{p} \right) = -1
		\]
		
		\textit{Trường hợp 3: Nếu \( 2 \in \mathcal{P} \)}, thì chọn \( p \) sao cho:
		\[
		\left( \frac{\prod_{q \in \mathcal{P} \setminus \{2\}} q}{p} \right) = 1
		\]
		
		Sử dụng định lý tương hỗ bậc hai, ta có thể dịch điều kiện trên thành điều kiện đồng dư của \( p \) modulo các số trong \( \mathcal{P} \).
		Từ định lý Dirichlet và định lý đồng dư Trung Hoa (CRT), tồn tại vô hạn số nguyên tố \( p \equiv 3 \Mod{8} \) thỏa các điều kiện này.
	\end{subproof}
	
	Chọn các số nguyên tố phân biệt \( p_1, p_2, \dots, p_{2014} \), sao cho:
    \begin{itemize}[topsep=0pt, partopsep=0pt, itemsep=0pt]
        \item Mỗi \( p_i > 1008^2 \),
        \item \( p_i \equiv 3 \Mod{8} \),
        \item \( \left( \frac{1007^2 + i}{p_i} \right) = -1. \)
    \end{itemize}
	
	Xét hệ congruence:
	\[
		x + i \equiv p_i \Mod{p_i^2} \quad \text{với } 1 \le i \le 2014 \tag{*}
	\]
	
	Theo định lý CRT, hệ này có nghiệm duy nhất modulo:
	\[
		M = p_1^2 p_2^2 \cdots p_{2014}^2
	\]
	
	Gọi nghiệm là \( x \equiv k \Mod{M} \). Khi đó, với mọi \( i \in [1, 2014] \), ta có:
	\[
		x + i \equiv p_i \Mod{p_i^2} \implies x + i \not\equiv \text{tổng hai bình phương}
	\]
	vì \( p_i \equiv 3 \Mod{4} \), nên một số chia hết cho \( p_i \) với lũy thừa lẻ thì không thể là tổng hai bình phương.
	
	Bây giờ, ta chứng minh \( k \) là tổng hai bình phương:
	\[
		\left( \frac{k - 1007^2}{p_i} \right) = \left( \frac{-i - 1007^2}{p_i} \right) = - \left( \frac{1007^2 + i}{p_i} \right) = 1,
	\]
	do đó, tồn tại \( x_i \) sao cho \( x_i^2 \equiv k - 1007^2 \Mod{p_i} \), và \( p_i \nmid x_i \).

	Sử dụng phương pháp nâng nghiệm Hensel, ta có thể tìm \( t_i \in \mathbb{Z} \) sao cho:
	\[
		p_i^2 \mid (x_i + p_i t_i)^2 - (k - 1007^2)
	\]
	
	Gọi \( a \equiv x_i + p_i t_i \Mod{p_i^2} \) với mỗi \( i \), và sử dụng CRT, tồn tại vô hạn \( a \in \mathbb{N} \) thỏa:
	\[
		a^2 = k - 1007^2 + \beta_a M \implies k = a^2 + 1007^2 + \beta_a M \implies k + 2015 = a^2 + 1008^2 + \beta_a M
	\]
	
	Vì vậy, với:
	\[
		n = k + \beta_a M = a^2 + 1007^2
	\]
	thì \( n \) và \( n + 2015 \) đều là tổng hai bình phương, nhưng các số \( n+1, n+2, \dots, n+2014 \) thì không.
	Do đó, ta tìm được hiệu \( b_{m+1} - b_m = 2015 \). Vì \( a \) có thể chọn tùy ý lớn, nên tồn tại vô hạn nhiều khoảng như vậy.
\end{soln}

\footnotetext{\href{https://artofproblemsolving.com/community/c6h1087616p5864559}{Dựa theo lời giải của \textbf{Ariscrim}.}}

\end{document}