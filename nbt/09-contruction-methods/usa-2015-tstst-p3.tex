\documentclass[../09-contruction-methods.tex]{subfiles}

\begin{document}

\begin{example*}[\gls{USA 2015 TSTST}/P3]\label{example:USA-2015-TSTST-P3}[\textbf{\nameref{definition:40M}}]
	Giả sử \( P \) là tập hợp tất cả các số nguyên tố, và \( M \) là một tập con không rỗng của \( P \).
	Giả sử rằng với mọi tập con không rỗng \( \{p_1, p_2, \ldots, p_k\} \) của \( M \), tất cả các ước số nguyên tố của \( p_1p_2\ldots p_k + 1 \) cũng thuộc \( M \).
	Chứng minh rằng \( M = P \).
\end{example*}

\begin{story*}
    Bài toán yêu cầu xây dựng hai số cố định \( a, b \) sao cho bất kỳ cặp số \( m, n \) nguyên tố cùng nhau nào cũng không thể nằm gần \( (a, b) \).
    Ta xét việc chọn \( a = b = N! \) với \( N \) lớn (ví dụ \( N = 1000 \)).
    Khi đó, mọi số nhỏ hơn hoặc bằng \( N \) đều chia hết \( a \) và \( b \),
    trong khi \( m, n \) nguyên tố cùng nhau không thể đồng thời chia hết cho cùng các thừa số nguyên tố của \( N! \).
    Vì vậy, \( m \ne a \) và \( n \ne b \), và sự khác biệt tuyệt đối với từng hoán vị là lớn.
    Ta kiểm soát được giá trị \( |a - m| + |b - n| \) bằng việc ép \( m, n \) không thể gần \( a, b \). Từ đó ta đảm bảo giá trị luôn vượt quá 1000.
\end{story*}

\bigbreak

\begin{soln}\footnotemark
    Giả sử ngược lại rằng tồn tại số nguyên tố \( p \notin M \).  
    Do điều kiện đề bài, ta biết rằng nếu lấy tích các số nguyên tố trong \( M \) rồi cộng \( 1 \), thì các ước số nguyên tố của kết quả luôn thuộc \( M \).  
    Ta sẽ xây dựng một dãy số trong \( M \) để dẫn đến mâu thuẫn với giả thiết \( p \notin M \).
    
    Xét các lớp dư modulo \( p \). Gọi \( X \) là tập các số nguyên tố trong \( M \) mà lớp dư modulo \( p \) của chúng xuất hiện vô hạn lần trong \( M \),  
    và \( Y = M \setminus X \), tức là tập các số nguyên tố trong \( M \) mà lớp dư modulo \( p \) chỉ xuất hiện hữu hạn lần.  
    Vì chỉ có hữu hạn lớp dư modulo \( p \), nên \( Y \) là hữu hạn.
    
    Đặt 
    \[
    t = 
    \begin{cases}
        1 & \text{nếu } Y = \emptyset, \\
        \prod\limits_{y \in Y} y & \text{nếu } Y \neq \emptyset.
    \end{cases}
    \]
    
    Rõ ràng \( p \nmid t \), vì \( p \notin M \) và \( t \) là tích các phần tử của \( M \).
    
    Bây giờ, ta xây dựng một dãy \( \{a_n\} \) như sau:
    
    \begin{itemize}[topsep=0pt, partopsep=0pt, itemsep=0pt]
        \item Đặt \( a_0 = 1 \).
        \item Với \( n \ge 0 \), xét \( ta_n + 1 \) và phân tích nó thành thừa số nguyên tố:
      	\[
      		ta_n + 1 = p_1^{\alpha_1} p_2^{\alpha_2} \ldots p_k^{\alpha_k}.
      	\]
    	Vì \( (t, ta_n + 1) = 1 \), nên mọi \( p_i \) đều không chia hết \( t \) và do đó không thuộc \( Y \), tức là \( p_i \in X \subseteq M \).
    	\item Với mỗi \( p_i \), do \( p_i \in X \) nên có vô hạn số nguyên tố trong \( M \) đồng dư với \( p_i \Mod{p} \).
    	Do đó, ta có thể chọn \( \alpha_i \) số nguyên tố phân biệt trong \( M \), mỗi số đồng dư với \( p_i \Mod{p} \), và tất cả các số này đều phân biệt.  
    	Gọi \( a_{n+1} \) là tích của các số nguyên tố được chọn.
    \end{itemize}
    
    Rõ ràng \( a_{n+1} \equiv ta_n + 1 \Mod{p} \). Vì \( a_0 = 1 \), nên ta có:
    \[
    	a_1 \equiv t + 1 \Mod{p}, \quad a_2 \equiv t(t + 1) + 1 = t^2 + t + 1 \Mod{p}, \quad \text{v.v.}
    \]
    
    Suy ra:
    \[
    	a_n \equiv t^n + t^{n-1} + \ldots + 1 \Mod{p}.
    \]
    
    Xét ba trường hợp:
    \begin{itemize}[topsep=0pt, partopsep=0pt, itemsep=0pt]
        \item Nếu \( t \equiv 0 \Mod{p} \) thì \( p \mid t \), mâu thuẫn với \( p \notin M \).
        \item Nếu \( t \equiv 1 \Mod{p} \) thì \( a_p \equiv p \equiv 0 \Mod{p} \).
        \item Nếu \( t \not\equiv 0,1 \Mod{p} \) thì theo công thức cấp số nhân:
        \[
        	a_{p-2} \equiv \frac{t^{p-1} - 1}{t - 1} \equiv 0 \Mod{p}.
        \]
    \end{itemize}
    
    Trong cả ba trường hợp, tồn tại \( n \) sao cho \( a_n \equiv 0 \Mod{p} \).
	Nhưng điều này vô lý, vì mỗi \( a_n \) là tích của các số nguyên tố thuộc \( M \), nên không thể chia hết cho \( p \notin M \).
    
    Vậy giả thiết ban đầu sai. Suy ra \( M = P \).
\end{soln}

\footnotetext{\href{https://artofproblemsolving.com/community/c6h1062614p18497986}{Lời giải của \textbf{Aiscrim}} do \textbf{Evan Chen} \href{https://web.evanchen.cc/exams/sols-TSTST-2015.pdf}{viết lại}.}

\end{document}