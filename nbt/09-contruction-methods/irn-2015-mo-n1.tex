\documentclass[../09-contruction-methods.tex]{subfiles}

\begin{document}

\begin{example*}[\gls{IRN 2015 MO}/N1]\label{example:IRN-2015-N1}[\textbf{unrated}]
	Chứng minh rằng tồn tại vô hạn số tự nhiên \( n \) sao cho \( n \) không thể viết được dưới dạng tổng của hai số nguyên dương
	mà tất cả các thừa số nguyên tố của chúng đều nhỏ hơn 1394.
\end{example*}

\begin{soln}(Cách 1)\footnotemark
	Gọi \( p_1, p_2, \dots, p_k \) là tất cả các số nguyên tố nhỏ hơn 1394.

	Ký hiệu số tự nhiên \textit{số \( P \)-mịn (smooth)} nếu tất cả các ước số nguyên tố đều thuộc tập \( \{p_1, \dots, p_k\} \).
	
	Với \( m \) nguyên dương, ta ước lượng số các số nguyên dương không vượt quá \( m \) và là số \( P \)-mịn như sau:
	\[
		x_m = (\lfloor \log_{p_1} m \rfloor + 1) \cdot (\lfloor \log_{p_2} m \rfloor + 1) \cdots (\lfloor \log_{p_k} m \rfloor + 1)
	\]
	
	Lý do: Mỗi số \( P \)-mịn không vượt quá \( m \) có thể biểu diễn dưới dạng:
	\[
		p_1^{a_1} p_2^{a_2} \cdots p_k^{a_k} \leq m,\  a_i \geq 0 \tag{1}
	\]
	
	Do đó số lượng tổ hợp các bộ số mũ \( (a_1, a_2, \dots, a_k) \) thỏa mãn (1) bị chặn bởi \( x_m \).
	
	Từ đó, số các cặp \( (a, b) \) là hai số nguyên dương \( P \)-mịn có tổng không vượt quá \( m \) bị chặn trên bởi \( x_m^2 \).
	
	Do đó, số lượng các số tự nhiên \( \leq m \) \textbf{có thể} viết được dưới dạng tổng của hai số \( P \)-mịn là không quá \( x_m^2 \).
	
	Suy ra, số lượng các số tự nhiên \( \leq m \) \textbf{không} thể biểu diễn dưới dạng tổng của hai số \( P \)-mịn là ít nhất:
	\[
		m - x_m^2.
	\]
	
	Giờ ta xét giới hạn khi \( m \to \infty \). Ta thấy:
	\[ x_m = \prod_{i=1}^{k} (\log_{p_i} m + 1) = O((\log m)^k) \implies x_m^2 = O((\log m)^{2k}) \implies \lim_{ m \to \infty} m - x_m^2 \to \infty. \]
	
	Vì vậy, tồn tại vô hạn số tự nhiên \( n \) mà không thể viết được thành tổng của hai số nguyên dương có tất cả các thừa số nguyên tố nhỏ hơn 1394.
\end{soln}

\footnotetext{\href{https://artofproblemsolving.com/community/c6h1138945p5340571}{Dựa theo lời giải của SCLT.}}

\end{document}