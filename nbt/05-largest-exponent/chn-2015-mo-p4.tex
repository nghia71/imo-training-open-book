\documentclass[../05-largest-exponent.tex]{subfiles}

\begin{document}

\begin{example*}[\gls{CHN 2015 MO}/P4]\label{example:CHN-2015-P4}[\textbf{unrated}]
	Xác định tất cả các số nguyên \( k \) sao cho tồn tại vô hạn số nguyên dương \( n \) không thỏa mãn:
	\[
		n + k \mid \binom{2n}{n}.
	\]
\end{example*}

\begin{soln}(Cách 1)\footnotemark
	\footnotetext{\samepage \href{https://artofproblemsolving.com/community/c6h618236p3687002}{Lời giải của yunxiu.}}

	Ta xét ba trường hợp sau.
	
	\textit{Trường hợp 1: Nếu \( k = 0 \)}, ta có thể chọn \( n = 2^\alpha \) với mọi số nguyên dương \( \alpha \geq 2 \).  
	Theo định lý Kummer, ta có
	\[
		v_2\left( \binom{2n}{n} \right) = 1 < \alpha = v_2(2^\alpha) = v_2(n + k).
	\]
		
	\textit{Trường hợp 2: Nếu \( k \neq 0,1 \)}, với mọi số nguyên dương \( \alpha \geq 3 + \log_2 |k| \), ta có thể chọn  
	\( n = 2^\alpha - k \). Trong hệ cơ số \( p \), \( n \) có nhiều nhất \( \alpha \) chữ số, với chữ số ít quan trọng nhất bằng 0.  
	Do đó, có nhiều nhất \( \alpha-1 \) lần nhớ khi cộng \( n \) vào chính nó\footnotemark, và do đó theo định lý Kummer, ta có  
	\[
		v_2\left( \binom{2n}{n} \right) \leq \alpha - 1 < \alpha = v_2(n + k).
	\]
		
	\textit{Trường hợp 3: Nếu \( k = 1 \)}, ta có  
	\[
		\frac{1}{n+1} \binom{2n}{n} = \binom{2n}{n} - \frac{n}{n+1} \binom{2n}{n} = \binom{2n}{n} - \binom{2n}{n+1}
	\]
	là một số nguyên, do đó \( (n+1) \mid \binom{2n}{n} \) với mọi \( n \).  

	Vậy, tất cả các số nguyên \( k \neq 1 \) đều thỏa mãn điều kiện.

	(\textit{Lưu ý rằng: \( \frac{1}{n+1} \binom{2n}{n} \) là một số Catalan.})
\end{soln}

\footnotetext{\href{https://math.dartmouth.edu/~carlp/catalan5.pdf}{Divisors of the middle binomial coefficient.}}

\end{document}