\documentclass[../03-arithmetic-functions.tex]{subfiles}

\begin{document}

\begin{exercise*}[\gls{TWN 2015 TST}2/Q2/P1]\label{example:GBR-2015-TST2-Q2-P3}[\textbf{\nameref{definition:20M}}]
    Với mỗi số nguyên dương \( n \), định nghĩa:
    \[
        a_n = \sum_{k=1}^{\infty} \left\lfloor \frac{n + 2^{k-1}}{2^k} \right\rfloor,
    \]
    trong đó \( \left\lfloor x \right\rfloor \) là phần nguyên của \( x \), tức là số nguyên lớn nhất không vượt quá \( x \).
\end{exercise*}

\begin{remark*}
    Gợi ý: Thử viết \( n \) trong cơ số 2 rồi phân tích biểu thức \( \left\lfloor \tfrac{n + 2^{k-1}}{2^k} \right\rfloor \).  
    Có thể tách từng chữ số nhị phân và xem cách các số hạng góp phần vào tổng.
\end{remark*}

\begin{story*}
    Bài toán yêu cầu hiểu tổng vô hạn của các phần nguyên liên quan đến biểu thức \( \frac{n + 2^{k-1}}{2^k} \).  
    Một hướng tiếp cận hiệu quả là viết \( n \) dưới dạng nhị phân:
    \[
        n = d_0 + 2d_1 + 4d_2 + \cdots + 2^m d_m,
    \]
    với \( d_i \in \{0, 1\} \). Khi đó, các số hạng \( \left\lfloor \frac{n + 2^{k-1}}{2^k} \right\rfloor \) lần lượt cộng lại từng chữ số \( d_i \), có tính chồng lấp kiểu “carry” như trong phép chia lùi. Tổng đó thực chất tính số lượng chữ số 1 trong biểu diễn nhị phân của \( n \), cộng với một phần tuyến tính từ các chữ số lớn hơn.  
    Phân tích đầy đủ sẽ chỉ ra rằng \( a_n = n \), hoặc liên quan mật thiết đến \( n \), bằng cách nhóm lại các phần tử của chuỗi.
\end{story*}

\end{document}