\documentclass[../03-arithmetic-functions.tex]{subfiles}

\begin{document}

\begin{example*}[\gls{ROU 2015 TST}/D2/P1]\label{example:ROU-2015-TST-D2-P1}[\textbf{\nameref{definition:25M}}]
    Cho \( a \in \mathbb{Z} \) và \( n \in \mathbb{N}_{>0} \). Chứng minh rằng:
    \[
        \sum_{k=1}^{n} a^{\gcd(k,n)}
    \]
    luôn chia hết cho \( n \), trong đó \( \gcd(k,n) \) là ước chung lớn nhất của \( k \) và \( n \).
\end{example*}

\begin{story*}
    Phương pháp: ta nhóm các số \(k\) có cùng \(\gcd(k,n)=d\) để viết tổng dưới dạng \(\sum_{d\mid n}\phi(d)\,a^{n/d}\).
    Sau đó, sử dụng việc khi \(n\) là lũy thừa nguyên tố, có thể áp dụng định lý Euler \((a^{\phi(p^k)} \equiv 1\Mod{p^k})\) để chứng minh tổng chia hết cho \(p^k\).
    Kế đến, duyệt qua tất cả các thừa số nguyên tố của \(n\) và kết hợp bằng nguyên lý đồng dư Trung Hoa, suy ra tổng chia hết cho \(n\).
\end{story*}

\bigbreak

\begin{soln}
    Ta chứng minh hai khẳng định sau.

    \begin{theorem*}[label:claimOne]
        \[
            \sum_{k=1}^n a^{\gcd(k,n)} \;=\; \sum_{d \mid n} \phi(d)\, a^{n/d}.
        \]
    \end{theorem*}

    \begin{subproof}
        Với mỗi \( d \mid n \), có đúng \(\phi\bigl(\tfrac{n}{d}\bigr)\) số \( k\) trong đoạn \([1,n]\) thỏa \(\gcd(k,n)=d\). Thay biến \( e = \tfrac{n}{d}\), thu được tổng \(\sum_{d\mid n}\phi(d)\,a^{n/d}\).
    \end{subproof}

    \begin{theorem*}[label:claimTwo]
        Nếu \( p \) là số nguyên tố và \( \gcd(a,p)=1 \), thì:
        \[
            a^{p^k} \;\equiv\; a^{p^{k-1}} \;\Mod{p^k}.
        \]
    \end{theorem*}

    \begin{subproof}
        Theo định lý Euler, \( a^{\phi(p^k)} \equiv 1 \Mod{p^k}\), với \(\phi(p^k)=p^{k-1}(p-1)\). Từ đó:
        \[
            a^{p^k} \;=\; a^{p\cdot p^{k-1}} \;\equiv\; a^{p^{k-1}} \;\Mod{p^k}.
        \]
    \end{subproof}

    \textit{Trường hợp 1:} \( n = p^s \), với \(p\) nguyên tố.  
    Từ \nameref{claimOne},
    \[
        \sum_{k=1}^{p^s} a^{\gcd(k,p^s)}
        \;=\;
        \sum_{d \mid p^s} \phi(d)\, a^{p^s/d}.
    \]
    Các ước \(d\) gồm \(p^0, p^1, \dots, p^s\), do đó:
    \[
        S \;=\; a^{p^s} + (p-1)\,a^{p^{s-1}} + (p^2-p)\,a^{p^{s-2}}
        \;+\dots+\; (p^s - p^{s-1})\,a.
    \]
    Ta chứng minh bằng quy nạp theo \(s\) rằng \(p^s \mid S\).  
    \(\bullet\) Cơ sở \(s=1\): \(a^p + (p-1)\,a = p\,a \equiv 0\Mod{p}.\)  
    \(\bullet\) Giả thiết quy nạp và \nameref{claimTwo} cho thấy mỗi bước đều bảo toàn tính chia hết \(p^s\).  

    \textit{Trường hợp 2:} \(n\) có ít nhất hai thừa số nguyên tố (một dạng “tổng quát”).  
    Giả sử \( n = p^s m\) với \(\gcd(p,m)=1\). Theo \nameref{claimOne}:
    \[
        \sum_{d \mid n} \phi(d)\, a^{n/d} 
        = 
        \sum_{\substack{d \mid n \\ p \nmid d}} \phi(d)\, a^{n/d} 
        \;+\;
        \sum_{\substack{d \mid n \\ p \mid d}} \phi(d)\, a^{n/d}.
    \]
    Phần đầu chia hết cho \(m = n/p^s\) (theo giả thiết quy nạp với số nhỏ hơn), phần sau chia hết cho \(p^s\) (theo bước lũy thừa nguyên tố). Cuối cùng, vì \(\gcd\bigl(m,p^s\bigr)=1\), nên suy ra \(n\mid\sum_{k=1}^n a^{\gcd(k,n)}\) nhờ nguyên lý đồng dư Trung Hoa.

    Kết luận: 
    \[
        n \;\Big|\; \sum_{k=1}^{n} a^{\gcd(k,n)}.
    \]
\end{soln}

\footnotetext{\href{https://artofproblemsolving.com/community/c6h1097351p4930928}{Dựa theo lời giải của \textbf{andria}.}}

\end{document}