\documentclass[../03-arithmetic-functions.tex]{subfiles}

\begin{document}

\begin{example*}[\gls{CHN 2015 TST}3/D2/P3]\label{example:CHN-2015-TST3-D2-P3}[\textbf{unrated}]
	Với mọi số tự nhiên \( n \), định nghĩa:
	\[
		f(n) = \tau(n!) - \tau((n-1)!),
	\]
	trong đó \( \tau(a) \) là số ước số dương của \( a \).  
	
	Chứng minh rằng tồn tại vô hạn số \( n \) là hợp số sao cho với mọi số tự nhiên \( m < n \), ta có:
	\[
		f(m) < f(n).
	\]
\end{example*}

\begin{soln}(Cách 1)\footnotemark
	Cho \( p \) là một số nguyên tố lẻ. Theo \nameref{theorem:bertrand}, tồn tại số nguyên tố giữa $p$ và $2p$.
	Giả sử $q$ là số nguyên tố lớn nhất trong các số nguyên tố giữa $p$ và $2p$.
	Ta chứng minh khẳng định sau
	\begin{claim*}
		$f(2p) > f(q).$
	\end{claim*}
	\begin{subproof}
		Ta có:
		\[
			\begin{aligned} 
				f(2p) &= \tau((2p)!) - \tau((2p-1)!) = \frac{3}{2} \cdot \tau(2(2p-1)!) - \tau((2p-1)!) \\ 
			 	&= 3\tau\left(\frac{2(2p-1)!}{q}\right) - 2\tau\left(\frac{(2p-1)!}{q}\right) 
			  	> \tau\left(\frac{(2p-1)!}{q}\right) \geq \tau((q-1)!) = f(q).	
			\end{aligned}
		\]
	\end{subproof}

	Gọi \( n \) là số nguyên dương nhỏ nhất thỏa mãn \( n \leq 2p \) và \( f(n) \) đạt giá trị lớn nhất trong dãy \( f(1), f(2), \dots, f(2p) \).  
			
	Nếu \( n \leq q \), thì \( f(n) \leq \tau((q-1)!) = f(q) < f(2p) \), mâu thuẫn.  
	Do đó, \( n > q \), và từ định nghĩa của \( q \), ta suy ra \( n \) là hợp số và $ f(n)>f(m) $ với mọi $ m<n $. 
	
	Hơn nữa, vì \( n \in [p+1, 2p+1] \), ta có vô hạn giá trị \( n \) khi \( p \) chạy qua tất cả các số nguyên tố lẻ.
\end{soln}

\footnotetext{\href{https://artofproblemsolving.com/community/c6h1069931p21493104}{Lời giải của chirita.andrei.}}

\end{document}