\documentclass[../../imo-training-open-book.tex]{subfiles}

\begin{document}

\section{Lý thuyết}

\begin{definition*}[Hàm số học]
    \label{definition:arithmetic-function}
    $f:\ \NN \to \CC$ là một hàm số học.
\end{definition*}

\begin{definition*}[Phần nguyên]
    \label{definition:floor}
    $\floor{\circ}:\ \RR \rightarrow \ZZ$ là một hàm thỏa mãn điều kiện $\floor{x} = n$, trong đó $n \in \ZZ,\ n \leq x < n+1.$
    $\floor{x}$ được gọi là phần nguyên, hàm sàn, hoặc sàn của $x$. 
\end{definition*}

\begin{definition*}[Hàm Trần]
    \label{definition:ceiling}
    $\ceiling{\circ}:\ \RR \rightarrow \ZZ$ là một hàm thỏa mãn điều kiện $\ceiling{x} = n$, trong đó $n \in \ZZ,\ n-1 \leq x \leq n.$
    $\ceiling{x}$ được gọi là hàm trần, hoặc trần của $x$.
\end{definition*}

\begin{definition*}[Phần thập phân]
    \label{definition:fractional}
    $\{ x \}:\ \RR \rightarrow [0,1)$ là một hàm thỏa mãn điều kiện $\{x\} = x - \floor{x}.$
    $\{x\}$ được gọi là phần thập phân của $x$.
\end{definition*}

\begin{definition*}[Hàm cộng tính]
    \label{definition:additive-function}
    Một hàm số học \( f:\ \NN \to \CC \) được gọi là \textbf{cộng tính} nếu thỏa mãn:
    \[
        f(mn) = f(m) + f(n),\quad \forall m,n \in \NN,\ (m,n)=1.
    \]
\end{definition*}

\begin{definition*}[Hàm nhân tính]
    \label{definition:multiplicative-function}
    Một hàm số học \( f:\ \NN \to \CC \) được gọi là \textbf{nhân tính} nếu thỏa mãn:
    \[
        f(mn) = f(m)f(n),\quad \forall m,n \in \NN,\ (m,n)=1.
    \]
\end{definition*}

\begin{definition*}[Hàm số ước số dương]
    \label{definition:tau-function}
    Với $n \in \ZZ^{+},\ n=p_1^{a_1}p_2^{a_2} \cdots p_k^{a_k}$, $\tau(n) = (1+a_1)(1+a_2) \cdots (1+a_k).$
    $d(n)$ cũng được dùng thay cho $\tau(n).$
\end{definition*}

\begin{definition*}[Hàm tổng lũy thừa ước số dương]
    \label{definition:sigma-function}
    Với $n \in \NN$, $\sigma_k$ là tổng các $k^{\text{h}}$ lũy thừa của các ước số dương của $n$.  
    \[
        \sigma_k(n) = \sum_{d \mid n} d^k.
    \]
    $d(n)$ hoặc $\tau(n)$ ký hiệu cho $\sigma_0(n)$, tức số ước của $n$, và $\sigma(n)$ ký hiệu cho $\sigma_1(n)$, tức tổng các ước số của $n$.
\end{definition*}

\begin{theorem*}[Công thức cho $\sigma(n)$]
    \label{theorem:sigma-function}
    Với $n \in \ZZ^{+}$, $n=p_1^{a_1}p_2^{a_2} \cdots p_k^{a_k}$, thì
    \[
        \sigma(n) = \frac{p_1^{a_1+1}-1}{p_1-1} \cdot \frac{p_2^{a_2+1}-1}{p_2-1} \cdots \frac{p_k^{a_k+1}-1}{p_k-1}.
    \]
\end{theorem*}

\begin{definition*}[Hàm Euler's Totient]
    \label{definition:euler-totient-function}
    Với $n \in \ZZ^{+}$, $n=p_1^{a_1}p_2^{a_2} \cdots p_k^{a_k}$, thì
    $\varphi(n)$ là số các số nguyên dương nhỏ hơn $n$ và nguyên tố cùng nhau với $n$.
\end{definition*}

\begin{theorem*}[Công Thức Cho $\varphi(n)$]
    \label{theorem:euler-totient-function}
    Với $n \in \ZZ^{+}$, $n=p_1^{a_1}p_2^{a_2} \cdots p_k^{a_k}$, thì
    \[
        \begin{aligned}
            \varphi(n) &= n\left(1-\frac{1}{p_1} \right) \left(1-\frac{1}{p_2} \right) \cdots \left(1-\frac{1}{p_k} \right),\\[1ex]
            \varphi(n) &= p_1^{a_1-1}p_2^{a_2-1} \cdots p_k^{a_k-1}(p_1-1)(p_2-1)\cdots (p_k-1).
        \end{aligned}
    \]
\end{theorem*}

\begin{lemma*}[Các hàm nhân tính cơ bản]
    \label{lemma:basic-additive-functions}
    Các hàm $d(n)$, $\sigma(n)$ và $\varphi(n)$ là các hàm nhân.
\end{lemma*}

\begin{theorem}[Bổ đề Bertrand]
    \label{theorem:bertrand}
    Với mọi số nguyên \( n \geq 1 \), luôn tồn tại một số nguyên tố \( p \) thỏa mãn:
    \[
        n < p < 2n.
    \]
\end{theorem}

\newpage

\section{Các ví dụ}

\subfile{./03-arithmetic-functions/chn-2015-tst3-d2-p3.tex} \newpage
\subfile{./03-arithmetic-functions/imo-2015-n1.tex} \newpage

\section{Bài tập}

\newpage

\end{document}