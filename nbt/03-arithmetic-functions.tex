\documentclass[../../imo-training-open-book.tex]{subfiles}

\begin{document}

\section{Lý thuyết}

\begin{definition*}[Hàm số học]
    \label{definition:arithmetic-function}
    $f:\ \NN \to \CC$ là một hàm số học.
\end{definition*}

\begin{definition*}[Phần nguyên]
    \label{definition:floor}
    $\floor{\circ}:\ \RR \rightarrow \ZZ$ là một hàm thỏa mãn điều kiện $\floor{x} = n$, trong đó $n \in \ZZ,\ n \leq x < n+1.$
    $\floor{x}$ được gọi là phần nguyên, hàm sàn, hoặc sàn của $x$. 
\end{definition*}

\begin{definition*}[Hàm Trần]
    \label{definition:ceiling}
    $\ceiling{\circ}:\ \RR \rightarrow \ZZ$ là một hàm thỏa mãn điều kiện $\ceiling{x} = n$, trong đó $n \in \ZZ,\ n-1 \leq x \leq n.$
    $\ceiling{x}$ được gọi là hàm trần, hoặc trần của $x$.
\end{definition*}

\begin{definition*}[Phần thập phân]
    \label{definition:fractional}
    $\{ x \}:\ \RR \rightarrow [0,1)$ là một hàm thỏa mãn điều kiện $\{x\} = x - \floor{x}.$
    $\{x\}$ được gọi là phần thập phân của $x$.
\end{definition*}

\begin{definition*}[Hàm cộng tính]
    \label{definition:additive-function}
    Một hàm số học \( f:\ \NN \to \CC \) được gọi là \textbf{cộng tính} nếu thỏa mãn:
    \[
        f(mn) = f(m) + f(n),\quad \forall m,n \in \NN,\ (m,n)=1.
    \]
\end{definition*}

\begin{definition*}[Hàm nhân tính]
    \label{definition:multiplicative-function}
    Một hàm số học \( f:\ \NN \to \CC \) được gọi là \textbf{nhân tính} nếu thỏa mãn:
    \[
        f(mn) = f(m)f(n),\quad \forall m,n \in \NN,\ (m,n)=1.
    \]
\end{definition*}

\begin{definition*}[Hàm số ước số dương]
    \label{definition:tau-function}
    Với $n \in \ZZ^{+},\ n=p_1^{a_1}p_2^{a_2} \cdots p_k^{a_k}$, $\tau(n) = (1+a_1)(1+a_2) \cdots (1+a_k).$
\end{definition*}

\begin{theorem}[Bổ đề Bertrand]
    \label{theorem:bertrand}
    Với mọi số nguyên \( n \geq 1 \), luôn tồn tại một số nguyên tố \( p \) thỏa mãn:
    \[
        n < p < 2n.
    \]
\end{theorem}

\newpage

\section{Các ví dụ}

\subfile{./03-arithmetic-functions/chn-2015-tst3-d2-p3.tex}
\subfile{./03-arithmetic-functions/imo-2015-n1.tex}

\newpage

\section{Bài tập}

\newpage

\end{document}