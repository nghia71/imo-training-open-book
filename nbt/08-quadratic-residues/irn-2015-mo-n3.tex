\documentclass[../08-quadratic-residues.tex]{subfiles}

\begin{document}

\begin{example*}[\gls{IRN 2015 MO}/N3]\label{example:IRN-2015-N3}[\textbf{unrated}]
	Cho \( p > 5 \) là một số nguyên tố. Gọi \( A = \{b_1, b_2, \dots, b_{\frac{p-1}{2}}\} \)
	là tập tất cả các bình phương đồng dư modulo \( p \), loại trừ 0.
	Chứng minh rằng không tồn tại các số tự nhiên \( a, c \) sao cho \( \gcd(ac, p) = 1 \) và tập
	\[
		B = \{ab_1 + c, ab_2 + c, \dots, ab_{\frac{p-1}{2}} + c\} \pmod{p}
	\]
	không giao \( A \), tức là \( A \cap B = \emptyset \mod{p} \).
\end{example*}

\begin{soln}(Cách 1)\footnotemark
	Ta làm việc trong trường \( \mathbb{F}_p = \mathbb{Z}/p\mathbb{Z} \). Gọi \( \left( \frac{\cdot}{p} \right) \) là ký hiệu Legendre.

	\textit{Trường hợp 1:} \( \left( \frac{a}{p} \right) = 1 \)
	
	Trong trường hợp này, do tính chất nhân của ký hiệu Legendre, ta có:
	\[
		\left( \frac{ab_i}{p} \right) = \left( \frac{a}{p} \right) \left( \frac{b_i}{p} \right) = 1 \quad \text{với mọi } i
	\]
	
	Tức là \( ab_i \) vẫn là các bình phương đồng dư. Khi đó, tập \( S = \{ab_1, ab_2, \dots, ab_{\frac{p-1}{2}}\} \) chính là \( A \).
	
	Bây giờ giả sử \( B = \{ab_i + c\} \) không giao với \( A \). Ta sẽ chứng minh điều này dẫn đến mâu thuẫn.
	
	Xét một phần tử \( r \in \mathbb{F}_p^* \) sao cho \( r^2 \not\equiv \pm c \mod{p} \). Vì \( p > 5 \), nên tồn tại ít nhất một giá trị như vậy.
	
	Đặt \( s = \frac{c}{r} \), và xét hai biểu thức:
	\[
		x_1 = \left( \frac{r + s}{2} \right)^2, \quad x_2 = \left( \frac{r - s}{2} \right)^2 \implies x_1 - x_2 = rs = c.
	\]

	Do đó, \( x_2 + c = x_1 \), nghĩa là nếu \( x_2 \in A \), thì \( x_2 + c \in A \), tức là tồn tại phần tử trong \( B \) cũng thuộc \( A \),
	mâu thuẫn với giả thiết rằng \( A \cap B = \emptyset \).
	
	\textit{Trường hợp 2:} \( \left( \frac{a}{p} \right) = -1 \)
	
	Khi đó, do \( b_i \) là bình phương, nên \( ab_i \) là không bình phương. Tức là tập \( S = \{ab_i\} \) chính là tập các phần tử không là bình phương modulo \( p \).
	
	Bây giờ giả sử \( B = \{ab_i + c\} \) rời nhau với \( A \), tức là mọi phần tử của \( B \) cũng là không bình phương.
	
	Ta xét dãy:
	\[
		S = \{-c, -2c, \dots, -\tfrac{p-1}{2}c\} \pmod{p}
	\]
	
	Đây là một hoán vị (theo hệ số) của các phần tử của \( S = aA \). Nếu giả thiết đúng thì dãy này gồm toàn bộ các phần tử không bình phương.
	
	Bây giờ xét tổng các phần tử trong dãy trên. Tổng này là:
	\[
		\sum_{i=1}^{\frac{p-1}{2}} (-ic) = -c \cdot \sum_{i=1}^{\frac{p-1}{2}} i = -c \cdot \frac{(p-1)(p+1)}{8}
	\]
	
	Gọi \( s \) là tổng các phần tử trong \( B \), thì \( s \equiv \sum ab_i + \frac{p-1}{2}c \pmod{p} \).
	Ta so sánh với tổng của các phần tử trong \( A \). Do \( A \) là đối xứng (vì với mỗi \( x \in \mathbb{F}_p^* \), \( x^2 \in A \)), tổng của \( A \) là cố định.
	
	Nếu phép dịch \( +c \) bảo toàn tập không bình phương, thì tổng của \( B \) phải có dạng giống như tổng của \( S \), tức là có giá trị khác tổng của \( A \).
	Nhưng nếu \( B \) không trùng \( A \) và vẫn là không bình phương, thì điều này dẫn đến mâu thuẫn về tổng.
	
	Mặt khác, nếu \( c^{-1}S = \{-1, -2, \dots, -\frac{p-1}{2} \} \), thì tập này gồm các phần tử \textit{liên tiếp}.
	Nhưng tập các phần tử không bình phương modulo \( p \) không thể là một cấp số cộng kiểu đó vì nó không đóng dưới phép cộng.
	Do đó, \( c^{-1}S \) phải là tập các bình phương, tức là:
	\[
		\left( \frac{-1}{p} \right) = 1, \quad \text{và } \left( \frac{ic}{p} \right) = 1
	\]
	Nhưng điều này dẫn đến mâu thuẫn vì \( S \) là tập không bình phương. Do đó, mâu thuẫn.

	Trong cả hai trường hợp \( \left( \frac{a}{p} \right) = \pm 1 \), giả thiết rằng \( A \cap B = \emptyset \) đều dẫn đến mâu thuẫn.
	Vậy không tồn tại \( a, c \in \mathbb{N} \) với \( \gcd(ac, p) = 1 \) sao cho \( B \) hoàn toàn nằm ngoài \( A \) modulo \( p \).
\end{soln}

\footnotetext{\href{https://artofproblemsolving.com/community/c6h1140077p5349209}{Dựa theo lời giải của Dukejukem và rafayaashary1.}}

\end{document}