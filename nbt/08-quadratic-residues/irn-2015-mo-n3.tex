\documentclass[../08-quadratic-residues.tex]{subfiles}

\begin{document}

\begin{example*}[\gls{IRN 2015 MO}/N3]\label{example:IRN-2015-MO-N3}[\textbf{\nameref{definition:35M}}]
	Cho \( p > 5 \) là một số nguyên tố. Gọi \( A = \{b_1, b_2, \dots, b_{\frac{p-1}{2}}\} \)
	là tập tất cả các bình phương đồng dư modulo \( p \), loại trừ 0.
	Chứng minh rằng không tồn tại các số tự nhiên \( a, c \) sao cho \( \gcd(ac, p) = 1 \) và tập
	\[
		B = \{ab_1 + c, ab_2 + c, \dots, ab_{\frac{p-1}{2}} + c\} \Mod{p}
	\]
	không giao \( A \), tức là \( A \cap B = \emptyset \mod{p} \).
\end{example*}

\begin{story*}
	Bài toán khai thác sâu các tính chất của ký hiệu Legendre trong trường hữu hạn \( \mathbb{F}_p \). Hai trường hợp \( a \) là bình phương và không bình phương dẫn đến các cấu trúc khác nhau cho tập \( aA + c \). Dùng tính chất bảo toàn tập qua phép nhân và phép tịnh tiến, kết hợp lý lẽ về tổng phần tử và tính đối xứng, ta suy ra rằng không thể tránh được giao với tập bình phương.
\end{story*}

\bigbreak

\begin{soln}\footnotemark
	Ta làm việc trong trường \( \mathbb{F}_p \). Gọi \( A \subset \mathbb{F}_p^* \) là tập các bình phương (modulo \(p\)), có đúng \( \frac{p-1}{2} \) phần tử.  
	Giả sử tồn tại \( a, c \in \mathbb{N} \) với \( \gcd(ac, p) = 1 \) sao cho tập \( B = \{ ab_i + c \pmod{p} \} \) không giao với \( A \), tức \( A \cap B = \emptyset \).

	\textbf{Trường hợp 1:} \( \left( \frac{a}{p} \right) = 1 \)  
	Vì tích của bình phương với bình phương vẫn là bình phương, \( aA = A \). Khi đó \( B = A + c \) là phép tịnh tiến của \( A \). Nếu \( B \cap A = \emptyset \), thì \( A \cap (A + c) = \emptyset \). Nhưng điều này mâu thuẫn với:
	\[
		|A| = \frac{p-1}{2},\quad |A + c| = |A|,\quad A \cup (A + c) \subset \mathbb{F}_p^*.
	\]
	Mà \( 2|A| = p - 1 \), nên \( A \cup (A + c) = \mathbb{F}_p^* \). Tức là tập các bình phương và không bình phương bị phân tách bởi phép cộng \( +c \). Điều này dẫn đến:
	\[
		\left( \frac{x}{p} \right) = - \left( \frac{x + c}{p} \right)\quad \text{với mọi } x \in \mathbb{F}_p^*.
	\]
	Tổng hai vế theo \( x \) trên \( \mathbb{F}_p^* \) dẫn đến tổng bằng 0, nhưng tổng bên trái bằng 0, nên vế phải cũng phải bằng 0. Điều này mâu thuẫn vì hàm Legendre không thể thay đổi dấu đều đặn như vậy.

	\textbf{Trường hợp 2:} \( \left( \frac{a}{p} \right) = -1 \)  
	Lúc này \( aA \) là tập các phần tử không phải bình phương. Khi đó \( B = aA + c \) cũng là tập gồm các phần tử không phải bình phương.  
	Nhưng số lượng phần tử không phải bình phương là \( \frac{p-1}{2} \), nên nếu dịch bởi \( +c \) mà không giao với \( A \), thì \( A \) và \( B \) là hai hoán vị rời nhau trong \( \mathbb{F}_p^* \). Như trên, tổng chẵn lẻ và tính chất đối xứng của Legendre dẫn đến mâu thuẫn.

	\textbf{Kết luận:} Trong cả hai trường hợp, giả thiết \( A \cap B = \emptyset \) dẫn đến mâu thuẫn. Vậy:
	\[
		\boxed{\text{Không tồn tại } a, c \in \mathbb{N} \text{ sao cho } \gcd(ac, p) = 1 \text{ và } A \cap B = \emptyset.}
	\]
\end{soln}

\footnotetext{\href{https://artofproblemsolving.com/community/c6h1140077p5349209}{Dựa theo lời giải của \textbf{Dukejukem} và \textbf{rafayaashary1}.}}

\end{document}