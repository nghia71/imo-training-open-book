\documentclass[../02-modular-arithmetic-b.tex]{subfiles}

\begin{document}

\begin{example*}[\gls{IND 2015 TST}4/P3]\label{example:IND-2015-TST4-P3}[\textbf{unrated}]\footnotemark
	Cho \( n > 1 \) là một số nguyên. Chứng minh rằng có vô hạn phần tử của dãy \( (a_k)_{k \ge 1} \), được xác định bởi
	\[
		a_k = \left\lfloor \frac{n^k}{k} \right\rfloor,
	\]
	là số lẻ. (Với số thực \( x \), ký hiệu \( \lfloor x \rfloor \) là phần nguyên của \( x \).)
\end{example*}

\begin{soln}(Cách 1)
	Nếu \( n \) là số lẻ, lấy \( k = n^m \) với \( m = 1, 2, \dots \). Khi đó:
	\[
		a_k = n^{n^m-m}\ \text{là số lẻ với mọi}\ m.
	\]
	
	Bây giờ giả sử \( n \) là số chẵn, viết \( n = 2t \) với \( t \ge 1 \) là số nguyên. Với mỗi \( m \ge 2 \), xét số nguyên:
	\[
		n^{2^m} - 2^m = 2^m \cdot \left(2^{2^m - m} \cdot t^{2^m} - 1 \right),
	\]
	vì \( 2^m - m > 1 \), nên biểu thức trong ngoặc có ước nguyên tố lẻ \( p \).
	
	Khi đó, đặt \( k = p \cdot 2^m \), ta có:
	\[
		n^k = (n^{2^m})^p \equiv (2^{m})^p = (2^p)^{m} \equiv 2^{m} \pmod{p},
	\]
	(vì \( 2^p \equiv 2 \pmod{p} \), theo định lý Fermat nhỏ).
	
	Mặt khác, từ bất đẳng thức:
	\[
		n^k - 2^m < n^k < n^k + 2^m(p - 1),
	\]
	ta suy ra phân số \( \frac{n^k}{k} \) nằm giữa hai số nguyên liên tiếp:
	\[
		\frac{n^k - 2^m}{p \cdot 2^m} \quad \text{và} \quad \frac{n^k + 2^m(p - 1)}{p \cdot 2^m}.
	\]
	
	Do đó:
	\[
		a_k = \left\lfloor \frac{n^k}{k} \right\rfloor = \frac{n^k - 2^m}{p \cdot 2^m}.
	\]
	
	Ta thấy:
	\[
		\frac{n^k - 2^m}{p \cdot 2^m} = \frac{\frac{n^k}{2^m} - 1}{p},
	\]
	và vì \( \frac{n^k}{2^m} - 1 \) là số nguyên lẻ (do \( k > m \)), nên \( a_k \) là số lẻ.
	
	Với mỗi \( m \) khác nhau ta thu được các \( k \) khác nhau vì số mũ của 2 trong phân tích thừa số nguyên tố của \( k \) khác nhau.
	Vậy có vô hạn giá trị \( k \) sao cho \( a_k \) là số lẻ.
\end{soln}

\footnotetext{\href{https://www.imo-official.org/problems/IMO2014SL.pdf}{IMO SL 2014 N4.}}

\end{document}