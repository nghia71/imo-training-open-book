\documentclass[../02-modular-arithmetic-b.tex]{subfiles}

\begin{document}

\begin{example*}[\gls{RUS 2015 TST}/D7/P5]\label{example:RUS-2015-D7-P5}\textbf{[unrated]}
	Cho số nguyên tố $p \ge 5$. Chứng minh rằng tồn tại số nguyên dương $a < p-1$ sao cho cả hai số $a^{p-1} - 1$ và $(a+1)^{p-1} - 1$ đều không chia hết cho $p^2$.
\end{example*}

\begin{soln}\footnotemark(Cách 1)
	Giả sử tồn tại $a$ sao cho $p^2 \mid a^{p-1} - 1$. Khi đó 
	\begin{claim*}
		$p^2 \nmid (p - a)^{p-1}$.
	\end{claim*}
	\begin{subproof}
		Thật vậy:
		\[
			(p - a)^{p-1} = \sum_{i=0}^{p-1} \binom{p-1}{i} p^{p-1-i} (-a)^i \equiv a^{p-1} - 1 - (p - 1)pa^{p-2} \equiv a^{p-1} - 1 + pa^{p-2} \Mod{p^2},
		\]
		do đó:
		\[
			(p - a)^{p-1} \equiv pa^{p-2} \not\equiv 0 \pmod{p^2}.
		\]
	\end{subproof}

	Giả sử phản chứng: \emph{với mọi} $1 \le a \le p-2$, ta luôn có:
	\[
		p^2 \mid a^{p-1} - 1 \quad \text{hoặc} \quad p^2 \mid (a+1)^{p-1} - 1.
	\]
	
	Vì $p^2 \mid 1^{p-1} - 1 = 0$, nên $a = 1$ thỏa mãn giả sử phản chứng.
	Do đó, theo khẳng định $p^2 \nmid (p-1)^{p-1} - 1$, từ đó theo giả sử phản chứng $p^2 \mid (p - 2)^{p-1} - 1.$

	Áp dụng khai triển nhị thức:
	\[
		(p - 2)^{p-1} \equiv 2^{p-1} - p(p-1)2^{p-2} \pmod{p^2}
	\]
	
	Vì $p^2 \mid (p - 2)^{p-1} - 1$, suy ra:
	\[
		p^2 \mid 2^{p-1} - 1 + p2^{p-2}.
	\]
	
	Nhân hai vế với $2^{p-1} + 1$, ta được:
	\[
		p^2 \mid (2^{p-1} + 1)(2^{p-1} - 1 + p2^{p-2}) = 4^{p-1} - 1 + p2^{p-2}(2^{p-1} + 1). \tag{1}
	\]

	Mặt khác, lần lượt theo giả sử phản chứng và khẳng định:
	\[
		p^2 \nmid 2^{p-1} - 1 \implies p^2 \mid 3^{p-1} - 1 \implies p^2 \nmid (p - 3)^{p-1} - 1 \implies p^2 \mid (p - 4)^{p-1} - 1.
	\]
	
	Suy ra:
	\[
		p^2 \mid 4^{p-1} - 1 + p4^{p-2}. \tag{2}
	\]

	So sánh (1) và (2), ta có:
	\[
		p \mid 4^{p-2} + 2^{p-2}.
	\]
	
	Nhưng, theo định lý Fermat nhỏ:
	\[
		4(4^{p-2} + 2^{p-2}) = 4^{p-1} + 2 \cdot 2^{p-1} \equiv 3 \pmod{p},\ \text{mâu thuẫn!}
	\]
	
	Vì vậy, giả thiết phản chứng sai.

	Vậy tồn tại ít nhất một số $a < p-1$ sao cho cả $a^{p-1} - 1$ và $(a+1)^{p-1} - 1$ đều không chia hết cho $p^2$.
\end{soln}

\footnotetext{\samepage \href{https://artofproblemsolving.com/community/c6h142625p33502347}{Lời giải của HoshimiyaMukuro.}}

\end{document}