\documentclass[../02-modular-arithmetic-b.tex]{subfiles}

\begin{document}

\begin{example*}[\gls{IRN 2015 MO}/N4]\label{example:IRN-2015-N4}[\textbf{unrated}]
	Cho các số nguyên dương \( a, b, c, d, k, \ell \) sao cho với mọi số tự nhiên \( n \), tập các thừa số nguyên tố của hai số
	\[
		n^k + a^n + c \quad \text{và} \quad n^\ell + b^n + d
	\]
	là giống nhau. Chứng minh rằng \( a = b \), \( c = d \), và \( k = \ell \).
\end{example*}

\begin{soln}(Cách 1)\footnotemark
	Giả sử tồn tại các số nguyên dương \( a, b, c, d, k, l \) thỏa mãn điều kiện đề bài, nhưng \( (a, b, c, d, k, l) \) không giống nhau.

	Ta xét số \( n \) có dạng:
	\[
		n = (kp - t)(p - 1) + s
	\]
	với \( p \) là một số nguyên tố đủ lớn, \( t, s \in \mathbb{N} \) cố định.
	
	Với cách chọn này, ta có:
	\[
		\begin{aligned}
			&n \equiv s \pmod{p - 1} \implies a^n \equiv a^s \pmod{p}, \quad b^n \equiv b^s \pmod{p}\\
			&n \equiv t + s \pmod{p} \implies n^k \equiv (t + s)^k \pmod{p}, \quad n^l \equiv (t + s)^l \pmod{p}
		\end{aligned}
	\]

	Vậy:
	\[
		n^k + a^n + c \equiv (t + s)^k + a^s + c \pmod{p},\ n^l + b^n + d \equiv (t + s)^l + b^s + d \pmod{p}
	\]
	
	Nếu \( p \mid n^k + a^n + c \), thì:
	\[
		(t + s)^k \equiv -a^s - c \pmod{p},\ (t + s)^l \equiv -b^s - d \pmod{p}
	\]
	
	Nâng hai vế lên lũy thừa bội chung:
	\[
		\begin{aligned}
			&((t + s)^k)^l = (-(a^s + c))^l,\ ((t + s)^l)^k = (-(b^s + d))^k\\
			&\implies (-(a^s + c))^l \equiv (-(b^s + d))^k \pmod{p} \implies p \mid (a^s + c)^l - (b^s + d)^k.
		\end{aligned}
	\]
	
	Giả sử:
	\[
		P(s) := (a^s + c)^l - (b^s + d)^k
	\]
	là đa thức khác hằng số. Thế thì \( P(s) \) có vô hạn giá trị khác 0 khi \( s \) thay đổi, và do đó có vô hạn thừa số nguyên tố.
	
	Nhưng mặt khác, với mỗi \( s \), ta có thể chọn \( t \) và \( p \) sao cho \( p \mid n^k + a^n + c \), nên \( p \mid P(s) \).
	Điều này chỉ có thể xảy ra nếu \( P(s) = 0 \) với mọi \( s \in \mathbb{N} \), tức là:
	\[
		(a^s + c)^l = (b^s + d)^k \quad \text{với mọi } s \in \mathbb{N}.
	\]
	
	
	Đặt \( j = \gcd(k, l) \), và viết \( k = j \cdot k', l = j \cdot l' \). Khi đó:
	\[
		(a^s + c)^{l'} = (b^s + d)^{k'}
		\implies a^s + c \text{ là } k'\text{-th power}, \quad b^s + d \text{ là } l'\text{-th power}
	\]
	
	Bây giờ, giả sử \( k' > 1 \). Với \( s \) đủ lớn, \( a^s \) sẽ chiếm ưu thế lớn hơn \( c \),
	nên \( a^s + c \) không thể là lũy thừa đúng của \( k' \) (do chênh lệch giữa hai số lũy thừa kề nhau rất lớn). Mâu thuẫn.
	
	Tương tự, giả sử \( l' > 1 \) cũng dẫn đến mâu thuẫn.
	Vì vậy, để phương trình đúng với mọi \( s \), ta phải có \( k = l \) và:
	\[
		a^s + c = b^s + d \implies a^s - b^s = d - c
	\]
	
	Nhưng với \( s \) đủ lớn và \( a \ne b \), hiệu \( a^s - b^s \) thay đổi và không thể cố định, mâu thuẫn với \( d - c \) cố định. Do đó, \( a = b \), suy ra \( c = d \).
\end{soln}

\footnotetext{\href{https://artofproblemsolving.com/community/c6h1138949p5340611}{Lời giải của mojyla222.}}
\end{document}