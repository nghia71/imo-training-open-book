\documentclass[../02-modular-arithmetic-b.tex]{subfiles}

\begin{document}

\begin{example*}[\gls{IND 2015 TST}3/P2]\label{example:IND-2015-TST3-P2}[\textbf{unrated}]\footnotemark
	Tìm tất cả các bộ ba \( (p, x, y) \) bao gồm một số nguyên tố \( p \) và hai số nguyên dương \( x \) và \( y \)
	sao cho \( x^{p-1} + y \) và \( x + y^{p-1} \) đều là lũy thừa của \( p \).
\end{example*}

\begin{soln}(Cách 1)
	Với \( p = 2 \), rõ ràng mọi cặp số nguyên dương \( x, y \) sao cho tổng của chúng là một lũy thừa của 2 đều thỏa mãn điều kiện bài toán.

	Do đó, từ đây giả sử \( p > 2 \), và ta đặt \( a, b \) là các số nguyên dương sao cho:
	\[
		x^{p-1} + y = p^a,\quad x + y^{p-1} = p^b
	\]
	
	Giả sử không mất tính tổng quát rằng \( x \le y \), do đó:
	\[
		p^a = x^{p-1} + y \le x + y^{p-1} = p^b \implies a \le b \implies p^a \mid p^b
	\]
	
	Xét:
	\[
		p^b = y^{p-1} + x = (p^a - x^{p-1})^{p-1} + x
	\]
	
	Lấy đồng dư theo modulo \( p^a \), và lưu ý răng $p-1$ chẵn, ta được:
	\[
		0 \equiv x^{(p-1)^2} + x \Mod{p^a}
	\]
	
	Nếu \( p \mid x \), thì \( p^a \mid x \), vì $x^{(p-1)^2-1} + 1$ không chia hết cho $p$.
	Tuy nhiên điều này là không thể vì \( x \le x^{p-1} < p^a \). Vậy \( p \nmid x \), do đó:
	\[
		p^a \mid x^{(p-1)^2-1} + 1 = x^{p{p-2}} + 1.
	\]
	
	Áp dụng định lý Fermat nhỏ, ta có \( x^{(p-1)^2} \equiv 1 \Mod{p} \), dẫn đến \( p \mid x + 1 \).
	
	Gọi \( p^r \) là lũy thừa cao nhất của \( p \) chia hết \( x + 1 \). Sử dụng khai triển nhị thức, ta viết:
	\[
		x^{p(p - 2)} = \sum_{k=0}^{p(p - 2)} \binom{p(p - 2)}{k} (-1)^{p(p - 2)-k}(x+1)^k
	\]
	
	Tất cả các hạng tử trong tổng đều chia hết cho $p^{3r}$ nên chia hết cho \( p^{r+2} \), ngoại trừ:
    \begin{itemize}[topsep=0pt, partopsep=0pt, itemsep=0pt]
        \item Hạng tử với \( k = 2 \): giá trị là $-\frac{p(p-2)(p^2-2p-1)}{2}(x+1)^2$, chia hết cho \( p^{2r + 1} \) nên chia hết cho \( p^{r + 2} \)
        \item Hạng tử với \( k = 1 \): giá trị là $p(p-2)(x+1)$, chia hết cho \( p^{r + 1} \) nhưng không chia hết cho \( p^{r + 2} \)
        \item Hạng tử với \( k = 0 \): giá trị là \( -1 \), 
    \end{itemize}
	
	Vậy lũy thừa lớn nhất của \( p \) chia hết \( x^{p(p-2)} + 1 \) là \( p^{r+1} \).

	Nhưng ta đã biết \( p^a \mid x^{p(p-2)} + 1 \), nên suy ra \( a \le r + 1 \).
	
	Mặt khác, \( p^r \le x + 1 \le x^{p-1} + y = p^a \), do đó \( a = r \) hoặc \( a = r + 1 \).
	
	Nếu \( a = r \), thì bất đẳng thức chỉ xảy ra khi \( x = y = 1 \), điều này mâu thuẫn với \( p > 2 \). Vậy \( a = r + 1 \), và do \( p^r \le x + 1 \), nên:
	\[
		x = \frac{x^2 + x}{x + 1} \le \frac{x^{p-1} + y}{x + 1} = \frac{p^a}{x + 1} \le \frac{p^a}{p^r} = p \implies x = p - 1
	\]
	
	Do đó \( r = 1 \), \( a = 2 \). Nếu \( p \ge 5 \), thì:
	\[
		p^a = x^{p - 1} + y > (p - 1)^4 = (p^2 - 2p + 1)^2 > (3p)^2 > p^2 = p^a,\ \text{mâu thuẫn.}
	\]
	
	Vậy trường hợp duy nhất là \( p = 3 \), và \( x = 2 \), \( y = p^a - x^{p - 1} = 9 - 4 = 5 \), là lời giải.
\end{soln}

\footnotetext{\href{https://www.imo-official.org/problems/IMO2014SL.pdf}{IMO SL 2014 N5.}}

\end{document}