\documentclass[../../imo-training-open-book.tex]{subfiles}

\begin{document}

\section{Lý thuyết}

\begin{theorem*}[Định lý Fermat về tổng hai số chính phương]
    \label{theorem:fermat-sums-of-two-squares-theorem}
    Một số nguyên tố lẻ \( p \) có thể được biểu diễn dưới dạng tổng của hai số chính phương nếu và chỉ nếu \( p \equiv 1 \pmod{4} \),
    tức là tồn tại các số nguyên \( x, y \) sao cho
    \[
        p = x^2 + y^2.
    \]
\end{theorem*}

\newpage

\section{Các ví dụ}

\subfile{./08-quadratic-residues/chn-2015-tst2-d2-p3.tex} \newpage
\subfile{./08-quadratic-residues/irn-2015-mo-n3.tex} \newpage

\section{Bài tập}

\newpage

\end{document}