\documentclass[../imo-training-open-book.tex]{subfiles}

\begin{document}

Lời nói đầu

Cuốn sách này được biên soạn dành cho giáo viên và học sinh luyện thi Đội tuyển Quốc gia Việt Nam dự thi IMO.
Tài liệu tập hợp \textit{các bài toán mới trong vòng 10 năm trở lại đây} từ các kỳ thi quan trọng như IMO Shortlist,
các cuộc thi quốc tế uy tín như MEMO, BMO, APMO, EGMO, cũng như các kỳ thi quốc gia của 20 nước hàng đầu thế giới.

Mỗi bài toán được \textit{xếp hạng theo thang độ khó MOHS}, đi kèm với \textit{danh sách các định lý, bổ đề, hằng đẳng thức quan trọng} cần thiết cho lời giải.
Các yếu tố này được liên kết trong một hệ thống đồ thị tri thức, giúp người đọc dễ dàng tra cứu và hiểu rõ mối liên hệ giữa các công cụ toán học.
Ngoài ra, mỗi bài toán còn được gắn \textit{thẻ thông tin chi tiết} về kỳ thi (năm, vòng), giúp thuận tiện cho việc tìm kiếm và tham khảo.

Để hỗ trợ người học, mỗi bài toán có một \textit{mã định danh duy nhất} (UUID), kèm theo \textit{gợi ý} khi gặp khó khăn.
Nếu có nhiều cách giải, tất cả sẽ được trình bày các chuyên đề liên quan đến cách giải để giúp người đọc mở rộng tư duy.

Cấu trúc sách gồm bốn phần chính tương ứng với bốn lĩnh vực quan trọng của toán học thi đấu: Đại số, Tổ hợp, Hình học và Số học.
Mỗi phần chia thành các chương theo từng chuyên đề cụ thể với các bài toán liên quan.

Đây là một cuốn sách \textit{mở, luôn được cập nhật và có sẵn trên Internet} để bất kỳ ai cũng có thể truy cập.
Người dùng có thể đóng góp bằng cách đề xuất bài toán mới hoặc thay đổi mức độ khó, gợi ý,
hoặc thêm lời giải mới cho bài toán bằng cách gửi \textit{một tệp duy nhất theo định dạng LaTeX quy định}.
Việc đóng góp tập trung vào nội dung mà không cần lo lắng về định dạng, tổ chức, mã LaTeX hay quy trình xuất bản.

Toàn bộ quá trình này được giám sát bởi các nhân sự được ủy quyền từ Hội Toán Học Việt Nam và Tạp chí Pi,
nhằm đảm bảo chất lượng và tính nhất quán của tài liệu.

Chúng tôi hy vọng tài liệu này sẽ trở thành một nguồn tham khảo hữu ích,
giúp giáo viên và học sinh tiến xa hơn trong hành trình chinh phục các kỳ thi toán quốc tế.

\flushright Ban Biên soạn

\end{document}