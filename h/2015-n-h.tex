\documentclass[./m.tex]{subfiles}

\begin{document}

\section{2015}

\begin{hint*}[\autoref{problem:BGR-2015-EGMO-TST-P4}](\gls{BGR 2015 EGMO TST}/P4)[\footnotemark]
	Thử chọn \((x,y)\) vừa đủ điều kiện chia, kết hợp với sự nguyên tố cùng nhau của \((x,y)\).  
	Xem định lý Euclid về vô hạn số nguyên tố để tạo dãy vô hạn đáp ứng yêu cầu.
	[\textbf{\nameref{definition:25M}}]
\end{hint*}

\footnotetext{\href{https://artofproblemsolving.com/community/c3943}{IMO SL 1992 P1.}}

\begin{hint*}[\autoref{problem:BGR-2015-EGMO-TST-P6}](\gls{BGR 2015 EGMO TST}/P6)
	Thử xây dựng một họ các bộ \( n \) số (có thể đồng dư hoặc biến thiên theo tham số) sao cho tổng lập phương thu được là \( (\text{một giá trị tham số})^3 \).  
	Các ví dụ quen thuộc là các “nhóm” số mà tổng lập phương khéo léo triệt tiêu và cộng dồn thành một khối lập phương.
	[\textbf{\nameref{definition:25M}}]
\end{hint*}

\vspace{1em}

\begin{hint*}[\autoref{problem:BMO-2015-P4}]
	Hãy chọn \( d \) có dạng \( 5(4k + 3) \) sao cho \( d \) không là chính phương và có ước nguyên tố \( \equiv 3 \Mod{4} \),
	rồi xét giá trị của \( n^2 d \) giữa hai bình phương liên tiếp.
\end{hint*}

\vspace{1em}

\begin{hint*}[\autoref{problem:BxMO-2015-P3}]
	Lưu ý rằng $381 = 3 \cdot 127$, $3811 = 37 \cdot 103$, và $38111 = 23 \cdot 1657.$
	Phân tích theo số lượng chữ số của số đã cho theo modulo \(3\).
\end{hint*}

\vspace{1em}

\begin{hint*}[\autoref{problem:CAN-2015-QRC-P3}]
	Trong tổng của các hoán vị vai trò của các chữ số là như nhau.
\end{hint*}

\vspace{1em}

\begin{hint*}[\autoref{problem:CHN-2015-TST1-D1-P2}]
	Giả sử ngược lại: từ một chỉ số \( K \) trở đi, các giá trị \( \operatorname{lcm}(a_k, a_{k+1}) \) đều không vượt quá \( ck \).  
	Chứng minh $\frac{1}{a_k} + \frac{1}{a_{k+1}} \ge \frac{3}{ck},$ rồi tìm cận dưới cho tổng các số \( \frac{1}{a_k} + \frac{1}{a_{k+1}} \).
\end{hint*}

\vspace{1em}

\begin{hint*}[\autoref{problem:CHN-2015-TST1-D2-P2}]
	Ta xét tập giá trị mà biểu thức \( ax + by + cz \) có thể đạt được khi \( x, y, z \in [-n, n] \), và so sánh với số lượng bộ ba như vậy.  
\end{hint*}

\vspace{1em}

\begin{hint*}[\autoref{problem:EGMO-2015-P3}]
	Giải quyết các trường hợp biên: $\min_{i} a_i \ge n^m - 1$ qua việc chỉ ra sự tồn tại các bộ ($b_i$) cụ thể.
	Sau đó, giả sử phản chứng với \(b_1, \ldots, b_m \in \{1, \ldots, n\}\) sao cho có một cặp $\gcd(a_i + b_i, a_j + b_j) = 1.$
\end{hint*}

\vspace{1em}

\begin{hint*}[\autoref{problem:EGMO-2015-P3-strong}]
	Sử dụng phương pháp trong bài \autoref{problem:EGMO-2015-P3}. Giới hạn \( n^{2^{m-1}} \) có thể được cải thiện thêm.
\end{hint*}

\vspace{1em}

\begin{hint*}[\autoref{problem:GBR-2015-TST-N3-P3}](\gls{GBR 2015 TST}/N3/P3)
    Mỗi đoạn con là khoảng giữa hai số nguyên không chia hết cho bất kỳ \( a_i \), tức là nằm ngoài tập hợp các bội số.
	\textbf{[\nameref{definition:20M}]}\index{20M}
\end{hint*}

\vspace{1em}

\begin{hint*}[\autoref{problem:GER-2015-MO-P2}]
	Phân tích \(\Mod{4}\) để xác định \(n\) là \(0\) hoặc \(1 \Mod{4}\). Kiểm tra các trường hợp cụ thể và loại trừ \(n = 4\).
	Xây dựng các ví dụ để chứng minh các số \(n\) thỏa mãn là những số \(n \equiv 0\) hoặc \(1 \Mod{4}\), ngoại trừ \(n = 4\).
\end{hint*}

\vspace{1em}

\begin{hint*}[\autoref{problem:HUN-2015-TST-KMA-633}](\gls{HUN 2015 TST}/KMA/633)
    Hãy sắp xếp các số đã cho theo thứ tự tăng dần, ước lượng kích thước của bội chung nhỏ nhất,
    và tìm cách chọn bốn số sở hữu lũy thừa nguyên tố lớn vượt mức \( n^{3.99} \).
	\textbf{[\nameref{definition:20M}]}\index{20M}
\end{hint*}

\vspace{1em}

\begin{hint*}[\autoref{problem:IND-2015-N2}]
	Biểu diễn thập phân vô hạn tuần hoàn của \( \frac{1}{n} \) gồm phần không tuần hoàn (các chữ số đầu tiên) và phần tuần hoàn.
	Phần không tuần hoàn chỉ xuất hiện nếu mẫu số chứa thừa số 2 hoặc 5.
\end{hint*}

\vspace{1em}

\begin{hint*}[\autoref{problem:IND-2015-N6}]
	Ta xét phần dư của số chính phương theo modulo 12 (chỉ có thể là \( 0, 1, 4, 9 \)),
	sau đó áp dụng nguyên lý Dirichlet để xét các phần dư xuất hiện nhiều lần.  
\end{hint*}

\vspace{1em}

\begin{hint*}[\autoref{problem:IND-2015-TST2-P1}]
	Tìm ngưỡng qua việc tính tổng các số \( 2^n - 2^k \), sau đó
	dùng quy nạp để chứng minh mọi số lớn hơn ngưỡng này đều biểu diễn được.
\end{hint*}

\vspace{1em}

\begin{hint*}[\autoref{problem:IRN-2015-N2}]
	Nếu có quan hệ chia hết giữa các phần tử trong tập \( M_n \) điều đó dẫn đến một bất đẳng thức giữa giá trị nhỏ nhất ở bước hiện tại
	và sai biệt lớn nhất giữa các phần tử của tập trước đó.
\end{hint*}

\vspace{1em}

\begin{hint*}[\autoref{problem:IRN-2015-TST-D3-P2}]
	Đặt \( b_n = \dfrac{a_n}{\text{lcm}(a_{n-2}, a_{n-3})} \), chứng minh dãy \( b_n \) nguyên và giảm dần.
\end{hint*}

\vspace{1em}

\begin{hint*}[\autoref{problem:JPN-2015-MO-P1}](\gls{JPN 2015 MO}1/P1)
    Hãy phân tích mẫu số \( n^3 + n^2 + n + 1 \) thành nhân tử, sau đó kiểm tra xem khi nào nó chia hết \( 10^n \).
	\textbf{[\nameref{definition:10M}]}
\end{hint*}

\vspace{1em}

\begin{hint*}[\autoref{problem:KOR-2015-MO-P8}]
	Cách 1: Khai thác tính đối xứng trong phần dư modulo và viết lại tổng theo số nhỏ hơn và lớn hơn \( \frac{n}{2} \).

	Cách 2: Dựa trên bất đẳng thức về tổng trung bình và kiểm tra các cấu hình của \( n \) theo modulo 4.
\end{hint*}

\vspace{1em}

\begin{hint*}[\autoref{problem:MEMO-2015-I-P4}]
	Đặt \( m = kn + r \), rồi phân tích tử số theo \( a^n + b^n \).
	Để phân số là số nguyên, cần \( a^r + b^r \) chia hết cho \( a^n + b^n \), từ đó tìm điều kiện cho \( r \).
\end{hint*}

\vspace{1em}

\begin{hint*}[\autoref{problem:ROU-2015-MO-G10-P2}](\gls{ROU 2015 MO}/G10/P2)
	Xem xét những giá trị của \(f(i)\) phải chứa đủ các ước nguyên tố của \(n\).
    Lưu ý nếu một trong số các \(f(i)\) luôn chọn được bội của tất cả số nguyên tố \((p_1, p_2, \dots)\) thì tích sẽ chia hết \(n\).
	\textbf{[\nameref{definition:20M}]}
\end{hint*}

\vspace{1em}

\begin{hint*}[\autoref{problem:ROU-2015-TST-D1-P4}]
	Sử dụng biểu thức \(a^2 = a + t\), rồi phân tích phần thập phân \(\varepsilon_n = a n - \lfloor a n \rfloor\) để suy ra sự phân bố đều của hiệu.
\end{hint*}

\vspace{1em}

\begin{hint*}[\autoref{problem:ROU-2015-TST-D2-P1}]
	Sử dụng đẳng thức \(\displaystyle \sum_{k=1}^{n} a^{\gcd(k,n)} = \sum_{d \mid n} \phi(d) a^{n/d}\) và áp dụng định lý Euler cùng với nguyên lý đồng dư Trung Hoa.
\end{hint*}

\vspace{1em}

\begin{hint*}[\autoref{problem:ROU-2015-TST-D3-P3}]
	Khai thác tính chất chia hết của \( \mathrm{lcm} \) và lập mâu thuẫn bằng cách xét cấu trúc của dãy liên tiếp.
\end{hint*}

\vspace{1em}

\begin{hint*}[\autoref{problem:RUS-2015-TST-D10-P2}]
	Sử dụng \textit{tổng} và \textit{tích} toàn bộ dãy modulo \( p \),
	rồi chuyển điều kiện về dạng \( AB \equiv -1 \Mod{p} \) với \( A \) là tổng và \( B \) là tích một tập con.
\end{hint*}

\vspace{1em}

\begin{hint*}[\autoref{problem:THA-2015-MO-P1}]
	Trừ hai công thức liên tiếp rồi khai thác tính nguyên tố của \( p \) để dẫn đến mâu thuẫn về chia hết.
\end{hint*}

\vspace{1em}

\begin{hint*}[\autoref{problem:TWN-2015-TST2-Q2-P1}]
	Thử viết \( n \) trong cơ số 2 rồi phân tích biểu thức \( \left\lfloor \tfrac{n + 2^{k-1}}{2^k} \right\rfloor \).  
    Có thể tách từng chữ số nhị phân và xem cách các số hạng góp phần vào tổng.
\end{hint*}

\end{document}