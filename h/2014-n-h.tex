\documentclass[./m.tex]{subfiles}

\begin{document}

\section{2014}

\begin{hint*}[\autoref{problem:ROU-2014-MO-G9-P2}](\gls{ROU 2014 MO}/G9/P2)
	Do \( a+\sqrt{a} > a \) và \( a-\sqrt{a} < a \), nên phần thập phân của chúng không thể trùng nhau khi nhân với số nguyên.
	Hiệu \( m(a+\sqrt{a}) - n(a-\sqrt{a}) \) không thể là số nguyên, vì vậy phần nguyên của chúng cũng khác nhau.
	\textbf{[\nameref{definition:10M}]}
\end{hint*}

\vspace{1em}

\begin{hint*}[\autoref{problem:ROU-2014-MO-G10-P1}](\gls{ROU 2014 MO}/G10/P1)
	Với mỗi số chia \( d \), tìm những giá trị \( k \) thỏa \( d \mid k \), \( k \le d^2 \), và tổng lại theo \( d \) thay vì \( k \).
	\textbf{[\nameref{definition:20M}]}
\end{hint*}

\end{document}