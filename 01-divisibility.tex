\documentclass[../imo-training-open-book.tex]{subfiles}

\begin{document}

\newpage

\section{Các ví dụ}

\subfile{./01-divisibility/bxmo-2015-p3.tex} \newpage
\subfile{./01-divisibility/can-2015-qrc-p3.tex} \newpage
\subfile{./01-divisibility/chn-2015-tst1-d1-p2.tex} \newpage
\subfile{./01-divisibility/chn-2015-tst1-d2-p2.tex} \newpage
\subfile{./01-divisibility/egmo-2015-p3.tex} \newpage
\subfile{./01-divisibility/imo-2023-p1.tex} \newpage
\subfile{./01-divisibility/ind-2015-mo-p2.tex} \newpage
\subfile{./01-divisibility/ind-2015-mo-p6.tex} \newpage
\subfile{./01-divisibility/ind-2015-tst2-p1.tex} \newpage
\subfile{./01-divisibility/irn-2015-mo-n2.tex} \newpage
\subfile{./01-divisibility/irn-2015-tst-d3-p2.tex} \newpage
\subfile{./01-divisibility/kor-2015-mo-p8.tex} \newpage
\subfile{./01-divisibility/memo-2015-i-p4.tex} \newpage
\subfile{./01-divisibility/rus-2015-tst-d10-p2.tex} \newpage
\subfile{./01-divisibility/tha-2015-mo-p1.tex} \newpage

\section{Bài tập}

\subfile{./01-divisibility/gbr-2015-tst-n3-p3.tex}
\subfile{./01-divisibility/hun-2015-tst-kma-633.tex}
\subfile{./01-divisibility/jpn-2015-mo-p1.tex}
\subfile{./01-divisibility/rou-2014-mo-g10-p1.tex}
\subfile{./01-divisibility/rou-2015-mo-g10-p2.tex}

\newpage

\section{Định lý, bổ đề, và hằng đẳng thức}

\begin{theorem*}[Tính chia hết của chuỗi chữ số \(1\)]
	Gọi \( R_n = \underbrace{11\ldots1}_n \) là số gồm \( n \) chữ số \( 1 \) (repunit). Khi đó:
	\begin{itemize}[topsep=0pt, partopsep=0pt, itemsep=0pt]
		\item Nếu \( n \equiv 0 \pmod{3} \) thì \( R_n \) chia hết cho \( 3 \).
		\item Nếu \( n \equiv 1 \pmod{3} \) thì \( R_n \) chia hết cho \( 3 \).
		\item Nếu \( n \equiv 2 \pmod{3} \) thì \( R_n \) chia hết cho \( 37 \).
	\end{itemize}
\end{theorem*}

\begin{lemma*}[Trung bình cộng của các hoán vị ba chữ số]\label{lemma:perm-average}
	Giả sử \( a, b, c \in \{0,1,\ldots,9\} \) là ba chữ số phân biệt. Khi liệt kê tất cả 6 hoán vị ba chữ số khác nhau từ \( a, b, c \), tổng của chúng bằng \( 222(a + b + c) \), do đó trung bình cộng bằng \( 37(a + b + c) \).
\end{lemma*}

\newpage

\end{document}